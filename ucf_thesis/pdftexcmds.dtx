% \iffalse meta-comment
%
% Copyright (C) 2007 by Heiko Oberdiek <oberdiek@uni-freiburg.de>
%
% This work may be distributed and/or modified under the
% conditions of the LaTeX Project Public License, either
% version 1.3 of this license or (at your option) any later
% version. The latest version of this license is in
%    http://www.latex-project.org/lppl.txt
% and version 1.3 or later is part of all distributions of
% LaTeX version 2005/12/01 or later.
%
% This work has the LPPL maintenance status "maintained".
%
% This Current Maintainer of this work is Heiko Oberdiek.
%
% This work consists of the main source file pdftexcmds.dtx
% and the derived files
%    pdftexcmds.sty, pdftexcmds.pdf, pdftexcmds.ins, pdftexcmds.drv,
%    oberdiek.pdftexcmds.lua, pdftexcmds.lua.
%
% Distribution:
%    CTAN:macros/latex/contrib/oberdiek/pdftexcmds.dtx
%    CTAN:macros/latex/contrib/oberdiek/pdftexcmds.pdf
%
% Unpacking:
%    (a) If pdftexcmds.ins is present:
%           tex pdftexcmds.ins
%    (b) Without pdftexcmds.ins:
%           tex pdftexcmds.dtx
%    (c) If you insist on using LaTeX
%           latex \let\install=y% \iffalse meta-comment
%
% Copyright (C) 2007 by Heiko Oberdiek <oberdiek@uni-freiburg.de>
%
% This work may be distributed and/or modified under the
% conditions of the LaTeX Project Public License, either
% version 1.3 of this license or (at your option) any later
% version. The latest version of this license is in
%    http://www.latex-project.org/lppl.txt
% and version 1.3 or later is part of all distributions of
% LaTeX version 2005/12/01 or later.
%
% This work has the LPPL maintenance status "maintained".
%
% This Current Maintainer of this work is Heiko Oberdiek.
%
% This work consists of the main source file pdftexcmds.dtx
% and the derived files
%    pdftexcmds.sty, pdftexcmds.pdf, pdftexcmds.ins, pdftexcmds.drv,
%    oberdiek.pdftexcmds.lua, pdftexcmds.lua.
%
% Distribution:
%    CTAN:macros/latex/contrib/oberdiek/pdftexcmds.dtx
%    CTAN:macros/latex/contrib/oberdiek/pdftexcmds.pdf
%
% Unpacking:
%    (a) If pdftexcmds.ins is present:
%           tex pdftexcmds.ins
%    (b) Without pdftexcmds.ins:
%           tex pdftexcmds.dtx
%    (c) If you insist on using LaTeX
%           latex \let\install=y% \iffalse meta-comment
%
% Copyright (C) 2007 by Heiko Oberdiek <oberdiek@uni-freiburg.de>
%
% This work may be distributed and/or modified under the
% conditions of the LaTeX Project Public License, either
% version 1.3 of this license or (at your option) any later
% version. The latest version of this license is in
%    http://www.latex-project.org/lppl.txt
% and version 1.3 or later is part of all distributions of
% LaTeX version 2005/12/01 or later.
%
% This work has the LPPL maintenance status "maintained".
%
% This Current Maintainer of this work is Heiko Oberdiek.
%
% This work consists of the main source file pdftexcmds.dtx
% and the derived files
%    pdftexcmds.sty, pdftexcmds.pdf, pdftexcmds.ins, pdftexcmds.drv,
%    oberdiek.pdftexcmds.lua, pdftexcmds.lua.
%
% Distribution:
%    CTAN:macros/latex/contrib/oberdiek/pdftexcmds.dtx
%    CTAN:macros/latex/contrib/oberdiek/pdftexcmds.pdf
%
% Unpacking:
%    (a) If pdftexcmds.ins is present:
%           tex pdftexcmds.ins
%    (b) Without pdftexcmds.ins:
%           tex pdftexcmds.dtx
%    (c) If you insist on using LaTeX
%           latex \let\install=y% \iffalse meta-comment
%
% Copyright (C) 2007 by Heiko Oberdiek <oberdiek@uni-freiburg.de>
%
% This work may be distributed and/or modified under the
% conditions of the LaTeX Project Public License, either
% version 1.3 of this license or (at your option) any later
% version. The latest version of this license is in
%    http://www.latex-project.org/lppl.txt
% and version 1.3 or later is part of all distributions of
% LaTeX version 2005/12/01 or later.
%
% This work has the LPPL maintenance status "maintained".
%
% This Current Maintainer of this work is Heiko Oberdiek.
%
% This work consists of the main source file pdftexcmds.dtx
% and the derived files
%    pdftexcmds.sty, pdftexcmds.pdf, pdftexcmds.ins, pdftexcmds.drv,
%    oberdiek.pdftexcmds.lua, pdftexcmds.lua.
%
% Distribution:
%    CTAN:macros/latex/contrib/oberdiek/pdftexcmds.dtx
%    CTAN:macros/latex/contrib/oberdiek/pdftexcmds.pdf
%
% Unpacking:
%    (a) If pdftexcmds.ins is present:
%           tex pdftexcmds.ins
%    (b) Without pdftexcmds.ins:
%           tex pdftexcmds.dtx
%    (c) If you insist on using LaTeX
%           latex \let\install=y\input{pdftexcmds.dtx}
%        (quote the arguments according to the demands of your shell)
%
% Documentation:
%    (a) If pdftexcmds.drv is present:
%           latex pdftexcmds.drv
%    (b) Without pdftexcmds.drv:
%           latex pdftexcmds.dtx; ...
%    The class ltxdoc loads the configuration file ltxdoc.cfg
%    if available. Here you can specify further options, e.g.
%    use A4 as paper format:
%       \PassOptionsToClass{a4paper}{article}
%
%    Programm calls to get the documentation (example):
%       pdflatex pdftexcmds.dtx
%       makeindex -s gind.ist pdftexcmds.idx
%       pdflatex pdftexcmds.dtx
%       makeindex -s gind.ist pdftexcmds.idx
%       pdflatex pdftexcmds.dtx
%
% Installation:
%    TDS:tex/generic/oberdiek/pdftexcmds.sty
%    TDS:scripts/oberdiek/oberdiek.pdftexcmds.lua
%    TDS:scripts/oberdiek/pdftexcmds.lua
%    TDS:doc/latex/oberdiek/pdftexcmds.pdf
%    TDS:source/latex/oberdiek/pdftexcmds.dtx
%
%<*ignore>
\begingroup
  \def\x{LaTeX2e}%
\expandafter\endgroup
\ifcase 0\ifx\install y1\fi\expandafter
         \ifx\csname processbatchFile\endcsname\relax\else1\fi
         \ifx\fmtname\x\else 1\fi\relax
\else\csname fi\endcsname
%</ignore>
%<*install>
\input docstrip.tex
\Msg{************************************************************************}
\Msg{* Installation}
\Msg{* Package: pdftexcmds 2007/12/12 v0.3 LuaTeX support for pdfTeX utility functions (HO)}
\Msg{************************************************************************}

\keepsilent
\askforoverwritefalse

\let\MetaPrefix\relax
\preamble

This is a generated file.

Copyright (C) 2007 by Heiko Oberdiek <oberdiek@uni-freiburg.de>

This work may be distributed and/or modified under the
conditions of the LaTeX Project Public License, either
version 1.3 of this license or (at your option) any later
version. The latest version of this license is in
   http://www.latex-project.org/lppl.txt
and version 1.3 or later is part of all distributions of
LaTeX version 2005/12/01 or later.

This work has the LPPL maintenance status "maintained".

This Current Maintainer of this work is Heiko Oberdiek.

This work consists of the main source file pdftexcmds.dtx
and the derived files
   pdftexcmds.sty, pdftexcmds.pdf, pdftexcmds.ins, pdftexcmds.drv,
   oberdiek.pdftexcmds.lua, pdftexcmds.lua.

\endpreamble
\let\MetaPrefix\DoubleperCent

\generate{%
  \file{pdftexcmds.ins}{\from{pdftexcmds.dtx}{install}}%
  \file{pdftexcmds.drv}{\from{pdftexcmds.dtx}{driver}}%
  \usedir{tex/generic/oberdiek}%
  \file{pdftexcmds.sty}{\from{pdftexcmds.dtx}{package}}%
}
\def\MetaPrefix{-- }
\def\defaultpostamble{%
  \MetaPrefix^^J%
  \MetaPrefix\space End of File `\outFileName'.%
}
\def\currentpostamble{\defaultpostamble}%
\generate{%
  \usedir{scripts/oberdiek}%
  \file{oberdiek.pdftexcmds.lua}{\from{pdftexcmds.dtx}{lua}}%
  \file{pdftexcmds.lua}{\from{pdftexcmds.dtx}{lua}}%
}

\obeyspaces
\Msg{************************************************************************}
\Msg{*}
\Msg{* To finish the installation you have to move the following}
\Msg{* file into a directory searched by TeX:}
\Msg{*}
\Msg{*     pdftexcmds.sty}
\Msg{*}
\Msg{* And install the following script files:}
\Msg{*}
\Msg{*     oberdiek.pdftexcmds.lua, pdftexcmds.lua}
\Msg{*}
\Msg{* To produce the documentation run the file `pdftexcmds.drv'}
\Msg{* through LaTeX.}
\Msg{*}
\Msg{* Happy TeXing!}
\Msg{*}
\Msg{************************************************************************}

\endbatchfile
%</install>
%<*ignore>
\fi
%</ignore>
%<*driver>
\NeedsTeXFormat{LaTeX2e}
\ProvidesFile{pdftexcmds.drv}%
  [2007/12/12 v0.3 LuaTeX support for pdfTeX utility functions (HO)]%
\documentclass{ltxdoc}
\usepackage{holtxdoc}[2007/11/11]
\usepackage{paralist}
\begin{document}
  \DocInput{pdftexcmds.dtx}%
\end{document}
%</driver>
% \fi
%
% \CheckSum{552}
%
% \CharacterTable
%  {Upper-case    \A\B\C\D\E\F\G\H\I\J\K\L\M\N\O\P\Q\R\S\T\U\V\W\X\Y\Z
%   Lower-case    \a\b\c\d\e\f\g\h\i\j\k\l\m\n\o\p\q\r\s\t\u\v\w\x\y\z
%   Digits        \0\1\2\3\4\5\6\7\8\9
%   Exclamation   \!     Double quote  \"     Hash (number) \#
%   Dollar        \$     Percent       \%     Ampersand     \&
%   Acute accent  \'     Left paren    \(     Right paren   \)
%   Asterisk      \*     Plus          \+     Comma         \,
%   Minus         \-     Point         \.     Solidus       \/
%   Colon         \:     Semicolon     \;     Less than     \<
%   Equals        \=     Greater than  \>     Question mark \?
%   Commercial at \@     Left bracket  \[     Backslash     \\
%   Right bracket \]     Circumflex    \^     Underscore    \_
%   Grave accent  \`     Left brace    \{     Vertical bar  \|
%   Right brace   \}     Tilde         \~}
%
% \GetFileInfo{pdftexcmds.drv}
%
% \title{The \xpackage{pdftexcmds} package}
% \date{2007/12/12 v0.3}
% \author{Heiko Oberdiek\\\xemail{oberdiek@uni-freiburg.de}}
%
% \maketitle
%
% \begin{abstract}
% \LuaTeX\ provides most of the commands of \pdfTeX\ 1.40. However
% a number of utility functions are removed. This package tries to fill
% the gap and implements some of the missing primitive using Lua.
% \end{abstract}
%
% \tableofcontents
%
% \section{Documentation}
%
% Some primitives of \pdfTeX\ are not defined by \LuaTeX.
% This package implements macro based solutions using Lua code
% for the following missing \pdfTeX\ primitives;
% \begin{compactitem}
% \item \cs{pdfstrcmp}
% \item \cs{pdfunescapehex}
% \item \cs{pdfescapehex}
% \item \cs{pdfescapename}
% \item \cs{pdfescapestring}
% \item \cs{pdffilesize}
% \item \cs{pdffilemoddate}
% \item \cs{pdffiledump}
% \item \cs{pdfmdfivesum}
% \item |\immediate\write18|
% \end{compactitem}
% The original names of the primitives cannot be used:
% \begin{itemize}
% \item
% The syntax for their arguments cannot easily
% simulated by macros. The primitives using key words
% such as |file| (\cs{pdfmdfivesum}) or |offset| and |length|
% (\cs{pdffiledump}) and uses \meta{general text} for the other
% arguments. Using token registers assignments, \meta{general text} could
% be catched. However, the simulated primitives are expandable
% and register assignments would destroy this important property.
% (\meta{general text} allows something like |\expandafter\bgroup ...}|.)
% \item
% The original primitives can be expanded using one expansion step.
% The new macros need two expansion steps because of the additional
% macro expansion. Example:
% \begin{quote}
%   |\expandafter\foo\pdffilemoddate{file}|\\
%   vs.
%   |\expandafter\expandafter\expandafter\foo\pdf@filemoddate{file}|.
% \end{quote}
% \end{itemize}
%
% \LuaTeX\ isn't stable yet and thus the status of this package is
% \emph{experimental}. Feedback is welcome.
%
% \subsection{General principles}
%
% \begin{description}
% \item[Naming convention:]
%   Usually this package defines a macro |\pdf@|\meta{cmd} if
%   \pdfTeX\ provides |\pdf|\meta{cmd}.
% \item[Arguments:] The order of arguments in |\pdf@|\meta{cmd}
%   is the same as for the corresponding primitive of \pdfTeX.
%   The arguments are ordinary undelimited \TeX\ arguments,
%   no \meta{general text} and without additional keywords.
% \item[Expandibility:]
%   The macro |\pdf@|\meta{cmd} is expandable if the
%   corresponding \pdfTeX\ primitive has this property.
%   Exact two expansion steps are necessary (first is the macro
%   expansion).
% \item[Without \LuaTeX:]
%   The macros |\pdf@|\meta{cmd} are mapped to the commands
%   of \pdfTeX\ if they are available. Otherwise they are undefined.
% \end{description}
%
% \subsection{Macros}
%
% \begin{declcs}{pdf@strcmp} \M{stringA} \M{stringB}
% \end{declcs}
% Same as |\pdfstrcmp{|\meta{stringA}|}{|\meta{stringB}|}|.
%
% \begin{declcs}{pdf@unescapehex} \M{string}
% \end{declcs}
% Same as |\pdfunescapehex{|\meta{string}|}|.
% The argument is a byte string given in hexadecimal notation.
% The result are character tokens from 0 until 255 with
% catcode 12 and the space with catcode 10.
%
% \begin{declcs}{pdf@escapehex} \M{string}\\
%   \cs{pdf@escapestring} \M{string}\\
%   \cs{pdf@escapename} \M{string}
% \end{declcs}
% Same as the primitives of \pdfTeX. However \pdfTeX\ does not
% know about characters with codes 256 and larger. Thus the
% string is treated as byte string, characters with more than
% eight bits are ignored.
%
% \begin{declcs}{pdf@filesize} \M{filename}
% \end{declcs}
% Same as |\pdffilesize{|\meta{filename}|}|.
%
% \begin{declcs}{pdf@filemoddate} \M{filename}
% \end{declcs}
% Same as |\pdffilemoddate{|\meta{filename}|}|.
%
% \begin{declcs}{pdf@filedump} \M{offset} \M{length} \M{filename}
% \end{declcs}
% Same as |\pdffiledump offset| \meta{offset} |length| \meta{length}
% |{|\meta{filename}|}|. Both \meta{offset} and \meta{length} must
% not be empty, but must be a valid \TeX\ number.
%
% \begin{declcs}{pdf@mdfivesum} \M{string}
% \end{declcs}
% Same as |\pdfmdfivesum{|\meta{string}|}|. Keyword |file| is supported
% by macro \cs{pdf@filemdfivesum}.
%
% \begin{declcs}{pdf@filemdfivesum} \M{filename}
% \end{declcs}
% Same as |\pdfmdfivesum file{|\meta{filename}|}|.
%
% \begin{declcs}{pdf@shellescape}
% \end{declcs}
% Same as |\pdfshellescape|. It expands to |1| if external
% commands can be executed and |0| otherwise. In \pdfTeX\ external
% commands must be enabled first by command line option or
% configuration option. In \LuaTeX\ option |--safer| disables
% the execution of external commands.
%
% \begin{declcs}{pdf@system} \M{cmdline}
% \end{declcs}
% It is a wrapper for |\immediate\write18| in \pdfTeX\ or
% |os.execute| in \LuaTeX.
%
% In theory |os.execute|
% returns a status number. But its meaning is quite
% undefined. Are there some reliable properties?
% Does it make sense to provide an user interface to
% this status exit code?
%
% \subsubsection{Experimental}
%
% \begin{declcs}{pdf@unescapehexnative} \M{string}\\
%   \cs{pdf@escapehexnative} \M{string}\\
%   \cs{pdf@escapenamenative} \M{string}\\
%   \cs{pdf@mdfivesumnative} \M{string}
% \end{declcs}
% The variants without |native| in the macro name are supposed to
% be compatible with \pdfTeX. However characters with more than
% eight bits are not supported and are ignored. If \LuaTeX\ is
% running, then its UTF-8 coded strings are used. Thus the full
% unicode character range is supported. However the result
% differs from \pdfTeX\ for characters with eight or more bits.
%
% \begin{declcs}{pdf@pipe} \M{cmdline}
% \end{declcs}
% It calls \meta{cmdline} and returns the output of the external
% program in the usual manner as byte string (catcode 12, space with
% catcode 10). The Lua documentation says, that the used |io.popen|
% may not be available on all platforms. Then macro \cs{pdf@pipe}
% is undefined.
%
% \StopEventually{
% }
%
% \section{Implementation}
%
%    \begin{macrocode}
%<*package>
%    \end{macrocode}
%
% \subsection{Reload check and package identification}
%    Reload check, especially if the package is not used with \LaTeX.
%    \begin{macrocode}
\begingroup
  \catcode44 12 % ,
  \catcode45 12 % -
  \catcode46 12 % .
  \catcode58 12 % :
  \catcode64 11 % @
  \expandafter\let\expandafter\x\csname ver@pdftexcmds.sty\endcsname
  \ifcase 0%
    \ifx\x\relax % plain
    \else
      \ifx\x\empty % LaTeX
      \else
        1%
      \fi
    \fi
  \else
    \catcode35 6 % #
    \catcode123 1 % {
    \catcode125 2 % }
    \expandafter\ifx\csname PackageInfo\endcsname\relax
      \def\x#1#2{%
        \immediate\write-1{Package #1 Info: #2.}%
      }%
    \else
      \def\x#1#2{\PackageInfo{#1}{#2, stopped}}%
    \fi
    \x{pdftexcmds}{The package is already loaded}%
    \endgroup
    \expandafter\endinput
  \fi
\endgroup
%    \end{macrocode}
%    Package identification:
%    \begin{macrocode}
\begingroup
  \catcode35 6 % #
  \catcode40 12 % (
  \catcode41 12 % )
  \catcode44 12 % ,
  \catcode45 12 % -
  \catcode46 12 % .
  \catcode47 12 % /
  \catcode58 12 % :
  \catcode64 11 % @
  \catcode123 1 % {
  \catcode125 2 % }
  \expandafter\ifx\csname ProvidesPackage\endcsname\relax
    \def\x#1#2#3[#4]{\endgroup
      \immediate\write-1{Package: #3 #4}%
      \xdef#1{#4}%
    }%
  \else
    \def\x#1#2[#3]{\endgroup
      #2[{#3}]%
      \ifx#1\relax
        \xdef#1{#3}%
      \fi
    }%
  \fi
\expandafter\x\csname ver@pdftexcmds.sty\endcsname
\ProvidesPackage{pdftexcmds}%
  [2007/12/12 v0.3 LuaTeX support for pdfTeX utility functions (HO)]
%    \end{macrocode}
%
% \subsection{Catcodes}
%
%    \begin{macrocode}
\begingroup
  \catcode123 1 % {
  \catcode125 2 % }
  \def\x{\endgroup
    \expandafter\edef\csname pdftexcmds@AtEnd\endcsname{%
      \catcode35 \the\catcode35\relax
      \catcode64 \the\catcode64\relax
      \catcode123 \the\catcode123\relax
      \catcode125 \the\catcode125\relax
    }%
  }%
\x
\catcode35 6 % #
\catcode64 11 % @
\catcode123 1 % {
\catcode125 2 % }
\def\TMP@EnsureCode#1#2{%
  \edef\pdftexcmds@AtEnd{%
    \pdftexcmds@AtEnd
    \catcode#1 \the\catcode#1\relax
  }%
  \catcode#1 #2\relax
}
\TMP@EnsureCode{10}{12}% ^^J
\TMP@EnsureCode{33}{12}% !
\TMP@EnsureCode{34}{12}% "
\TMP@EnsureCode{39}{12}% '
\TMP@EnsureCode{40}{12}% (
\TMP@EnsureCode{41}{12}% )
\TMP@EnsureCode{42}{12}% *
\TMP@EnsureCode{43}{12}% +
\TMP@EnsureCode{44}{12}% ,
\TMP@EnsureCode{45}{12}% -
\TMP@EnsureCode{46}{12}% .
\TMP@EnsureCode{47}{12}% /
\TMP@EnsureCode{58}{12}% :
\TMP@EnsureCode{60}{12}% <
\TMP@EnsureCode{61}{12}% =
\TMP@EnsureCode{62}{12}% >
\TMP@EnsureCode{94}{7}% ^ (superscript)
\TMP@EnsureCode{95}{12}% _ (other)
\TMP@EnsureCode{126}{12}% ~ (other)
%    \end{macrocode}
%
% \subsection{Load package \xpackage{infwarerr}}
%
%    \begin{macrocode}
\begingroup\expandafter\expandafter\expandafter\endgroup
\expandafter\ifx\csname RequirePackage\endcsname\relax
  \input infwarerr.sty\relax
\else
  \RequirePackage{infwarerr}[2007/09/09]%
\fi
%    \end{macrocode}
%
% \subsection{Without \LuaTeX}
%
%    \begin{macrocode}
\begingroup\expandafter\expandafter\expandafter\endgroup
\expandafter\ifx\csname directlua\endcsname\relax
  \@PackageInfo{pdftexcmds}{LuaTeX not detected}%
  \def\pdftexcmds@nopdftex{%
    \@PackageInfoNoLine{pdftexcmds}{pdfTeX >= 1.30 not detected}%
    \let\pdftexcmds@nopdftex\relax
  }%
  \def\pdftexcmds@temp#1{%
    \begingroup\expandafter\expandafter\expandafter\endgroup
    \expandafter\ifx\csname pdf#1\endcsname\relax
      \pdftexcmds@nopdftex
    \else
      \expandafter\def\csname pdf@#1\expandafter\endcsname
      \expandafter##\expandafter{%
        \csname pdf#1\endcsname
      }%
    \fi
  }%
  \pdftexcmds@temp{strcmp}%
  \pdftexcmds@temp{escapehex}%
  \let\pdf@escapehexnative\pdf@escapehex
  \pdftexcmds@temp{unescapehex}%
  \let\pdf@unescapehexnative\pdf@unescapehex
  \pdftexcmds@temp{escapestring}%
  \pdftexcmds@temp{escapename}%
  \pdftexcmds@temp{filesize}%
  \pdftexcmds@temp{filemoddate}%
  \begingroup\expandafter\expandafter\expandafter\endgroup
  \expandafter\ifx\csname pdfshellescape\endcsname\relax
    \pdftexcmds@nopdftex
  \else
    \def\pdf@shellescape{%
      \pdfshellescape
    }%
  \fi
  \begingroup\expandafter\expandafter\expandafter\endgroup
  \expandafter\ifx\csname pdffiledump\endcsname\relax
    \pdftexcmds@nopdftex
  \else
    \def\pdf@filedump#1#2#3{%
      \pdffiledump offset#1 length#2{#3}%
    }%
  \fi
  \begingroup\expandafter\expandafter\expandafter\endgroup
  \expandafter\ifx\csname pdfmdfivesum\endcsname\relax
    \pdftexcmds@nopdftex
  \else
    \def\pdf@mdfivesum#{\pdfmdfivesum}%
    \let\pdf@mdfivesumnative\pdf@mdfivesum
    \def\pdf@filemdfivesum#{\pdfmdfivesum file}%
  \fi
  \def\pdf@system#{%
    \immediate\write18%
  }%
  \pdftexcmds@AtEnd
  \expandafter\endinput
\fi
%    \end{macrocode}
%
% \subsection{Load module}
%
%    \begin{macrocode}
\begingroup\expandafter\expandafter\expandafter\endgroup
\expandafter\ifx\csname RequirePackage\endcsname\relax
  \input luatex-loader.sty\relax
\else
  \RequirePackage{luatex-loader}[2007/12/12]%
\fi
\directlua0{%
  require("oberdiek.pdftexcmds")%
}
%    \end{macrocode}
%
% \subsection{Lua functions}
%
%    \begin{macro}{\pdftexcmds@toks}
%    \begin{macrocode}
\begingroup\expandafter\expandafter\expandafter\endgroup
\expandafter\ifx\csname newtoks\endcsname\relax
  \toksdef\pdftexcmds@toks=0 %
\else
  \csname newtoks\endcsname\pdftexcmds@toks
\fi
%    \end{macrocode}
%    \end{macro}
%
%    \begin{macro}{\pdf@strcmp}
%    \begin{macrocode}
\long\def\pdf@strcmp#1#2{%
  \directlua0{%
    oberdiek.pdftexcmds.strcmp("\luaescapestring{#1}",%
        "\luaescapestring{#2}")%
  }%
}%
%    \end{macrocode}
%    \end{macro}
%    \begin{macro}{\pdf@escapehex}
%    \begin{macrocode}
\long\def\pdf@escapehex#1{%
  \directlua0{%
    oberdiek.pdftexcmds.escapehex("\luaescapestring{#1}", "byte")%
  }%
}%
%    \end{macrocode}
%    \end{macro}
%    \begin{macro}{\pdf@escapehexnative}
%    \begin{macrocode}
\long\def\pdf@escapehexnative#1{%
  \directlua0{%
    oberdiek.pdftexcmds.escapehex("\luaescapestring{#1}")%
  }%
}%
%    \end{macrocode}
%    \end{macro}
%    \begin{macro}{\pdf@unescapehex}
%    \begin{macrocode}
\def\pdf@unescapehex#1{%
  \the\expandafter\pdftexcmds@toks
  \directlua0{%
    oberdiek.pdftexcmds.toks="pdftexcmds@toks"%
    oberdiek.pdftexcmds.unescapehex("\luaescapestring{#1}", "byte")%
  }%
}%
%    \end{macrocode}
%    \end{macro}
%    \begin{macro}{\pdf@unescapehexnative}
%    \begin{macrocode}
\def\pdf@unescapehexnative#1{%
  \the\expandafter\pdftexcmds@toks
  \directlua0{%
    oberdiek.pdftexcmds.toks="pdftexcmds@toks"%
    oberdiek.pdftexcmds.unescapehex("\luaescapestring{#1}")%
  }%
}%
%    \end{macrocode}
%    \end{macro}
%    \begin{macro}{\pdf@escapestring}
%    \begin{macrocode}
\long\def\pdf@escapestring#1{%
  \directlua0{%
    oberdiek.pdftexcmds.escapestring("\luaescapestring{#1}", "byte")%
  }%
}
%    \end{macrocode}
%    \end{macro}
%    \begin{macro}{\pdf@escapename}
%    \begin{macrocode}
\long\def\pdf@escapename#1{%
  \directlua0{%
    oberdiek.pdftexcmds.escapename("\luaescapestring{#1}", "byte")%
  }%
}
%    \end{macrocode}
%    \end{macro}
%    \begin{macro}{\pdf@escapenamenative}
%    \begin{macrocode}
\long\def\pdf@escapenamenative#1{%
  \directlua0{%
    oberdiek.pdftexcmds.escapename("\luaescapestring{#1}")%
  }%
}
%    \end{macrocode}
%    \end{macro}
%    \begin{macro}{\pdf@filesize}
%    \begin{macrocode}
\def\pdf@filesize#1{%
  \directlua0{%
    oberdiek.pdftexcmds.filesize("\luaescapestring{#1}")%
  }%
}
%    \end{macrocode}
%    \end{macro}
%    \begin{macro}{\pdf@filemoddate}
%    \begin{macrocode}
\def\pdf@filemoddate#1{%
  \directlua0{%
    oberdiek.pdftexcmds.filemoddate("\luaescapestring{#1}")%
  }%
}
%    \end{macrocode}
%    \end{macro}
%    \begin{macro}{\pdf@filedump}
%    \begin{macrocode}
\def\pdf@filedump#1#2#3{%
  \directlua0{%
    oberdiek.pdftexcmds.filedump("\luaescapestring{\number#1}",%
        "\luaescapestring{\number#2}",%
        "\luaescapestring{#3}")%
  }%
}%
%    \end{macrocode}
%    \end{macro}
%    \begin{macro}{\pdf@mdfivesum}
%    \begin{macrocode}
\long\def\pdf@mdfivesum#1{%
  \directlua0{%
    oberdiek.pdftexcmds.mdfivesum("\luaescapestring{#1}", "byte")%
  }%
}%
%    \end{macrocode}
%    \end{macro}
%    \begin{macro}{\pdf@mdfivesumnative}
%    \begin{macrocode}
\long\def\pdf@mdfivesumnative#1{%
  \directlua0{%
    oberdiek.pdftexcmds.mdfivesum("\luaescapestring{#1}")%
  }%
}%
%    \end{macrocode}
%    \end{macro}
%    \begin{macro}{\pdf@filemdfivesum}
%    \begin{macrocode}
\def\pdf@filemdfivesum#1{%
  \directlua0{%
    oberdiek.pdftexcmds.filemdfivesum("\luaescapestring{#1}")%
  }%
}%
%    \end{macrocode}
%    \end{macro}
%
%    \begin{macro}{\pdf@shellescape}
%    \begin{macrocode}
\def\pdf@shellescape{%
  \directlua0{%
    oberdiek.pdftexcmds.shellescape()%
  }%
}
%    \end{macrocode}
%    \end{macro}
%
%    \begin{macro}{\pdf@system}
%    \begin{macrocode}
\def\pdf@system#1{%
  \directlua0{%
    oberdiek.pdftexcmds.system("\luaescapestring{#1}")%
  }%
}
%    \end{macrocode}
%    \end{macro}
%
%    \begin{macro}{\pdf@lastsystemstatus}
%    \begin{macrocode}
\def\pdf@lastsystemstatus{%
  \directlua0{%
    oberdiek.pdftexcmds.lastsystemstatus()%
  }%
}
%    \end{macrocode}
%    \end{macro}
%    \begin{macro}{\pdf@lastsystemexit}
%    \begin{macrocode}
\def\pdf@lastsystemexit{%
  \directlua0{%
    oberdiek.pdftexcmds.lastsystemexit()%
  }%
}
%    \end{macrocode}
%    \end{macro}
%
%    \begin{macro}{\pdf@pipe}
%    Check availability of |io.popen| first.
%    \begin{macrocode}
\ifnum0%
    \directlua0{%
      if io.popen then %
        tex.write("1")%
      end%
    }%
    =1 %
  \def\pdf@pipe#1{%
    \the\expandafter\pdftexcmds@toks
    \directlua0{%
      oberdiek.pdftexcmds.toks="pdftexcmds@toks"%
      oberdiek.pdftexcmds.pipe("\luaescapestring{#1}")%
    }%
  }%
\fi
%    \end{macrocode}
%    \end{macro}
%
%    \begin{macrocode}
\pdftexcmds@AtEnd
%</package>
%    \end{macrocode}
%
% \subsection{Lua module}
%
%    \begin{macrocode}
%<*lua>
%    \end{macrocode}
%
%    \begin{macrocode}
module("oberdiek.pdftexcmds", package.seeall)
local systemexitstatus
function strcmp(A, B)
  if A == B then
    tex.write("0")
  elseif A < B then
    tex.write("-1")
  else
    tex.write("1")
  end
end
local function utf8_to_byte(str)
  local i = 0
  local n = string.len(str)
  local t = {}
  while i < n do
    i = i + 1
    local a = string.byte(str, i)
    if a < 128 then
      table.insert(t, string.char(a))
    else
      if a >= 192 and i < n then
        i = i + 1
        local b = string.byte(str, i)
        if b < 128 or b >= 192 then
          i = i - 1
        elseif a == 194 then
          table.insert(t, string.char(b))
        elseif a == 195 then
          table.insert(t, string.char(b + 64))
        end
      end
    end
  end
  return table.concat(t)
end
function escapehex(str, mode)
  if mode == "byte" then
    str = utf8_to_byte(str)
  end
  tex.write((string.gsub(str, ".",
    function (ch)
      return string.format("%02X", string.byte(ch))
    end
  )))
end
%    \end{macrocode}
%    See procedure |unescapehex| in file \xfile{utils.c} of \pdfTeX.
%    Caution: |tex.write| ignores leading spaces.
%    \begin{macrocode}
function unescapehex(str, mode)
  local a = 0
  local first = true
  local result = {}
  for i = 1, string.len(str), 1 do
    local ch = string.byte(str, i)
    if ch >= 48 and ch <= 57 then
      ch = ch - 48
    elseif ch >= 65 and ch <= 70 then
      ch = ch - 55
    elseif ch >= 97 and ch <= 102 then
      ch = ch - 87
    else
      ch = nil
    end
    if ch then
      if first then
        a = ch * 16
        first = false
      else
        table.insert(result, a + ch)
        first = true
      end
    end
  end
  if not first then
    table.insert(result, a)
  end
  if mode == "byte" then
    local utf8 = {}
    for i, a in ipairs(result) do
      if a < 128 then
        table.insert(utf8, a)
      else
        if a < 192 then
          table.insert(utf8, 194)
          a = a - 128
        else
          table.insert(utf8, 195)
          a = a - 192
        end
        table.insert(utf8, a + 128)
      end
    end
    result = utf8
  end
  tex.settoks(toks, string.char(unpack(result)))
end
%    \end{macrocode}
%    See procedure |escapestring| in file \xfile{utils.c} of \pdfTeX.
%    \begin{macrocode}
function escapestring(str, mode)
  if mode == "byte" then
    str = utf8_to_byte(str)
  end
  tex.write((string.gsub(str, ".",
    function (ch)
      local b = string.byte(ch)
      if b < 33 or b > 126 then
        return string.format("\\%.3o", b)
      end
      if b == 40 or b == 41 or b == 92 then
        return "\\" .. ch
      end
%    \end{macrocode}
%    Lua 5.1 returns the match in case of return value |nil|.
%    \begin{macrocode}
      return nil
    end
  )))
end
%    \end{macrocode}
%    See procedure |escapename| in file \xfile{utils.c} of \pdfTeX.
%    \begin{macrocode}
function escapename(str, mode)
  if mode == "byte" then
    str = utf8_to_byte(str)
  end
  tex.write((string.gsub(str, ".",
    function (ch)
      local b = string.byte(ch)
      if b == 0 then
%    \end{macrocode}
%    In Lua 5.0 |nil| could be used for the empty string,
%    But |nil| returns the match in Lua 5.1, thus we use
%    the empty string explicitly.
%    \begin{macrocode}
        return ""
      end
      if b <= 32 or b >= 127
          or b == 35 or b == 37 or b == 40 or b == 41
          or b == 47 or b == 60 or b == 62 or b == 91
          or b == 93 or b == 123 or b == 125 then
        return string.format("#%.2X", b)
      else
%    \end{macrocode}
%    Lua 5.1 returns the match in case of return value |nil|.
%    \begin{macrocode}
        return nil
      end
    end
  )))
end
function filesize(filename)
  local foundfile = kpse.find_file(filename, "tex", true)
  if foundfile then
    local size = lfs.attributes(foundfile, "size")
    if size then
      tex.write(size)
    end
  end
end
%    \end{macrocode}
%    See procedure |makepdftime| in file \xfile{utils.c} of \pdfTeX.
%    \begin{macrocode}
function filemoddate(filename)
  local foundfile = kpse.find_file(filename, "tex", true)
  if foundfile then
    local date = lfs.attributes(foundfile, "modification")
    if date then
      local d = os.date("*t", date)
      if d.sec >= 60 then
        d.sec = 59
      end
      local u = os.date("!*t", date)
      local off = 60 * (d.hour - u.hour) + d.min - u.min
      if d.year ~= u.year then
        if d.year > u.year then
          off = off + 1440
        else
          off = off - 1440
        end
      elseif d.yday ~= u.yday then
        if d.yday > u.yday then
          off = off + 1440
        else
          off = off - 1440
        end
      end
      local timezone
      if off == 0 then
        timezone = "Z"
      else
        local hours = math.floor(off / 60)
        local mins = math.abs(off - hours * 60)
        timezone = string.format("%+03d'%02d'", hours, mins)
      end
      tex.write(string.format("D:%04d%02d%02d%02d%02d%02d%s",
          d.year, d.month, d.day, d.hour, d.min, d.sec, timezone))
    end
  end
end
function filedump(offset, length, filename)
  length = tonumber(length)
  if length and length > 0 then
    local foundfile = kpse.find_file(filename, "tex", true)
    if foundfile then
      offset = tonumber(offset)
      if not offset then
        offset = 0
      end
      local filehandle = io.open(foundfile, "r")
      if filehandle then
        if offset > 0 then
          filehandle:seek("set", offset)
        end
        local dump = filehandle:read(length)
        escapehex(dump)
      end
    end
  end
end
function mdfivesum(str, mode)
  if mode == "byte" then
    str = utf8_to_byte(str)
  end
  escapehex(md5.sum(str))
end
function filemdfivesum(filename)
  local foundfile = kpse.find_file(filename, "tex", true)
  if foundfile then
    local filehandle = io.open(foundfile, "r")
    if filehandle then
      local contents = filehandle:read("*a")
      escapehex(md5.sum(contents))
    end
  end
end
function shellescape()
  if os.execute then
    tex.write("1")
  else
    tex.write("0")
  end
end
function system(cmdline)
  systemexitstatus = nil
  texio.write_nl("log", "system(" .. cmdline .. ") ")
  if os.execute then
    texio.write("log", "executed.")
    systemexitstatus = os.execute(cmdline)
  else
    texio.write("log", "disabled.")
  end
end
function lastsystemstatus()
  local result = tonumber(systemexitstatus)
  if result then
    local x = math.floor(result / 256)
    tex.write(result - 256 * math.floor(result / 256))
  end
end
function lastsystemexit()
  local result = tonumber(systemexitstatus)
  if result then
    tex.write(math.floor(result / 256))
  end
end
function pipe(cmdline)
  local result
  systemexitstatus = nil
  texio.write_nl("log", "pipe(" .. cmdline ..") ")
  if io.popen then
    texio.write("log", "executed.")
    local handle = io.popen(cmdline, "r")
    if handle then
      result = handle:read("*a")
      handle:close()
    end
  else
    texio.write("log", "disabled.")
  end
  if result then
    tex.settoks(toks, result)
  else
    tex.settoks(toks, "")
  end
end
%    \end{macrocode}
%    \begin{macrocode}
%</lua>
%    \end{macrocode}
%
% \section{Test}
%
% \subsection{Catcode checks for loading}
%
%    \begin{macrocode}
%<*test1>
%    \end{macrocode}
%    \begin{macrocode}
\catcode`\{=1 %
\catcode`\}=2 %
\catcode`\#=6 %
\catcode`\@=11 %
\expandafter\ifx\csname count@\endcsname\relax
  \countdef\count@=255 %
\fi
\expandafter\ifx\csname @gobble\endcsname\relax
  \long\def\@gobble#1{}%
\fi
\expandafter\ifx\csname @firstofone\endcsname\relax
  \long\def\@firstofone#1{#1}%
\fi
\expandafter\ifx\csname loop\endcsname\relax
  \expandafter\@firstofone
\else
  \expandafter\@gobble
\fi
{%
  \def\loop#1\repeat{%
    \def\body{#1}%
    \iterate
  }%
  \def\iterate{%
    \body
      \let\next\iterate
    \else
      \let\next\relax
    \fi
    \next
  }%
  \let\repeat=\fi
}%
\def\RestoreCatcodes{}
\count@=0 %
\loop
  \edef\RestoreCatcodes{%
    \RestoreCatcodes
    \catcode\the\count@=\the\catcode\count@\relax
  }%
\ifnum\count@<255 %
  \advance\count@ 1 %
\repeat

\def\RangeCatcodeInvalid#1#2{%
  \count@=#1\relax
  \loop
    \catcode\count@=15 %
  \ifnum\count@<#2\relax
    \advance\count@ 1 %
  \repeat
}
\expandafter\ifx\csname LoadCommand\endcsname\relax
  \def\LoadCommand{\input pdftexcmds.sty\relax}%
\fi
\def\Test{%
  \RangeCatcodeInvalid{0}{47}%
  \RangeCatcodeInvalid{58}{64}%
  \RangeCatcodeInvalid{91}{96}%
  \RangeCatcodeInvalid{123}{255}%
  \catcode`\@=12 %
  \catcode`\\=0 %
  \catcode`\{=1 %
  \catcode`\}=2 %
  \catcode`\#=6 %
  \catcode`\[=12 %
  \catcode`\]=12 %
  \catcode`\%=14 %
  \catcode`\ =10 %
  \catcode13=5 %
  \LoadCommand
  \RestoreCatcodes
}
\Test
\csname @@end\endcsname
\end
%    \end{macrocode}
%    \begin{macrocode}
%</test1>
%    \end{macrocode}
%
% \section{Installation}
%
% \subsection{Download}
%
% \paragraph{Package.} This package is available on
% CTAN\footnote{\url{ftp://ftp.ctan.org/tex-archive/}}:
% \begin{description}
% \item[\CTAN{macros/latex/contrib/oberdiek/pdftexcmds.dtx}] The source file.
% \item[\CTAN{macros/latex/contrib/oberdiek/pdftexcmds.pdf}] Documentation.
% \end{description}
%
%
% \paragraph{Bundle.} All the packages of the bundle `oberdiek'
% are also available in a TDS compliant ZIP archive. There
% the packages are already unpacked and the documentation files
% are generated. The files and directories obey the TDS standard.
% \begin{description}
% \item[\CTAN{macros/latex/contrib/oberdiek/oberdiek-tds.zip}]
% \end{description}
% \emph{TDS} refers to the standard ``A Directory Structure
% for \TeX\ Files'' (\CTAN{tds/tds.pdf}). Directories
% with \xfile{texmf} in their name are usually organized this way.
%
% \subsection{Bundle installation}
%
% \paragraph{Unpacking.} Unpack the \xfile{oberdiek-tds.zip} in the
% TDS tree (also known as \xfile{texmf} tree) of your choice.
% Example (linux):
% \begin{quote}
%   |unzip oberdiek-tds.zip -d ~/texmf|
% \end{quote}
%
% \paragraph{Script installation.}
% Check the directory \xfile{TDS:scripts/oberdiek/} for
% scripts that need further installation steps.
% Package \xpackage{attachfile2} comes with the Perl script
% \xfile{pdfatfi.pl} that should be installed in such a way
% that it can be called as \texttt{pdfatfi}.
% Example (linux):
% \begin{quote}
%   |chmod +x scripts/oberdiek/pdfatfi.pl|\\
%   |cp scripts/oberdiek/pdfatfi.pl /usr/local/bin/|
% \end{quote}
%
% \subsection{Package installation}
%
% \paragraph{Unpacking.} The \xfile{.dtx} file is a self-extracting
% \docstrip\ archive. The files are extracted by running the
% \xfile{.dtx} through \plainTeX:
% \begin{quote}
%   \verb|tex pdftexcmds.dtx|
% \end{quote}
%
% \paragraph{TDS.} Now the different files must be moved into
% the different directories in your installation TDS tree
% (also known as \xfile{texmf} tree):
% \begin{quote}
% \def\t{^^A
% \begin{tabular}{@{}>{\ttfamily}l@{ $\rightarrow$ }>{\ttfamily}l@{}}
%   pdftexcmds.sty & tex/generic/oberdiek/pdftexcmds.sty\\
%   oberdiek.pdftexcmds.lua & scripts/oberdiek/oberdiek.pdftexcmds.lua\\
%   pdftexcmds.lua & scripts/oberdiek/pdftexcmds.lua\\
%   pdftexcmds.pdf & doc/latex/oberdiek/pdftexcmds.pdf\\
%   pdftexcmds.dtx & source/latex/oberdiek/pdftexcmds.dtx\\
% \end{tabular}^^A
% }^^A
% \sbox0{\t}^^A
% \ifdim\wd0>\linewidth
%   \begingroup
%     \advance\linewidth by\leftmargin
%     \advance\linewidth by\rightmargin
%   \edef\x{\endgroup
%     \def\noexpand\lw{\the\linewidth}^^A
%   }\x
%   \def\lwbox{^^A
%     \leavevmode
%     \hbox to \linewidth{^^A
%       \kern-\leftmargin\relax
%       \hss
%       \usebox0
%       \hss
%       \kern-\rightmargin\relax
%     }^^A
%   }^^A
%   \ifdim\wd0>\lw
%     \sbox0{\small\t}^^A
%     \ifdim\wd0>\linewidth
%       \ifdim\wd0>\lw
%         \sbox0{\footnotesize\t}^^A
%         \ifdim\wd0>\linewidth
%           \ifdim\wd0>\lw
%             \sbox0{\scriptsize\t}^^A
%             \ifdim\wd0>\linewidth
%               \ifdim\wd0>\lw
%                 \sbox0{\tiny\t}^^A
%                 \ifdim\wd0>\linewidth
%                   \lwbox
%                 \else
%                   \usebox0
%                 \fi
%               \else
%                 \lwbox
%               \fi
%             \else
%               \usebox0
%             \fi
%           \else
%             \lwbox
%           \fi
%         \else
%           \usebox0
%         \fi
%       \else
%         \lwbox
%       \fi
%     \else
%       \usebox0
%     \fi
%   \else
%     \lwbox
%   \fi
% \else
%   \usebox0
% \fi
% \end{quote}
% If you have a \xfile{docstrip.cfg} that configures and enables \docstrip's
% TDS installing feature, then some files can already be in the right
% place, see the documentation of \docstrip.
%
% \subsection{Refresh file name databases}
%
% If your \TeX~distribution
% (\teTeX, \mikTeX, \dots) relies on file name databases, you must refresh
% these. For example, \teTeX\ users run \verb|texhash| or
% \verb|mktexlsr|.
%
% \subsection{Some details for the interested}
%
% \paragraph{Attached source.}
%
% The PDF documentation on CTAN also includes the
% \xfile{.dtx} source file. It can be extracted by
% AcrobatReader 6 or higher. Another option is \textsf{pdftk},
% e.g. unpack the file into the current directory:
% \begin{quote}
%   \verb|pdftk pdftexcmds.pdf unpack_files output .|
% \end{quote}
%
% \paragraph{Unpacking with \LaTeX.}
% The \xfile{.dtx} chooses its action depending on the format:
% \begin{description}
% \item[\plainTeX:] Run \docstrip\ and extract the files.
% \item[\LaTeX:] Generate the documentation.
% \end{description}
% If you insist on using \LaTeX\ for \docstrip\ (really,
% \docstrip\ does not need \LaTeX), then inform the autodetect routine
% about your intention:
% \begin{quote}
%   \verb|latex \let\install=y\input{pdftexcmds.dtx}|
% \end{quote}
% Do not forget to quote the argument according to the demands
% of your shell.
%
% \paragraph{Generating the documentation.}
% You can use both the \xfile{.dtx} or the \xfile{.drv} to generate
% the documentation. The process can be configured by the
% configuration file \xfile{ltxdoc.cfg}. For instance, put this
% line into this file, if you want to have A4 as paper format:
% \begin{quote}
%   \verb|\PassOptionsToClass{a4paper}{article}|
% \end{quote}
% An example follows how to generate the
% documentation with pdf\LaTeX:
% \begin{quote}
%\begin{verbatim}
%pdflatex pdftexcmds.dtx
%makeindex -s gind.ist pdftexcmds.idx
%pdflatex pdftexcmds.dtx
%makeindex -s gind.ist pdftexcmds.idx
%pdflatex pdftexcmds.dtx
%\end{verbatim}
% \end{quote}
%
% \begin{History}
%   \begin{Version}{2007/11/11 v0.1}
%   \item
%     First version.
%   \end{Version}
%   \begin{Version}{2007/11/12 v0.2}
%   \item
%     Short description fixed.
%   \end{Version}
%   \begin{Version}{2007/12/12 v0.3}
%   \item
%     Organization of Lua code as module.
%   \end{Version}
% \end{History}
%
% \PrintIndex
%
% \Finale
\endinput

%        (quote the arguments according to the demands of your shell)
%
% Documentation:
%    (a) If pdftexcmds.drv is present:
%           latex pdftexcmds.drv
%    (b) Without pdftexcmds.drv:
%           latex pdftexcmds.dtx; ...
%    The class ltxdoc loads the configuration file ltxdoc.cfg
%    if available. Here you can specify further options, e.g.
%    use A4 as paper format:
%       \PassOptionsToClass{a4paper}{article}
%
%    Programm calls to get the documentation (example):
%       pdflatex pdftexcmds.dtx
%       makeindex -s gind.ist pdftexcmds.idx
%       pdflatex pdftexcmds.dtx
%       makeindex -s gind.ist pdftexcmds.idx
%       pdflatex pdftexcmds.dtx
%
% Installation:
%    TDS:tex/generic/oberdiek/pdftexcmds.sty
%    TDS:scripts/oberdiek/oberdiek.pdftexcmds.lua
%    TDS:scripts/oberdiek/pdftexcmds.lua
%    TDS:doc/latex/oberdiek/pdftexcmds.pdf
%    TDS:source/latex/oberdiek/pdftexcmds.dtx
%
%<*ignore>
\begingroup
  \def\x{LaTeX2e}%
\expandafter\endgroup
\ifcase 0\ifx\install y1\fi\expandafter
         \ifx\csname processbatchFile\endcsname\relax\else1\fi
         \ifx\fmtname\x\else 1\fi\relax
\else\csname fi\endcsname
%</ignore>
%<*install>
\input docstrip.tex
\Msg{************************************************************************}
\Msg{* Installation}
\Msg{* Package: pdftexcmds 2007/12/12 v0.3 LuaTeX support for pdfTeX utility functions (HO)}
\Msg{************************************************************************}

\keepsilent
\askforoverwritefalse

\let\MetaPrefix\relax
\preamble

This is a generated file.

Copyright (C) 2007 by Heiko Oberdiek <oberdiek@uni-freiburg.de>

This work may be distributed and/or modified under the
conditions of the LaTeX Project Public License, either
version 1.3 of this license or (at your option) any later
version. The latest version of this license is in
   http://www.latex-project.org/lppl.txt
and version 1.3 or later is part of all distributions of
LaTeX version 2005/12/01 or later.

This work has the LPPL maintenance status "maintained".

This Current Maintainer of this work is Heiko Oberdiek.

This work consists of the main source file pdftexcmds.dtx
and the derived files
   pdftexcmds.sty, pdftexcmds.pdf, pdftexcmds.ins, pdftexcmds.drv,
   oberdiek.pdftexcmds.lua, pdftexcmds.lua.

\endpreamble
\let\MetaPrefix\DoubleperCent

\generate{%
  \file{pdftexcmds.ins}{\from{pdftexcmds.dtx}{install}}%
  \file{pdftexcmds.drv}{\from{pdftexcmds.dtx}{driver}}%
  \usedir{tex/generic/oberdiek}%
  \file{pdftexcmds.sty}{\from{pdftexcmds.dtx}{package}}%
}
\def\MetaPrefix{-- }
\def\defaultpostamble{%
  \MetaPrefix^^J%
  \MetaPrefix\space End of File `\outFileName'.%
}
\def\currentpostamble{\defaultpostamble}%
\generate{%
  \usedir{scripts/oberdiek}%
  \file{oberdiek.pdftexcmds.lua}{\from{pdftexcmds.dtx}{lua}}%
  \file{pdftexcmds.lua}{\from{pdftexcmds.dtx}{lua}}%
}

\obeyspaces
\Msg{************************************************************************}
\Msg{*}
\Msg{* To finish the installation you have to move the following}
\Msg{* file into a directory searched by TeX:}
\Msg{*}
\Msg{*     pdftexcmds.sty}
\Msg{*}
\Msg{* And install the following script files:}
\Msg{*}
\Msg{*     oberdiek.pdftexcmds.lua, pdftexcmds.lua}
\Msg{*}
\Msg{* To produce the documentation run the file `pdftexcmds.drv'}
\Msg{* through LaTeX.}
\Msg{*}
\Msg{* Happy TeXing!}
\Msg{*}
\Msg{************************************************************************}

\endbatchfile
%</install>
%<*ignore>
\fi
%</ignore>
%<*driver>
\NeedsTeXFormat{LaTeX2e}
\ProvidesFile{pdftexcmds.drv}%
  [2007/12/12 v0.3 LuaTeX support for pdfTeX utility functions (HO)]%
\documentclass{ltxdoc}
\usepackage{holtxdoc}[2007/11/11]
\usepackage{paralist}
\begin{document}
  \DocInput{pdftexcmds.dtx}%
\end{document}
%</driver>
% \fi
%
% \CheckSum{552}
%
% \CharacterTable
%  {Upper-case    \A\B\C\D\E\F\G\H\I\J\K\L\M\N\O\P\Q\R\S\T\U\V\W\X\Y\Z
%   Lower-case    \a\b\c\d\e\f\g\h\i\j\k\l\m\n\o\p\q\r\s\t\u\v\w\x\y\z
%   Digits        \0\1\2\3\4\5\6\7\8\9
%   Exclamation   \!     Double quote  \"     Hash (number) \#
%   Dollar        \$     Percent       \%     Ampersand     \&
%   Acute accent  \'     Left paren    \(     Right paren   \)
%   Asterisk      \*     Plus          \+     Comma         \,
%   Minus         \-     Point         \.     Solidus       \/
%   Colon         \:     Semicolon     \;     Less than     \<
%   Equals        \=     Greater than  \>     Question mark \?
%   Commercial at \@     Left bracket  \[     Backslash     \\
%   Right bracket \]     Circumflex    \^     Underscore    \_
%   Grave accent  \`     Left brace    \{     Vertical bar  \|
%   Right brace   \}     Tilde         \~}
%
% \GetFileInfo{pdftexcmds.drv}
%
% \title{The \xpackage{pdftexcmds} package}
% \date{2007/12/12 v0.3}
% \author{Heiko Oberdiek\\\xemail{oberdiek@uni-freiburg.de}}
%
% \maketitle
%
% \begin{abstract}
% \LuaTeX\ provides most of the commands of \pdfTeX\ 1.40. However
% a number of utility functions are removed. This package tries to fill
% the gap and implements some of the missing primitive using Lua.
% \end{abstract}
%
% \tableofcontents
%
% \section{Documentation}
%
% Some primitives of \pdfTeX\ are not defined by \LuaTeX.
% This package implements macro based solutions using Lua code
% for the following missing \pdfTeX\ primitives;
% \begin{compactitem}
% \item \cs{pdfstrcmp}
% \item \cs{pdfunescapehex}
% \item \cs{pdfescapehex}
% \item \cs{pdfescapename}
% \item \cs{pdfescapestring}
% \item \cs{pdffilesize}
% \item \cs{pdffilemoddate}
% \item \cs{pdffiledump}
% \item \cs{pdfmdfivesum}
% \item |\immediate\write18|
% \end{compactitem}
% The original names of the primitives cannot be used:
% \begin{itemize}
% \item
% The syntax for their arguments cannot easily
% simulated by macros. The primitives using key words
% such as |file| (\cs{pdfmdfivesum}) or |offset| and |length|
% (\cs{pdffiledump}) and uses \meta{general text} for the other
% arguments. Using token registers assignments, \meta{general text} could
% be catched. However, the simulated primitives are expandable
% and register assignments would destroy this important property.
% (\meta{general text} allows something like |\expandafter\bgroup ...}|.)
% \item
% The original primitives can be expanded using one expansion step.
% The new macros need two expansion steps because of the additional
% macro expansion. Example:
% \begin{quote}
%   |\expandafter\foo\pdffilemoddate{file}|\\
%   vs.
%   |\expandafter\expandafter\expandafter\foo\pdf@filemoddate{file}|.
% \end{quote}
% \end{itemize}
%
% \LuaTeX\ isn't stable yet and thus the status of this package is
% \emph{experimental}. Feedback is welcome.
%
% \subsection{General principles}
%
% \begin{description}
% \item[Naming convention:]
%   Usually this package defines a macro |\pdf@|\meta{cmd} if
%   \pdfTeX\ provides |\pdf|\meta{cmd}.
% \item[Arguments:] The order of arguments in |\pdf@|\meta{cmd}
%   is the same as for the corresponding primitive of \pdfTeX.
%   The arguments are ordinary undelimited \TeX\ arguments,
%   no \meta{general text} and without additional keywords.
% \item[Expandibility:]
%   The macro |\pdf@|\meta{cmd} is expandable if the
%   corresponding \pdfTeX\ primitive has this property.
%   Exact two expansion steps are necessary (first is the macro
%   expansion).
% \item[Without \LuaTeX:]
%   The macros |\pdf@|\meta{cmd} are mapped to the commands
%   of \pdfTeX\ if they are available. Otherwise they are undefined.
% \end{description}
%
% \subsection{Macros}
%
% \begin{declcs}{pdf@strcmp} \M{stringA} \M{stringB}
% \end{declcs}
% Same as |\pdfstrcmp{|\meta{stringA}|}{|\meta{stringB}|}|.
%
% \begin{declcs}{pdf@unescapehex} \M{string}
% \end{declcs}
% Same as |\pdfunescapehex{|\meta{string}|}|.
% The argument is a byte string given in hexadecimal notation.
% The result are character tokens from 0 until 255 with
% catcode 12 and the space with catcode 10.
%
% \begin{declcs}{pdf@escapehex} \M{string}\\
%   \cs{pdf@escapestring} \M{string}\\
%   \cs{pdf@escapename} \M{string}
% \end{declcs}
% Same as the primitives of \pdfTeX. However \pdfTeX\ does not
% know about characters with codes 256 and larger. Thus the
% string is treated as byte string, characters with more than
% eight bits are ignored.
%
% \begin{declcs}{pdf@filesize} \M{filename}
% \end{declcs}
% Same as |\pdffilesize{|\meta{filename}|}|.
%
% \begin{declcs}{pdf@filemoddate} \M{filename}
% \end{declcs}
% Same as |\pdffilemoddate{|\meta{filename}|}|.
%
% \begin{declcs}{pdf@filedump} \M{offset} \M{length} \M{filename}
% \end{declcs}
% Same as |\pdffiledump offset| \meta{offset} |length| \meta{length}
% |{|\meta{filename}|}|. Both \meta{offset} and \meta{length} must
% not be empty, but must be a valid \TeX\ number.
%
% \begin{declcs}{pdf@mdfivesum} \M{string}
% \end{declcs}
% Same as |\pdfmdfivesum{|\meta{string}|}|. Keyword |file| is supported
% by macro \cs{pdf@filemdfivesum}.
%
% \begin{declcs}{pdf@filemdfivesum} \M{filename}
% \end{declcs}
% Same as |\pdfmdfivesum file{|\meta{filename}|}|.
%
% \begin{declcs}{pdf@shellescape}
% \end{declcs}
% Same as |\pdfshellescape|. It expands to |1| if external
% commands can be executed and |0| otherwise. In \pdfTeX\ external
% commands must be enabled first by command line option or
% configuration option. In \LuaTeX\ option |--safer| disables
% the execution of external commands.
%
% \begin{declcs}{pdf@system} \M{cmdline}
% \end{declcs}
% It is a wrapper for |\immediate\write18| in \pdfTeX\ or
% |os.execute| in \LuaTeX.
%
% In theory |os.execute|
% returns a status number. But its meaning is quite
% undefined. Are there some reliable properties?
% Does it make sense to provide an user interface to
% this status exit code?
%
% \subsubsection{Experimental}
%
% \begin{declcs}{pdf@unescapehexnative} \M{string}\\
%   \cs{pdf@escapehexnative} \M{string}\\
%   \cs{pdf@escapenamenative} \M{string}\\
%   \cs{pdf@mdfivesumnative} \M{string}
% \end{declcs}
% The variants without |native| in the macro name are supposed to
% be compatible with \pdfTeX. However characters with more than
% eight bits are not supported and are ignored. If \LuaTeX\ is
% running, then its UTF-8 coded strings are used. Thus the full
% unicode character range is supported. However the result
% differs from \pdfTeX\ for characters with eight or more bits.
%
% \begin{declcs}{pdf@pipe} \M{cmdline}
% \end{declcs}
% It calls \meta{cmdline} and returns the output of the external
% program in the usual manner as byte string (catcode 12, space with
% catcode 10). The Lua documentation says, that the used |io.popen|
% may not be available on all platforms. Then macro \cs{pdf@pipe}
% is undefined.
%
% \StopEventually{
% }
%
% \section{Implementation}
%
%    \begin{macrocode}
%<*package>
%    \end{macrocode}
%
% \subsection{Reload check and package identification}
%    Reload check, especially if the package is not used with \LaTeX.
%    \begin{macrocode}
\begingroup
  \catcode44 12 % ,
  \catcode45 12 % -
  \catcode46 12 % .
  \catcode58 12 % :
  \catcode64 11 % @
  \expandafter\let\expandafter\x\csname ver@pdftexcmds.sty\endcsname
  \ifcase 0%
    \ifx\x\relax % plain
    \else
      \ifx\x\empty % LaTeX
      \else
        1%
      \fi
    \fi
  \else
    \catcode35 6 % #
    \catcode123 1 % {
    \catcode125 2 % }
    \expandafter\ifx\csname PackageInfo\endcsname\relax
      \def\x#1#2{%
        \immediate\write-1{Package #1 Info: #2.}%
      }%
    \else
      \def\x#1#2{\PackageInfo{#1}{#2, stopped}}%
    \fi
    \x{pdftexcmds}{The package is already loaded}%
    \endgroup
    \expandafter\endinput
  \fi
\endgroup
%    \end{macrocode}
%    Package identification:
%    \begin{macrocode}
\begingroup
  \catcode35 6 % #
  \catcode40 12 % (
  \catcode41 12 % )
  \catcode44 12 % ,
  \catcode45 12 % -
  \catcode46 12 % .
  \catcode47 12 % /
  \catcode58 12 % :
  \catcode64 11 % @
  \catcode123 1 % {
  \catcode125 2 % }
  \expandafter\ifx\csname ProvidesPackage\endcsname\relax
    \def\x#1#2#3[#4]{\endgroup
      \immediate\write-1{Package: #3 #4}%
      \xdef#1{#4}%
    }%
  \else
    \def\x#1#2[#3]{\endgroup
      #2[{#3}]%
      \ifx#1\relax
        \xdef#1{#3}%
      \fi
    }%
  \fi
\expandafter\x\csname ver@pdftexcmds.sty\endcsname
\ProvidesPackage{pdftexcmds}%
  [2007/12/12 v0.3 LuaTeX support for pdfTeX utility functions (HO)]
%    \end{macrocode}
%
% \subsection{Catcodes}
%
%    \begin{macrocode}
\begingroup
  \catcode123 1 % {
  \catcode125 2 % }
  \def\x{\endgroup
    \expandafter\edef\csname pdftexcmds@AtEnd\endcsname{%
      \catcode35 \the\catcode35\relax
      \catcode64 \the\catcode64\relax
      \catcode123 \the\catcode123\relax
      \catcode125 \the\catcode125\relax
    }%
  }%
\x
\catcode35 6 % #
\catcode64 11 % @
\catcode123 1 % {
\catcode125 2 % }
\def\TMP@EnsureCode#1#2{%
  \edef\pdftexcmds@AtEnd{%
    \pdftexcmds@AtEnd
    \catcode#1 \the\catcode#1\relax
  }%
  \catcode#1 #2\relax
}
\TMP@EnsureCode{10}{12}% ^^J
\TMP@EnsureCode{33}{12}% !
\TMP@EnsureCode{34}{12}% "
\TMP@EnsureCode{39}{12}% '
\TMP@EnsureCode{40}{12}% (
\TMP@EnsureCode{41}{12}% )
\TMP@EnsureCode{42}{12}% *
\TMP@EnsureCode{43}{12}% +
\TMP@EnsureCode{44}{12}% ,
\TMP@EnsureCode{45}{12}% -
\TMP@EnsureCode{46}{12}% .
\TMP@EnsureCode{47}{12}% /
\TMP@EnsureCode{58}{12}% :
\TMP@EnsureCode{60}{12}% <
\TMP@EnsureCode{61}{12}% =
\TMP@EnsureCode{62}{12}% >
\TMP@EnsureCode{94}{7}% ^ (superscript)
\TMP@EnsureCode{95}{12}% _ (other)
\TMP@EnsureCode{126}{12}% ~ (other)
%    \end{macrocode}
%
% \subsection{Load package \xpackage{infwarerr}}
%
%    \begin{macrocode}
\begingroup\expandafter\expandafter\expandafter\endgroup
\expandafter\ifx\csname RequirePackage\endcsname\relax
  \input infwarerr.sty\relax
\else
  \RequirePackage{infwarerr}[2007/09/09]%
\fi
%    \end{macrocode}
%
% \subsection{Without \LuaTeX}
%
%    \begin{macrocode}
\begingroup\expandafter\expandafter\expandafter\endgroup
\expandafter\ifx\csname directlua\endcsname\relax
  \@PackageInfo{pdftexcmds}{LuaTeX not detected}%
  \def\pdftexcmds@nopdftex{%
    \@PackageInfoNoLine{pdftexcmds}{pdfTeX >= 1.30 not detected}%
    \let\pdftexcmds@nopdftex\relax
  }%
  \def\pdftexcmds@temp#1{%
    \begingroup\expandafter\expandafter\expandafter\endgroup
    \expandafter\ifx\csname pdf#1\endcsname\relax
      \pdftexcmds@nopdftex
    \else
      \expandafter\def\csname pdf@#1\expandafter\endcsname
      \expandafter##\expandafter{%
        \csname pdf#1\endcsname
      }%
    \fi
  }%
  \pdftexcmds@temp{strcmp}%
  \pdftexcmds@temp{escapehex}%
  \let\pdf@escapehexnative\pdf@escapehex
  \pdftexcmds@temp{unescapehex}%
  \let\pdf@unescapehexnative\pdf@unescapehex
  \pdftexcmds@temp{escapestring}%
  \pdftexcmds@temp{escapename}%
  \pdftexcmds@temp{filesize}%
  \pdftexcmds@temp{filemoddate}%
  \begingroup\expandafter\expandafter\expandafter\endgroup
  \expandafter\ifx\csname pdfshellescape\endcsname\relax
    \pdftexcmds@nopdftex
  \else
    \def\pdf@shellescape{%
      \pdfshellescape
    }%
  \fi
  \begingroup\expandafter\expandafter\expandafter\endgroup
  \expandafter\ifx\csname pdffiledump\endcsname\relax
    \pdftexcmds@nopdftex
  \else
    \def\pdf@filedump#1#2#3{%
      \pdffiledump offset#1 length#2{#3}%
    }%
  \fi
  \begingroup\expandafter\expandafter\expandafter\endgroup
  \expandafter\ifx\csname pdfmdfivesum\endcsname\relax
    \pdftexcmds@nopdftex
  \else
    \def\pdf@mdfivesum#{\pdfmdfivesum}%
    \let\pdf@mdfivesumnative\pdf@mdfivesum
    \def\pdf@filemdfivesum#{\pdfmdfivesum file}%
  \fi
  \def\pdf@system#{%
    \immediate\write18%
  }%
  \pdftexcmds@AtEnd
  \expandafter\endinput
\fi
%    \end{macrocode}
%
% \subsection{Load module}
%
%    \begin{macrocode}
\begingroup\expandafter\expandafter\expandafter\endgroup
\expandafter\ifx\csname RequirePackage\endcsname\relax
  \input luatex-loader.sty\relax
\else
  \RequirePackage{luatex-loader}[2007/12/12]%
\fi
\directlua0{%
  require("oberdiek.pdftexcmds")%
}
%    \end{macrocode}
%
% \subsection{Lua functions}
%
%    \begin{macro}{\pdftexcmds@toks}
%    \begin{macrocode}
\begingroup\expandafter\expandafter\expandafter\endgroup
\expandafter\ifx\csname newtoks\endcsname\relax
  \toksdef\pdftexcmds@toks=0 %
\else
  \csname newtoks\endcsname\pdftexcmds@toks
\fi
%    \end{macrocode}
%    \end{macro}
%
%    \begin{macro}{\pdf@strcmp}
%    \begin{macrocode}
\long\def\pdf@strcmp#1#2{%
  \directlua0{%
    oberdiek.pdftexcmds.strcmp("\luaescapestring{#1}",%
        "\luaescapestring{#2}")%
  }%
}%
%    \end{macrocode}
%    \end{macro}
%    \begin{macro}{\pdf@escapehex}
%    \begin{macrocode}
\long\def\pdf@escapehex#1{%
  \directlua0{%
    oberdiek.pdftexcmds.escapehex("\luaescapestring{#1}", "byte")%
  }%
}%
%    \end{macrocode}
%    \end{macro}
%    \begin{macro}{\pdf@escapehexnative}
%    \begin{macrocode}
\long\def\pdf@escapehexnative#1{%
  \directlua0{%
    oberdiek.pdftexcmds.escapehex("\luaescapestring{#1}")%
  }%
}%
%    \end{macrocode}
%    \end{macro}
%    \begin{macro}{\pdf@unescapehex}
%    \begin{macrocode}
\def\pdf@unescapehex#1{%
  \the\expandafter\pdftexcmds@toks
  \directlua0{%
    oberdiek.pdftexcmds.toks="pdftexcmds@toks"%
    oberdiek.pdftexcmds.unescapehex("\luaescapestring{#1}", "byte")%
  }%
}%
%    \end{macrocode}
%    \end{macro}
%    \begin{macro}{\pdf@unescapehexnative}
%    \begin{macrocode}
\def\pdf@unescapehexnative#1{%
  \the\expandafter\pdftexcmds@toks
  \directlua0{%
    oberdiek.pdftexcmds.toks="pdftexcmds@toks"%
    oberdiek.pdftexcmds.unescapehex("\luaescapestring{#1}")%
  }%
}%
%    \end{macrocode}
%    \end{macro}
%    \begin{macro}{\pdf@escapestring}
%    \begin{macrocode}
\long\def\pdf@escapestring#1{%
  \directlua0{%
    oberdiek.pdftexcmds.escapestring("\luaescapestring{#1}", "byte")%
  }%
}
%    \end{macrocode}
%    \end{macro}
%    \begin{macro}{\pdf@escapename}
%    \begin{macrocode}
\long\def\pdf@escapename#1{%
  \directlua0{%
    oberdiek.pdftexcmds.escapename("\luaescapestring{#1}", "byte")%
  }%
}
%    \end{macrocode}
%    \end{macro}
%    \begin{macro}{\pdf@escapenamenative}
%    \begin{macrocode}
\long\def\pdf@escapenamenative#1{%
  \directlua0{%
    oberdiek.pdftexcmds.escapename("\luaescapestring{#1}")%
  }%
}
%    \end{macrocode}
%    \end{macro}
%    \begin{macro}{\pdf@filesize}
%    \begin{macrocode}
\def\pdf@filesize#1{%
  \directlua0{%
    oberdiek.pdftexcmds.filesize("\luaescapestring{#1}")%
  }%
}
%    \end{macrocode}
%    \end{macro}
%    \begin{macro}{\pdf@filemoddate}
%    \begin{macrocode}
\def\pdf@filemoddate#1{%
  \directlua0{%
    oberdiek.pdftexcmds.filemoddate("\luaescapestring{#1}")%
  }%
}
%    \end{macrocode}
%    \end{macro}
%    \begin{macro}{\pdf@filedump}
%    \begin{macrocode}
\def\pdf@filedump#1#2#3{%
  \directlua0{%
    oberdiek.pdftexcmds.filedump("\luaescapestring{\number#1}",%
        "\luaescapestring{\number#2}",%
        "\luaescapestring{#3}")%
  }%
}%
%    \end{macrocode}
%    \end{macro}
%    \begin{macro}{\pdf@mdfivesum}
%    \begin{macrocode}
\long\def\pdf@mdfivesum#1{%
  \directlua0{%
    oberdiek.pdftexcmds.mdfivesum("\luaescapestring{#1}", "byte")%
  }%
}%
%    \end{macrocode}
%    \end{macro}
%    \begin{macro}{\pdf@mdfivesumnative}
%    \begin{macrocode}
\long\def\pdf@mdfivesumnative#1{%
  \directlua0{%
    oberdiek.pdftexcmds.mdfivesum("\luaescapestring{#1}")%
  }%
}%
%    \end{macrocode}
%    \end{macro}
%    \begin{macro}{\pdf@filemdfivesum}
%    \begin{macrocode}
\def\pdf@filemdfivesum#1{%
  \directlua0{%
    oberdiek.pdftexcmds.filemdfivesum("\luaescapestring{#1}")%
  }%
}%
%    \end{macrocode}
%    \end{macro}
%
%    \begin{macro}{\pdf@shellescape}
%    \begin{macrocode}
\def\pdf@shellescape{%
  \directlua0{%
    oberdiek.pdftexcmds.shellescape()%
  }%
}
%    \end{macrocode}
%    \end{macro}
%
%    \begin{macro}{\pdf@system}
%    \begin{macrocode}
\def\pdf@system#1{%
  \directlua0{%
    oberdiek.pdftexcmds.system("\luaescapestring{#1}")%
  }%
}
%    \end{macrocode}
%    \end{macro}
%
%    \begin{macro}{\pdf@lastsystemstatus}
%    \begin{macrocode}
\def\pdf@lastsystemstatus{%
  \directlua0{%
    oberdiek.pdftexcmds.lastsystemstatus()%
  }%
}
%    \end{macrocode}
%    \end{macro}
%    \begin{macro}{\pdf@lastsystemexit}
%    \begin{macrocode}
\def\pdf@lastsystemexit{%
  \directlua0{%
    oberdiek.pdftexcmds.lastsystemexit()%
  }%
}
%    \end{macrocode}
%    \end{macro}
%
%    \begin{macro}{\pdf@pipe}
%    Check availability of |io.popen| first.
%    \begin{macrocode}
\ifnum0%
    \directlua0{%
      if io.popen then %
        tex.write("1")%
      end%
    }%
    =1 %
  \def\pdf@pipe#1{%
    \the\expandafter\pdftexcmds@toks
    \directlua0{%
      oberdiek.pdftexcmds.toks="pdftexcmds@toks"%
      oberdiek.pdftexcmds.pipe("\luaescapestring{#1}")%
    }%
  }%
\fi
%    \end{macrocode}
%    \end{macro}
%
%    \begin{macrocode}
\pdftexcmds@AtEnd
%</package>
%    \end{macrocode}
%
% \subsection{Lua module}
%
%    \begin{macrocode}
%<*lua>
%    \end{macrocode}
%
%    \begin{macrocode}
module("oberdiek.pdftexcmds", package.seeall)
local systemexitstatus
function strcmp(A, B)
  if A == B then
    tex.write("0")
  elseif A < B then
    tex.write("-1")
  else
    tex.write("1")
  end
end
local function utf8_to_byte(str)
  local i = 0
  local n = string.len(str)
  local t = {}
  while i < n do
    i = i + 1
    local a = string.byte(str, i)
    if a < 128 then
      table.insert(t, string.char(a))
    else
      if a >= 192 and i < n then
        i = i + 1
        local b = string.byte(str, i)
        if b < 128 or b >= 192 then
          i = i - 1
        elseif a == 194 then
          table.insert(t, string.char(b))
        elseif a == 195 then
          table.insert(t, string.char(b + 64))
        end
      end
    end
  end
  return table.concat(t)
end
function escapehex(str, mode)
  if mode == "byte" then
    str = utf8_to_byte(str)
  end
  tex.write((string.gsub(str, ".",
    function (ch)
      return string.format("%02X", string.byte(ch))
    end
  )))
end
%    \end{macrocode}
%    See procedure |unescapehex| in file \xfile{utils.c} of \pdfTeX.
%    Caution: |tex.write| ignores leading spaces.
%    \begin{macrocode}
function unescapehex(str, mode)
  local a = 0
  local first = true
  local result = {}
  for i = 1, string.len(str), 1 do
    local ch = string.byte(str, i)
    if ch >= 48 and ch <= 57 then
      ch = ch - 48
    elseif ch >= 65 and ch <= 70 then
      ch = ch - 55
    elseif ch >= 97 and ch <= 102 then
      ch = ch - 87
    else
      ch = nil
    end
    if ch then
      if first then
        a = ch * 16
        first = false
      else
        table.insert(result, a + ch)
        first = true
      end
    end
  end
  if not first then
    table.insert(result, a)
  end
  if mode == "byte" then
    local utf8 = {}
    for i, a in ipairs(result) do
      if a < 128 then
        table.insert(utf8, a)
      else
        if a < 192 then
          table.insert(utf8, 194)
          a = a - 128
        else
          table.insert(utf8, 195)
          a = a - 192
        end
        table.insert(utf8, a + 128)
      end
    end
    result = utf8
  end
  tex.settoks(toks, string.char(unpack(result)))
end
%    \end{macrocode}
%    See procedure |escapestring| in file \xfile{utils.c} of \pdfTeX.
%    \begin{macrocode}
function escapestring(str, mode)
  if mode == "byte" then
    str = utf8_to_byte(str)
  end
  tex.write((string.gsub(str, ".",
    function (ch)
      local b = string.byte(ch)
      if b < 33 or b > 126 then
        return string.format("\\%.3o", b)
      end
      if b == 40 or b == 41 or b == 92 then
        return "\\" .. ch
      end
%    \end{macrocode}
%    Lua 5.1 returns the match in case of return value |nil|.
%    \begin{macrocode}
      return nil
    end
  )))
end
%    \end{macrocode}
%    See procedure |escapename| in file \xfile{utils.c} of \pdfTeX.
%    \begin{macrocode}
function escapename(str, mode)
  if mode == "byte" then
    str = utf8_to_byte(str)
  end
  tex.write((string.gsub(str, ".",
    function (ch)
      local b = string.byte(ch)
      if b == 0 then
%    \end{macrocode}
%    In Lua 5.0 |nil| could be used for the empty string,
%    But |nil| returns the match in Lua 5.1, thus we use
%    the empty string explicitly.
%    \begin{macrocode}
        return ""
      end
      if b <= 32 or b >= 127
          or b == 35 or b == 37 or b == 40 or b == 41
          or b == 47 or b == 60 or b == 62 or b == 91
          or b == 93 or b == 123 or b == 125 then
        return string.format("#%.2X", b)
      else
%    \end{macrocode}
%    Lua 5.1 returns the match in case of return value |nil|.
%    \begin{macrocode}
        return nil
      end
    end
  )))
end
function filesize(filename)
  local foundfile = kpse.find_file(filename, "tex", true)
  if foundfile then
    local size = lfs.attributes(foundfile, "size")
    if size then
      tex.write(size)
    end
  end
end
%    \end{macrocode}
%    See procedure |makepdftime| in file \xfile{utils.c} of \pdfTeX.
%    \begin{macrocode}
function filemoddate(filename)
  local foundfile = kpse.find_file(filename, "tex", true)
  if foundfile then
    local date = lfs.attributes(foundfile, "modification")
    if date then
      local d = os.date("*t", date)
      if d.sec >= 60 then
        d.sec = 59
      end
      local u = os.date("!*t", date)
      local off = 60 * (d.hour - u.hour) + d.min - u.min
      if d.year ~= u.year then
        if d.year > u.year then
          off = off + 1440
        else
          off = off - 1440
        end
      elseif d.yday ~= u.yday then
        if d.yday > u.yday then
          off = off + 1440
        else
          off = off - 1440
        end
      end
      local timezone
      if off == 0 then
        timezone = "Z"
      else
        local hours = math.floor(off / 60)
        local mins = math.abs(off - hours * 60)
        timezone = string.format("%+03d'%02d'", hours, mins)
      end
      tex.write(string.format("D:%04d%02d%02d%02d%02d%02d%s",
          d.year, d.month, d.day, d.hour, d.min, d.sec, timezone))
    end
  end
end
function filedump(offset, length, filename)
  length = tonumber(length)
  if length and length > 0 then
    local foundfile = kpse.find_file(filename, "tex", true)
    if foundfile then
      offset = tonumber(offset)
      if not offset then
        offset = 0
      end
      local filehandle = io.open(foundfile, "r")
      if filehandle then
        if offset > 0 then
          filehandle:seek("set", offset)
        end
        local dump = filehandle:read(length)
        escapehex(dump)
      end
    end
  end
end
function mdfivesum(str, mode)
  if mode == "byte" then
    str = utf8_to_byte(str)
  end
  escapehex(md5.sum(str))
end
function filemdfivesum(filename)
  local foundfile = kpse.find_file(filename, "tex", true)
  if foundfile then
    local filehandle = io.open(foundfile, "r")
    if filehandle then
      local contents = filehandle:read("*a")
      escapehex(md5.sum(contents))
    end
  end
end
function shellescape()
  if os.execute then
    tex.write("1")
  else
    tex.write("0")
  end
end
function system(cmdline)
  systemexitstatus = nil
  texio.write_nl("log", "system(" .. cmdline .. ") ")
  if os.execute then
    texio.write("log", "executed.")
    systemexitstatus = os.execute(cmdline)
  else
    texio.write("log", "disabled.")
  end
end
function lastsystemstatus()
  local result = tonumber(systemexitstatus)
  if result then
    local x = math.floor(result / 256)
    tex.write(result - 256 * math.floor(result / 256))
  end
end
function lastsystemexit()
  local result = tonumber(systemexitstatus)
  if result then
    tex.write(math.floor(result / 256))
  end
end
function pipe(cmdline)
  local result
  systemexitstatus = nil
  texio.write_nl("log", "pipe(" .. cmdline ..") ")
  if io.popen then
    texio.write("log", "executed.")
    local handle = io.popen(cmdline, "r")
    if handle then
      result = handle:read("*a")
      handle:close()
    end
  else
    texio.write("log", "disabled.")
  end
  if result then
    tex.settoks(toks, result)
  else
    tex.settoks(toks, "")
  end
end
%    \end{macrocode}
%    \begin{macrocode}
%</lua>
%    \end{macrocode}
%
% \section{Test}
%
% \subsection{Catcode checks for loading}
%
%    \begin{macrocode}
%<*test1>
%    \end{macrocode}
%    \begin{macrocode}
\catcode`\{=1 %
\catcode`\}=2 %
\catcode`\#=6 %
\catcode`\@=11 %
\expandafter\ifx\csname count@\endcsname\relax
  \countdef\count@=255 %
\fi
\expandafter\ifx\csname @gobble\endcsname\relax
  \long\def\@gobble#1{}%
\fi
\expandafter\ifx\csname @firstofone\endcsname\relax
  \long\def\@firstofone#1{#1}%
\fi
\expandafter\ifx\csname loop\endcsname\relax
  \expandafter\@firstofone
\else
  \expandafter\@gobble
\fi
{%
  \def\loop#1\repeat{%
    \def\body{#1}%
    \iterate
  }%
  \def\iterate{%
    \body
      \let\next\iterate
    \else
      \let\next\relax
    \fi
    \next
  }%
  \let\repeat=\fi
}%
\def\RestoreCatcodes{}
\count@=0 %
\loop
  \edef\RestoreCatcodes{%
    \RestoreCatcodes
    \catcode\the\count@=\the\catcode\count@\relax
  }%
\ifnum\count@<255 %
  \advance\count@ 1 %
\repeat

\def\RangeCatcodeInvalid#1#2{%
  \count@=#1\relax
  \loop
    \catcode\count@=15 %
  \ifnum\count@<#2\relax
    \advance\count@ 1 %
  \repeat
}
\expandafter\ifx\csname LoadCommand\endcsname\relax
  \def\LoadCommand{\input pdftexcmds.sty\relax}%
\fi
\def\Test{%
  \RangeCatcodeInvalid{0}{47}%
  \RangeCatcodeInvalid{58}{64}%
  \RangeCatcodeInvalid{91}{96}%
  \RangeCatcodeInvalid{123}{255}%
  \catcode`\@=12 %
  \catcode`\\=0 %
  \catcode`\{=1 %
  \catcode`\}=2 %
  \catcode`\#=6 %
  \catcode`\[=12 %
  \catcode`\]=12 %
  \catcode`\%=14 %
  \catcode`\ =10 %
  \catcode13=5 %
  \LoadCommand
  \RestoreCatcodes
}
\Test
\csname @@end\endcsname
\end
%    \end{macrocode}
%    \begin{macrocode}
%</test1>
%    \end{macrocode}
%
% \section{Installation}
%
% \subsection{Download}
%
% \paragraph{Package.} This package is available on
% CTAN\footnote{\url{ftp://ftp.ctan.org/tex-archive/}}:
% \begin{description}
% \item[\CTAN{macros/latex/contrib/oberdiek/pdftexcmds.dtx}] The source file.
% \item[\CTAN{macros/latex/contrib/oberdiek/pdftexcmds.pdf}] Documentation.
% \end{description}
%
%
% \paragraph{Bundle.} All the packages of the bundle `oberdiek'
% are also available in a TDS compliant ZIP archive. There
% the packages are already unpacked and the documentation files
% are generated. The files and directories obey the TDS standard.
% \begin{description}
% \item[\CTAN{macros/latex/contrib/oberdiek/oberdiek-tds.zip}]
% \end{description}
% \emph{TDS} refers to the standard ``A Directory Structure
% for \TeX\ Files'' (\CTAN{tds/tds.pdf}). Directories
% with \xfile{texmf} in their name are usually organized this way.
%
% \subsection{Bundle installation}
%
% \paragraph{Unpacking.} Unpack the \xfile{oberdiek-tds.zip} in the
% TDS tree (also known as \xfile{texmf} tree) of your choice.
% Example (linux):
% \begin{quote}
%   |unzip oberdiek-tds.zip -d ~/texmf|
% \end{quote}
%
% \paragraph{Script installation.}
% Check the directory \xfile{TDS:scripts/oberdiek/} for
% scripts that need further installation steps.
% Package \xpackage{attachfile2} comes with the Perl script
% \xfile{pdfatfi.pl} that should be installed in such a way
% that it can be called as \texttt{pdfatfi}.
% Example (linux):
% \begin{quote}
%   |chmod +x scripts/oberdiek/pdfatfi.pl|\\
%   |cp scripts/oberdiek/pdfatfi.pl /usr/local/bin/|
% \end{quote}
%
% \subsection{Package installation}
%
% \paragraph{Unpacking.} The \xfile{.dtx} file is a self-extracting
% \docstrip\ archive. The files are extracted by running the
% \xfile{.dtx} through \plainTeX:
% \begin{quote}
%   \verb|tex pdftexcmds.dtx|
% \end{quote}
%
% \paragraph{TDS.} Now the different files must be moved into
% the different directories in your installation TDS tree
% (also known as \xfile{texmf} tree):
% \begin{quote}
% \def\t{^^A
% \begin{tabular}{@{}>{\ttfamily}l@{ $\rightarrow$ }>{\ttfamily}l@{}}
%   pdftexcmds.sty & tex/generic/oberdiek/pdftexcmds.sty\\
%   oberdiek.pdftexcmds.lua & scripts/oberdiek/oberdiek.pdftexcmds.lua\\
%   pdftexcmds.lua & scripts/oberdiek/pdftexcmds.lua\\
%   pdftexcmds.pdf & doc/latex/oberdiek/pdftexcmds.pdf\\
%   pdftexcmds.dtx & source/latex/oberdiek/pdftexcmds.dtx\\
% \end{tabular}^^A
% }^^A
% \sbox0{\t}^^A
% \ifdim\wd0>\linewidth
%   \begingroup
%     \advance\linewidth by\leftmargin
%     \advance\linewidth by\rightmargin
%   \edef\x{\endgroup
%     \def\noexpand\lw{\the\linewidth}^^A
%   }\x
%   \def\lwbox{^^A
%     \leavevmode
%     \hbox to \linewidth{^^A
%       \kern-\leftmargin\relax
%       \hss
%       \usebox0
%       \hss
%       \kern-\rightmargin\relax
%     }^^A
%   }^^A
%   \ifdim\wd0>\lw
%     \sbox0{\small\t}^^A
%     \ifdim\wd0>\linewidth
%       \ifdim\wd0>\lw
%         \sbox0{\footnotesize\t}^^A
%         \ifdim\wd0>\linewidth
%           \ifdim\wd0>\lw
%             \sbox0{\scriptsize\t}^^A
%             \ifdim\wd0>\linewidth
%               \ifdim\wd0>\lw
%                 \sbox0{\tiny\t}^^A
%                 \ifdim\wd0>\linewidth
%                   \lwbox
%                 \else
%                   \usebox0
%                 \fi
%               \else
%                 \lwbox
%               \fi
%             \else
%               \usebox0
%             \fi
%           \else
%             \lwbox
%           \fi
%         \else
%           \usebox0
%         \fi
%       \else
%         \lwbox
%       \fi
%     \else
%       \usebox0
%     \fi
%   \else
%     \lwbox
%   \fi
% \else
%   \usebox0
% \fi
% \end{quote}
% If you have a \xfile{docstrip.cfg} that configures and enables \docstrip's
% TDS installing feature, then some files can already be in the right
% place, see the documentation of \docstrip.
%
% \subsection{Refresh file name databases}
%
% If your \TeX~distribution
% (\teTeX, \mikTeX, \dots) relies on file name databases, you must refresh
% these. For example, \teTeX\ users run \verb|texhash| or
% \verb|mktexlsr|.
%
% \subsection{Some details for the interested}
%
% \paragraph{Attached source.}
%
% The PDF documentation on CTAN also includes the
% \xfile{.dtx} source file. It can be extracted by
% AcrobatReader 6 or higher. Another option is \textsf{pdftk},
% e.g. unpack the file into the current directory:
% \begin{quote}
%   \verb|pdftk pdftexcmds.pdf unpack_files output .|
% \end{quote}
%
% \paragraph{Unpacking with \LaTeX.}
% The \xfile{.dtx} chooses its action depending on the format:
% \begin{description}
% \item[\plainTeX:] Run \docstrip\ and extract the files.
% \item[\LaTeX:] Generate the documentation.
% \end{description}
% If you insist on using \LaTeX\ for \docstrip\ (really,
% \docstrip\ does not need \LaTeX), then inform the autodetect routine
% about your intention:
% \begin{quote}
%   \verb|latex \let\install=y% \iffalse meta-comment
%
% Copyright (C) 2007 by Heiko Oberdiek <oberdiek@uni-freiburg.de>
%
% This work may be distributed and/or modified under the
% conditions of the LaTeX Project Public License, either
% version 1.3 of this license or (at your option) any later
% version. The latest version of this license is in
%    http://www.latex-project.org/lppl.txt
% and version 1.3 or later is part of all distributions of
% LaTeX version 2005/12/01 or later.
%
% This work has the LPPL maintenance status "maintained".
%
% This Current Maintainer of this work is Heiko Oberdiek.
%
% This work consists of the main source file pdftexcmds.dtx
% and the derived files
%    pdftexcmds.sty, pdftexcmds.pdf, pdftexcmds.ins, pdftexcmds.drv,
%    oberdiek.pdftexcmds.lua, pdftexcmds.lua.
%
% Distribution:
%    CTAN:macros/latex/contrib/oberdiek/pdftexcmds.dtx
%    CTAN:macros/latex/contrib/oberdiek/pdftexcmds.pdf
%
% Unpacking:
%    (a) If pdftexcmds.ins is present:
%           tex pdftexcmds.ins
%    (b) Without pdftexcmds.ins:
%           tex pdftexcmds.dtx
%    (c) If you insist on using LaTeX
%           latex \let\install=y\input{pdftexcmds.dtx}
%        (quote the arguments according to the demands of your shell)
%
% Documentation:
%    (a) If pdftexcmds.drv is present:
%           latex pdftexcmds.drv
%    (b) Without pdftexcmds.drv:
%           latex pdftexcmds.dtx; ...
%    The class ltxdoc loads the configuration file ltxdoc.cfg
%    if available. Here you can specify further options, e.g.
%    use A4 as paper format:
%       \PassOptionsToClass{a4paper}{article}
%
%    Programm calls to get the documentation (example):
%       pdflatex pdftexcmds.dtx
%       makeindex -s gind.ist pdftexcmds.idx
%       pdflatex pdftexcmds.dtx
%       makeindex -s gind.ist pdftexcmds.idx
%       pdflatex pdftexcmds.dtx
%
% Installation:
%    TDS:tex/generic/oberdiek/pdftexcmds.sty
%    TDS:scripts/oberdiek/oberdiek.pdftexcmds.lua
%    TDS:scripts/oberdiek/pdftexcmds.lua
%    TDS:doc/latex/oberdiek/pdftexcmds.pdf
%    TDS:source/latex/oberdiek/pdftexcmds.dtx
%
%<*ignore>
\begingroup
  \def\x{LaTeX2e}%
\expandafter\endgroup
\ifcase 0\ifx\install y1\fi\expandafter
         \ifx\csname processbatchFile\endcsname\relax\else1\fi
         \ifx\fmtname\x\else 1\fi\relax
\else\csname fi\endcsname
%</ignore>
%<*install>
\input docstrip.tex
\Msg{************************************************************************}
\Msg{* Installation}
\Msg{* Package: pdftexcmds 2007/12/12 v0.3 LuaTeX support for pdfTeX utility functions (HO)}
\Msg{************************************************************************}

\keepsilent
\askforoverwritefalse

\let\MetaPrefix\relax
\preamble

This is a generated file.

Copyright (C) 2007 by Heiko Oberdiek <oberdiek@uni-freiburg.de>

This work may be distributed and/or modified under the
conditions of the LaTeX Project Public License, either
version 1.3 of this license or (at your option) any later
version. The latest version of this license is in
   http://www.latex-project.org/lppl.txt
and version 1.3 or later is part of all distributions of
LaTeX version 2005/12/01 or later.

This work has the LPPL maintenance status "maintained".

This Current Maintainer of this work is Heiko Oberdiek.

This work consists of the main source file pdftexcmds.dtx
and the derived files
   pdftexcmds.sty, pdftexcmds.pdf, pdftexcmds.ins, pdftexcmds.drv,
   oberdiek.pdftexcmds.lua, pdftexcmds.lua.

\endpreamble
\let\MetaPrefix\DoubleperCent

\generate{%
  \file{pdftexcmds.ins}{\from{pdftexcmds.dtx}{install}}%
  \file{pdftexcmds.drv}{\from{pdftexcmds.dtx}{driver}}%
  \usedir{tex/generic/oberdiek}%
  \file{pdftexcmds.sty}{\from{pdftexcmds.dtx}{package}}%
}
\def\MetaPrefix{-- }
\def\defaultpostamble{%
  \MetaPrefix^^J%
  \MetaPrefix\space End of File `\outFileName'.%
}
\def\currentpostamble{\defaultpostamble}%
\generate{%
  \usedir{scripts/oberdiek}%
  \file{oberdiek.pdftexcmds.lua}{\from{pdftexcmds.dtx}{lua}}%
  \file{pdftexcmds.lua}{\from{pdftexcmds.dtx}{lua}}%
}

\obeyspaces
\Msg{************************************************************************}
\Msg{*}
\Msg{* To finish the installation you have to move the following}
\Msg{* file into a directory searched by TeX:}
\Msg{*}
\Msg{*     pdftexcmds.sty}
\Msg{*}
\Msg{* And install the following script files:}
\Msg{*}
\Msg{*     oberdiek.pdftexcmds.lua, pdftexcmds.lua}
\Msg{*}
\Msg{* To produce the documentation run the file `pdftexcmds.drv'}
\Msg{* through LaTeX.}
\Msg{*}
\Msg{* Happy TeXing!}
\Msg{*}
\Msg{************************************************************************}

\endbatchfile
%</install>
%<*ignore>
\fi
%</ignore>
%<*driver>
\NeedsTeXFormat{LaTeX2e}
\ProvidesFile{pdftexcmds.drv}%
  [2007/12/12 v0.3 LuaTeX support for pdfTeX utility functions (HO)]%
\documentclass{ltxdoc}
\usepackage{holtxdoc}[2007/11/11]
\usepackage{paralist}
\begin{document}
  \DocInput{pdftexcmds.dtx}%
\end{document}
%</driver>
% \fi
%
% \CheckSum{552}
%
% \CharacterTable
%  {Upper-case    \A\B\C\D\E\F\G\H\I\J\K\L\M\N\O\P\Q\R\S\T\U\V\W\X\Y\Z
%   Lower-case    \a\b\c\d\e\f\g\h\i\j\k\l\m\n\o\p\q\r\s\t\u\v\w\x\y\z
%   Digits        \0\1\2\3\4\5\6\7\8\9
%   Exclamation   \!     Double quote  \"     Hash (number) \#
%   Dollar        \$     Percent       \%     Ampersand     \&
%   Acute accent  \'     Left paren    \(     Right paren   \)
%   Asterisk      \*     Plus          \+     Comma         \,
%   Minus         \-     Point         \.     Solidus       \/
%   Colon         \:     Semicolon     \;     Less than     \<
%   Equals        \=     Greater than  \>     Question mark \?
%   Commercial at \@     Left bracket  \[     Backslash     \\
%   Right bracket \]     Circumflex    \^     Underscore    \_
%   Grave accent  \`     Left brace    \{     Vertical bar  \|
%   Right brace   \}     Tilde         \~}
%
% \GetFileInfo{pdftexcmds.drv}
%
% \title{The \xpackage{pdftexcmds} package}
% \date{2007/12/12 v0.3}
% \author{Heiko Oberdiek\\\xemail{oberdiek@uni-freiburg.de}}
%
% \maketitle
%
% \begin{abstract}
% \LuaTeX\ provides most of the commands of \pdfTeX\ 1.40. However
% a number of utility functions are removed. This package tries to fill
% the gap and implements some of the missing primitive using Lua.
% \end{abstract}
%
% \tableofcontents
%
% \section{Documentation}
%
% Some primitives of \pdfTeX\ are not defined by \LuaTeX.
% This package implements macro based solutions using Lua code
% for the following missing \pdfTeX\ primitives;
% \begin{compactitem}
% \item \cs{pdfstrcmp}
% \item \cs{pdfunescapehex}
% \item \cs{pdfescapehex}
% \item \cs{pdfescapename}
% \item \cs{pdfescapestring}
% \item \cs{pdffilesize}
% \item \cs{pdffilemoddate}
% \item \cs{pdffiledump}
% \item \cs{pdfmdfivesum}
% \item |\immediate\write18|
% \end{compactitem}
% The original names of the primitives cannot be used:
% \begin{itemize}
% \item
% The syntax for their arguments cannot easily
% simulated by macros. The primitives using key words
% such as |file| (\cs{pdfmdfivesum}) or |offset| and |length|
% (\cs{pdffiledump}) and uses \meta{general text} for the other
% arguments. Using token registers assignments, \meta{general text} could
% be catched. However, the simulated primitives are expandable
% and register assignments would destroy this important property.
% (\meta{general text} allows something like |\expandafter\bgroup ...}|.)
% \item
% The original primitives can be expanded using one expansion step.
% The new macros need two expansion steps because of the additional
% macro expansion. Example:
% \begin{quote}
%   |\expandafter\foo\pdffilemoddate{file}|\\
%   vs.
%   |\expandafter\expandafter\expandafter\foo\pdf@filemoddate{file}|.
% \end{quote}
% \end{itemize}
%
% \LuaTeX\ isn't stable yet and thus the status of this package is
% \emph{experimental}. Feedback is welcome.
%
% \subsection{General principles}
%
% \begin{description}
% \item[Naming convention:]
%   Usually this package defines a macro |\pdf@|\meta{cmd} if
%   \pdfTeX\ provides |\pdf|\meta{cmd}.
% \item[Arguments:] The order of arguments in |\pdf@|\meta{cmd}
%   is the same as for the corresponding primitive of \pdfTeX.
%   The arguments are ordinary undelimited \TeX\ arguments,
%   no \meta{general text} and without additional keywords.
% \item[Expandibility:]
%   The macro |\pdf@|\meta{cmd} is expandable if the
%   corresponding \pdfTeX\ primitive has this property.
%   Exact two expansion steps are necessary (first is the macro
%   expansion).
% \item[Without \LuaTeX:]
%   The macros |\pdf@|\meta{cmd} are mapped to the commands
%   of \pdfTeX\ if they are available. Otherwise they are undefined.
% \end{description}
%
% \subsection{Macros}
%
% \begin{declcs}{pdf@strcmp} \M{stringA} \M{stringB}
% \end{declcs}
% Same as |\pdfstrcmp{|\meta{stringA}|}{|\meta{stringB}|}|.
%
% \begin{declcs}{pdf@unescapehex} \M{string}
% \end{declcs}
% Same as |\pdfunescapehex{|\meta{string}|}|.
% The argument is a byte string given in hexadecimal notation.
% The result are character tokens from 0 until 255 with
% catcode 12 and the space with catcode 10.
%
% \begin{declcs}{pdf@escapehex} \M{string}\\
%   \cs{pdf@escapestring} \M{string}\\
%   \cs{pdf@escapename} \M{string}
% \end{declcs}
% Same as the primitives of \pdfTeX. However \pdfTeX\ does not
% know about characters with codes 256 and larger. Thus the
% string is treated as byte string, characters with more than
% eight bits are ignored.
%
% \begin{declcs}{pdf@filesize} \M{filename}
% \end{declcs}
% Same as |\pdffilesize{|\meta{filename}|}|.
%
% \begin{declcs}{pdf@filemoddate} \M{filename}
% \end{declcs}
% Same as |\pdffilemoddate{|\meta{filename}|}|.
%
% \begin{declcs}{pdf@filedump} \M{offset} \M{length} \M{filename}
% \end{declcs}
% Same as |\pdffiledump offset| \meta{offset} |length| \meta{length}
% |{|\meta{filename}|}|. Both \meta{offset} and \meta{length} must
% not be empty, but must be a valid \TeX\ number.
%
% \begin{declcs}{pdf@mdfivesum} \M{string}
% \end{declcs}
% Same as |\pdfmdfivesum{|\meta{string}|}|. Keyword |file| is supported
% by macro \cs{pdf@filemdfivesum}.
%
% \begin{declcs}{pdf@filemdfivesum} \M{filename}
% \end{declcs}
% Same as |\pdfmdfivesum file{|\meta{filename}|}|.
%
% \begin{declcs}{pdf@shellescape}
% \end{declcs}
% Same as |\pdfshellescape|. It expands to |1| if external
% commands can be executed and |0| otherwise. In \pdfTeX\ external
% commands must be enabled first by command line option or
% configuration option. In \LuaTeX\ option |--safer| disables
% the execution of external commands.
%
% \begin{declcs}{pdf@system} \M{cmdline}
% \end{declcs}
% It is a wrapper for |\immediate\write18| in \pdfTeX\ or
% |os.execute| in \LuaTeX.
%
% In theory |os.execute|
% returns a status number. But its meaning is quite
% undefined. Are there some reliable properties?
% Does it make sense to provide an user interface to
% this status exit code?
%
% \subsubsection{Experimental}
%
% \begin{declcs}{pdf@unescapehexnative} \M{string}\\
%   \cs{pdf@escapehexnative} \M{string}\\
%   \cs{pdf@escapenamenative} \M{string}\\
%   \cs{pdf@mdfivesumnative} \M{string}
% \end{declcs}
% The variants without |native| in the macro name are supposed to
% be compatible with \pdfTeX. However characters with more than
% eight bits are not supported and are ignored. If \LuaTeX\ is
% running, then its UTF-8 coded strings are used. Thus the full
% unicode character range is supported. However the result
% differs from \pdfTeX\ for characters with eight or more bits.
%
% \begin{declcs}{pdf@pipe} \M{cmdline}
% \end{declcs}
% It calls \meta{cmdline} and returns the output of the external
% program in the usual manner as byte string (catcode 12, space with
% catcode 10). The Lua documentation says, that the used |io.popen|
% may not be available on all platforms. Then macro \cs{pdf@pipe}
% is undefined.
%
% \StopEventually{
% }
%
% \section{Implementation}
%
%    \begin{macrocode}
%<*package>
%    \end{macrocode}
%
% \subsection{Reload check and package identification}
%    Reload check, especially if the package is not used with \LaTeX.
%    \begin{macrocode}
\begingroup
  \catcode44 12 % ,
  \catcode45 12 % -
  \catcode46 12 % .
  \catcode58 12 % :
  \catcode64 11 % @
  \expandafter\let\expandafter\x\csname ver@pdftexcmds.sty\endcsname
  \ifcase 0%
    \ifx\x\relax % plain
    \else
      \ifx\x\empty % LaTeX
      \else
        1%
      \fi
    \fi
  \else
    \catcode35 6 % #
    \catcode123 1 % {
    \catcode125 2 % }
    \expandafter\ifx\csname PackageInfo\endcsname\relax
      \def\x#1#2{%
        \immediate\write-1{Package #1 Info: #2.}%
      }%
    \else
      \def\x#1#2{\PackageInfo{#1}{#2, stopped}}%
    \fi
    \x{pdftexcmds}{The package is already loaded}%
    \endgroup
    \expandafter\endinput
  \fi
\endgroup
%    \end{macrocode}
%    Package identification:
%    \begin{macrocode}
\begingroup
  \catcode35 6 % #
  \catcode40 12 % (
  \catcode41 12 % )
  \catcode44 12 % ,
  \catcode45 12 % -
  \catcode46 12 % .
  \catcode47 12 % /
  \catcode58 12 % :
  \catcode64 11 % @
  \catcode123 1 % {
  \catcode125 2 % }
  \expandafter\ifx\csname ProvidesPackage\endcsname\relax
    \def\x#1#2#3[#4]{\endgroup
      \immediate\write-1{Package: #3 #4}%
      \xdef#1{#4}%
    }%
  \else
    \def\x#1#2[#3]{\endgroup
      #2[{#3}]%
      \ifx#1\relax
        \xdef#1{#3}%
      \fi
    }%
  \fi
\expandafter\x\csname ver@pdftexcmds.sty\endcsname
\ProvidesPackage{pdftexcmds}%
  [2007/12/12 v0.3 LuaTeX support for pdfTeX utility functions (HO)]
%    \end{macrocode}
%
% \subsection{Catcodes}
%
%    \begin{macrocode}
\begingroup
  \catcode123 1 % {
  \catcode125 2 % }
  \def\x{\endgroup
    \expandafter\edef\csname pdftexcmds@AtEnd\endcsname{%
      \catcode35 \the\catcode35\relax
      \catcode64 \the\catcode64\relax
      \catcode123 \the\catcode123\relax
      \catcode125 \the\catcode125\relax
    }%
  }%
\x
\catcode35 6 % #
\catcode64 11 % @
\catcode123 1 % {
\catcode125 2 % }
\def\TMP@EnsureCode#1#2{%
  \edef\pdftexcmds@AtEnd{%
    \pdftexcmds@AtEnd
    \catcode#1 \the\catcode#1\relax
  }%
  \catcode#1 #2\relax
}
\TMP@EnsureCode{10}{12}% ^^J
\TMP@EnsureCode{33}{12}% !
\TMP@EnsureCode{34}{12}% "
\TMP@EnsureCode{39}{12}% '
\TMP@EnsureCode{40}{12}% (
\TMP@EnsureCode{41}{12}% )
\TMP@EnsureCode{42}{12}% *
\TMP@EnsureCode{43}{12}% +
\TMP@EnsureCode{44}{12}% ,
\TMP@EnsureCode{45}{12}% -
\TMP@EnsureCode{46}{12}% .
\TMP@EnsureCode{47}{12}% /
\TMP@EnsureCode{58}{12}% :
\TMP@EnsureCode{60}{12}% <
\TMP@EnsureCode{61}{12}% =
\TMP@EnsureCode{62}{12}% >
\TMP@EnsureCode{94}{7}% ^ (superscript)
\TMP@EnsureCode{95}{12}% _ (other)
\TMP@EnsureCode{126}{12}% ~ (other)
%    \end{macrocode}
%
% \subsection{Load package \xpackage{infwarerr}}
%
%    \begin{macrocode}
\begingroup\expandafter\expandafter\expandafter\endgroup
\expandafter\ifx\csname RequirePackage\endcsname\relax
  \input infwarerr.sty\relax
\else
  \RequirePackage{infwarerr}[2007/09/09]%
\fi
%    \end{macrocode}
%
% \subsection{Without \LuaTeX}
%
%    \begin{macrocode}
\begingroup\expandafter\expandafter\expandafter\endgroup
\expandafter\ifx\csname directlua\endcsname\relax
  \@PackageInfo{pdftexcmds}{LuaTeX not detected}%
  \def\pdftexcmds@nopdftex{%
    \@PackageInfoNoLine{pdftexcmds}{pdfTeX >= 1.30 not detected}%
    \let\pdftexcmds@nopdftex\relax
  }%
  \def\pdftexcmds@temp#1{%
    \begingroup\expandafter\expandafter\expandafter\endgroup
    \expandafter\ifx\csname pdf#1\endcsname\relax
      \pdftexcmds@nopdftex
    \else
      \expandafter\def\csname pdf@#1\expandafter\endcsname
      \expandafter##\expandafter{%
        \csname pdf#1\endcsname
      }%
    \fi
  }%
  \pdftexcmds@temp{strcmp}%
  \pdftexcmds@temp{escapehex}%
  \let\pdf@escapehexnative\pdf@escapehex
  \pdftexcmds@temp{unescapehex}%
  \let\pdf@unescapehexnative\pdf@unescapehex
  \pdftexcmds@temp{escapestring}%
  \pdftexcmds@temp{escapename}%
  \pdftexcmds@temp{filesize}%
  \pdftexcmds@temp{filemoddate}%
  \begingroup\expandafter\expandafter\expandafter\endgroup
  \expandafter\ifx\csname pdfshellescape\endcsname\relax
    \pdftexcmds@nopdftex
  \else
    \def\pdf@shellescape{%
      \pdfshellescape
    }%
  \fi
  \begingroup\expandafter\expandafter\expandafter\endgroup
  \expandafter\ifx\csname pdffiledump\endcsname\relax
    \pdftexcmds@nopdftex
  \else
    \def\pdf@filedump#1#2#3{%
      \pdffiledump offset#1 length#2{#3}%
    }%
  \fi
  \begingroup\expandafter\expandafter\expandafter\endgroup
  \expandafter\ifx\csname pdfmdfivesum\endcsname\relax
    \pdftexcmds@nopdftex
  \else
    \def\pdf@mdfivesum#{\pdfmdfivesum}%
    \let\pdf@mdfivesumnative\pdf@mdfivesum
    \def\pdf@filemdfivesum#{\pdfmdfivesum file}%
  \fi
  \def\pdf@system#{%
    \immediate\write18%
  }%
  \pdftexcmds@AtEnd
  \expandafter\endinput
\fi
%    \end{macrocode}
%
% \subsection{Load module}
%
%    \begin{macrocode}
\begingroup\expandafter\expandafter\expandafter\endgroup
\expandafter\ifx\csname RequirePackage\endcsname\relax
  \input luatex-loader.sty\relax
\else
  \RequirePackage{luatex-loader}[2007/12/12]%
\fi
\directlua0{%
  require("oberdiek.pdftexcmds")%
}
%    \end{macrocode}
%
% \subsection{Lua functions}
%
%    \begin{macro}{\pdftexcmds@toks}
%    \begin{macrocode}
\begingroup\expandafter\expandafter\expandafter\endgroup
\expandafter\ifx\csname newtoks\endcsname\relax
  \toksdef\pdftexcmds@toks=0 %
\else
  \csname newtoks\endcsname\pdftexcmds@toks
\fi
%    \end{macrocode}
%    \end{macro}
%
%    \begin{macro}{\pdf@strcmp}
%    \begin{macrocode}
\long\def\pdf@strcmp#1#2{%
  \directlua0{%
    oberdiek.pdftexcmds.strcmp("\luaescapestring{#1}",%
        "\luaescapestring{#2}")%
  }%
}%
%    \end{macrocode}
%    \end{macro}
%    \begin{macro}{\pdf@escapehex}
%    \begin{macrocode}
\long\def\pdf@escapehex#1{%
  \directlua0{%
    oberdiek.pdftexcmds.escapehex("\luaescapestring{#1}", "byte")%
  }%
}%
%    \end{macrocode}
%    \end{macro}
%    \begin{macro}{\pdf@escapehexnative}
%    \begin{macrocode}
\long\def\pdf@escapehexnative#1{%
  \directlua0{%
    oberdiek.pdftexcmds.escapehex("\luaescapestring{#1}")%
  }%
}%
%    \end{macrocode}
%    \end{macro}
%    \begin{macro}{\pdf@unescapehex}
%    \begin{macrocode}
\def\pdf@unescapehex#1{%
  \the\expandafter\pdftexcmds@toks
  \directlua0{%
    oberdiek.pdftexcmds.toks="pdftexcmds@toks"%
    oberdiek.pdftexcmds.unescapehex("\luaescapestring{#1}", "byte")%
  }%
}%
%    \end{macrocode}
%    \end{macro}
%    \begin{macro}{\pdf@unescapehexnative}
%    \begin{macrocode}
\def\pdf@unescapehexnative#1{%
  \the\expandafter\pdftexcmds@toks
  \directlua0{%
    oberdiek.pdftexcmds.toks="pdftexcmds@toks"%
    oberdiek.pdftexcmds.unescapehex("\luaescapestring{#1}")%
  }%
}%
%    \end{macrocode}
%    \end{macro}
%    \begin{macro}{\pdf@escapestring}
%    \begin{macrocode}
\long\def\pdf@escapestring#1{%
  \directlua0{%
    oberdiek.pdftexcmds.escapestring("\luaescapestring{#1}", "byte")%
  }%
}
%    \end{macrocode}
%    \end{macro}
%    \begin{macro}{\pdf@escapename}
%    \begin{macrocode}
\long\def\pdf@escapename#1{%
  \directlua0{%
    oberdiek.pdftexcmds.escapename("\luaescapestring{#1}", "byte")%
  }%
}
%    \end{macrocode}
%    \end{macro}
%    \begin{macro}{\pdf@escapenamenative}
%    \begin{macrocode}
\long\def\pdf@escapenamenative#1{%
  \directlua0{%
    oberdiek.pdftexcmds.escapename("\luaescapestring{#1}")%
  }%
}
%    \end{macrocode}
%    \end{macro}
%    \begin{macro}{\pdf@filesize}
%    \begin{macrocode}
\def\pdf@filesize#1{%
  \directlua0{%
    oberdiek.pdftexcmds.filesize("\luaescapestring{#1}")%
  }%
}
%    \end{macrocode}
%    \end{macro}
%    \begin{macro}{\pdf@filemoddate}
%    \begin{macrocode}
\def\pdf@filemoddate#1{%
  \directlua0{%
    oberdiek.pdftexcmds.filemoddate("\luaescapestring{#1}")%
  }%
}
%    \end{macrocode}
%    \end{macro}
%    \begin{macro}{\pdf@filedump}
%    \begin{macrocode}
\def\pdf@filedump#1#2#3{%
  \directlua0{%
    oberdiek.pdftexcmds.filedump("\luaescapestring{\number#1}",%
        "\luaescapestring{\number#2}",%
        "\luaescapestring{#3}")%
  }%
}%
%    \end{macrocode}
%    \end{macro}
%    \begin{macro}{\pdf@mdfivesum}
%    \begin{macrocode}
\long\def\pdf@mdfivesum#1{%
  \directlua0{%
    oberdiek.pdftexcmds.mdfivesum("\luaescapestring{#1}", "byte")%
  }%
}%
%    \end{macrocode}
%    \end{macro}
%    \begin{macro}{\pdf@mdfivesumnative}
%    \begin{macrocode}
\long\def\pdf@mdfivesumnative#1{%
  \directlua0{%
    oberdiek.pdftexcmds.mdfivesum("\luaescapestring{#1}")%
  }%
}%
%    \end{macrocode}
%    \end{macro}
%    \begin{macro}{\pdf@filemdfivesum}
%    \begin{macrocode}
\def\pdf@filemdfivesum#1{%
  \directlua0{%
    oberdiek.pdftexcmds.filemdfivesum("\luaescapestring{#1}")%
  }%
}%
%    \end{macrocode}
%    \end{macro}
%
%    \begin{macro}{\pdf@shellescape}
%    \begin{macrocode}
\def\pdf@shellescape{%
  \directlua0{%
    oberdiek.pdftexcmds.shellescape()%
  }%
}
%    \end{macrocode}
%    \end{macro}
%
%    \begin{macro}{\pdf@system}
%    \begin{macrocode}
\def\pdf@system#1{%
  \directlua0{%
    oberdiek.pdftexcmds.system("\luaescapestring{#1}")%
  }%
}
%    \end{macrocode}
%    \end{macro}
%
%    \begin{macro}{\pdf@lastsystemstatus}
%    \begin{macrocode}
\def\pdf@lastsystemstatus{%
  \directlua0{%
    oberdiek.pdftexcmds.lastsystemstatus()%
  }%
}
%    \end{macrocode}
%    \end{macro}
%    \begin{macro}{\pdf@lastsystemexit}
%    \begin{macrocode}
\def\pdf@lastsystemexit{%
  \directlua0{%
    oberdiek.pdftexcmds.lastsystemexit()%
  }%
}
%    \end{macrocode}
%    \end{macro}
%
%    \begin{macro}{\pdf@pipe}
%    Check availability of |io.popen| first.
%    \begin{macrocode}
\ifnum0%
    \directlua0{%
      if io.popen then %
        tex.write("1")%
      end%
    }%
    =1 %
  \def\pdf@pipe#1{%
    \the\expandafter\pdftexcmds@toks
    \directlua0{%
      oberdiek.pdftexcmds.toks="pdftexcmds@toks"%
      oberdiek.pdftexcmds.pipe("\luaescapestring{#1}")%
    }%
  }%
\fi
%    \end{macrocode}
%    \end{macro}
%
%    \begin{macrocode}
\pdftexcmds@AtEnd
%</package>
%    \end{macrocode}
%
% \subsection{Lua module}
%
%    \begin{macrocode}
%<*lua>
%    \end{macrocode}
%
%    \begin{macrocode}
module("oberdiek.pdftexcmds", package.seeall)
local systemexitstatus
function strcmp(A, B)
  if A == B then
    tex.write("0")
  elseif A < B then
    tex.write("-1")
  else
    tex.write("1")
  end
end
local function utf8_to_byte(str)
  local i = 0
  local n = string.len(str)
  local t = {}
  while i < n do
    i = i + 1
    local a = string.byte(str, i)
    if a < 128 then
      table.insert(t, string.char(a))
    else
      if a >= 192 and i < n then
        i = i + 1
        local b = string.byte(str, i)
        if b < 128 or b >= 192 then
          i = i - 1
        elseif a == 194 then
          table.insert(t, string.char(b))
        elseif a == 195 then
          table.insert(t, string.char(b + 64))
        end
      end
    end
  end
  return table.concat(t)
end
function escapehex(str, mode)
  if mode == "byte" then
    str = utf8_to_byte(str)
  end
  tex.write((string.gsub(str, ".",
    function (ch)
      return string.format("%02X", string.byte(ch))
    end
  )))
end
%    \end{macrocode}
%    See procedure |unescapehex| in file \xfile{utils.c} of \pdfTeX.
%    Caution: |tex.write| ignores leading spaces.
%    \begin{macrocode}
function unescapehex(str, mode)
  local a = 0
  local first = true
  local result = {}
  for i = 1, string.len(str), 1 do
    local ch = string.byte(str, i)
    if ch >= 48 and ch <= 57 then
      ch = ch - 48
    elseif ch >= 65 and ch <= 70 then
      ch = ch - 55
    elseif ch >= 97 and ch <= 102 then
      ch = ch - 87
    else
      ch = nil
    end
    if ch then
      if first then
        a = ch * 16
        first = false
      else
        table.insert(result, a + ch)
        first = true
      end
    end
  end
  if not first then
    table.insert(result, a)
  end
  if mode == "byte" then
    local utf8 = {}
    for i, a in ipairs(result) do
      if a < 128 then
        table.insert(utf8, a)
      else
        if a < 192 then
          table.insert(utf8, 194)
          a = a - 128
        else
          table.insert(utf8, 195)
          a = a - 192
        end
        table.insert(utf8, a + 128)
      end
    end
    result = utf8
  end
  tex.settoks(toks, string.char(unpack(result)))
end
%    \end{macrocode}
%    See procedure |escapestring| in file \xfile{utils.c} of \pdfTeX.
%    \begin{macrocode}
function escapestring(str, mode)
  if mode == "byte" then
    str = utf8_to_byte(str)
  end
  tex.write((string.gsub(str, ".",
    function (ch)
      local b = string.byte(ch)
      if b < 33 or b > 126 then
        return string.format("\\%.3o", b)
      end
      if b == 40 or b == 41 or b == 92 then
        return "\\" .. ch
      end
%    \end{macrocode}
%    Lua 5.1 returns the match in case of return value |nil|.
%    \begin{macrocode}
      return nil
    end
  )))
end
%    \end{macrocode}
%    See procedure |escapename| in file \xfile{utils.c} of \pdfTeX.
%    \begin{macrocode}
function escapename(str, mode)
  if mode == "byte" then
    str = utf8_to_byte(str)
  end
  tex.write((string.gsub(str, ".",
    function (ch)
      local b = string.byte(ch)
      if b == 0 then
%    \end{macrocode}
%    In Lua 5.0 |nil| could be used for the empty string,
%    But |nil| returns the match in Lua 5.1, thus we use
%    the empty string explicitly.
%    \begin{macrocode}
        return ""
      end
      if b <= 32 or b >= 127
          or b == 35 or b == 37 or b == 40 or b == 41
          or b == 47 or b == 60 or b == 62 or b == 91
          or b == 93 or b == 123 or b == 125 then
        return string.format("#%.2X", b)
      else
%    \end{macrocode}
%    Lua 5.1 returns the match in case of return value |nil|.
%    \begin{macrocode}
        return nil
      end
    end
  )))
end
function filesize(filename)
  local foundfile = kpse.find_file(filename, "tex", true)
  if foundfile then
    local size = lfs.attributes(foundfile, "size")
    if size then
      tex.write(size)
    end
  end
end
%    \end{macrocode}
%    See procedure |makepdftime| in file \xfile{utils.c} of \pdfTeX.
%    \begin{macrocode}
function filemoddate(filename)
  local foundfile = kpse.find_file(filename, "tex", true)
  if foundfile then
    local date = lfs.attributes(foundfile, "modification")
    if date then
      local d = os.date("*t", date)
      if d.sec >= 60 then
        d.sec = 59
      end
      local u = os.date("!*t", date)
      local off = 60 * (d.hour - u.hour) + d.min - u.min
      if d.year ~= u.year then
        if d.year > u.year then
          off = off + 1440
        else
          off = off - 1440
        end
      elseif d.yday ~= u.yday then
        if d.yday > u.yday then
          off = off + 1440
        else
          off = off - 1440
        end
      end
      local timezone
      if off == 0 then
        timezone = "Z"
      else
        local hours = math.floor(off / 60)
        local mins = math.abs(off - hours * 60)
        timezone = string.format("%+03d'%02d'", hours, mins)
      end
      tex.write(string.format("D:%04d%02d%02d%02d%02d%02d%s",
          d.year, d.month, d.day, d.hour, d.min, d.sec, timezone))
    end
  end
end
function filedump(offset, length, filename)
  length = tonumber(length)
  if length and length > 0 then
    local foundfile = kpse.find_file(filename, "tex", true)
    if foundfile then
      offset = tonumber(offset)
      if not offset then
        offset = 0
      end
      local filehandle = io.open(foundfile, "r")
      if filehandle then
        if offset > 0 then
          filehandle:seek("set", offset)
        end
        local dump = filehandle:read(length)
        escapehex(dump)
      end
    end
  end
end
function mdfivesum(str, mode)
  if mode == "byte" then
    str = utf8_to_byte(str)
  end
  escapehex(md5.sum(str))
end
function filemdfivesum(filename)
  local foundfile = kpse.find_file(filename, "tex", true)
  if foundfile then
    local filehandle = io.open(foundfile, "r")
    if filehandle then
      local contents = filehandle:read("*a")
      escapehex(md5.sum(contents))
    end
  end
end
function shellescape()
  if os.execute then
    tex.write("1")
  else
    tex.write("0")
  end
end
function system(cmdline)
  systemexitstatus = nil
  texio.write_nl("log", "system(" .. cmdline .. ") ")
  if os.execute then
    texio.write("log", "executed.")
    systemexitstatus = os.execute(cmdline)
  else
    texio.write("log", "disabled.")
  end
end
function lastsystemstatus()
  local result = tonumber(systemexitstatus)
  if result then
    local x = math.floor(result / 256)
    tex.write(result - 256 * math.floor(result / 256))
  end
end
function lastsystemexit()
  local result = tonumber(systemexitstatus)
  if result then
    tex.write(math.floor(result / 256))
  end
end
function pipe(cmdline)
  local result
  systemexitstatus = nil
  texio.write_nl("log", "pipe(" .. cmdline ..") ")
  if io.popen then
    texio.write("log", "executed.")
    local handle = io.popen(cmdline, "r")
    if handle then
      result = handle:read("*a")
      handle:close()
    end
  else
    texio.write("log", "disabled.")
  end
  if result then
    tex.settoks(toks, result)
  else
    tex.settoks(toks, "")
  end
end
%    \end{macrocode}
%    \begin{macrocode}
%</lua>
%    \end{macrocode}
%
% \section{Test}
%
% \subsection{Catcode checks for loading}
%
%    \begin{macrocode}
%<*test1>
%    \end{macrocode}
%    \begin{macrocode}
\catcode`\{=1 %
\catcode`\}=2 %
\catcode`\#=6 %
\catcode`\@=11 %
\expandafter\ifx\csname count@\endcsname\relax
  \countdef\count@=255 %
\fi
\expandafter\ifx\csname @gobble\endcsname\relax
  \long\def\@gobble#1{}%
\fi
\expandafter\ifx\csname @firstofone\endcsname\relax
  \long\def\@firstofone#1{#1}%
\fi
\expandafter\ifx\csname loop\endcsname\relax
  \expandafter\@firstofone
\else
  \expandafter\@gobble
\fi
{%
  \def\loop#1\repeat{%
    \def\body{#1}%
    \iterate
  }%
  \def\iterate{%
    \body
      \let\next\iterate
    \else
      \let\next\relax
    \fi
    \next
  }%
  \let\repeat=\fi
}%
\def\RestoreCatcodes{}
\count@=0 %
\loop
  \edef\RestoreCatcodes{%
    \RestoreCatcodes
    \catcode\the\count@=\the\catcode\count@\relax
  }%
\ifnum\count@<255 %
  \advance\count@ 1 %
\repeat

\def\RangeCatcodeInvalid#1#2{%
  \count@=#1\relax
  \loop
    \catcode\count@=15 %
  \ifnum\count@<#2\relax
    \advance\count@ 1 %
  \repeat
}
\expandafter\ifx\csname LoadCommand\endcsname\relax
  \def\LoadCommand{\input pdftexcmds.sty\relax}%
\fi
\def\Test{%
  \RangeCatcodeInvalid{0}{47}%
  \RangeCatcodeInvalid{58}{64}%
  \RangeCatcodeInvalid{91}{96}%
  \RangeCatcodeInvalid{123}{255}%
  \catcode`\@=12 %
  \catcode`\\=0 %
  \catcode`\{=1 %
  \catcode`\}=2 %
  \catcode`\#=6 %
  \catcode`\[=12 %
  \catcode`\]=12 %
  \catcode`\%=14 %
  \catcode`\ =10 %
  \catcode13=5 %
  \LoadCommand
  \RestoreCatcodes
}
\Test
\csname @@end\endcsname
\end
%    \end{macrocode}
%    \begin{macrocode}
%</test1>
%    \end{macrocode}
%
% \section{Installation}
%
% \subsection{Download}
%
% \paragraph{Package.} This package is available on
% CTAN\footnote{\url{ftp://ftp.ctan.org/tex-archive/}}:
% \begin{description}
% \item[\CTAN{macros/latex/contrib/oberdiek/pdftexcmds.dtx}] The source file.
% \item[\CTAN{macros/latex/contrib/oberdiek/pdftexcmds.pdf}] Documentation.
% \end{description}
%
%
% \paragraph{Bundle.} All the packages of the bundle `oberdiek'
% are also available in a TDS compliant ZIP archive. There
% the packages are already unpacked and the documentation files
% are generated. The files and directories obey the TDS standard.
% \begin{description}
% \item[\CTAN{macros/latex/contrib/oberdiek/oberdiek-tds.zip}]
% \end{description}
% \emph{TDS} refers to the standard ``A Directory Structure
% for \TeX\ Files'' (\CTAN{tds/tds.pdf}). Directories
% with \xfile{texmf} in their name are usually organized this way.
%
% \subsection{Bundle installation}
%
% \paragraph{Unpacking.} Unpack the \xfile{oberdiek-tds.zip} in the
% TDS tree (also known as \xfile{texmf} tree) of your choice.
% Example (linux):
% \begin{quote}
%   |unzip oberdiek-tds.zip -d ~/texmf|
% \end{quote}
%
% \paragraph{Script installation.}
% Check the directory \xfile{TDS:scripts/oberdiek/} for
% scripts that need further installation steps.
% Package \xpackage{attachfile2} comes with the Perl script
% \xfile{pdfatfi.pl} that should be installed in such a way
% that it can be called as \texttt{pdfatfi}.
% Example (linux):
% \begin{quote}
%   |chmod +x scripts/oberdiek/pdfatfi.pl|\\
%   |cp scripts/oberdiek/pdfatfi.pl /usr/local/bin/|
% \end{quote}
%
% \subsection{Package installation}
%
% \paragraph{Unpacking.} The \xfile{.dtx} file is a self-extracting
% \docstrip\ archive. The files are extracted by running the
% \xfile{.dtx} through \plainTeX:
% \begin{quote}
%   \verb|tex pdftexcmds.dtx|
% \end{quote}
%
% \paragraph{TDS.} Now the different files must be moved into
% the different directories in your installation TDS tree
% (also known as \xfile{texmf} tree):
% \begin{quote}
% \def\t{^^A
% \begin{tabular}{@{}>{\ttfamily}l@{ $\rightarrow$ }>{\ttfamily}l@{}}
%   pdftexcmds.sty & tex/generic/oberdiek/pdftexcmds.sty\\
%   oberdiek.pdftexcmds.lua & scripts/oberdiek/oberdiek.pdftexcmds.lua\\
%   pdftexcmds.lua & scripts/oberdiek/pdftexcmds.lua\\
%   pdftexcmds.pdf & doc/latex/oberdiek/pdftexcmds.pdf\\
%   pdftexcmds.dtx & source/latex/oberdiek/pdftexcmds.dtx\\
% \end{tabular}^^A
% }^^A
% \sbox0{\t}^^A
% \ifdim\wd0>\linewidth
%   \begingroup
%     \advance\linewidth by\leftmargin
%     \advance\linewidth by\rightmargin
%   \edef\x{\endgroup
%     \def\noexpand\lw{\the\linewidth}^^A
%   }\x
%   \def\lwbox{^^A
%     \leavevmode
%     \hbox to \linewidth{^^A
%       \kern-\leftmargin\relax
%       \hss
%       \usebox0
%       \hss
%       \kern-\rightmargin\relax
%     }^^A
%   }^^A
%   \ifdim\wd0>\lw
%     \sbox0{\small\t}^^A
%     \ifdim\wd0>\linewidth
%       \ifdim\wd0>\lw
%         \sbox0{\footnotesize\t}^^A
%         \ifdim\wd0>\linewidth
%           \ifdim\wd0>\lw
%             \sbox0{\scriptsize\t}^^A
%             \ifdim\wd0>\linewidth
%               \ifdim\wd0>\lw
%                 \sbox0{\tiny\t}^^A
%                 \ifdim\wd0>\linewidth
%                   \lwbox
%                 \else
%                   \usebox0
%                 \fi
%               \else
%                 \lwbox
%               \fi
%             \else
%               \usebox0
%             \fi
%           \else
%             \lwbox
%           \fi
%         \else
%           \usebox0
%         \fi
%       \else
%         \lwbox
%       \fi
%     \else
%       \usebox0
%     \fi
%   \else
%     \lwbox
%   \fi
% \else
%   \usebox0
% \fi
% \end{quote}
% If you have a \xfile{docstrip.cfg} that configures and enables \docstrip's
% TDS installing feature, then some files can already be in the right
% place, see the documentation of \docstrip.
%
% \subsection{Refresh file name databases}
%
% If your \TeX~distribution
% (\teTeX, \mikTeX, \dots) relies on file name databases, you must refresh
% these. For example, \teTeX\ users run \verb|texhash| or
% \verb|mktexlsr|.
%
% \subsection{Some details for the interested}
%
% \paragraph{Attached source.}
%
% The PDF documentation on CTAN also includes the
% \xfile{.dtx} source file. It can be extracted by
% AcrobatReader 6 or higher. Another option is \textsf{pdftk},
% e.g. unpack the file into the current directory:
% \begin{quote}
%   \verb|pdftk pdftexcmds.pdf unpack_files output .|
% \end{quote}
%
% \paragraph{Unpacking with \LaTeX.}
% The \xfile{.dtx} chooses its action depending on the format:
% \begin{description}
% \item[\plainTeX:] Run \docstrip\ and extract the files.
% \item[\LaTeX:] Generate the documentation.
% \end{description}
% If you insist on using \LaTeX\ for \docstrip\ (really,
% \docstrip\ does not need \LaTeX), then inform the autodetect routine
% about your intention:
% \begin{quote}
%   \verb|latex \let\install=y\input{pdftexcmds.dtx}|
% \end{quote}
% Do not forget to quote the argument according to the demands
% of your shell.
%
% \paragraph{Generating the documentation.}
% You can use both the \xfile{.dtx} or the \xfile{.drv} to generate
% the documentation. The process can be configured by the
% configuration file \xfile{ltxdoc.cfg}. For instance, put this
% line into this file, if you want to have A4 as paper format:
% \begin{quote}
%   \verb|\PassOptionsToClass{a4paper}{article}|
% \end{quote}
% An example follows how to generate the
% documentation with pdf\LaTeX:
% \begin{quote}
%\begin{verbatim}
%pdflatex pdftexcmds.dtx
%makeindex -s gind.ist pdftexcmds.idx
%pdflatex pdftexcmds.dtx
%makeindex -s gind.ist pdftexcmds.idx
%pdflatex pdftexcmds.dtx
%\end{verbatim}
% \end{quote}
%
% \begin{History}
%   \begin{Version}{2007/11/11 v0.1}
%   \item
%     First version.
%   \end{Version}
%   \begin{Version}{2007/11/12 v0.2}
%   \item
%     Short description fixed.
%   \end{Version}
%   \begin{Version}{2007/12/12 v0.3}
%   \item
%     Organization of Lua code as module.
%   \end{Version}
% \end{History}
%
% \PrintIndex
%
% \Finale
\endinput
|
% \end{quote}
% Do not forget to quote the argument according to the demands
% of your shell.
%
% \paragraph{Generating the documentation.}
% You can use both the \xfile{.dtx} or the \xfile{.drv} to generate
% the documentation. The process can be configured by the
% configuration file \xfile{ltxdoc.cfg}. For instance, put this
% line into this file, if you want to have A4 as paper format:
% \begin{quote}
%   \verb|\PassOptionsToClass{a4paper}{article}|
% \end{quote}
% An example follows how to generate the
% documentation with pdf\LaTeX:
% \begin{quote}
%\begin{verbatim}
%pdflatex pdftexcmds.dtx
%makeindex -s gind.ist pdftexcmds.idx
%pdflatex pdftexcmds.dtx
%makeindex -s gind.ist pdftexcmds.idx
%pdflatex pdftexcmds.dtx
%\end{verbatim}
% \end{quote}
%
% \begin{History}
%   \begin{Version}{2007/11/11 v0.1}
%   \item
%     First version.
%   \end{Version}
%   \begin{Version}{2007/11/12 v0.2}
%   \item
%     Short description fixed.
%   \end{Version}
%   \begin{Version}{2007/12/12 v0.3}
%   \item
%     Organization of Lua code as module.
%   \end{Version}
% \end{History}
%
% \PrintIndex
%
% \Finale
\endinput

%        (quote the arguments according to the demands of your shell)
%
% Documentation:
%    (a) If pdftexcmds.drv is present:
%           latex pdftexcmds.drv
%    (b) Without pdftexcmds.drv:
%           latex pdftexcmds.dtx; ...
%    The class ltxdoc loads the configuration file ltxdoc.cfg
%    if available. Here you can specify further options, e.g.
%    use A4 as paper format:
%       \PassOptionsToClass{a4paper}{article}
%
%    Programm calls to get the documentation (example):
%       pdflatex pdftexcmds.dtx
%       makeindex -s gind.ist pdftexcmds.idx
%       pdflatex pdftexcmds.dtx
%       makeindex -s gind.ist pdftexcmds.idx
%       pdflatex pdftexcmds.dtx
%
% Installation:
%    TDS:tex/generic/oberdiek/pdftexcmds.sty
%    TDS:scripts/oberdiek/oberdiek.pdftexcmds.lua
%    TDS:scripts/oberdiek/pdftexcmds.lua
%    TDS:doc/latex/oberdiek/pdftexcmds.pdf
%    TDS:source/latex/oberdiek/pdftexcmds.dtx
%
%<*ignore>
\begingroup
  \def\x{LaTeX2e}%
\expandafter\endgroup
\ifcase 0\ifx\install y1\fi\expandafter
         \ifx\csname processbatchFile\endcsname\relax\else1\fi
         \ifx\fmtname\x\else 1\fi\relax
\else\csname fi\endcsname
%</ignore>
%<*install>
\input docstrip.tex
\Msg{************************************************************************}
\Msg{* Installation}
\Msg{* Package: pdftexcmds 2007/12/12 v0.3 LuaTeX support for pdfTeX utility functions (HO)}
\Msg{************************************************************************}

\keepsilent
\askforoverwritefalse

\let\MetaPrefix\relax
\preamble

This is a generated file.

Copyright (C) 2007 by Heiko Oberdiek <oberdiek@uni-freiburg.de>

This work may be distributed and/or modified under the
conditions of the LaTeX Project Public License, either
version 1.3 of this license or (at your option) any later
version. The latest version of this license is in
   http://www.latex-project.org/lppl.txt
and version 1.3 or later is part of all distributions of
LaTeX version 2005/12/01 or later.

This work has the LPPL maintenance status "maintained".

This Current Maintainer of this work is Heiko Oberdiek.

This work consists of the main source file pdftexcmds.dtx
and the derived files
   pdftexcmds.sty, pdftexcmds.pdf, pdftexcmds.ins, pdftexcmds.drv,
   oberdiek.pdftexcmds.lua, pdftexcmds.lua.

\endpreamble
\let\MetaPrefix\DoubleperCent

\generate{%
  \file{pdftexcmds.ins}{\from{pdftexcmds.dtx}{install}}%
  \file{pdftexcmds.drv}{\from{pdftexcmds.dtx}{driver}}%
  \usedir{tex/generic/oberdiek}%
  \file{pdftexcmds.sty}{\from{pdftexcmds.dtx}{package}}%
}
\def\MetaPrefix{-- }
\def\defaultpostamble{%
  \MetaPrefix^^J%
  \MetaPrefix\space End of File `\outFileName'.%
}
\def\currentpostamble{\defaultpostamble}%
\generate{%
  \usedir{scripts/oberdiek}%
  \file{oberdiek.pdftexcmds.lua}{\from{pdftexcmds.dtx}{lua}}%
  \file{pdftexcmds.lua}{\from{pdftexcmds.dtx}{lua}}%
}

\obeyspaces
\Msg{************************************************************************}
\Msg{*}
\Msg{* To finish the installation you have to move the following}
\Msg{* file into a directory searched by TeX:}
\Msg{*}
\Msg{*     pdftexcmds.sty}
\Msg{*}
\Msg{* And install the following script files:}
\Msg{*}
\Msg{*     oberdiek.pdftexcmds.lua, pdftexcmds.lua}
\Msg{*}
\Msg{* To produce the documentation run the file `pdftexcmds.drv'}
\Msg{* through LaTeX.}
\Msg{*}
\Msg{* Happy TeXing!}
\Msg{*}
\Msg{************************************************************************}

\endbatchfile
%</install>
%<*ignore>
\fi
%</ignore>
%<*driver>
\NeedsTeXFormat{LaTeX2e}
\ProvidesFile{pdftexcmds.drv}%
  [2007/12/12 v0.3 LuaTeX support for pdfTeX utility functions (HO)]%
\documentclass{ltxdoc}
\usepackage{holtxdoc}[2007/11/11]
\usepackage{paralist}
\begin{document}
  \DocInput{pdftexcmds.dtx}%
\end{document}
%</driver>
% \fi
%
% \CheckSum{552}
%
% \CharacterTable
%  {Upper-case    \A\B\C\D\E\F\G\H\I\J\K\L\M\N\O\P\Q\R\S\T\U\V\W\X\Y\Z
%   Lower-case    \a\b\c\d\e\f\g\h\i\j\k\l\m\n\o\p\q\r\s\t\u\v\w\x\y\z
%   Digits        \0\1\2\3\4\5\6\7\8\9
%   Exclamation   \!     Double quote  \"     Hash (number) \#
%   Dollar        \$     Percent       \%     Ampersand     \&
%   Acute accent  \'     Left paren    \(     Right paren   \)
%   Asterisk      \*     Plus          \+     Comma         \,
%   Minus         \-     Point         \.     Solidus       \/
%   Colon         \:     Semicolon     \;     Less than     \<
%   Equals        \=     Greater than  \>     Question mark \?
%   Commercial at \@     Left bracket  \[     Backslash     \\
%   Right bracket \]     Circumflex    \^     Underscore    \_
%   Grave accent  \`     Left brace    \{     Vertical bar  \|
%   Right brace   \}     Tilde         \~}
%
% \GetFileInfo{pdftexcmds.drv}
%
% \title{The \xpackage{pdftexcmds} package}
% \date{2007/12/12 v0.3}
% \author{Heiko Oberdiek\\\xemail{oberdiek@uni-freiburg.de}}
%
% \maketitle
%
% \begin{abstract}
% \LuaTeX\ provides most of the commands of \pdfTeX\ 1.40. However
% a number of utility functions are removed. This package tries to fill
% the gap and implements some of the missing primitive using Lua.
% \end{abstract}
%
% \tableofcontents
%
% \section{Documentation}
%
% Some primitives of \pdfTeX\ are not defined by \LuaTeX.
% This package implements macro based solutions using Lua code
% for the following missing \pdfTeX\ primitives;
% \begin{compactitem}
% \item \cs{pdfstrcmp}
% \item \cs{pdfunescapehex}
% \item \cs{pdfescapehex}
% \item \cs{pdfescapename}
% \item \cs{pdfescapestring}
% \item \cs{pdffilesize}
% \item \cs{pdffilemoddate}
% \item \cs{pdffiledump}
% \item \cs{pdfmdfivesum}
% \item |\immediate\write18|
% \end{compactitem}
% The original names of the primitives cannot be used:
% \begin{itemize}
% \item
% The syntax for their arguments cannot easily
% simulated by macros. The primitives using key words
% such as |file| (\cs{pdfmdfivesum}) or |offset| and |length|
% (\cs{pdffiledump}) and uses \meta{general text} for the other
% arguments. Using token registers assignments, \meta{general text} could
% be catched. However, the simulated primitives are expandable
% and register assignments would destroy this important property.
% (\meta{general text} allows something like |\expandafter\bgroup ...}|.)
% \item
% The original primitives can be expanded using one expansion step.
% The new macros need two expansion steps because of the additional
% macro expansion. Example:
% \begin{quote}
%   |\expandafter\foo\pdffilemoddate{file}|\\
%   vs.
%   |\expandafter\expandafter\expandafter\foo\pdf@filemoddate{file}|.
% \end{quote}
% \end{itemize}
%
% \LuaTeX\ isn't stable yet and thus the status of this package is
% \emph{experimental}. Feedback is welcome.
%
% \subsection{General principles}
%
% \begin{description}
% \item[Naming convention:]
%   Usually this package defines a macro |\pdf@|\meta{cmd} if
%   \pdfTeX\ provides |\pdf|\meta{cmd}.
% \item[Arguments:] The order of arguments in |\pdf@|\meta{cmd}
%   is the same as for the corresponding primitive of \pdfTeX.
%   The arguments are ordinary undelimited \TeX\ arguments,
%   no \meta{general text} and without additional keywords.
% \item[Expandibility:]
%   The macro |\pdf@|\meta{cmd} is expandable if the
%   corresponding \pdfTeX\ primitive has this property.
%   Exact two expansion steps are necessary (first is the macro
%   expansion).
% \item[Without \LuaTeX:]
%   The macros |\pdf@|\meta{cmd} are mapped to the commands
%   of \pdfTeX\ if they are available. Otherwise they are undefined.
% \end{description}
%
% \subsection{Macros}
%
% \begin{declcs}{pdf@strcmp} \M{stringA} \M{stringB}
% \end{declcs}
% Same as |\pdfstrcmp{|\meta{stringA}|}{|\meta{stringB}|}|.
%
% \begin{declcs}{pdf@unescapehex} \M{string}
% \end{declcs}
% Same as |\pdfunescapehex{|\meta{string}|}|.
% The argument is a byte string given in hexadecimal notation.
% The result are character tokens from 0 until 255 with
% catcode 12 and the space with catcode 10.
%
% \begin{declcs}{pdf@escapehex} \M{string}\\
%   \cs{pdf@escapestring} \M{string}\\
%   \cs{pdf@escapename} \M{string}
% \end{declcs}
% Same as the primitives of \pdfTeX. However \pdfTeX\ does not
% know about characters with codes 256 and larger. Thus the
% string is treated as byte string, characters with more than
% eight bits are ignored.
%
% \begin{declcs}{pdf@filesize} \M{filename}
% \end{declcs}
% Same as |\pdffilesize{|\meta{filename}|}|.
%
% \begin{declcs}{pdf@filemoddate} \M{filename}
% \end{declcs}
% Same as |\pdffilemoddate{|\meta{filename}|}|.
%
% \begin{declcs}{pdf@filedump} \M{offset} \M{length} \M{filename}
% \end{declcs}
% Same as |\pdffiledump offset| \meta{offset} |length| \meta{length}
% |{|\meta{filename}|}|. Both \meta{offset} and \meta{length} must
% not be empty, but must be a valid \TeX\ number.
%
% \begin{declcs}{pdf@mdfivesum} \M{string}
% \end{declcs}
% Same as |\pdfmdfivesum{|\meta{string}|}|. Keyword |file| is supported
% by macro \cs{pdf@filemdfivesum}.
%
% \begin{declcs}{pdf@filemdfivesum} \M{filename}
% \end{declcs}
% Same as |\pdfmdfivesum file{|\meta{filename}|}|.
%
% \begin{declcs}{pdf@shellescape}
% \end{declcs}
% Same as |\pdfshellescape|. It expands to |1| if external
% commands can be executed and |0| otherwise. In \pdfTeX\ external
% commands must be enabled first by command line option or
% configuration option. In \LuaTeX\ option |--safer| disables
% the execution of external commands.
%
% \begin{declcs}{pdf@system} \M{cmdline}
% \end{declcs}
% It is a wrapper for |\immediate\write18| in \pdfTeX\ or
% |os.execute| in \LuaTeX.
%
% In theory |os.execute|
% returns a status number. But its meaning is quite
% undefined. Are there some reliable properties?
% Does it make sense to provide an user interface to
% this status exit code?
%
% \subsubsection{Experimental}
%
% \begin{declcs}{pdf@unescapehexnative} \M{string}\\
%   \cs{pdf@escapehexnative} \M{string}\\
%   \cs{pdf@escapenamenative} \M{string}\\
%   \cs{pdf@mdfivesumnative} \M{string}
% \end{declcs}
% The variants without |native| in the macro name are supposed to
% be compatible with \pdfTeX. However characters with more than
% eight bits are not supported and are ignored. If \LuaTeX\ is
% running, then its UTF-8 coded strings are used. Thus the full
% unicode character range is supported. However the result
% differs from \pdfTeX\ for characters with eight or more bits.
%
% \begin{declcs}{pdf@pipe} \M{cmdline}
% \end{declcs}
% It calls \meta{cmdline} and returns the output of the external
% program in the usual manner as byte string (catcode 12, space with
% catcode 10). The Lua documentation says, that the used |io.popen|
% may not be available on all platforms. Then macro \cs{pdf@pipe}
% is undefined.
%
% \StopEventually{
% }
%
% \section{Implementation}
%
%    \begin{macrocode}
%<*package>
%    \end{macrocode}
%
% \subsection{Reload check and package identification}
%    Reload check, especially if the package is not used with \LaTeX.
%    \begin{macrocode}
\begingroup
  \catcode44 12 % ,
  \catcode45 12 % -
  \catcode46 12 % .
  \catcode58 12 % :
  \catcode64 11 % @
  \expandafter\let\expandafter\x\csname ver@pdftexcmds.sty\endcsname
  \ifcase 0%
    \ifx\x\relax % plain
    \else
      \ifx\x\empty % LaTeX
      \else
        1%
      \fi
    \fi
  \else
    \catcode35 6 % #
    \catcode123 1 % {
    \catcode125 2 % }
    \expandafter\ifx\csname PackageInfo\endcsname\relax
      \def\x#1#2{%
        \immediate\write-1{Package #1 Info: #2.}%
      }%
    \else
      \def\x#1#2{\PackageInfo{#1}{#2, stopped}}%
    \fi
    \x{pdftexcmds}{The package is already loaded}%
    \endgroup
    \expandafter\endinput
  \fi
\endgroup
%    \end{macrocode}
%    Package identification:
%    \begin{macrocode}
\begingroup
  \catcode35 6 % #
  \catcode40 12 % (
  \catcode41 12 % )
  \catcode44 12 % ,
  \catcode45 12 % -
  \catcode46 12 % .
  \catcode47 12 % /
  \catcode58 12 % :
  \catcode64 11 % @
  \catcode123 1 % {
  \catcode125 2 % }
  \expandafter\ifx\csname ProvidesPackage\endcsname\relax
    \def\x#1#2#3[#4]{\endgroup
      \immediate\write-1{Package: #3 #4}%
      \xdef#1{#4}%
    }%
  \else
    \def\x#1#2[#3]{\endgroup
      #2[{#3}]%
      \ifx#1\relax
        \xdef#1{#3}%
      \fi
    }%
  \fi
\expandafter\x\csname ver@pdftexcmds.sty\endcsname
\ProvidesPackage{pdftexcmds}%
  [2007/12/12 v0.3 LuaTeX support for pdfTeX utility functions (HO)]
%    \end{macrocode}
%
% \subsection{Catcodes}
%
%    \begin{macrocode}
\begingroup
  \catcode123 1 % {
  \catcode125 2 % }
  \def\x{\endgroup
    \expandafter\edef\csname pdftexcmds@AtEnd\endcsname{%
      \catcode35 \the\catcode35\relax
      \catcode64 \the\catcode64\relax
      \catcode123 \the\catcode123\relax
      \catcode125 \the\catcode125\relax
    }%
  }%
\x
\catcode35 6 % #
\catcode64 11 % @
\catcode123 1 % {
\catcode125 2 % }
\def\TMP@EnsureCode#1#2{%
  \edef\pdftexcmds@AtEnd{%
    \pdftexcmds@AtEnd
    \catcode#1 \the\catcode#1\relax
  }%
  \catcode#1 #2\relax
}
\TMP@EnsureCode{10}{12}% ^^J
\TMP@EnsureCode{33}{12}% !
\TMP@EnsureCode{34}{12}% "
\TMP@EnsureCode{39}{12}% '
\TMP@EnsureCode{40}{12}% (
\TMP@EnsureCode{41}{12}% )
\TMP@EnsureCode{42}{12}% *
\TMP@EnsureCode{43}{12}% +
\TMP@EnsureCode{44}{12}% ,
\TMP@EnsureCode{45}{12}% -
\TMP@EnsureCode{46}{12}% .
\TMP@EnsureCode{47}{12}% /
\TMP@EnsureCode{58}{12}% :
\TMP@EnsureCode{60}{12}% <
\TMP@EnsureCode{61}{12}% =
\TMP@EnsureCode{62}{12}% >
\TMP@EnsureCode{94}{7}% ^ (superscript)
\TMP@EnsureCode{95}{12}% _ (other)
\TMP@EnsureCode{126}{12}% ~ (other)
%    \end{macrocode}
%
% \subsection{Load package \xpackage{infwarerr}}
%
%    \begin{macrocode}
\begingroup\expandafter\expandafter\expandafter\endgroup
\expandafter\ifx\csname RequirePackage\endcsname\relax
  \input infwarerr.sty\relax
\else
  \RequirePackage{infwarerr}[2007/09/09]%
\fi
%    \end{macrocode}
%
% \subsection{Without \LuaTeX}
%
%    \begin{macrocode}
\begingroup\expandafter\expandafter\expandafter\endgroup
\expandafter\ifx\csname directlua\endcsname\relax
  \@PackageInfo{pdftexcmds}{LuaTeX not detected}%
  \def\pdftexcmds@nopdftex{%
    \@PackageInfoNoLine{pdftexcmds}{pdfTeX >= 1.30 not detected}%
    \let\pdftexcmds@nopdftex\relax
  }%
  \def\pdftexcmds@temp#1{%
    \begingroup\expandafter\expandafter\expandafter\endgroup
    \expandafter\ifx\csname pdf#1\endcsname\relax
      \pdftexcmds@nopdftex
    \else
      \expandafter\def\csname pdf@#1\expandafter\endcsname
      \expandafter##\expandafter{%
        \csname pdf#1\endcsname
      }%
    \fi
  }%
  \pdftexcmds@temp{strcmp}%
  \pdftexcmds@temp{escapehex}%
  \let\pdf@escapehexnative\pdf@escapehex
  \pdftexcmds@temp{unescapehex}%
  \let\pdf@unescapehexnative\pdf@unescapehex
  \pdftexcmds@temp{escapestring}%
  \pdftexcmds@temp{escapename}%
  \pdftexcmds@temp{filesize}%
  \pdftexcmds@temp{filemoddate}%
  \begingroup\expandafter\expandafter\expandafter\endgroup
  \expandafter\ifx\csname pdfshellescape\endcsname\relax
    \pdftexcmds@nopdftex
  \else
    \def\pdf@shellescape{%
      \pdfshellescape
    }%
  \fi
  \begingroup\expandafter\expandafter\expandafter\endgroup
  \expandafter\ifx\csname pdffiledump\endcsname\relax
    \pdftexcmds@nopdftex
  \else
    \def\pdf@filedump#1#2#3{%
      \pdffiledump offset#1 length#2{#3}%
    }%
  \fi
  \begingroup\expandafter\expandafter\expandafter\endgroup
  \expandafter\ifx\csname pdfmdfivesum\endcsname\relax
    \pdftexcmds@nopdftex
  \else
    \def\pdf@mdfivesum#{\pdfmdfivesum}%
    \let\pdf@mdfivesumnative\pdf@mdfivesum
    \def\pdf@filemdfivesum#{\pdfmdfivesum file}%
  \fi
  \def\pdf@system#{%
    \immediate\write18%
  }%
  \pdftexcmds@AtEnd
  \expandafter\endinput
\fi
%    \end{macrocode}
%
% \subsection{Load module}
%
%    \begin{macrocode}
\begingroup\expandafter\expandafter\expandafter\endgroup
\expandafter\ifx\csname RequirePackage\endcsname\relax
  \input luatex-loader.sty\relax
\else
  \RequirePackage{luatex-loader}[2007/12/12]%
\fi
\directlua0{%
  require("oberdiek.pdftexcmds")%
}
%    \end{macrocode}
%
% \subsection{Lua functions}
%
%    \begin{macro}{\pdftexcmds@toks}
%    \begin{macrocode}
\begingroup\expandafter\expandafter\expandafter\endgroup
\expandafter\ifx\csname newtoks\endcsname\relax
  \toksdef\pdftexcmds@toks=0 %
\else
  \csname newtoks\endcsname\pdftexcmds@toks
\fi
%    \end{macrocode}
%    \end{macro}
%
%    \begin{macro}{\pdf@strcmp}
%    \begin{macrocode}
\long\def\pdf@strcmp#1#2{%
  \directlua0{%
    oberdiek.pdftexcmds.strcmp("\luaescapestring{#1}",%
        "\luaescapestring{#2}")%
  }%
}%
%    \end{macrocode}
%    \end{macro}
%    \begin{macro}{\pdf@escapehex}
%    \begin{macrocode}
\long\def\pdf@escapehex#1{%
  \directlua0{%
    oberdiek.pdftexcmds.escapehex("\luaescapestring{#1}", "byte")%
  }%
}%
%    \end{macrocode}
%    \end{macro}
%    \begin{macro}{\pdf@escapehexnative}
%    \begin{macrocode}
\long\def\pdf@escapehexnative#1{%
  \directlua0{%
    oberdiek.pdftexcmds.escapehex("\luaescapestring{#1}")%
  }%
}%
%    \end{macrocode}
%    \end{macro}
%    \begin{macro}{\pdf@unescapehex}
%    \begin{macrocode}
\def\pdf@unescapehex#1{%
  \the\expandafter\pdftexcmds@toks
  \directlua0{%
    oberdiek.pdftexcmds.toks="pdftexcmds@toks"%
    oberdiek.pdftexcmds.unescapehex("\luaescapestring{#1}", "byte")%
  }%
}%
%    \end{macrocode}
%    \end{macro}
%    \begin{macro}{\pdf@unescapehexnative}
%    \begin{macrocode}
\def\pdf@unescapehexnative#1{%
  \the\expandafter\pdftexcmds@toks
  \directlua0{%
    oberdiek.pdftexcmds.toks="pdftexcmds@toks"%
    oberdiek.pdftexcmds.unescapehex("\luaescapestring{#1}")%
  }%
}%
%    \end{macrocode}
%    \end{macro}
%    \begin{macro}{\pdf@escapestring}
%    \begin{macrocode}
\long\def\pdf@escapestring#1{%
  \directlua0{%
    oberdiek.pdftexcmds.escapestring("\luaescapestring{#1}", "byte")%
  }%
}
%    \end{macrocode}
%    \end{macro}
%    \begin{macro}{\pdf@escapename}
%    \begin{macrocode}
\long\def\pdf@escapename#1{%
  \directlua0{%
    oberdiek.pdftexcmds.escapename("\luaescapestring{#1}", "byte")%
  }%
}
%    \end{macrocode}
%    \end{macro}
%    \begin{macro}{\pdf@escapenamenative}
%    \begin{macrocode}
\long\def\pdf@escapenamenative#1{%
  \directlua0{%
    oberdiek.pdftexcmds.escapename("\luaescapestring{#1}")%
  }%
}
%    \end{macrocode}
%    \end{macro}
%    \begin{macro}{\pdf@filesize}
%    \begin{macrocode}
\def\pdf@filesize#1{%
  \directlua0{%
    oberdiek.pdftexcmds.filesize("\luaescapestring{#1}")%
  }%
}
%    \end{macrocode}
%    \end{macro}
%    \begin{macro}{\pdf@filemoddate}
%    \begin{macrocode}
\def\pdf@filemoddate#1{%
  \directlua0{%
    oberdiek.pdftexcmds.filemoddate("\luaescapestring{#1}")%
  }%
}
%    \end{macrocode}
%    \end{macro}
%    \begin{macro}{\pdf@filedump}
%    \begin{macrocode}
\def\pdf@filedump#1#2#3{%
  \directlua0{%
    oberdiek.pdftexcmds.filedump("\luaescapestring{\number#1}",%
        "\luaescapestring{\number#2}",%
        "\luaescapestring{#3}")%
  }%
}%
%    \end{macrocode}
%    \end{macro}
%    \begin{macro}{\pdf@mdfivesum}
%    \begin{macrocode}
\long\def\pdf@mdfivesum#1{%
  \directlua0{%
    oberdiek.pdftexcmds.mdfivesum("\luaescapestring{#1}", "byte")%
  }%
}%
%    \end{macrocode}
%    \end{macro}
%    \begin{macro}{\pdf@mdfivesumnative}
%    \begin{macrocode}
\long\def\pdf@mdfivesumnative#1{%
  \directlua0{%
    oberdiek.pdftexcmds.mdfivesum("\luaescapestring{#1}")%
  }%
}%
%    \end{macrocode}
%    \end{macro}
%    \begin{macro}{\pdf@filemdfivesum}
%    \begin{macrocode}
\def\pdf@filemdfivesum#1{%
  \directlua0{%
    oberdiek.pdftexcmds.filemdfivesum("\luaescapestring{#1}")%
  }%
}%
%    \end{macrocode}
%    \end{macro}
%
%    \begin{macro}{\pdf@shellescape}
%    \begin{macrocode}
\def\pdf@shellescape{%
  \directlua0{%
    oberdiek.pdftexcmds.shellescape()%
  }%
}
%    \end{macrocode}
%    \end{macro}
%
%    \begin{macro}{\pdf@system}
%    \begin{macrocode}
\def\pdf@system#1{%
  \directlua0{%
    oberdiek.pdftexcmds.system("\luaescapestring{#1}")%
  }%
}
%    \end{macrocode}
%    \end{macro}
%
%    \begin{macro}{\pdf@lastsystemstatus}
%    \begin{macrocode}
\def\pdf@lastsystemstatus{%
  \directlua0{%
    oberdiek.pdftexcmds.lastsystemstatus()%
  }%
}
%    \end{macrocode}
%    \end{macro}
%    \begin{macro}{\pdf@lastsystemexit}
%    \begin{macrocode}
\def\pdf@lastsystemexit{%
  \directlua0{%
    oberdiek.pdftexcmds.lastsystemexit()%
  }%
}
%    \end{macrocode}
%    \end{macro}
%
%    \begin{macro}{\pdf@pipe}
%    Check availability of |io.popen| first.
%    \begin{macrocode}
\ifnum0%
    \directlua0{%
      if io.popen then %
        tex.write("1")%
      end%
    }%
    =1 %
  \def\pdf@pipe#1{%
    \the\expandafter\pdftexcmds@toks
    \directlua0{%
      oberdiek.pdftexcmds.toks="pdftexcmds@toks"%
      oberdiek.pdftexcmds.pipe("\luaescapestring{#1}")%
    }%
  }%
\fi
%    \end{macrocode}
%    \end{macro}
%
%    \begin{macrocode}
\pdftexcmds@AtEnd
%</package>
%    \end{macrocode}
%
% \subsection{Lua module}
%
%    \begin{macrocode}
%<*lua>
%    \end{macrocode}
%
%    \begin{macrocode}
module("oberdiek.pdftexcmds", package.seeall)
local systemexitstatus
function strcmp(A, B)
  if A == B then
    tex.write("0")
  elseif A < B then
    tex.write("-1")
  else
    tex.write("1")
  end
end
local function utf8_to_byte(str)
  local i = 0
  local n = string.len(str)
  local t = {}
  while i < n do
    i = i + 1
    local a = string.byte(str, i)
    if a < 128 then
      table.insert(t, string.char(a))
    else
      if a >= 192 and i < n then
        i = i + 1
        local b = string.byte(str, i)
        if b < 128 or b >= 192 then
          i = i - 1
        elseif a == 194 then
          table.insert(t, string.char(b))
        elseif a == 195 then
          table.insert(t, string.char(b + 64))
        end
      end
    end
  end
  return table.concat(t)
end
function escapehex(str, mode)
  if mode == "byte" then
    str = utf8_to_byte(str)
  end
  tex.write((string.gsub(str, ".",
    function (ch)
      return string.format("%02X", string.byte(ch))
    end
  )))
end
%    \end{macrocode}
%    See procedure |unescapehex| in file \xfile{utils.c} of \pdfTeX.
%    Caution: |tex.write| ignores leading spaces.
%    \begin{macrocode}
function unescapehex(str, mode)
  local a = 0
  local first = true
  local result = {}
  for i = 1, string.len(str), 1 do
    local ch = string.byte(str, i)
    if ch >= 48 and ch <= 57 then
      ch = ch - 48
    elseif ch >= 65 and ch <= 70 then
      ch = ch - 55
    elseif ch >= 97 and ch <= 102 then
      ch = ch - 87
    else
      ch = nil
    end
    if ch then
      if first then
        a = ch * 16
        first = false
      else
        table.insert(result, a + ch)
        first = true
      end
    end
  end
  if not first then
    table.insert(result, a)
  end
  if mode == "byte" then
    local utf8 = {}
    for i, a in ipairs(result) do
      if a < 128 then
        table.insert(utf8, a)
      else
        if a < 192 then
          table.insert(utf8, 194)
          a = a - 128
        else
          table.insert(utf8, 195)
          a = a - 192
        end
        table.insert(utf8, a + 128)
      end
    end
    result = utf8
  end
  tex.settoks(toks, string.char(unpack(result)))
end
%    \end{macrocode}
%    See procedure |escapestring| in file \xfile{utils.c} of \pdfTeX.
%    \begin{macrocode}
function escapestring(str, mode)
  if mode == "byte" then
    str = utf8_to_byte(str)
  end
  tex.write((string.gsub(str, ".",
    function (ch)
      local b = string.byte(ch)
      if b < 33 or b > 126 then
        return string.format("\\%.3o", b)
      end
      if b == 40 or b == 41 or b == 92 then
        return "\\" .. ch
      end
%    \end{macrocode}
%    Lua 5.1 returns the match in case of return value |nil|.
%    \begin{macrocode}
      return nil
    end
  )))
end
%    \end{macrocode}
%    See procedure |escapename| in file \xfile{utils.c} of \pdfTeX.
%    \begin{macrocode}
function escapename(str, mode)
  if mode == "byte" then
    str = utf8_to_byte(str)
  end
  tex.write((string.gsub(str, ".",
    function (ch)
      local b = string.byte(ch)
      if b == 0 then
%    \end{macrocode}
%    In Lua 5.0 |nil| could be used for the empty string,
%    But |nil| returns the match in Lua 5.1, thus we use
%    the empty string explicitly.
%    \begin{macrocode}
        return ""
      end
      if b <= 32 or b >= 127
          or b == 35 or b == 37 or b == 40 or b == 41
          or b == 47 or b == 60 or b == 62 or b == 91
          or b == 93 or b == 123 or b == 125 then
        return string.format("#%.2X", b)
      else
%    \end{macrocode}
%    Lua 5.1 returns the match in case of return value |nil|.
%    \begin{macrocode}
        return nil
      end
    end
  )))
end
function filesize(filename)
  local foundfile = kpse.find_file(filename, "tex", true)
  if foundfile then
    local size = lfs.attributes(foundfile, "size")
    if size then
      tex.write(size)
    end
  end
end
%    \end{macrocode}
%    See procedure |makepdftime| in file \xfile{utils.c} of \pdfTeX.
%    \begin{macrocode}
function filemoddate(filename)
  local foundfile = kpse.find_file(filename, "tex", true)
  if foundfile then
    local date = lfs.attributes(foundfile, "modification")
    if date then
      local d = os.date("*t", date)
      if d.sec >= 60 then
        d.sec = 59
      end
      local u = os.date("!*t", date)
      local off = 60 * (d.hour - u.hour) + d.min - u.min
      if d.year ~= u.year then
        if d.year > u.year then
          off = off + 1440
        else
          off = off - 1440
        end
      elseif d.yday ~= u.yday then
        if d.yday > u.yday then
          off = off + 1440
        else
          off = off - 1440
        end
      end
      local timezone
      if off == 0 then
        timezone = "Z"
      else
        local hours = math.floor(off / 60)
        local mins = math.abs(off - hours * 60)
        timezone = string.format("%+03d'%02d'", hours, mins)
      end
      tex.write(string.format("D:%04d%02d%02d%02d%02d%02d%s",
          d.year, d.month, d.day, d.hour, d.min, d.sec, timezone))
    end
  end
end
function filedump(offset, length, filename)
  length = tonumber(length)
  if length and length > 0 then
    local foundfile = kpse.find_file(filename, "tex", true)
    if foundfile then
      offset = tonumber(offset)
      if not offset then
        offset = 0
      end
      local filehandle = io.open(foundfile, "r")
      if filehandle then
        if offset > 0 then
          filehandle:seek("set", offset)
        end
        local dump = filehandle:read(length)
        escapehex(dump)
      end
    end
  end
end
function mdfivesum(str, mode)
  if mode == "byte" then
    str = utf8_to_byte(str)
  end
  escapehex(md5.sum(str))
end
function filemdfivesum(filename)
  local foundfile = kpse.find_file(filename, "tex", true)
  if foundfile then
    local filehandle = io.open(foundfile, "r")
    if filehandle then
      local contents = filehandle:read("*a")
      escapehex(md5.sum(contents))
    end
  end
end
function shellescape()
  if os.execute then
    tex.write("1")
  else
    tex.write("0")
  end
end
function system(cmdline)
  systemexitstatus = nil
  texio.write_nl("log", "system(" .. cmdline .. ") ")
  if os.execute then
    texio.write("log", "executed.")
    systemexitstatus = os.execute(cmdline)
  else
    texio.write("log", "disabled.")
  end
end
function lastsystemstatus()
  local result = tonumber(systemexitstatus)
  if result then
    local x = math.floor(result / 256)
    tex.write(result - 256 * math.floor(result / 256))
  end
end
function lastsystemexit()
  local result = tonumber(systemexitstatus)
  if result then
    tex.write(math.floor(result / 256))
  end
end
function pipe(cmdline)
  local result
  systemexitstatus = nil
  texio.write_nl("log", "pipe(" .. cmdline ..") ")
  if io.popen then
    texio.write("log", "executed.")
    local handle = io.popen(cmdline, "r")
    if handle then
      result = handle:read("*a")
      handle:close()
    end
  else
    texio.write("log", "disabled.")
  end
  if result then
    tex.settoks(toks, result)
  else
    tex.settoks(toks, "")
  end
end
%    \end{macrocode}
%    \begin{macrocode}
%</lua>
%    \end{macrocode}
%
% \section{Test}
%
% \subsection{Catcode checks for loading}
%
%    \begin{macrocode}
%<*test1>
%    \end{macrocode}
%    \begin{macrocode}
\catcode`\{=1 %
\catcode`\}=2 %
\catcode`\#=6 %
\catcode`\@=11 %
\expandafter\ifx\csname count@\endcsname\relax
  \countdef\count@=255 %
\fi
\expandafter\ifx\csname @gobble\endcsname\relax
  \long\def\@gobble#1{}%
\fi
\expandafter\ifx\csname @firstofone\endcsname\relax
  \long\def\@firstofone#1{#1}%
\fi
\expandafter\ifx\csname loop\endcsname\relax
  \expandafter\@firstofone
\else
  \expandafter\@gobble
\fi
{%
  \def\loop#1\repeat{%
    \def\body{#1}%
    \iterate
  }%
  \def\iterate{%
    \body
      \let\next\iterate
    \else
      \let\next\relax
    \fi
    \next
  }%
  \let\repeat=\fi
}%
\def\RestoreCatcodes{}
\count@=0 %
\loop
  \edef\RestoreCatcodes{%
    \RestoreCatcodes
    \catcode\the\count@=\the\catcode\count@\relax
  }%
\ifnum\count@<255 %
  \advance\count@ 1 %
\repeat

\def\RangeCatcodeInvalid#1#2{%
  \count@=#1\relax
  \loop
    \catcode\count@=15 %
  \ifnum\count@<#2\relax
    \advance\count@ 1 %
  \repeat
}
\expandafter\ifx\csname LoadCommand\endcsname\relax
  \def\LoadCommand{\input pdftexcmds.sty\relax}%
\fi
\def\Test{%
  \RangeCatcodeInvalid{0}{47}%
  \RangeCatcodeInvalid{58}{64}%
  \RangeCatcodeInvalid{91}{96}%
  \RangeCatcodeInvalid{123}{255}%
  \catcode`\@=12 %
  \catcode`\\=0 %
  \catcode`\{=1 %
  \catcode`\}=2 %
  \catcode`\#=6 %
  \catcode`\[=12 %
  \catcode`\]=12 %
  \catcode`\%=14 %
  \catcode`\ =10 %
  \catcode13=5 %
  \LoadCommand
  \RestoreCatcodes
}
\Test
\csname @@end\endcsname
\end
%    \end{macrocode}
%    \begin{macrocode}
%</test1>
%    \end{macrocode}
%
% \section{Installation}
%
% \subsection{Download}
%
% \paragraph{Package.} This package is available on
% CTAN\footnote{\url{ftp://ftp.ctan.org/tex-archive/}}:
% \begin{description}
% \item[\CTAN{macros/latex/contrib/oberdiek/pdftexcmds.dtx}] The source file.
% \item[\CTAN{macros/latex/contrib/oberdiek/pdftexcmds.pdf}] Documentation.
% \end{description}
%
%
% \paragraph{Bundle.} All the packages of the bundle `oberdiek'
% are also available in a TDS compliant ZIP archive. There
% the packages are already unpacked and the documentation files
% are generated. The files and directories obey the TDS standard.
% \begin{description}
% \item[\CTAN{macros/latex/contrib/oberdiek/oberdiek-tds.zip}]
% \end{description}
% \emph{TDS} refers to the standard ``A Directory Structure
% for \TeX\ Files'' (\CTAN{tds/tds.pdf}). Directories
% with \xfile{texmf} in their name are usually organized this way.
%
% \subsection{Bundle installation}
%
% \paragraph{Unpacking.} Unpack the \xfile{oberdiek-tds.zip} in the
% TDS tree (also known as \xfile{texmf} tree) of your choice.
% Example (linux):
% \begin{quote}
%   |unzip oberdiek-tds.zip -d ~/texmf|
% \end{quote}
%
% \paragraph{Script installation.}
% Check the directory \xfile{TDS:scripts/oberdiek/} for
% scripts that need further installation steps.
% Package \xpackage{attachfile2} comes with the Perl script
% \xfile{pdfatfi.pl} that should be installed in such a way
% that it can be called as \texttt{pdfatfi}.
% Example (linux):
% \begin{quote}
%   |chmod +x scripts/oberdiek/pdfatfi.pl|\\
%   |cp scripts/oberdiek/pdfatfi.pl /usr/local/bin/|
% \end{quote}
%
% \subsection{Package installation}
%
% \paragraph{Unpacking.} The \xfile{.dtx} file is a self-extracting
% \docstrip\ archive. The files are extracted by running the
% \xfile{.dtx} through \plainTeX:
% \begin{quote}
%   \verb|tex pdftexcmds.dtx|
% \end{quote}
%
% \paragraph{TDS.} Now the different files must be moved into
% the different directories in your installation TDS tree
% (also known as \xfile{texmf} tree):
% \begin{quote}
% \def\t{^^A
% \begin{tabular}{@{}>{\ttfamily}l@{ $\rightarrow$ }>{\ttfamily}l@{}}
%   pdftexcmds.sty & tex/generic/oberdiek/pdftexcmds.sty\\
%   oberdiek.pdftexcmds.lua & scripts/oberdiek/oberdiek.pdftexcmds.lua\\
%   pdftexcmds.lua & scripts/oberdiek/pdftexcmds.lua\\
%   pdftexcmds.pdf & doc/latex/oberdiek/pdftexcmds.pdf\\
%   pdftexcmds.dtx & source/latex/oberdiek/pdftexcmds.dtx\\
% \end{tabular}^^A
% }^^A
% \sbox0{\t}^^A
% \ifdim\wd0>\linewidth
%   \begingroup
%     \advance\linewidth by\leftmargin
%     \advance\linewidth by\rightmargin
%   \edef\x{\endgroup
%     \def\noexpand\lw{\the\linewidth}^^A
%   }\x
%   \def\lwbox{^^A
%     \leavevmode
%     \hbox to \linewidth{^^A
%       \kern-\leftmargin\relax
%       \hss
%       \usebox0
%       \hss
%       \kern-\rightmargin\relax
%     }^^A
%   }^^A
%   \ifdim\wd0>\lw
%     \sbox0{\small\t}^^A
%     \ifdim\wd0>\linewidth
%       \ifdim\wd0>\lw
%         \sbox0{\footnotesize\t}^^A
%         \ifdim\wd0>\linewidth
%           \ifdim\wd0>\lw
%             \sbox0{\scriptsize\t}^^A
%             \ifdim\wd0>\linewidth
%               \ifdim\wd0>\lw
%                 \sbox0{\tiny\t}^^A
%                 \ifdim\wd0>\linewidth
%                   \lwbox
%                 \else
%                   \usebox0
%                 \fi
%               \else
%                 \lwbox
%               \fi
%             \else
%               \usebox0
%             \fi
%           \else
%             \lwbox
%           \fi
%         \else
%           \usebox0
%         \fi
%       \else
%         \lwbox
%       \fi
%     \else
%       \usebox0
%     \fi
%   \else
%     \lwbox
%   \fi
% \else
%   \usebox0
% \fi
% \end{quote}
% If you have a \xfile{docstrip.cfg} that configures and enables \docstrip's
% TDS installing feature, then some files can already be in the right
% place, see the documentation of \docstrip.
%
% \subsection{Refresh file name databases}
%
% If your \TeX~distribution
% (\teTeX, \mikTeX, \dots) relies on file name databases, you must refresh
% these. For example, \teTeX\ users run \verb|texhash| or
% \verb|mktexlsr|.
%
% \subsection{Some details for the interested}
%
% \paragraph{Attached source.}
%
% The PDF documentation on CTAN also includes the
% \xfile{.dtx} source file. It can be extracted by
% AcrobatReader 6 or higher. Another option is \textsf{pdftk},
% e.g. unpack the file into the current directory:
% \begin{quote}
%   \verb|pdftk pdftexcmds.pdf unpack_files output .|
% \end{quote}
%
% \paragraph{Unpacking with \LaTeX.}
% The \xfile{.dtx} chooses its action depending on the format:
% \begin{description}
% \item[\plainTeX:] Run \docstrip\ and extract the files.
% \item[\LaTeX:] Generate the documentation.
% \end{description}
% If you insist on using \LaTeX\ for \docstrip\ (really,
% \docstrip\ does not need \LaTeX), then inform the autodetect routine
% about your intention:
% \begin{quote}
%   \verb|latex \let\install=y% \iffalse meta-comment
%
% Copyright (C) 2007 by Heiko Oberdiek <oberdiek@uni-freiburg.de>
%
% This work may be distributed and/or modified under the
% conditions of the LaTeX Project Public License, either
% version 1.3 of this license or (at your option) any later
% version. The latest version of this license is in
%    http://www.latex-project.org/lppl.txt
% and version 1.3 or later is part of all distributions of
% LaTeX version 2005/12/01 or later.
%
% This work has the LPPL maintenance status "maintained".
%
% This Current Maintainer of this work is Heiko Oberdiek.
%
% This work consists of the main source file pdftexcmds.dtx
% and the derived files
%    pdftexcmds.sty, pdftexcmds.pdf, pdftexcmds.ins, pdftexcmds.drv,
%    oberdiek.pdftexcmds.lua, pdftexcmds.lua.
%
% Distribution:
%    CTAN:macros/latex/contrib/oberdiek/pdftexcmds.dtx
%    CTAN:macros/latex/contrib/oberdiek/pdftexcmds.pdf
%
% Unpacking:
%    (a) If pdftexcmds.ins is present:
%           tex pdftexcmds.ins
%    (b) Without pdftexcmds.ins:
%           tex pdftexcmds.dtx
%    (c) If you insist on using LaTeX
%           latex \let\install=y% \iffalse meta-comment
%
% Copyright (C) 2007 by Heiko Oberdiek <oberdiek@uni-freiburg.de>
%
% This work may be distributed and/or modified under the
% conditions of the LaTeX Project Public License, either
% version 1.3 of this license or (at your option) any later
% version. The latest version of this license is in
%    http://www.latex-project.org/lppl.txt
% and version 1.3 or later is part of all distributions of
% LaTeX version 2005/12/01 or later.
%
% This work has the LPPL maintenance status "maintained".
%
% This Current Maintainer of this work is Heiko Oberdiek.
%
% This work consists of the main source file pdftexcmds.dtx
% and the derived files
%    pdftexcmds.sty, pdftexcmds.pdf, pdftexcmds.ins, pdftexcmds.drv,
%    oberdiek.pdftexcmds.lua, pdftexcmds.lua.
%
% Distribution:
%    CTAN:macros/latex/contrib/oberdiek/pdftexcmds.dtx
%    CTAN:macros/latex/contrib/oberdiek/pdftexcmds.pdf
%
% Unpacking:
%    (a) If pdftexcmds.ins is present:
%           tex pdftexcmds.ins
%    (b) Without pdftexcmds.ins:
%           tex pdftexcmds.dtx
%    (c) If you insist on using LaTeX
%           latex \let\install=y\input{pdftexcmds.dtx}
%        (quote the arguments according to the demands of your shell)
%
% Documentation:
%    (a) If pdftexcmds.drv is present:
%           latex pdftexcmds.drv
%    (b) Without pdftexcmds.drv:
%           latex pdftexcmds.dtx; ...
%    The class ltxdoc loads the configuration file ltxdoc.cfg
%    if available. Here you can specify further options, e.g.
%    use A4 as paper format:
%       \PassOptionsToClass{a4paper}{article}
%
%    Programm calls to get the documentation (example):
%       pdflatex pdftexcmds.dtx
%       makeindex -s gind.ist pdftexcmds.idx
%       pdflatex pdftexcmds.dtx
%       makeindex -s gind.ist pdftexcmds.idx
%       pdflatex pdftexcmds.dtx
%
% Installation:
%    TDS:tex/generic/oberdiek/pdftexcmds.sty
%    TDS:scripts/oberdiek/oberdiek.pdftexcmds.lua
%    TDS:scripts/oberdiek/pdftexcmds.lua
%    TDS:doc/latex/oberdiek/pdftexcmds.pdf
%    TDS:source/latex/oberdiek/pdftexcmds.dtx
%
%<*ignore>
\begingroup
  \def\x{LaTeX2e}%
\expandafter\endgroup
\ifcase 0\ifx\install y1\fi\expandafter
         \ifx\csname processbatchFile\endcsname\relax\else1\fi
         \ifx\fmtname\x\else 1\fi\relax
\else\csname fi\endcsname
%</ignore>
%<*install>
\input docstrip.tex
\Msg{************************************************************************}
\Msg{* Installation}
\Msg{* Package: pdftexcmds 2007/12/12 v0.3 LuaTeX support for pdfTeX utility functions (HO)}
\Msg{************************************************************************}

\keepsilent
\askforoverwritefalse

\let\MetaPrefix\relax
\preamble

This is a generated file.

Copyright (C) 2007 by Heiko Oberdiek <oberdiek@uni-freiburg.de>

This work may be distributed and/or modified under the
conditions of the LaTeX Project Public License, either
version 1.3 of this license or (at your option) any later
version. The latest version of this license is in
   http://www.latex-project.org/lppl.txt
and version 1.3 or later is part of all distributions of
LaTeX version 2005/12/01 or later.

This work has the LPPL maintenance status "maintained".

This Current Maintainer of this work is Heiko Oberdiek.

This work consists of the main source file pdftexcmds.dtx
and the derived files
   pdftexcmds.sty, pdftexcmds.pdf, pdftexcmds.ins, pdftexcmds.drv,
   oberdiek.pdftexcmds.lua, pdftexcmds.lua.

\endpreamble
\let\MetaPrefix\DoubleperCent

\generate{%
  \file{pdftexcmds.ins}{\from{pdftexcmds.dtx}{install}}%
  \file{pdftexcmds.drv}{\from{pdftexcmds.dtx}{driver}}%
  \usedir{tex/generic/oberdiek}%
  \file{pdftexcmds.sty}{\from{pdftexcmds.dtx}{package}}%
}
\def\MetaPrefix{-- }
\def\defaultpostamble{%
  \MetaPrefix^^J%
  \MetaPrefix\space End of File `\outFileName'.%
}
\def\currentpostamble{\defaultpostamble}%
\generate{%
  \usedir{scripts/oberdiek}%
  \file{oberdiek.pdftexcmds.lua}{\from{pdftexcmds.dtx}{lua}}%
  \file{pdftexcmds.lua}{\from{pdftexcmds.dtx}{lua}}%
}

\obeyspaces
\Msg{************************************************************************}
\Msg{*}
\Msg{* To finish the installation you have to move the following}
\Msg{* file into a directory searched by TeX:}
\Msg{*}
\Msg{*     pdftexcmds.sty}
\Msg{*}
\Msg{* And install the following script files:}
\Msg{*}
\Msg{*     oberdiek.pdftexcmds.lua, pdftexcmds.lua}
\Msg{*}
\Msg{* To produce the documentation run the file `pdftexcmds.drv'}
\Msg{* through LaTeX.}
\Msg{*}
\Msg{* Happy TeXing!}
\Msg{*}
\Msg{************************************************************************}

\endbatchfile
%</install>
%<*ignore>
\fi
%</ignore>
%<*driver>
\NeedsTeXFormat{LaTeX2e}
\ProvidesFile{pdftexcmds.drv}%
  [2007/12/12 v0.3 LuaTeX support for pdfTeX utility functions (HO)]%
\documentclass{ltxdoc}
\usepackage{holtxdoc}[2007/11/11]
\usepackage{paralist}
\begin{document}
  \DocInput{pdftexcmds.dtx}%
\end{document}
%</driver>
% \fi
%
% \CheckSum{552}
%
% \CharacterTable
%  {Upper-case    \A\B\C\D\E\F\G\H\I\J\K\L\M\N\O\P\Q\R\S\T\U\V\W\X\Y\Z
%   Lower-case    \a\b\c\d\e\f\g\h\i\j\k\l\m\n\o\p\q\r\s\t\u\v\w\x\y\z
%   Digits        \0\1\2\3\4\5\6\7\8\9
%   Exclamation   \!     Double quote  \"     Hash (number) \#
%   Dollar        \$     Percent       \%     Ampersand     \&
%   Acute accent  \'     Left paren    \(     Right paren   \)
%   Asterisk      \*     Plus          \+     Comma         \,
%   Minus         \-     Point         \.     Solidus       \/
%   Colon         \:     Semicolon     \;     Less than     \<
%   Equals        \=     Greater than  \>     Question mark \?
%   Commercial at \@     Left bracket  \[     Backslash     \\
%   Right bracket \]     Circumflex    \^     Underscore    \_
%   Grave accent  \`     Left brace    \{     Vertical bar  \|
%   Right brace   \}     Tilde         \~}
%
% \GetFileInfo{pdftexcmds.drv}
%
% \title{The \xpackage{pdftexcmds} package}
% \date{2007/12/12 v0.3}
% \author{Heiko Oberdiek\\\xemail{oberdiek@uni-freiburg.de}}
%
% \maketitle
%
% \begin{abstract}
% \LuaTeX\ provides most of the commands of \pdfTeX\ 1.40. However
% a number of utility functions are removed. This package tries to fill
% the gap and implements some of the missing primitive using Lua.
% \end{abstract}
%
% \tableofcontents
%
% \section{Documentation}
%
% Some primitives of \pdfTeX\ are not defined by \LuaTeX.
% This package implements macro based solutions using Lua code
% for the following missing \pdfTeX\ primitives;
% \begin{compactitem}
% \item \cs{pdfstrcmp}
% \item \cs{pdfunescapehex}
% \item \cs{pdfescapehex}
% \item \cs{pdfescapename}
% \item \cs{pdfescapestring}
% \item \cs{pdffilesize}
% \item \cs{pdffilemoddate}
% \item \cs{pdffiledump}
% \item \cs{pdfmdfivesum}
% \item |\immediate\write18|
% \end{compactitem}
% The original names of the primitives cannot be used:
% \begin{itemize}
% \item
% The syntax for their arguments cannot easily
% simulated by macros. The primitives using key words
% such as |file| (\cs{pdfmdfivesum}) or |offset| and |length|
% (\cs{pdffiledump}) and uses \meta{general text} for the other
% arguments. Using token registers assignments, \meta{general text} could
% be catched. However, the simulated primitives are expandable
% and register assignments would destroy this important property.
% (\meta{general text} allows something like |\expandafter\bgroup ...}|.)
% \item
% The original primitives can be expanded using one expansion step.
% The new macros need two expansion steps because of the additional
% macro expansion. Example:
% \begin{quote}
%   |\expandafter\foo\pdffilemoddate{file}|\\
%   vs.
%   |\expandafter\expandafter\expandafter\foo\pdf@filemoddate{file}|.
% \end{quote}
% \end{itemize}
%
% \LuaTeX\ isn't stable yet and thus the status of this package is
% \emph{experimental}. Feedback is welcome.
%
% \subsection{General principles}
%
% \begin{description}
% \item[Naming convention:]
%   Usually this package defines a macro |\pdf@|\meta{cmd} if
%   \pdfTeX\ provides |\pdf|\meta{cmd}.
% \item[Arguments:] The order of arguments in |\pdf@|\meta{cmd}
%   is the same as for the corresponding primitive of \pdfTeX.
%   The arguments are ordinary undelimited \TeX\ arguments,
%   no \meta{general text} and without additional keywords.
% \item[Expandibility:]
%   The macro |\pdf@|\meta{cmd} is expandable if the
%   corresponding \pdfTeX\ primitive has this property.
%   Exact two expansion steps are necessary (first is the macro
%   expansion).
% \item[Without \LuaTeX:]
%   The macros |\pdf@|\meta{cmd} are mapped to the commands
%   of \pdfTeX\ if they are available. Otherwise they are undefined.
% \end{description}
%
% \subsection{Macros}
%
% \begin{declcs}{pdf@strcmp} \M{stringA} \M{stringB}
% \end{declcs}
% Same as |\pdfstrcmp{|\meta{stringA}|}{|\meta{stringB}|}|.
%
% \begin{declcs}{pdf@unescapehex} \M{string}
% \end{declcs}
% Same as |\pdfunescapehex{|\meta{string}|}|.
% The argument is a byte string given in hexadecimal notation.
% The result are character tokens from 0 until 255 with
% catcode 12 and the space with catcode 10.
%
% \begin{declcs}{pdf@escapehex} \M{string}\\
%   \cs{pdf@escapestring} \M{string}\\
%   \cs{pdf@escapename} \M{string}
% \end{declcs}
% Same as the primitives of \pdfTeX. However \pdfTeX\ does not
% know about characters with codes 256 and larger. Thus the
% string is treated as byte string, characters with more than
% eight bits are ignored.
%
% \begin{declcs}{pdf@filesize} \M{filename}
% \end{declcs}
% Same as |\pdffilesize{|\meta{filename}|}|.
%
% \begin{declcs}{pdf@filemoddate} \M{filename}
% \end{declcs}
% Same as |\pdffilemoddate{|\meta{filename}|}|.
%
% \begin{declcs}{pdf@filedump} \M{offset} \M{length} \M{filename}
% \end{declcs}
% Same as |\pdffiledump offset| \meta{offset} |length| \meta{length}
% |{|\meta{filename}|}|. Both \meta{offset} and \meta{length} must
% not be empty, but must be a valid \TeX\ number.
%
% \begin{declcs}{pdf@mdfivesum} \M{string}
% \end{declcs}
% Same as |\pdfmdfivesum{|\meta{string}|}|. Keyword |file| is supported
% by macro \cs{pdf@filemdfivesum}.
%
% \begin{declcs}{pdf@filemdfivesum} \M{filename}
% \end{declcs}
% Same as |\pdfmdfivesum file{|\meta{filename}|}|.
%
% \begin{declcs}{pdf@shellescape}
% \end{declcs}
% Same as |\pdfshellescape|. It expands to |1| if external
% commands can be executed and |0| otherwise. In \pdfTeX\ external
% commands must be enabled first by command line option or
% configuration option. In \LuaTeX\ option |--safer| disables
% the execution of external commands.
%
% \begin{declcs}{pdf@system} \M{cmdline}
% \end{declcs}
% It is a wrapper for |\immediate\write18| in \pdfTeX\ or
% |os.execute| in \LuaTeX.
%
% In theory |os.execute|
% returns a status number. But its meaning is quite
% undefined. Are there some reliable properties?
% Does it make sense to provide an user interface to
% this status exit code?
%
% \subsubsection{Experimental}
%
% \begin{declcs}{pdf@unescapehexnative} \M{string}\\
%   \cs{pdf@escapehexnative} \M{string}\\
%   \cs{pdf@escapenamenative} \M{string}\\
%   \cs{pdf@mdfivesumnative} \M{string}
% \end{declcs}
% The variants without |native| in the macro name are supposed to
% be compatible with \pdfTeX. However characters with more than
% eight bits are not supported and are ignored. If \LuaTeX\ is
% running, then its UTF-8 coded strings are used. Thus the full
% unicode character range is supported. However the result
% differs from \pdfTeX\ for characters with eight or more bits.
%
% \begin{declcs}{pdf@pipe} \M{cmdline}
% \end{declcs}
% It calls \meta{cmdline} and returns the output of the external
% program in the usual manner as byte string (catcode 12, space with
% catcode 10). The Lua documentation says, that the used |io.popen|
% may not be available on all platforms. Then macro \cs{pdf@pipe}
% is undefined.
%
% \StopEventually{
% }
%
% \section{Implementation}
%
%    \begin{macrocode}
%<*package>
%    \end{macrocode}
%
% \subsection{Reload check and package identification}
%    Reload check, especially if the package is not used with \LaTeX.
%    \begin{macrocode}
\begingroup
  \catcode44 12 % ,
  \catcode45 12 % -
  \catcode46 12 % .
  \catcode58 12 % :
  \catcode64 11 % @
  \expandafter\let\expandafter\x\csname ver@pdftexcmds.sty\endcsname
  \ifcase 0%
    \ifx\x\relax % plain
    \else
      \ifx\x\empty % LaTeX
      \else
        1%
      \fi
    \fi
  \else
    \catcode35 6 % #
    \catcode123 1 % {
    \catcode125 2 % }
    \expandafter\ifx\csname PackageInfo\endcsname\relax
      \def\x#1#2{%
        \immediate\write-1{Package #1 Info: #2.}%
      }%
    \else
      \def\x#1#2{\PackageInfo{#1}{#2, stopped}}%
    \fi
    \x{pdftexcmds}{The package is already loaded}%
    \endgroup
    \expandafter\endinput
  \fi
\endgroup
%    \end{macrocode}
%    Package identification:
%    \begin{macrocode}
\begingroup
  \catcode35 6 % #
  \catcode40 12 % (
  \catcode41 12 % )
  \catcode44 12 % ,
  \catcode45 12 % -
  \catcode46 12 % .
  \catcode47 12 % /
  \catcode58 12 % :
  \catcode64 11 % @
  \catcode123 1 % {
  \catcode125 2 % }
  \expandafter\ifx\csname ProvidesPackage\endcsname\relax
    \def\x#1#2#3[#4]{\endgroup
      \immediate\write-1{Package: #3 #4}%
      \xdef#1{#4}%
    }%
  \else
    \def\x#1#2[#3]{\endgroup
      #2[{#3}]%
      \ifx#1\relax
        \xdef#1{#3}%
      \fi
    }%
  \fi
\expandafter\x\csname ver@pdftexcmds.sty\endcsname
\ProvidesPackage{pdftexcmds}%
  [2007/12/12 v0.3 LuaTeX support for pdfTeX utility functions (HO)]
%    \end{macrocode}
%
% \subsection{Catcodes}
%
%    \begin{macrocode}
\begingroup
  \catcode123 1 % {
  \catcode125 2 % }
  \def\x{\endgroup
    \expandafter\edef\csname pdftexcmds@AtEnd\endcsname{%
      \catcode35 \the\catcode35\relax
      \catcode64 \the\catcode64\relax
      \catcode123 \the\catcode123\relax
      \catcode125 \the\catcode125\relax
    }%
  }%
\x
\catcode35 6 % #
\catcode64 11 % @
\catcode123 1 % {
\catcode125 2 % }
\def\TMP@EnsureCode#1#2{%
  \edef\pdftexcmds@AtEnd{%
    \pdftexcmds@AtEnd
    \catcode#1 \the\catcode#1\relax
  }%
  \catcode#1 #2\relax
}
\TMP@EnsureCode{10}{12}% ^^J
\TMP@EnsureCode{33}{12}% !
\TMP@EnsureCode{34}{12}% "
\TMP@EnsureCode{39}{12}% '
\TMP@EnsureCode{40}{12}% (
\TMP@EnsureCode{41}{12}% )
\TMP@EnsureCode{42}{12}% *
\TMP@EnsureCode{43}{12}% +
\TMP@EnsureCode{44}{12}% ,
\TMP@EnsureCode{45}{12}% -
\TMP@EnsureCode{46}{12}% .
\TMP@EnsureCode{47}{12}% /
\TMP@EnsureCode{58}{12}% :
\TMP@EnsureCode{60}{12}% <
\TMP@EnsureCode{61}{12}% =
\TMP@EnsureCode{62}{12}% >
\TMP@EnsureCode{94}{7}% ^ (superscript)
\TMP@EnsureCode{95}{12}% _ (other)
\TMP@EnsureCode{126}{12}% ~ (other)
%    \end{macrocode}
%
% \subsection{Load package \xpackage{infwarerr}}
%
%    \begin{macrocode}
\begingroup\expandafter\expandafter\expandafter\endgroup
\expandafter\ifx\csname RequirePackage\endcsname\relax
  \input infwarerr.sty\relax
\else
  \RequirePackage{infwarerr}[2007/09/09]%
\fi
%    \end{macrocode}
%
% \subsection{Without \LuaTeX}
%
%    \begin{macrocode}
\begingroup\expandafter\expandafter\expandafter\endgroup
\expandafter\ifx\csname directlua\endcsname\relax
  \@PackageInfo{pdftexcmds}{LuaTeX not detected}%
  \def\pdftexcmds@nopdftex{%
    \@PackageInfoNoLine{pdftexcmds}{pdfTeX >= 1.30 not detected}%
    \let\pdftexcmds@nopdftex\relax
  }%
  \def\pdftexcmds@temp#1{%
    \begingroup\expandafter\expandafter\expandafter\endgroup
    \expandafter\ifx\csname pdf#1\endcsname\relax
      \pdftexcmds@nopdftex
    \else
      \expandafter\def\csname pdf@#1\expandafter\endcsname
      \expandafter##\expandafter{%
        \csname pdf#1\endcsname
      }%
    \fi
  }%
  \pdftexcmds@temp{strcmp}%
  \pdftexcmds@temp{escapehex}%
  \let\pdf@escapehexnative\pdf@escapehex
  \pdftexcmds@temp{unescapehex}%
  \let\pdf@unescapehexnative\pdf@unescapehex
  \pdftexcmds@temp{escapestring}%
  \pdftexcmds@temp{escapename}%
  \pdftexcmds@temp{filesize}%
  \pdftexcmds@temp{filemoddate}%
  \begingroup\expandafter\expandafter\expandafter\endgroup
  \expandafter\ifx\csname pdfshellescape\endcsname\relax
    \pdftexcmds@nopdftex
  \else
    \def\pdf@shellescape{%
      \pdfshellescape
    }%
  \fi
  \begingroup\expandafter\expandafter\expandafter\endgroup
  \expandafter\ifx\csname pdffiledump\endcsname\relax
    \pdftexcmds@nopdftex
  \else
    \def\pdf@filedump#1#2#3{%
      \pdffiledump offset#1 length#2{#3}%
    }%
  \fi
  \begingroup\expandafter\expandafter\expandafter\endgroup
  \expandafter\ifx\csname pdfmdfivesum\endcsname\relax
    \pdftexcmds@nopdftex
  \else
    \def\pdf@mdfivesum#{\pdfmdfivesum}%
    \let\pdf@mdfivesumnative\pdf@mdfivesum
    \def\pdf@filemdfivesum#{\pdfmdfivesum file}%
  \fi
  \def\pdf@system#{%
    \immediate\write18%
  }%
  \pdftexcmds@AtEnd
  \expandafter\endinput
\fi
%    \end{macrocode}
%
% \subsection{Load module}
%
%    \begin{macrocode}
\begingroup\expandafter\expandafter\expandafter\endgroup
\expandafter\ifx\csname RequirePackage\endcsname\relax
  \input luatex-loader.sty\relax
\else
  \RequirePackage{luatex-loader}[2007/12/12]%
\fi
\directlua0{%
  require("oberdiek.pdftexcmds")%
}
%    \end{macrocode}
%
% \subsection{Lua functions}
%
%    \begin{macro}{\pdftexcmds@toks}
%    \begin{macrocode}
\begingroup\expandafter\expandafter\expandafter\endgroup
\expandafter\ifx\csname newtoks\endcsname\relax
  \toksdef\pdftexcmds@toks=0 %
\else
  \csname newtoks\endcsname\pdftexcmds@toks
\fi
%    \end{macrocode}
%    \end{macro}
%
%    \begin{macro}{\pdf@strcmp}
%    \begin{macrocode}
\long\def\pdf@strcmp#1#2{%
  \directlua0{%
    oberdiek.pdftexcmds.strcmp("\luaescapestring{#1}",%
        "\luaescapestring{#2}")%
  }%
}%
%    \end{macrocode}
%    \end{macro}
%    \begin{macro}{\pdf@escapehex}
%    \begin{macrocode}
\long\def\pdf@escapehex#1{%
  \directlua0{%
    oberdiek.pdftexcmds.escapehex("\luaescapestring{#1}", "byte")%
  }%
}%
%    \end{macrocode}
%    \end{macro}
%    \begin{macro}{\pdf@escapehexnative}
%    \begin{macrocode}
\long\def\pdf@escapehexnative#1{%
  \directlua0{%
    oberdiek.pdftexcmds.escapehex("\luaescapestring{#1}")%
  }%
}%
%    \end{macrocode}
%    \end{macro}
%    \begin{macro}{\pdf@unescapehex}
%    \begin{macrocode}
\def\pdf@unescapehex#1{%
  \the\expandafter\pdftexcmds@toks
  \directlua0{%
    oberdiek.pdftexcmds.toks="pdftexcmds@toks"%
    oberdiek.pdftexcmds.unescapehex("\luaescapestring{#1}", "byte")%
  }%
}%
%    \end{macrocode}
%    \end{macro}
%    \begin{macro}{\pdf@unescapehexnative}
%    \begin{macrocode}
\def\pdf@unescapehexnative#1{%
  \the\expandafter\pdftexcmds@toks
  \directlua0{%
    oberdiek.pdftexcmds.toks="pdftexcmds@toks"%
    oberdiek.pdftexcmds.unescapehex("\luaescapestring{#1}")%
  }%
}%
%    \end{macrocode}
%    \end{macro}
%    \begin{macro}{\pdf@escapestring}
%    \begin{macrocode}
\long\def\pdf@escapestring#1{%
  \directlua0{%
    oberdiek.pdftexcmds.escapestring("\luaescapestring{#1}", "byte")%
  }%
}
%    \end{macrocode}
%    \end{macro}
%    \begin{macro}{\pdf@escapename}
%    \begin{macrocode}
\long\def\pdf@escapename#1{%
  \directlua0{%
    oberdiek.pdftexcmds.escapename("\luaescapestring{#1}", "byte")%
  }%
}
%    \end{macrocode}
%    \end{macro}
%    \begin{macro}{\pdf@escapenamenative}
%    \begin{macrocode}
\long\def\pdf@escapenamenative#1{%
  \directlua0{%
    oberdiek.pdftexcmds.escapename("\luaescapestring{#1}")%
  }%
}
%    \end{macrocode}
%    \end{macro}
%    \begin{macro}{\pdf@filesize}
%    \begin{macrocode}
\def\pdf@filesize#1{%
  \directlua0{%
    oberdiek.pdftexcmds.filesize("\luaescapestring{#1}")%
  }%
}
%    \end{macrocode}
%    \end{macro}
%    \begin{macro}{\pdf@filemoddate}
%    \begin{macrocode}
\def\pdf@filemoddate#1{%
  \directlua0{%
    oberdiek.pdftexcmds.filemoddate("\luaescapestring{#1}")%
  }%
}
%    \end{macrocode}
%    \end{macro}
%    \begin{macro}{\pdf@filedump}
%    \begin{macrocode}
\def\pdf@filedump#1#2#3{%
  \directlua0{%
    oberdiek.pdftexcmds.filedump("\luaescapestring{\number#1}",%
        "\luaescapestring{\number#2}",%
        "\luaescapestring{#3}")%
  }%
}%
%    \end{macrocode}
%    \end{macro}
%    \begin{macro}{\pdf@mdfivesum}
%    \begin{macrocode}
\long\def\pdf@mdfivesum#1{%
  \directlua0{%
    oberdiek.pdftexcmds.mdfivesum("\luaescapestring{#1}", "byte")%
  }%
}%
%    \end{macrocode}
%    \end{macro}
%    \begin{macro}{\pdf@mdfivesumnative}
%    \begin{macrocode}
\long\def\pdf@mdfivesumnative#1{%
  \directlua0{%
    oberdiek.pdftexcmds.mdfivesum("\luaescapestring{#1}")%
  }%
}%
%    \end{macrocode}
%    \end{macro}
%    \begin{macro}{\pdf@filemdfivesum}
%    \begin{macrocode}
\def\pdf@filemdfivesum#1{%
  \directlua0{%
    oberdiek.pdftexcmds.filemdfivesum("\luaescapestring{#1}")%
  }%
}%
%    \end{macrocode}
%    \end{macro}
%
%    \begin{macro}{\pdf@shellescape}
%    \begin{macrocode}
\def\pdf@shellescape{%
  \directlua0{%
    oberdiek.pdftexcmds.shellescape()%
  }%
}
%    \end{macrocode}
%    \end{macro}
%
%    \begin{macro}{\pdf@system}
%    \begin{macrocode}
\def\pdf@system#1{%
  \directlua0{%
    oberdiek.pdftexcmds.system("\luaescapestring{#1}")%
  }%
}
%    \end{macrocode}
%    \end{macro}
%
%    \begin{macro}{\pdf@lastsystemstatus}
%    \begin{macrocode}
\def\pdf@lastsystemstatus{%
  \directlua0{%
    oberdiek.pdftexcmds.lastsystemstatus()%
  }%
}
%    \end{macrocode}
%    \end{macro}
%    \begin{macro}{\pdf@lastsystemexit}
%    \begin{macrocode}
\def\pdf@lastsystemexit{%
  \directlua0{%
    oberdiek.pdftexcmds.lastsystemexit()%
  }%
}
%    \end{macrocode}
%    \end{macro}
%
%    \begin{macro}{\pdf@pipe}
%    Check availability of |io.popen| first.
%    \begin{macrocode}
\ifnum0%
    \directlua0{%
      if io.popen then %
        tex.write("1")%
      end%
    }%
    =1 %
  \def\pdf@pipe#1{%
    \the\expandafter\pdftexcmds@toks
    \directlua0{%
      oberdiek.pdftexcmds.toks="pdftexcmds@toks"%
      oberdiek.pdftexcmds.pipe("\luaescapestring{#1}")%
    }%
  }%
\fi
%    \end{macrocode}
%    \end{macro}
%
%    \begin{macrocode}
\pdftexcmds@AtEnd
%</package>
%    \end{macrocode}
%
% \subsection{Lua module}
%
%    \begin{macrocode}
%<*lua>
%    \end{macrocode}
%
%    \begin{macrocode}
module("oberdiek.pdftexcmds", package.seeall)
local systemexitstatus
function strcmp(A, B)
  if A == B then
    tex.write("0")
  elseif A < B then
    tex.write("-1")
  else
    tex.write("1")
  end
end
local function utf8_to_byte(str)
  local i = 0
  local n = string.len(str)
  local t = {}
  while i < n do
    i = i + 1
    local a = string.byte(str, i)
    if a < 128 then
      table.insert(t, string.char(a))
    else
      if a >= 192 and i < n then
        i = i + 1
        local b = string.byte(str, i)
        if b < 128 or b >= 192 then
          i = i - 1
        elseif a == 194 then
          table.insert(t, string.char(b))
        elseif a == 195 then
          table.insert(t, string.char(b + 64))
        end
      end
    end
  end
  return table.concat(t)
end
function escapehex(str, mode)
  if mode == "byte" then
    str = utf8_to_byte(str)
  end
  tex.write((string.gsub(str, ".",
    function (ch)
      return string.format("%02X", string.byte(ch))
    end
  )))
end
%    \end{macrocode}
%    See procedure |unescapehex| in file \xfile{utils.c} of \pdfTeX.
%    Caution: |tex.write| ignores leading spaces.
%    \begin{macrocode}
function unescapehex(str, mode)
  local a = 0
  local first = true
  local result = {}
  for i = 1, string.len(str), 1 do
    local ch = string.byte(str, i)
    if ch >= 48 and ch <= 57 then
      ch = ch - 48
    elseif ch >= 65 and ch <= 70 then
      ch = ch - 55
    elseif ch >= 97 and ch <= 102 then
      ch = ch - 87
    else
      ch = nil
    end
    if ch then
      if first then
        a = ch * 16
        first = false
      else
        table.insert(result, a + ch)
        first = true
      end
    end
  end
  if not first then
    table.insert(result, a)
  end
  if mode == "byte" then
    local utf8 = {}
    for i, a in ipairs(result) do
      if a < 128 then
        table.insert(utf8, a)
      else
        if a < 192 then
          table.insert(utf8, 194)
          a = a - 128
        else
          table.insert(utf8, 195)
          a = a - 192
        end
        table.insert(utf8, a + 128)
      end
    end
    result = utf8
  end
  tex.settoks(toks, string.char(unpack(result)))
end
%    \end{macrocode}
%    See procedure |escapestring| in file \xfile{utils.c} of \pdfTeX.
%    \begin{macrocode}
function escapestring(str, mode)
  if mode == "byte" then
    str = utf8_to_byte(str)
  end
  tex.write((string.gsub(str, ".",
    function (ch)
      local b = string.byte(ch)
      if b < 33 or b > 126 then
        return string.format("\\%.3o", b)
      end
      if b == 40 or b == 41 or b == 92 then
        return "\\" .. ch
      end
%    \end{macrocode}
%    Lua 5.1 returns the match in case of return value |nil|.
%    \begin{macrocode}
      return nil
    end
  )))
end
%    \end{macrocode}
%    See procedure |escapename| in file \xfile{utils.c} of \pdfTeX.
%    \begin{macrocode}
function escapename(str, mode)
  if mode == "byte" then
    str = utf8_to_byte(str)
  end
  tex.write((string.gsub(str, ".",
    function (ch)
      local b = string.byte(ch)
      if b == 0 then
%    \end{macrocode}
%    In Lua 5.0 |nil| could be used for the empty string,
%    But |nil| returns the match in Lua 5.1, thus we use
%    the empty string explicitly.
%    \begin{macrocode}
        return ""
      end
      if b <= 32 or b >= 127
          or b == 35 or b == 37 or b == 40 or b == 41
          or b == 47 or b == 60 or b == 62 or b == 91
          or b == 93 or b == 123 or b == 125 then
        return string.format("#%.2X", b)
      else
%    \end{macrocode}
%    Lua 5.1 returns the match in case of return value |nil|.
%    \begin{macrocode}
        return nil
      end
    end
  )))
end
function filesize(filename)
  local foundfile = kpse.find_file(filename, "tex", true)
  if foundfile then
    local size = lfs.attributes(foundfile, "size")
    if size then
      tex.write(size)
    end
  end
end
%    \end{macrocode}
%    See procedure |makepdftime| in file \xfile{utils.c} of \pdfTeX.
%    \begin{macrocode}
function filemoddate(filename)
  local foundfile = kpse.find_file(filename, "tex", true)
  if foundfile then
    local date = lfs.attributes(foundfile, "modification")
    if date then
      local d = os.date("*t", date)
      if d.sec >= 60 then
        d.sec = 59
      end
      local u = os.date("!*t", date)
      local off = 60 * (d.hour - u.hour) + d.min - u.min
      if d.year ~= u.year then
        if d.year > u.year then
          off = off + 1440
        else
          off = off - 1440
        end
      elseif d.yday ~= u.yday then
        if d.yday > u.yday then
          off = off + 1440
        else
          off = off - 1440
        end
      end
      local timezone
      if off == 0 then
        timezone = "Z"
      else
        local hours = math.floor(off / 60)
        local mins = math.abs(off - hours * 60)
        timezone = string.format("%+03d'%02d'", hours, mins)
      end
      tex.write(string.format("D:%04d%02d%02d%02d%02d%02d%s",
          d.year, d.month, d.day, d.hour, d.min, d.sec, timezone))
    end
  end
end
function filedump(offset, length, filename)
  length = tonumber(length)
  if length and length > 0 then
    local foundfile = kpse.find_file(filename, "tex", true)
    if foundfile then
      offset = tonumber(offset)
      if not offset then
        offset = 0
      end
      local filehandle = io.open(foundfile, "r")
      if filehandle then
        if offset > 0 then
          filehandle:seek("set", offset)
        end
        local dump = filehandle:read(length)
        escapehex(dump)
      end
    end
  end
end
function mdfivesum(str, mode)
  if mode == "byte" then
    str = utf8_to_byte(str)
  end
  escapehex(md5.sum(str))
end
function filemdfivesum(filename)
  local foundfile = kpse.find_file(filename, "tex", true)
  if foundfile then
    local filehandle = io.open(foundfile, "r")
    if filehandle then
      local contents = filehandle:read("*a")
      escapehex(md5.sum(contents))
    end
  end
end
function shellescape()
  if os.execute then
    tex.write("1")
  else
    tex.write("0")
  end
end
function system(cmdline)
  systemexitstatus = nil
  texio.write_nl("log", "system(" .. cmdline .. ") ")
  if os.execute then
    texio.write("log", "executed.")
    systemexitstatus = os.execute(cmdline)
  else
    texio.write("log", "disabled.")
  end
end
function lastsystemstatus()
  local result = tonumber(systemexitstatus)
  if result then
    local x = math.floor(result / 256)
    tex.write(result - 256 * math.floor(result / 256))
  end
end
function lastsystemexit()
  local result = tonumber(systemexitstatus)
  if result then
    tex.write(math.floor(result / 256))
  end
end
function pipe(cmdline)
  local result
  systemexitstatus = nil
  texio.write_nl("log", "pipe(" .. cmdline ..") ")
  if io.popen then
    texio.write("log", "executed.")
    local handle = io.popen(cmdline, "r")
    if handle then
      result = handle:read("*a")
      handle:close()
    end
  else
    texio.write("log", "disabled.")
  end
  if result then
    tex.settoks(toks, result)
  else
    tex.settoks(toks, "")
  end
end
%    \end{macrocode}
%    \begin{macrocode}
%</lua>
%    \end{macrocode}
%
% \section{Test}
%
% \subsection{Catcode checks for loading}
%
%    \begin{macrocode}
%<*test1>
%    \end{macrocode}
%    \begin{macrocode}
\catcode`\{=1 %
\catcode`\}=2 %
\catcode`\#=6 %
\catcode`\@=11 %
\expandafter\ifx\csname count@\endcsname\relax
  \countdef\count@=255 %
\fi
\expandafter\ifx\csname @gobble\endcsname\relax
  \long\def\@gobble#1{}%
\fi
\expandafter\ifx\csname @firstofone\endcsname\relax
  \long\def\@firstofone#1{#1}%
\fi
\expandafter\ifx\csname loop\endcsname\relax
  \expandafter\@firstofone
\else
  \expandafter\@gobble
\fi
{%
  \def\loop#1\repeat{%
    \def\body{#1}%
    \iterate
  }%
  \def\iterate{%
    \body
      \let\next\iterate
    \else
      \let\next\relax
    \fi
    \next
  }%
  \let\repeat=\fi
}%
\def\RestoreCatcodes{}
\count@=0 %
\loop
  \edef\RestoreCatcodes{%
    \RestoreCatcodes
    \catcode\the\count@=\the\catcode\count@\relax
  }%
\ifnum\count@<255 %
  \advance\count@ 1 %
\repeat

\def\RangeCatcodeInvalid#1#2{%
  \count@=#1\relax
  \loop
    \catcode\count@=15 %
  \ifnum\count@<#2\relax
    \advance\count@ 1 %
  \repeat
}
\expandafter\ifx\csname LoadCommand\endcsname\relax
  \def\LoadCommand{\input pdftexcmds.sty\relax}%
\fi
\def\Test{%
  \RangeCatcodeInvalid{0}{47}%
  \RangeCatcodeInvalid{58}{64}%
  \RangeCatcodeInvalid{91}{96}%
  \RangeCatcodeInvalid{123}{255}%
  \catcode`\@=12 %
  \catcode`\\=0 %
  \catcode`\{=1 %
  \catcode`\}=2 %
  \catcode`\#=6 %
  \catcode`\[=12 %
  \catcode`\]=12 %
  \catcode`\%=14 %
  \catcode`\ =10 %
  \catcode13=5 %
  \LoadCommand
  \RestoreCatcodes
}
\Test
\csname @@end\endcsname
\end
%    \end{macrocode}
%    \begin{macrocode}
%</test1>
%    \end{macrocode}
%
% \section{Installation}
%
% \subsection{Download}
%
% \paragraph{Package.} This package is available on
% CTAN\footnote{\url{ftp://ftp.ctan.org/tex-archive/}}:
% \begin{description}
% \item[\CTAN{macros/latex/contrib/oberdiek/pdftexcmds.dtx}] The source file.
% \item[\CTAN{macros/latex/contrib/oberdiek/pdftexcmds.pdf}] Documentation.
% \end{description}
%
%
% \paragraph{Bundle.} All the packages of the bundle `oberdiek'
% are also available in a TDS compliant ZIP archive. There
% the packages are already unpacked and the documentation files
% are generated. The files and directories obey the TDS standard.
% \begin{description}
% \item[\CTAN{macros/latex/contrib/oberdiek/oberdiek-tds.zip}]
% \end{description}
% \emph{TDS} refers to the standard ``A Directory Structure
% for \TeX\ Files'' (\CTAN{tds/tds.pdf}). Directories
% with \xfile{texmf} in their name are usually organized this way.
%
% \subsection{Bundle installation}
%
% \paragraph{Unpacking.} Unpack the \xfile{oberdiek-tds.zip} in the
% TDS tree (also known as \xfile{texmf} tree) of your choice.
% Example (linux):
% \begin{quote}
%   |unzip oberdiek-tds.zip -d ~/texmf|
% \end{quote}
%
% \paragraph{Script installation.}
% Check the directory \xfile{TDS:scripts/oberdiek/} for
% scripts that need further installation steps.
% Package \xpackage{attachfile2} comes with the Perl script
% \xfile{pdfatfi.pl} that should be installed in such a way
% that it can be called as \texttt{pdfatfi}.
% Example (linux):
% \begin{quote}
%   |chmod +x scripts/oberdiek/pdfatfi.pl|\\
%   |cp scripts/oberdiek/pdfatfi.pl /usr/local/bin/|
% \end{quote}
%
% \subsection{Package installation}
%
% \paragraph{Unpacking.} The \xfile{.dtx} file is a self-extracting
% \docstrip\ archive. The files are extracted by running the
% \xfile{.dtx} through \plainTeX:
% \begin{quote}
%   \verb|tex pdftexcmds.dtx|
% \end{quote}
%
% \paragraph{TDS.} Now the different files must be moved into
% the different directories in your installation TDS tree
% (also known as \xfile{texmf} tree):
% \begin{quote}
% \def\t{^^A
% \begin{tabular}{@{}>{\ttfamily}l@{ $\rightarrow$ }>{\ttfamily}l@{}}
%   pdftexcmds.sty & tex/generic/oberdiek/pdftexcmds.sty\\
%   oberdiek.pdftexcmds.lua & scripts/oberdiek/oberdiek.pdftexcmds.lua\\
%   pdftexcmds.lua & scripts/oberdiek/pdftexcmds.lua\\
%   pdftexcmds.pdf & doc/latex/oberdiek/pdftexcmds.pdf\\
%   pdftexcmds.dtx & source/latex/oberdiek/pdftexcmds.dtx\\
% \end{tabular}^^A
% }^^A
% \sbox0{\t}^^A
% \ifdim\wd0>\linewidth
%   \begingroup
%     \advance\linewidth by\leftmargin
%     \advance\linewidth by\rightmargin
%   \edef\x{\endgroup
%     \def\noexpand\lw{\the\linewidth}^^A
%   }\x
%   \def\lwbox{^^A
%     \leavevmode
%     \hbox to \linewidth{^^A
%       \kern-\leftmargin\relax
%       \hss
%       \usebox0
%       \hss
%       \kern-\rightmargin\relax
%     }^^A
%   }^^A
%   \ifdim\wd0>\lw
%     \sbox0{\small\t}^^A
%     \ifdim\wd0>\linewidth
%       \ifdim\wd0>\lw
%         \sbox0{\footnotesize\t}^^A
%         \ifdim\wd0>\linewidth
%           \ifdim\wd0>\lw
%             \sbox0{\scriptsize\t}^^A
%             \ifdim\wd0>\linewidth
%               \ifdim\wd0>\lw
%                 \sbox0{\tiny\t}^^A
%                 \ifdim\wd0>\linewidth
%                   \lwbox
%                 \else
%                   \usebox0
%                 \fi
%               \else
%                 \lwbox
%               \fi
%             \else
%               \usebox0
%             \fi
%           \else
%             \lwbox
%           \fi
%         \else
%           \usebox0
%         \fi
%       \else
%         \lwbox
%       \fi
%     \else
%       \usebox0
%     \fi
%   \else
%     \lwbox
%   \fi
% \else
%   \usebox0
% \fi
% \end{quote}
% If you have a \xfile{docstrip.cfg} that configures and enables \docstrip's
% TDS installing feature, then some files can already be in the right
% place, see the documentation of \docstrip.
%
% \subsection{Refresh file name databases}
%
% If your \TeX~distribution
% (\teTeX, \mikTeX, \dots) relies on file name databases, you must refresh
% these. For example, \teTeX\ users run \verb|texhash| or
% \verb|mktexlsr|.
%
% \subsection{Some details for the interested}
%
% \paragraph{Attached source.}
%
% The PDF documentation on CTAN also includes the
% \xfile{.dtx} source file. It can be extracted by
% AcrobatReader 6 or higher. Another option is \textsf{pdftk},
% e.g. unpack the file into the current directory:
% \begin{quote}
%   \verb|pdftk pdftexcmds.pdf unpack_files output .|
% \end{quote}
%
% \paragraph{Unpacking with \LaTeX.}
% The \xfile{.dtx} chooses its action depending on the format:
% \begin{description}
% \item[\plainTeX:] Run \docstrip\ and extract the files.
% \item[\LaTeX:] Generate the documentation.
% \end{description}
% If you insist on using \LaTeX\ for \docstrip\ (really,
% \docstrip\ does not need \LaTeX), then inform the autodetect routine
% about your intention:
% \begin{quote}
%   \verb|latex \let\install=y\input{pdftexcmds.dtx}|
% \end{quote}
% Do not forget to quote the argument according to the demands
% of your shell.
%
% \paragraph{Generating the documentation.}
% You can use both the \xfile{.dtx} or the \xfile{.drv} to generate
% the documentation. The process can be configured by the
% configuration file \xfile{ltxdoc.cfg}. For instance, put this
% line into this file, if you want to have A4 as paper format:
% \begin{quote}
%   \verb|\PassOptionsToClass{a4paper}{article}|
% \end{quote}
% An example follows how to generate the
% documentation with pdf\LaTeX:
% \begin{quote}
%\begin{verbatim}
%pdflatex pdftexcmds.dtx
%makeindex -s gind.ist pdftexcmds.idx
%pdflatex pdftexcmds.dtx
%makeindex -s gind.ist pdftexcmds.idx
%pdflatex pdftexcmds.dtx
%\end{verbatim}
% \end{quote}
%
% \begin{History}
%   \begin{Version}{2007/11/11 v0.1}
%   \item
%     First version.
%   \end{Version}
%   \begin{Version}{2007/11/12 v0.2}
%   \item
%     Short description fixed.
%   \end{Version}
%   \begin{Version}{2007/12/12 v0.3}
%   \item
%     Organization of Lua code as module.
%   \end{Version}
% \end{History}
%
% \PrintIndex
%
% \Finale
\endinput

%        (quote the arguments according to the demands of your shell)
%
% Documentation:
%    (a) If pdftexcmds.drv is present:
%           latex pdftexcmds.drv
%    (b) Without pdftexcmds.drv:
%           latex pdftexcmds.dtx; ...
%    The class ltxdoc loads the configuration file ltxdoc.cfg
%    if available. Here you can specify further options, e.g.
%    use A4 as paper format:
%       \PassOptionsToClass{a4paper}{article}
%
%    Programm calls to get the documentation (example):
%       pdflatex pdftexcmds.dtx
%       makeindex -s gind.ist pdftexcmds.idx
%       pdflatex pdftexcmds.dtx
%       makeindex -s gind.ist pdftexcmds.idx
%       pdflatex pdftexcmds.dtx
%
% Installation:
%    TDS:tex/generic/oberdiek/pdftexcmds.sty
%    TDS:scripts/oberdiek/oberdiek.pdftexcmds.lua
%    TDS:scripts/oberdiek/pdftexcmds.lua
%    TDS:doc/latex/oberdiek/pdftexcmds.pdf
%    TDS:source/latex/oberdiek/pdftexcmds.dtx
%
%<*ignore>
\begingroup
  \def\x{LaTeX2e}%
\expandafter\endgroup
\ifcase 0\ifx\install y1\fi\expandafter
         \ifx\csname processbatchFile\endcsname\relax\else1\fi
         \ifx\fmtname\x\else 1\fi\relax
\else\csname fi\endcsname
%</ignore>
%<*install>
\input docstrip.tex
\Msg{************************************************************************}
\Msg{* Installation}
\Msg{* Package: pdftexcmds 2007/12/12 v0.3 LuaTeX support for pdfTeX utility functions (HO)}
\Msg{************************************************************************}

\keepsilent
\askforoverwritefalse

\let\MetaPrefix\relax
\preamble

This is a generated file.

Copyright (C) 2007 by Heiko Oberdiek <oberdiek@uni-freiburg.de>

This work may be distributed and/or modified under the
conditions of the LaTeX Project Public License, either
version 1.3 of this license or (at your option) any later
version. The latest version of this license is in
   http://www.latex-project.org/lppl.txt
and version 1.3 or later is part of all distributions of
LaTeX version 2005/12/01 or later.

This work has the LPPL maintenance status "maintained".

This Current Maintainer of this work is Heiko Oberdiek.

This work consists of the main source file pdftexcmds.dtx
and the derived files
   pdftexcmds.sty, pdftexcmds.pdf, pdftexcmds.ins, pdftexcmds.drv,
   oberdiek.pdftexcmds.lua, pdftexcmds.lua.

\endpreamble
\let\MetaPrefix\DoubleperCent

\generate{%
  \file{pdftexcmds.ins}{\from{pdftexcmds.dtx}{install}}%
  \file{pdftexcmds.drv}{\from{pdftexcmds.dtx}{driver}}%
  \usedir{tex/generic/oberdiek}%
  \file{pdftexcmds.sty}{\from{pdftexcmds.dtx}{package}}%
}
\def\MetaPrefix{-- }
\def\defaultpostamble{%
  \MetaPrefix^^J%
  \MetaPrefix\space End of File `\outFileName'.%
}
\def\currentpostamble{\defaultpostamble}%
\generate{%
  \usedir{scripts/oberdiek}%
  \file{oberdiek.pdftexcmds.lua}{\from{pdftexcmds.dtx}{lua}}%
  \file{pdftexcmds.lua}{\from{pdftexcmds.dtx}{lua}}%
}

\obeyspaces
\Msg{************************************************************************}
\Msg{*}
\Msg{* To finish the installation you have to move the following}
\Msg{* file into a directory searched by TeX:}
\Msg{*}
\Msg{*     pdftexcmds.sty}
\Msg{*}
\Msg{* And install the following script files:}
\Msg{*}
\Msg{*     oberdiek.pdftexcmds.lua, pdftexcmds.lua}
\Msg{*}
\Msg{* To produce the documentation run the file `pdftexcmds.drv'}
\Msg{* through LaTeX.}
\Msg{*}
\Msg{* Happy TeXing!}
\Msg{*}
\Msg{************************************************************************}

\endbatchfile
%</install>
%<*ignore>
\fi
%</ignore>
%<*driver>
\NeedsTeXFormat{LaTeX2e}
\ProvidesFile{pdftexcmds.drv}%
  [2007/12/12 v0.3 LuaTeX support for pdfTeX utility functions (HO)]%
\documentclass{ltxdoc}
\usepackage{holtxdoc}[2007/11/11]
\usepackage{paralist}
\begin{document}
  \DocInput{pdftexcmds.dtx}%
\end{document}
%</driver>
% \fi
%
% \CheckSum{552}
%
% \CharacterTable
%  {Upper-case    \A\B\C\D\E\F\G\H\I\J\K\L\M\N\O\P\Q\R\S\T\U\V\W\X\Y\Z
%   Lower-case    \a\b\c\d\e\f\g\h\i\j\k\l\m\n\o\p\q\r\s\t\u\v\w\x\y\z
%   Digits        \0\1\2\3\4\5\6\7\8\9
%   Exclamation   \!     Double quote  \"     Hash (number) \#
%   Dollar        \$     Percent       \%     Ampersand     \&
%   Acute accent  \'     Left paren    \(     Right paren   \)
%   Asterisk      \*     Plus          \+     Comma         \,
%   Minus         \-     Point         \.     Solidus       \/
%   Colon         \:     Semicolon     \;     Less than     \<
%   Equals        \=     Greater than  \>     Question mark \?
%   Commercial at \@     Left bracket  \[     Backslash     \\
%   Right bracket \]     Circumflex    \^     Underscore    \_
%   Grave accent  \`     Left brace    \{     Vertical bar  \|
%   Right brace   \}     Tilde         \~}
%
% \GetFileInfo{pdftexcmds.drv}
%
% \title{The \xpackage{pdftexcmds} package}
% \date{2007/12/12 v0.3}
% \author{Heiko Oberdiek\\\xemail{oberdiek@uni-freiburg.de}}
%
% \maketitle
%
% \begin{abstract}
% \LuaTeX\ provides most of the commands of \pdfTeX\ 1.40. However
% a number of utility functions are removed. This package tries to fill
% the gap and implements some of the missing primitive using Lua.
% \end{abstract}
%
% \tableofcontents
%
% \section{Documentation}
%
% Some primitives of \pdfTeX\ are not defined by \LuaTeX.
% This package implements macro based solutions using Lua code
% for the following missing \pdfTeX\ primitives;
% \begin{compactitem}
% \item \cs{pdfstrcmp}
% \item \cs{pdfunescapehex}
% \item \cs{pdfescapehex}
% \item \cs{pdfescapename}
% \item \cs{pdfescapestring}
% \item \cs{pdffilesize}
% \item \cs{pdffilemoddate}
% \item \cs{pdffiledump}
% \item \cs{pdfmdfivesum}
% \item |\immediate\write18|
% \end{compactitem}
% The original names of the primitives cannot be used:
% \begin{itemize}
% \item
% The syntax for their arguments cannot easily
% simulated by macros. The primitives using key words
% such as |file| (\cs{pdfmdfivesum}) or |offset| and |length|
% (\cs{pdffiledump}) and uses \meta{general text} for the other
% arguments. Using token registers assignments, \meta{general text} could
% be catched. However, the simulated primitives are expandable
% and register assignments would destroy this important property.
% (\meta{general text} allows something like |\expandafter\bgroup ...}|.)
% \item
% The original primitives can be expanded using one expansion step.
% The new macros need two expansion steps because of the additional
% macro expansion. Example:
% \begin{quote}
%   |\expandafter\foo\pdffilemoddate{file}|\\
%   vs.
%   |\expandafter\expandafter\expandafter\foo\pdf@filemoddate{file}|.
% \end{quote}
% \end{itemize}
%
% \LuaTeX\ isn't stable yet and thus the status of this package is
% \emph{experimental}. Feedback is welcome.
%
% \subsection{General principles}
%
% \begin{description}
% \item[Naming convention:]
%   Usually this package defines a macro |\pdf@|\meta{cmd} if
%   \pdfTeX\ provides |\pdf|\meta{cmd}.
% \item[Arguments:] The order of arguments in |\pdf@|\meta{cmd}
%   is the same as for the corresponding primitive of \pdfTeX.
%   The arguments are ordinary undelimited \TeX\ arguments,
%   no \meta{general text} and without additional keywords.
% \item[Expandibility:]
%   The macro |\pdf@|\meta{cmd} is expandable if the
%   corresponding \pdfTeX\ primitive has this property.
%   Exact two expansion steps are necessary (first is the macro
%   expansion).
% \item[Without \LuaTeX:]
%   The macros |\pdf@|\meta{cmd} are mapped to the commands
%   of \pdfTeX\ if they are available. Otherwise they are undefined.
% \end{description}
%
% \subsection{Macros}
%
% \begin{declcs}{pdf@strcmp} \M{stringA} \M{stringB}
% \end{declcs}
% Same as |\pdfstrcmp{|\meta{stringA}|}{|\meta{stringB}|}|.
%
% \begin{declcs}{pdf@unescapehex} \M{string}
% \end{declcs}
% Same as |\pdfunescapehex{|\meta{string}|}|.
% The argument is a byte string given in hexadecimal notation.
% The result are character tokens from 0 until 255 with
% catcode 12 and the space with catcode 10.
%
% \begin{declcs}{pdf@escapehex} \M{string}\\
%   \cs{pdf@escapestring} \M{string}\\
%   \cs{pdf@escapename} \M{string}
% \end{declcs}
% Same as the primitives of \pdfTeX. However \pdfTeX\ does not
% know about characters with codes 256 and larger. Thus the
% string is treated as byte string, characters with more than
% eight bits are ignored.
%
% \begin{declcs}{pdf@filesize} \M{filename}
% \end{declcs}
% Same as |\pdffilesize{|\meta{filename}|}|.
%
% \begin{declcs}{pdf@filemoddate} \M{filename}
% \end{declcs}
% Same as |\pdffilemoddate{|\meta{filename}|}|.
%
% \begin{declcs}{pdf@filedump} \M{offset} \M{length} \M{filename}
% \end{declcs}
% Same as |\pdffiledump offset| \meta{offset} |length| \meta{length}
% |{|\meta{filename}|}|. Both \meta{offset} and \meta{length} must
% not be empty, but must be a valid \TeX\ number.
%
% \begin{declcs}{pdf@mdfivesum} \M{string}
% \end{declcs}
% Same as |\pdfmdfivesum{|\meta{string}|}|. Keyword |file| is supported
% by macro \cs{pdf@filemdfivesum}.
%
% \begin{declcs}{pdf@filemdfivesum} \M{filename}
% \end{declcs}
% Same as |\pdfmdfivesum file{|\meta{filename}|}|.
%
% \begin{declcs}{pdf@shellescape}
% \end{declcs}
% Same as |\pdfshellescape|. It expands to |1| if external
% commands can be executed and |0| otherwise. In \pdfTeX\ external
% commands must be enabled first by command line option or
% configuration option. In \LuaTeX\ option |--safer| disables
% the execution of external commands.
%
% \begin{declcs}{pdf@system} \M{cmdline}
% \end{declcs}
% It is a wrapper for |\immediate\write18| in \pdfTeX\ or
% |os.execute| in \LuaTeX.
%
% In theory |os.execute|
% returns a status number. But its meaning is quite
% undefined. Are there some reliable properties?
% Does it make sense to provide an user interface to
% this status exit code?
%
% \subsubsection{Experimental}
%
% \begin{declcs}{pdf@unescapehexnative} \M{string}\\
%   \cs{pdf@escapehexnative} \M{string}\\
%   \cs{pdf@escapenamenative} \M{string}\\
%   \cs{pdf@mdfivesumnative} \M{string}
% \end{declcs}
% The variants without |native| in the macro name are supposed to
% be compatible with \pdfTeX. However characters with more than
% eight bits are not supported and are ignored. If \LuaTeX\ is
% running, then its UTF-8 coded strings are used. Thus the full
% unicode character range is supported. However the result
% differs from \pdfTeX\ for characters with eight or more bits.
%
% \begin{declcs}{pdf@pipe} \M{cmdline}
% \end{declcs}
% It calls \meta{cmdline} and returns the output of the external
% program in the usual manner as byte string (catcode 12, space with
% catcode 10). The Lua documentation says, that the used |io.popen|
% may not be available on all platforms. Then macro \cs{pdf@pipe}
% is undefined.
%
% \StopEventually{
% }
%
% \section{Implementation}
%
%    \begin{macrocode}
%<*package>
%    \end{macrocode}
%
% \subsection{Reload check and package identification}
%    Reload check, especially if the package is not used with \LaTeX.
%    \begin{macrocode}
\begingroup
  \catcode44 12 % ,
  \catcode45 12 % -
  \catcode46 12 % .
  \catcode58 12 % :
  \catcode64 11 % @
  \expandafter\let\expandafter\x\csname ver@pdftexcmds.sty\endcsname
  \ifcase 0%
    \ifx\x\relax % plain
    \else
      \ifx\x\empty % LaTeX
      \else
        1%
      \fi
    \fi
  \else
    \catcode35 6 % #
    \catcode123 1 % {
    \catcode125 2 % }
    \expandafter\ifx\csname PackageInfo\endcsname\relax
      \def\x#1#2{%
        \immediate\write-1{Package #1 Info: #2.}%
      }%
    \else
      \def\x#1#2{\PackageInfo{#1}{#2, stopped}}%
    \fi
    \x{pdftexcmds}{The package is already loaded}%
    \endgroup
    \expandafter\endinput
  \fi
\endgroup
%    \end{macrocode}
%    Package identification:
%    \begin{macrocode}
\begingroup
  \catcode35 6 % #
  \catcode40 12 % (
  \catcode41 12 % )
  \catcode44 12 % ,
  \catcode45 12 % -
  \catcode46 12 % .
  \catcode47 12 % /
  \catcode58 12 % :
  \catcode64 11 % @
  \catcode123 1 % {
  \catcode125 2 % }
  \expandafter\ifx\csname ProvidesPackage\endcsname\relax
    \def\x#1#2#3[#4]{\endgroup
      \immediate\write-1{Package: #3 #4}%
      \xdef#1{#4}%
    }%
  \else
    \def\x#1#2[#3]{\endgroup
      #2[{#3}]%
      \ifx#1\relax
        \xdef#1{#3}%
      \fi
    }%
  \fi
\expandafter\x\csname ver@pdftexcmds.sty\endcsname
\ProvidesPackage{pdftexcmds}%
  [2007/12/12 v0.3 LuaTeX support for pdfTeX utility functions (HO)]
%    \end{macrocode}
%
% \subsection{Catcodes}
%
%    \begin{macrocode}
\begingroup
  \catcode123 1 % {
  \catcode125 2 % }
  \def\x{\endgroup
    \expandafter\edef\csname pdftexcmds@AtEnd\endcsname{%
      \catcode35 \the\catcode35\relax
      \catcode64 \the\catcode64\relax
      \catcode123 \the\catcode123\relax
      \catcode125 \the\catcode125\relax
    }%
  }%
\x
\catcode35 6 % #
\catcode64 11 % @
\catcode123 1 % {
\catcode125 2 % }
\def\TMP@EnsureCode#1#2{%
  \edef\pdftexcmds@AtEnd{%
    \pdftexcmds@AtEnd
    \catcode#1 \the\catcode#1\relax
  }%
  \catcode#1 #2\relax
}
\TMP@EnsureCode{10}{12}% ^^J
\TMP@EnsureCode{33}{12}% !
\TMP@EnsureCode{34}{12}% "
\TMP@EnsureCode{39}{12}% '
\TMP@EnsureCode{40}{12}% (
\TMP@EnsureCode{41}{12}% )
\TMP@EnsureCode{42}{12}% *
\TMP@EnsureCode{43}{12}% +
\TMP@EnsureCode{44}{12}% ,
\TMP@EnsureCode{45}{12}% -
\TMP@EnsureCode{46}{12}% .
\TMP@EnsureCode{47}{12}% /
\TMP@EnsureCode{58}{12}% :
\TMP@EnsureCode{60}{12}% <
\TMP@EnsureCode{61}{12}% =
\TMP@EnsureCode{62}{12}% >
\TMP@EnsureCode{94}{7}% ^ (superscript)
\TMP@EnsureCode{95}{12}% _ (other)
\TMP@EnsureCode{126}{12}% ~ (other)
%    \end{macrocode}
%
% \subsection{Load package \xpackage{infwarerr}}
%
%    \begin{macrocode}
\begingroup\expandafter\expandafter\expandafter\endgroup
\expandafter\ifx\csname RequirePackage\endcsname\relax
  \input infwarerr.sty\relax
\else
  \RequirePackage{infwarerr}[2007/09/09]%
\fi
%    \end{macrocode}
%
% \subsection{Without \LuaTeX}
%
%    \begin{macrocode}
\begingroup\expandafter\expandafter\expandafter\endgroup
\expandafter\ifx\csname directlua\endcsname\relax
  \@PackageInfo{pdftexcmds}{LuaTeX not detected}%
  \def\pdftexcmds@nopdftex{%
    \@PackageInfoNoLine{pdftexcmds}{pdfTeX >= 1.30 not detected}%
    \let\pdftexcmds@nopdftex\relax
  }%
  \def\pdftexcmds@temp#1{%
    \begingroup\expandafter\expandafter\expandafter\endgroup
    \expandafter\ifx\csname pdf#1\endcsname\relax
      \pdftexcmds@nopdftex
    \else
      \expandafter\def\csname pdf@#1\expandafter\endcsname
      \expandafter##\expandafter{%
        \csname pdf#1\endcsname
      }%
    \fi
  }%
  \pdftexcmds@temp{strcmp}%
  \pdftexcmds@temp{escapehex}%
  \let\pdf@escapehexnative\pdf@escapehex
  \pdftexcmds@temp{unescapehex}%
  \let\pdf@unescapehexnative\pdf@unescapehex
  \pdftexcmds@temp{escapestring}%
  \pdftexcmds@temp{escapename}%
  \pdftexcmds@temp{filesize}%
  \pdftexcmds@temp{filemoddate}%
  \begingroup\expandafter\expandafter\expandafter\endgroup
  \expandafter\ifx\csname pdfshellescape\endcsname\relax
    \pdftexcmds@nopdftex
  \else
    \def\pdf@shellescape{%
      \pdfshellescape
    }%
  \fi
  \begingroup\expandafter\expandafter\expandafter\endgroup
  \expandafter\ifx\csname pdffiledump\endcsname\relax
    \pdftexcmds@nopdftex
  \else
    \def\pdf@filedump#1#2#3{%
      \pdffiledump offset#1 length#2{#3}%
    }%
  \fi
  \begingroup\expandafter\expandafter\expandafter\endgroup
  \expandafter\ifx\csname pdfmdfivesum\endcsname\relax
    \pdftexcmds@nopdftex
  \else
    \def\pdf@mdfivesum#{\pdfmdfivesum}%
    \let\pdf@mdfivesumnative\pdf@mdfivesum
    \def\pdf@filemdfivesum#{\pdfmdfivesum file}%
  \fi
  \def\pdf@system#{%
    \immediate\write18%
  }%
  \pdftexcmds@AtEnd
  \expandafter\endinput
\fi
%    \end{macrocode}
%
% \subsection{Load module}
%
%    \begin{macrocode}
\begingroup\expandafter\expandafter\expandafter\endgroup
\expandafter\ifx\csname RequirePackage\endcsname\relax
  \input luatex-loader.sty\relax
\else
  \RequirePackage{luatex-loader}[2007/12/12]%
\fi
\directlua0{%
  require("oberdiek.pdftexcmds")%
}
%    \end{macrocode}
%
% \subsection{Lua functions}
%
%    \begin{macro}{\pdftexcmds@toks}
%    \begin{macrocode}
\begingroup\expandafter\expandafter\expandafter\endgroup
\expandafter\ifx\csname newtoks\endcsname\relax
  \toksdef\pdftexcmds@toks=0 %
\else
  \csname newtoks\endcsname\pdftexcmds@toks
\fi
%    \end{macrocode}
%    \end{macro}
%
%    \begin{macro}{\pdf@strcmp}
%    \begin{macrocode}
\long\def\pdf@strcmp#1#2{%
  \directlua0{%
    oberdiek.pdftexcmds.strcmp("\luaescapestring{#1}",%
        "\luaescapestring{#2}")%
  }%
}%
%    \end{macrocode}
%    \end{macro}
%    \begin{macro}{\pdf@escapehex}
%    \begin{macrocode}
\long\def\pdf@escapehex#1{%
  \directlua0{%
    oberdiek.pdftexcmds.escapehex("\luaescapestring{#1}", "byte")%
  }%
}%
%    \end{macrocode}
%    \end{macro}
%    \begin{macro}{\pdf@escapehexnative}
%    \begin{macrocode}
\long\def\pdf@escapehexnative#1{%
  \directlua0{%
    oberdiek.pdftexcmds.escapehex("\luaescapestring{#1}")%
  }%
}%
%    \end{macrocode}
%    \end{macro}
%    \begin{macro}{\pdf@unescapehex}
%    \begin{macrocode}
\def\pdf@unescapehex#1{%
  \the\expandafter\pdftexcmds@toks
  \directlua0{%
    oberdiek.pdftexcmds.toks="pdftexcmds@toks"%
    oberdiek.pdftexcmds.unescapehex("\luaescapestring{#1}", "byte")%
  }%
}%
%    \end{macrocode}
%    \end{macro}
%    \begin{macro}{\pdf@unescapehexnative}
%    \begin{macrocode}
\def\pdf@unescapehexnative#1{%
  \the\expandafter\pdftexcmds@toks
  \directlua0{%
    oberdiek.pdftexcmds.toks="pdftexcmds@toks"%
    oberdiek.pdftexcmds.unescapehex("\luaescapestring{#1}")%
  }%
}%
%    \end{macrocode}
%    \end{macro}
%    \begin{macro}{\pdf@escapestring}
%    \begin{macrocode}
\long\def\pdf@escapestring#1{%
  \directlua0{%
    oberdiek.pdftexcmds.escapestring("\luaescapestring{#1}", "byte")%
  }%
}
%    \end{macrocode}
%    \end{macro}
%    \begin{macro}{\pdf@escapename}
%    \begin{macrocode}
\long\def\pdf@escapename#1{%
  \directlua0{%
    oberdiek.pdftexcmds.escapename("\luaescapestring{#1}", "byte")%
  }%
}
%    \end{macrocode}
%    \end{macro}
%    \begin{macro}{\pdf@escapenamenative}
%    \begin{macrocode}
\long\def\pdf@escapenamenative#1{%
  \directlua0{%
    oberdiek.pdftexcmds.escapename("\luaescapestring{#1}")%
  }%
}
%    \end{macrocode}
%    \end{macro}
%    \begin{macro}{\pdf@filesize}
%    \begin{macrocode}
\def\pdf@filesize#1{%
  \directlua0{%
    oberdiek.pdftexcmds.filesize("\luaescapestring{#1}")%
  }%
}
%    \end{macrocode}
%    \end{macro}
%    \begin{macro}{\pdf@filemoddate}
%    \begin{macrocode}
\def\pdf@filemoddate#1{%
  \directlua0{%
    oberdiek.pdftexcmds.filemoddate("\luaescapestring{#1}")%
  }%
}
%    \end{macrocode}
%    \end{macro}
%    \begin{macro}{\pdf@filedump}
%    \begin{macrocode}
\def\pdf@filedump#1#2#3{%
  \directlua0{%
    oberdiek.pdftexcmds.filedump("\luaescapestring{\number#1}",%
        "\luaescapestring{\number#2}",%
        "\luaescapestring{#3}")%
  }%
}%
%    \end{macrocode}
%    \end{macro}
%    \begin{macro}{\pdf@mdfivesum}
%    \begin{macrocode}
\long\def\pdf@mdfivesum#1{%
  \directlua0{%
    oberdiek.pdftexcmds.mdfivesum("\luaescapestring{#1}", "byte")%
  }%
}%
%    \end{macrocode}
%    \end{macro}
%    \begin{macro}{\pdf@mdfivesumnative}
%    \begin{macrocode}
\long\def\pdf@mdfivesumnative#1{%
  \directlua0{%
    oberdiek.pdftexcmds.mdfivesum("\luaescapestring{#1}")%
  }%
}%
%    \end{macrocode}
%    \end{macro}
%    \begin{macro}{\pdf@filemdfivesum}
%    \begin{macrocode}
\def\pdf@filemdfivesum#1{%
  \directlua0{%
    oberdiek.pdftexcmds.filemdfivesum("\luaescapestring{#1}")%
  }%
}%
%    \end{macrocode}
%    \end{macro}
%
%    \begin{macro}{\pdf@shellescape}
%    \begin{macrocode}
\def\pdf@shellescape{%
  \directlua0{%
    oberdiek.pdftexcmds.shellescape()%
  }%
}
%    \end{macrocode}
%    \end{macro}
%
%    \begin{macro}{\pdf@system}
%    \begin{macrocode}
\def\pdf@system#1{%
  \directlua0{%
    oberdiek.pdftexcmds.system("\luaescapestring{#1}")%
  }%
}
%    \end{macrocode}
%    \end{macro}
%
%    \begin{macro}{\pdf@lastsystemstatus}
%    \begin{macrocode}
\def\pdf@lastsystemstatus{%
  \directlua0{%
    oberdiek.pdftexcmds.lastsystemstatus()%
  }%
}
%    \end{macrocode}
%    \end{macro}
%    \begin{macro}{\pdf@lastsystemexit}
%    \begin{macrocode}
\def\pdf@lastsystemexit{%
  \directlua0{%
    oberdiek.pdftexcmds.lastsystemexit()%
  }%
}
%    \end{macrocode}
%    \end{macro}
%
%    \begin{macro}{\pdf@pipe}
%    Check availability of |io.popen| first.
%    \begin{macrocode}
\ifnum0%
    \directlua0{%
      if io.popen then %
        tex.write("1")%
      end%
    }%
    =1 %
  \def\pdf@pipe#1{%
    \the\expandafter\pdftexcmds@toks
    \directlua0{%
      oberdiek.pdftexcmds.toks="pdftexcmds@toks"%
      oberdiek.pdftexcmds.pipe("\luaescapestring{#1}")%
    }%
  }%
\fi
%    \end{macrocode}
%    \end{macro}
%
%    \begin{macrocode}
\pdftexcmds@AtEnd
%</package>
%    \end{macrocode}
%
% \subsection{Lua module}
%
%    \begin{macrocode}
%<*lua>
%    \end{macrocode}
%
%    \begin{macrocode}
module("oberdiek.pdftexcmds", package.seeall)
local systemexitstatus
function strcmp(A, B)
  if A == B then
    tex.write("0")
  elseif A < B then
    tex.write("-1")
  else
    tex.write("1")
  end
end
local function utf8_to_byte(str)
  local i = 0
  local n = string.len(str)
  local t = {}
  while i < n do
    i = i + 1
    local a = string.byte(str, i)
    if a < 128 then
      table.insert(t, string.char(a))
    else
      if a >= 192 and i < n then
        i = i + 1
        local b = string.byte(str, i)
        if b < 128 or b >= 192 then
          i = i - 1
        elseif a == 194 then
          table.insert(t, string.char(b))
        elseif a == 195 then
          table.insert(t, string.char(b + 64))
        end
      end
    end
  end
  return table.concat(t)
end
function escapehex(str, mode)
  if mode == "byte" then
    str = utf8_to_byte(str)
  end
  tex.write((string.gsub(str, ".",
    function (ch)
      return string.format("%02X", string.byte(ch))
    end
  )))
end
%    \end{macrocode}
%    See procedure |unescapehex| in file \xfile{utils.c} of \pdfTeX.
%    Caution: |tex.write| ignores leading spaces.
%    \begin{macrocode}
function unescapehex(str, mode)
  local a = 0
  local first = true
  local result = {}
  for i = 1, string.len(str), 1 do
    local ch = string.byte(str, i)
    if ch >= 48 and ch <= 57 then
      ch = ch - 48
    elseif ch >= 65 and ch <= 70 then
      ch = ch - 55
    elseif ch >= 97 and ch <= 102 then
      ch = ch - 87
    else
      ch = nil
    end
    if ch then
      if first then
        a = ch * 16
        first = false
      else
        table.insert(result, a + ch)
        first = true
      end
    end
  end
  if not first then
    table.insert(result, a)
  end
  if mode == "byte" then
    local utf8 = {}
    for i, a in ipairs(result) do
      if a < 128 then
        table.insert(utf8, a)
      else
        if a < 192 then
          table.insert(utf8, 194)
          a = a - 128
        else
          table.insert(utf8, 195)
          a = a - 192
        end
        table.insert(utf8, a + 128)
      end
    end
    result = utf8
  end
  tex.settoks(toks, string.char(unpack(result)))
end
%    \end{macrocode}
%    See procedure |escapestring| in file \xfile{utils.c} of \pdfTeX.
%    \begin{macrocode}
function escapestring(str, mode)
  if mode == "byte" then
    str = utf8_to_byte(str)
  end
  tex.write((string.gsub(str, ".",
    function (ch)
      local b = string.byte(ch)
      if b < 33 or b > 126 then
        return string.format("\\%.3o", b)
      end
      if b == 40 or b == 41 or b == 92 then
        return "\\" .. ch
      end
%    \end{macrocode}
%    Lua 5.1 returns the match in case of return value |nil|.
%    \begin{macrocode}
      return nil
    end
  )))
end
%    \end{macrocode}
%    See procedure |escapename| in file \xfile{utils.c} of \pdfTeX.
%    \begin{macrocode}
function escapename(str, mode)
  if mode == "byte" then
    str = utf8_to_byte(str)
  end
  tex.write((string.gsub(str, ".",
    function (ch)
      local b = string.byte(ch)
      if b == 0 then
%    \end{macrocode}
%    In Lua 5.0 |nil| could be used for the empty string,
%    But |nil| returns the match in Lua 5.1, thus we use
%    the empty string explicitly.
%    \begin{macrocode}
        return ""
      end
      if b <= 32 or b >= 127
          or b == 35 or b == 37 or b == 40 or b == 41
          or b == 47 or b == 60 or b == 62 or b == 91
          or b == 93 or b == 123 or b == 125 then
        return string.format("#%.2X", b)
      else
%    \end{macrocode}
%    Lua 5.1 returns the match in case of return value |nil|.
%    \begin{macrocode}
        return nil
      end
    end
  )))
end
function filesize(filename)
  local foundfile = kpse.find_file(filename, "tex", true)
  if foundfile then
    local size = lfs.attributes(foundfile, "size")
    if size then
      tex.write(size)
    end
  end
end
%    \end{macrocode}
%    See procedure |makepdftime| in file \xfile{utils.c} of \pdfTeX.
%    \begin{macrocode}
function filemoddate(filename)
  local foundfile = kpse.find_file(filename, "tex", true)
  if foundfile then
    local date = lfs.attributes(foundfile, "modification")
    if date then
      local d = os.date("*t", date)
      if d.sec >= 60 then
        d.sec = 59
      end
      local u = os.date("!*t", date)
      local off = 60 * (d.hour - u.hour) + d.min - u.min
      if d.year ~= u.year then
        if d.year > u.year then
          off = off + 1440
        else
          off = off - 1440
        end
      elseif d.yday ~= u.yday then
        if d.yday > u.yday then
          off = off + 1440
        else
          off = off - 1440
        end
      end
      local timezone
      if off == 0 then
        timezone = "Z"
      else
        local hours = math.floor(off / 60)
        local mins = math.abs(off - hours * 60)
        timezone = string.format("%+03d'%02d'", hours, mins)
      end
      tex.write(string.format("D:%04d%02d%02d%02d%02d%02d%s",
          d.year, d.month, d.day, d.hour, d.min, d.sec, timezone))
    end
  end
end
function filedump(offset, length, filename)
  length = tonumber(length)
  if length and length > 0 then
    local foundfile = kpse.find_file(filename, "tex", true)
    if foundfile then
      offset = tonumber(offset)
      if not offset then
        offset = 0
      end
      local filehandle = io.open(foundfile, "r")
      if filehandle then
        if offset > 0 then
          filehandle:seek("set", offset)
        end
        local dump = filehandle:read(length)
        escapehex(dump)
      end
    end
  end
end
function mdfivesum(str, mode)
  if mode == "byte" then
    str = utf8_to_byte(str)
  end
  escapehex(md5.sum(str))
end
function filemdfivesum(filename)
  local foundfile = kpse.find_file(filename, "tex", true)
  if foundfile then
    local filehandle = io.open(foundfile, "r")
    if filehandle then
      local contents = filehandle:read("*a")
      escapehex(md5.sum(contents))
    end
  end
end
function shellescape()
  if os.execute then
    tex.write("1")
  else
    tex.write("0")
  end
end
function system(cmdline)
  systemexitstatus = nil
  texio.write_nl("log", "system(" .. cmdline .. ") ")
  if os.execute then
    texio.write("log", "executed.")
    systemexitstatus = os.execute(cmdline)
  else
    texio.write("log", "disabled.")
  end
end
function lastsystemstatus()
  local result = tonumber(systemexitstatus)
  if result then
    local x = math.floor(result / 256)
    tex.write(result - 256 * math.floor(result / 256))
  end
end
function lastsystemexit()
  local result = tonumber(systemexitstatus)
  if result then
    tex.write(math.floor(result / 256))
  end
end
function pipe(cmdline)
  local result
  systemexitstatus = nil
  texio.write_nl("log", "pipe(" .. cmdline ..") ")
  if io.popen then
    texio.write("log", "executed.")
    local handle = io.popen(cmdline, "r")
    if handle then
      result = handle:read("*a")
      handle:close()
    end
  else
    texio.write("log", "disabled.")
  end
  if result then
    tex.settoks(toks, result)
  else
    tex.settoks(toks, "")
  end
end
%    \end{macrocode}
%    \begin{macrocode}
%</lua>
%    \end{macrocode}
%
% \section{Test}
%
% \subsection{Catcode checks for loading}
%
%    \begin{macrocode}
%<*test1>
%    \end{macrocode}
%    \begin{macrocode}
\catcode`\{=1 %
\catcode`\}=2 %
\catcode`\#=6 %
\catcode`\@=11 %
\expandafter\ifx\csname count@\endcsname\relax
  \countdef\count@=255 %
\fi
\expandafter\ifx\csname @gobble\endcsname\relax
  \long\def\@gobble#1{}%
\fi
\expandafter\ifx\csname @firstofone\endcsname\relax
  \long\def\@firstofone#1{#1}%
\fi
\expandafter\ifx\csname loop\endcsname\relax
  \expandafter\@firstofone
\else
  \expandafter\@gobble
\fi
{%
  \def\loop#1\repeat{%
    \def\body{#1}%
    \iterate
  }%
  \def\iterate{%
    \body
      \let\next\iterate
    \else
      \let\next\relax
    \fi
    \next
  }%
  \let\repeat=\fi
}%
\def\RestoreCatcodes{}
\count@=0 %
\loop
  \edef\RestoreCatcodes{%
    \RestoreCatcodes
    \catcode\the\count@=\the\catcode\count@\relax
  }%
\ifnum\count@<255 %
  \advance\count@ 1 %
\repeat

\def\RangeCatcodeInvalid#1#2{%
  \count@=#1\relax
  \loop
    \catcode\count@=15 %
  \ifnum\count@<#2\relax
    \advance\count@ 1 %
  \repeat
}
\expandafter\ifx\csname LoadCommand\endcsname\relax
  \def\LoadCommand{\input pdftexcmds.sty\relax}%
\fi
\def\Test{%
  \RangeCatcodeInvalid{0}{47}%
  \RangeCatcodeInvalid{58}{64}%
  \RangeCatcodeInvalid{91}{96}%
  \RangeCatcodeInvalid{123}{255}%
  \catcode`\@=12 %
  \catcode`\\=0 %
  \catcode`\{=1 %
  \catcode`\}=2 %
  \catcode`\#=6 %
  \catcode`\[=12 %
  \catcode`\]=12 %
  \catcode`\%=14 %
  \catcode`\ =10 %
  \catcode13=5 %
  \LoadCommand
  \RestoreCatcodes
}
\Test
\csname @@end\endcsname
\end
%    \end{macrocode}
%    \begin{macrocode}
%</test1>
%    \end{macrocode}
%
% \section{Installation}
%
% \subsection{Download}
%
% \paragraph{Package.} This package is available on
% CTAN\footnote{\url{ftp://ftp.ctan.org/tex-archive/}}:
% \begin{description}
% \item[\CTAN{macros/latex/contrib/oberdiek/pdftexcmds.dtx}] The source file.
% \item[\CTAN{macros/latex/contrib/oberdiek/pdftexcmds.pdf}] Documentation.
% \end{description}
%
%
% \paragraph{Bundle.} All the packages of the bundle `oberdiek'
% are also available in a TDS compliant ZIP archive. There
% the packages are already unpacked and the documentation files
% are generated. The files and directories obey the TDS standard.
% \begin{description}
% \item[\CTAN{macros/latex/contrib/oberdiek/oberdiek-tds.zip}]
% \end{description}
% \emph{TDS} refers to the standard ``A Directory Structure
% for \TeX\ Files'' (\CTAN{tds/tds.pdf}). Directories
% with \xfile{texmf} in their name are usually organized this way.
%
% \subsection{Bundle installation}
%
% \paragraph{Unpacking.} Unpack the \xfile{oberdiek-tds.zip} in the
% TDS tree (also known as \xfile{texmf} tree) of your choice.
% Example (linux):
% \begin{quote}
%   |unzip oberdiek-tds.zip -d ~/texmf|
% \end{quote}
%
% \paragraph{Script installation.}
% Check the directory \xfile{TDS:scripts/oberdiek/} for
% scripts that need further installation steps.
% Package \xpackage{attachfile2} comes with the Perl script
% \xfile{pdfatfi.pl} that should be installed in such a way
% that it can be called as \texttt{pdfatfi}.
% Example (linux):
% \begin{quote}
%   |chmod +x scripts/oberdiek/pdfatfi.pl|\\
%   |cp scripts/oberdiek/pdfatfi.pl /usr/local/bin/|
% \end{quote}
%
% \subsection{Package installation}
%
% \paragraph{Unpacking.} The \xfile{.dtx} file is a self-extracting
% \docstrip\ archive. The files are extracted by running the
% \xfile{.dtx} through \plainTeX:
% \begin{quote}
%   \verb|tex pdftexcmds.dtx|
% \end{quote}
%
% \paragraph{TDS.} Now the different files must be moved into
% the different directories in your installation TDS tree
% (also known as \xfile{texmf} tree):
% \begin{quote}
% \def\t{^^A
% \begin{tabular}{@{}>{\ttfamily}l@{ $\rightarrow$ }>{\ttfamily}l@{}}
%   pdftexcmds.sty & tex/generic/oberdiek/pdftexcmds.sty\\
%   oberdiek.pdftexcmds.lua & scripts/oberdiek/oberdiek.pdftexcmds.lua\\
%   pdftexcmds.lua & scripts/oberdiek/pdftexcmds.lua\\
%   pdftexcmds.pdf & doc/latex/oberdiek/pdftexcmds.pdf\\
%   pdftexcmds.dtx & source/latex/oberdiek/pdftexcmds.dtx\\
% \end{tabular}^^A
% }^^A
% \sbox0{\t}^^A
% \ifdim\wd0>\linewidth
%   \begingroup
%     \advance\linewidth by\leftmargin
%     \advance\linewidth by\rightmargin
%   \edef\x{\endgroup
%     \def\noexpand\lw{\the\linewidth}^^A
%   }\x
%   \def\lwbox{^^A
%     \leavevmode
%     \hbox to \linewidth{^^A
%       \kern-\leftmargin\relax
%       \hss
%       \usebox0
%       \hss
%       \kern-\rightmargin\relax
%     }^^A
%   }^^A
%   \ifdim\wd0>\lw
%     \sbox0{\small\t}^^A
%     \ifdim\wd0>\linewidth
%       \ifdim\wd0>\lw
%         \sbox0{\footnotesize\t}^^A
%         \ifdim\wd0>\linewidth
%           \ifdim\wd0>\lw
%             \sbox0{\scriptsize\t}^^A
%             \ifdim\wd0>\linewidth
%               \ifdim\wd0>\lw
%                 \sbox0{\tiny\t}^^A
%                 \ifdim\wd0>\linewidth
%                   \lwbox
%                 \else
%                   \usebox0
%                 \fi
%               \else
%                 \lwbox
%               \fi
%             \else
%               \usebox0
%             \fi
%           \else
%             \lwbox
%           \fi
%         \else
%           \usebox0
%         \fi
%       \else
%         \lwbox
%       \fi
%     \else
%       \usebox0
%     \fi
%   \else
%     \lwbox
%   \fi
% \else
%   \usebox0
% \fi
% \end{quote}
% If you have a \xfile{docstrip.cfg} that configures and enables \docstrip's
% TDS installing feature, then some files can already be in the right
% place, see the documentation of \docstrip.
%
% \subsection{Refresh file name databases}
%
% If your \TeX~distribution
% (\teTeX, \mikTeX, \dots) relies on file name databases, you must refresh
% these. For example, \teTeX\ users run \verb|texhash| or
% \verb|mktexlsr|.
%
% \subsection{Some details for the interested}
%
% \paragraph{Attached source.}
%
% The PDF documentation on CTAN also includes the
% \xfile{.dtx} source file. It can be extracted by
% AcrobatReader 6 or higher. Another option is \textsf{pdftk},
% e.g. unpack the file into the current directory:
% \begin{quote}
%   \verb|pdftk pdftexcmds.pdf unpack_files output .|
% \end{quote}
%
% \paragraph{Unpacking with \LaTeX.}
% The \xfile{.dtx} chooses its action depending on the format:
% \begin{description}
% \item[\plainTeX:] Run \docstrip\ and extract the files.
% \item[\LaTeX:] Generate the documentation.
% \end{description}
% If you insist on using \LaTeX\ for \docstrip\ (really,
% \docstrip\ does not need \LaTeX), then inform the autodetect routine
% about your intention:
% \begin{quote}
%   \verb|latex \let\install=y% \iffalse meta-comment
%
% Copyright (C) 2007 by Heiko Oberdiek <oberdiek@uni-freiburg.de>
%
% This work may be distributed and/or modified under the
% conditions of the LaTeX Project Public License, either
% version 1.3 of this license or (at your option) any later
% version. The latest version of this license is in
%    http://www.latex-project.org/lppl.txt
% and version 1.3 or later is part of all distributions of
% LaTeX version 2005/12/01 or later.
%
% This work has the LPPL maintenance status "maintained".
%
% This Current Maintainer of this work is Heiko Oberdiek.
%
% This work consists of the main source file pdftexcmds.dtx
% and the derived files
%    pdftexcmds.sty, pdftexcmds.pdf, pdftexcmds.ins, pdftexcmds.drv,
%    oberdiek.pdftexcmds.lua, pdftexcmds.lua.
%
% Distribution:
%    CTAN:macros/latex/contrib/oberdiek/pdftexcmds.dtx
%    CTAN:macros/latex/contrib/oberdiek/pdftexcmds.pdf
%
% Unpacking:
%    (a) If pdftexcmds.ins is present:
%           tex pdftexcmds.ins
%    (b) Without pdftexcmds.ins:
%           tex pdftexcmds.dtx
%    (c) If you insist on using LaTeX
%           latex \let\install=y\input{pdftexcmds.dtx}
%        (quote the arguments according to the demands of your shell)
%
% Documentation:
%    (a) If pdftexcmds.drv is present:
%           latex pdftexcmds.drv
%    (b) Without pdftexcmds.drv:
%           latex pdftexcmds.dtx; ...
%    The class ltxdoc loads the configuration file ltxdoc.cfg
%    if available. Here you can specify further options, e.g.
%    use A4 as paper format:
%       \PassOptionsToClass{a4paper}{article}
%
%    Programm calls to get the documentation (example):
%       pdflatex pdftexcmds.dtx
%       makeindex -s gind.ist pdftexcmds.idx
%       pdflatex pdftexcmds.dtx
%       makeindex -s gind.ist pdftexcmds.idx
%       pdflatex pdftexcmds.dtx
%
% Installation:
%    TDS:tex/generic/oberdiek/pdftexcmds.sty
%    TDS:scripts/oberdiek/oberdiek.pdftexcmds.lua
%    TDS:scripts/oberdiek/pdftexcmds.lua
%    TDS:doc/latex/oberdiek/pdftexcmds.pdf
%    TDS:source/latex/oberdiek/pdftexcmds.dtx
%
%<*ignore>
\begingroup
  \def\x{LaTeX2e}%
\expandafter\endgroup
\ifcase 0\ifx\install y1\fi\expandafter
         \ifx\csname processbatchFile\endcsname\relax\else1\fi
         \ifx\fmtname\x\else 1\fi\relax
\else\csname fi\endcsname
%</ignore>
%<*install>
\input docstrip.tex
\Msg{************************************************************************}
\Msg{* Installation}
\Msg{* Package: pdftexcmds 2007/12/12 v0.3 LuaTeX support for pdfTeX utility functions (HO)}
\Msg{************************************************************************}

\keepsilent
\askforoverwritefalse

\let\MetaPrefix\relax
\preamble

This is a generated file.

Copyright (C) 2007 by Heiko Oberdiek <oberdiek@uni-freiburg.de>

This work may be distributed and/or modified under the
conditions of the LaTeX Project Public License, either
version 1.3 of this license or (at your option) any later
version. The latest version of this license is in
   http://www.latex-project.org/lppl.txt
and version 1.3 or later is part of all distributions of
LaTeX version 2005/12/01 or later.

This work has the LPPL maintenance status "maintained".

This Current Maintainer of this work is Heiko Oberdiek.

This work consists of the main source file pdftexcmds.dtx
and the derived files
   pdftexcmds.sty, pdftexcmds.pdf, pdftexcmds.ins, pdftexcmds.drv,
   oberdiek.pdftexcmds.lua, pdftexcmds.lua.

\endpreamble
\let\MetaPrefix\DoubleperCent

\generate{%
  \file{pdftexcmds.ins}{\from{pdftexcmds.dtx}{install}}%
  \file{pdftexcmds.drv}{\from{pdftexcmds.dtx}{driver}}%
  \usedir{tex/generic/oberdiek}%
  \file{pdftexcmds.sty}{\from{pdftexcmds.dtx}{package}}%
}
\def\MetaPrefix{-- }
\def\defaultpostamble{%
  \MetaPrefix^^J%
  \MetaPrefix\space End of File `\outFileName'.%
}
\def\currentpostamble{\defaultpostamble}%
\generate{%
  \usedir{scripts/oberdiek}%
  \file{oberdiek.pdftexcmds.lua}{\from{pdftexcmds.dtx}{lua}}%
  \file{pdftexcmds.lua}{\from{pdftexcmds.dtx}{lua}}%
}

\obeyspaces
\Msg{************************************************************************}
\Msg{*}
\Msg{* To finish the installation you have to move the following}
\Msg{* file into a directory searched by TeX:}
\Msg{*}
\Msg{*     pdftexcmds.sty}
\Msg{*}
\Msg{* And install the following script files:}
\Msg{*}
\Msg{*     oberdiek.pdftexcmds.lua, pdftexcmds.lua}
\Msg{*}
\Msg{* To produce the documentation run the file `pdftexcmds.drv'}
\Msg{* through LaTeX.}
\Msg{*}
\Msg{* Happy TeXing!}
\Msg{*}
\Msg{************************************************************************}

\endbatchfile
%</install>
%<*ignore>
\fi
%</ignore>
%<*driver>
\NeedsTeXFormat{LaTeX2e}
\ProvidesFile{pdftexcmds.drv}%
  [2007/12/12 v0.3 LuaTeX support for pdfTeX utility functions (HO)]%
\documentclass{ltxdoc}
\usepackage{holtxdoc}[2007/11/11]
\usepackage{paralist}
\begin{document}
  \DocInput{pdftexcmds.dtx}%
\end{document}
%</driver>
% \fi
%
% \CheckSum{552}
%
% \CharacterTable
%  {Upper-case    \A\B\C\D\E\F\G\H\I\J\K\L\M\N\O\P\Q\R\S\T\U\V\W\X\Y\Z
%   Lower-case    \a\b\c\d\e\f\g\h\i\j\k\l\m\n\o\p\q\r\s\t\u\v\w\x\y\z
%   Digits        \0\1\2\3\4\5\6\7\8\9
%   Exclamation   \!     Double quote  \"     Hash (number) \#
%   Dollar        \$     Percent       \%     Ampersand     \&
%   Acute accent  \'     Left paren    \(     Right paren   \)
%   Asterisk      \*     Plus          \+     Comma         \,
%   Minus         \-     Point         \.     Solidus       \/
%   Colon         \:     Semicolon     \;     Less than     \<
%   Equals        \=     Greater than  \>     Question mark \?
%   Commercial at \@     Left bracket  \[     Backslash     \\
%   Right bracket \]     Circumflex    \^     Underscore    \_
%   Grave accent  \`     Left brace    \{     Vertical bar  \|
%   Right brace   \}     Tilde         \~}
%
% \GetFileInfo{pdftexcmds.drv}
%
% \title{The \xpackage{pdftexcmds} package}
% \date{2007/12/12 v0.3}
% \author{Heiko Oberdiek\\\xemail{oberdiek@uni-freiburg.de}}
%
% \maketitle
%
% \begin{abstract}
% \LuaTeX\ provides most of the commands of \pdfTeX\ 1.40. However
% a number of utility functions are removed. This package tries to fill
% the gap and implements some of the missing primitive using Lua.
% \end{abstract}
%
% \tableofcontents
%
% \section{Documentation}
%
% Some primitives of \pdfTeX\ are not defined by \LuaTeX.
% This package implements macro based solutions using Lua code
% for the following missing \pdfTeX\ primitives;
% \begin{compactitem}
% \item \cs{pdfstrcmp}
% \item \cs{pdfunescapehex}
% \item \cs{pdfescapehex}
% \item \cs{pdfescapename}
% \item \cs{pdfescapestring}
% \item \cs{pdffilesize}
% \item \cs{pdffilemoddate}
% \item \cs{pdffiledump}
% \item \cs{pdfmdfivesum}
% \item |\immediate\write18|
% \end{compactitem}
% The original names of the primitives cannot be used:
% \begin{itemize}
% \item
% The syntax for their arguments cannot easily
% simulated by macros. The primitives using key words
% such as |file| (\cs{pdfmdfivesum}) or |offset| and |length|
% (\cs{pdffiledump}) and uses \meta{general text} for the other
% arguments. Using token registers assignments, \meta{general text} could
% be catched. However, the simulated primitives are expandable
% and register assignments would destroy this important property.
% (\meta{general text} allows something like |\expandafter\bgroup ...}|.)
% \item
% The original primitives can be expanded using one expansion step.
% The new macros need two expansion steps because of the additional
% macro expansion. Example:
% \begin{quote}
%   |\expandafter\foo\pdffilemoddate{file}|\\
%   vs.
%   |\expandafter\expandafter\expandafter\foo\pdf@filemoddate{file}|.
% \end{quote}
% \end{itemize}
%
% \LuaTeX\ isn't stable yet and thus the status of this package is
% \emph{experimental}. Feedback is welcome.
%
% \subsection{General principles}
%
% \begin{description}
% \item[Naming convention:]
%   Usually this package defines a macro |\pdf@|\meta{cmd} if
%   \pdfTeX\ provides |\pdf|\meta{cmd}.
% \item[Arguments:] The order of arguments in |\pdf@|\meta{cmd}
%   is the same as for the corresponding primitive of \pdfTeX.
%   The arguments are ordinary undelimited \TeX\ arguments,
%   no \meta{general text} and without additional keywords.
% \item[Expandibility:]
%   The macro |\pdf@|\meta{cmd} is expandable if the
%   corresponding \pdfTeX\ primitive has this property.
%   Exact two expansion steps are necessary (first is the macro
%   expansion).
% \item[Without \LuaTeX:]
%   The macros |\pdf@|\meta{cmd} are mapped to the commands
%   of \pdfTeX\ if they are available. Otherwise they are undefined.
% \end{description}
%
% \subsection{Macros}
%
% \begin{declcs}{pdf@strcmp} \M{stringA} \M{stringB}
% \end{declcs}
% Same as |\pdfstrcmp{|\meta{stringA}|}{|\meta{stringB}|}|.
%
% \begin{declcs}{pdf@unescapehex} \M{string}
% \end{declcs}
% Same as |\pdfunescapehex{|\meta{string}|}|.
% The argument is a byte string given in hexadecimal notation.
% The result are character tokens from 0 until 255 with
% catcode 12 and the space with catcode 10.
%
% \begin{declcs}{pdf@escapehex} \M{string}\\
%   \cs{pdf@escapestring} \M{string}\\
%   \cs{pdf@escapename} \M{string}
% \end{declcs}
% Same as the primitives of \pdfTeX. However \pdfTeX\ does not
% know about characters with codes 256 and larger. Thus the
% string is treated as byte string, characters with more than
% eight bits are ignored.
%
% \begin{declcs}{pdf@filesize} \M{filename}
% \end{declcs}
% Same as |\pdffilesize{|\meta{filename}|}|.
%
% \begin{declcs}{pdf@filemoddate} \M{filename}
% \end{declcs}
% Same as |\pdffilemoddate{|\meta{filename}|}|.
%
% \begin{declcs}{pdf@filedump} \M{offset} \M{length} \M{filename}
% \end{declcs}
% Same as |\pdffiledump offset| \meta{offset} |length| \meta{length}
% |{|\meta{filename}|}|. Both \meta{offset} and \meta{length} must
% not be empty, but must be a valid \TeX\ number.
%
% \begin{declcs}{pdf@mdfivesum} \M{string}
% \end{declcs}
% Same as |\pdfmdfivesum{|\meta{string}|}|. Keyword |file| is supported
% by macro \cs{pdf@filemdfivesum}.
%
% \begin{declcs}{pdf@filemdfivesum} \M{filename}
% \end{declcs}
% Same as |\pdfmdfivesum file{|\meta{filename}|}|.
%
% \begin{declcs}{pdf@shellescape}
% \end{declcs}
% Same as |\pdfshellescape|. It expands to |1| if external
% commands can be executed and |0| otherwise. In \pdfTeX\ external
% commands must be enabled first by command line option or
% configuration option. In \LuaTeX\ option |--safer| disables
% the execution of external commands.
%
% \begin{declcs}{pdf@system} \M{cmdline}
% \end{declcs}
% It is a wrapper for |\immediate\write18| in \pdfTeX\ or
% |os.execute| in \LuaTeX.
%
% In theory |os.execute|
% returns a status number. But its meaning is quite
% undefined. Are there some reliable properties?
% Does it make sense to provide an user interface to
% this status exit code?
%
% \subsubsection{Experimental}
%
% \begin{declcs}{pdf@unescapehexnative} \M{string}\\
%   \cs{pdf@escapehexnative} \M{string}\\
%   \cs{pdf@escapenamenative} \M{string}\\
%   \cs{pdf@mdfivesumnative} \M{string}
% \end{declcs}
% The variants without |native| in the macro name are supposed to
% be compatible with \pdfTeX. However characters with more than
% eight bits are not supported and are ignored. If \LuaTeX\ is
% running, then its UTF-8 coded strings are used. Thus the full
% unicode character range is supported. However the result
% differs from \pdfTeX\ for characters with eight or more bits.
%
% \begin{declcs}{pdf@pipe} \M{cmdline}
% \end{declcs}
% It calls \meta{cmdline} and returns the output of the external
% program in the usual manner as byte string (catcode 12, space with
% catcode 10). The Lua documentation says, that the used |io.popen|
% may not be available on all platforms. Then macro \cs{pdf@pipe}
% is undefined.
%
% \StopEventually{
% }
%
% \section{Implementation}
%
%    \begin{macrocode}
%<*package>
%    \end{macrocode}
%
% \subsection{Reload check and package identification}
%    Reload check, especially if the package is not used with \LaTeX.
%    \begin{macrocode}
\begingroup
  \catcode44 12 % ,
  \catcode45 12 % -
  \catcode46 12 % .
  \catcode58 12 % :
  \catcode64 11 % @
  \expandafter\let\expandafter\x\csname ver@pdftexcmds.sty\endcsname
  \ifcase 0%
    \ifx\x\relax % plain
    \else
      \ifx\x\empty % LaTeX
      \else
        1%
      \fi
    \fi
  \else
    \catcode35 6 % #
    \catcode123 1 % {
    \catcode125 2 % }
    \expandafter\ifx\csname PackageInfo\endcsname\relax
      \def\x#1#2{%
        \immediate\write-1{Package #1 Info: #2.}%
      }%
    \else
      \def\x#1#2{\PackageInfo{#1}{#2, stopped}}%
    \fi
    \x{pdftexcmds}{The package is already loaded}%
    \endgroup
    \expandafter\endinput
  \fi
\endgroup
%    \end{macrocode}
%    Package identification:
%    \begin{macrocode}
\begingroup
  \catcode35 6 % #
  \catcode40 12 % (
  \catcode41 12 % )
  \catcode44 12 % ,
  \catcode45 12 % -
  \catcode46 12 % .
  \catcode47 12 % /
  \catcode58 12 % :
  \catcode64 11 % @
  \catcode123 1 % {
  \catcode125 2 % }
  \expandafter\ifx\csname ProvidesPackage\endcsname\relax
    \def\x#1#2#3[#4]{\endgroup
      \immediate\write-1{Package: #3 #4}%
      \xdef#1{#4}%
    }%
  \else
    \def\x#1#2[#3]{\endgroup
      #2[{#3}]%
      \ifx#1\relax
        \xdef#1{#3}%
      \fi
    }%
  \fi
\expandafter\x\csname ver@pdftexcmds.sty\endcsname
\ProvidesPackage{pdftexcmds}%
  [2007/12/12 v0.3 LuaTeX support for pdfTeX utility functions (HO)]
%    \end{macrocode}
%
% \subsection{Catcodes}
%
%    \begin{macrocode}
\begingroup
  \catcode123 1 % {
  \catcode125 2 % }
  \def\x{\endgroup
    \expandafter\edef\csname pdftexcmds@AtEnd\endcsname{%
      \catcode35 \the\catcode35\relax
      \catcode64 \the\catcode64\relax
      \catcode123 \the\catcode123\relax
      \catcode125 \the\catcode125\relax
    }%
  }%
\x
\catcode35 6 % #
\catcode64 11 % @
\catcode123 1 % {
\catcode125 2 % }
\def\TMP@EnsureCode#1#2{%
  \edef\pdftexcmds@AtEnd{%
    \pdftexcmds@AtEnd
    \catcode#1 \the\catcode#1\relax
  }%
  \catcode#1 #2\relax
}
\TMP@EnsureCode{10}{12}% ^^J
\TMP@EnsureCode{33}{12}% !
\TMP@EnsureCode{34}{12}% "
\TMP@EnsureCode{39}{12}% '
\TMP@EnsureCode{40}{12}% (
\TMP@EnsureCode{41}{12}% )
\TMP@EnsureCode{42}{12}% *
\TMP@EnsureCode{43}{12}% +
\TMP@EnsureCode{44}{12}% ,
\TMP@EnsureCode{45}{12}% -
\TMP@EnsureCode{46}{12}% .
\TMP@EnsureCode{47}{12}% /
\TMP@EnsureCode{58}{12}% :
\TMP@EnsureCode{60}{12}% <
\TMP@EnsureCode{61}{12}% =
\TMP@EnsureCode{62}{12}% >
\TMP@EnsureCode{94}{7}% ^ (superscript)
\TMP@EnsureCode{95}{12}% _ (other)
\TMP@EnsureCode{126}{12}% ~ (other)
%    \end{macrocode}
%
% \subsection{Load package \xpackage{infwarerr}}
%
%    \begin{macrocode}
\begingroup\expandafter\expandafter\expandafter\endgroup
\expandafter\ifx\csname RequirePackage\endcsname\relax
  \input infwarerr.sty\relax
\else
  \RequirePackage{infwarerr}[2007/09/09]%
\fi
%    \end{macrocode}
%
% \subsection{Without \LuaTeX}
%
%    \begin{macrocode}
\begingroup\expandafter\expandafter\expandafter\endgroup
\expandafter\ifx\csname directlua\endcsname\relax
  \@PackageInfo{pdftexcmds}{LuaTeX not detected}%
  \def\pdftexcmds@nopdftex{%
    \@PackageInfoNoLine{pdftexcmds}{pdfTeX >= 1.30 not detected}%
    \let\pdftexcmds@nopdftex\relax
  }%
  \def\pdftexcmds@temp#1{%
    \begingroup\expandafter\expandafter\expandafter\endgroup
    \expandafter\ifx\csname pdf#1\endcsname\relax
      \pdftexcmds@nopdftex
    \else
      \expandafter\def\csname pdf@#1\expandafter\endcsname
      \expandafter##\expandafter{%
        \csname pdf#1\endcsname
      }%
    \fi
  }%
  \pdftexcmds@temp{strcmp}%
  \pdftexcmds@temp{escapehex}%
  \let\pdf@escapehexnative\pdf@escapehex
  \pdftexcmds@temp{unescapehex}%
  \let\pdf@unescapehexnative\pdf@unescapehex
  \pdftexcmds@temp{escapestring}%
  \pdftexcmds@temp{escapename}%
  \pdftexcmds@temp{filesize}%
  \pdftexcmds@temp{filemoddate}%
  \begingroup\expandafter\expandafter\expandafter\endgroup
  \expandafter\ifx\csname pdfshellescape\endcsname\relax
    \pdftexcmds@nopdftex
  \else
    \def\pdf@shellescape{%
      \pdfshellescape
    }%
  \fi
  \begingroup\expandafter\expandafter\expandafter\endgroup
  \expandafter\ifx\csname pdffiledump\endcsname\relax
    \pdftexcmds@nopdftex
  \else
    \def\pdf@filedump#1#2#3{%
      \pdffiledump offset#1 length#2{#3}%
    }%
  \fi
  \begingroup\expandafter\expandafter\expandafter\endgroup
  \expandafter\ifx\csname pdfmdfivesum\endcsname\relax
    \pdftexcmds@nopdftex
  \else
    \def\pdf@mdfivesum#{\pdfmdfivesum}%
    \let\pdf@mdfivesumnative\pdf@mdfivesum
    \def\pdf@filemdfivesum#{\pdfmdfivesum file}%
  \fi
  \def\pdf@system#{%
    \immediate\write18%
  }%
  \pdftexcmds@AtEnd
  \expandafter\endinput
\fi
%    \end{macrocode}
%
% \subsection{Load module}
%
%    \begin{macrocode}
\begingroup\expandafter\expandafter\expandafter\endgroup
\expandafter\ifx\csname RequirePackage\endcsname\relax
  \input luatex-loader.sty\relax
\else
  \RequirePackage{luatex-loader}[2007/12/12]%
\fi
\directlua0{%
  require("oberdiek.pdftexcmds")%
}
%    \end{macrocode}
%
% \subsection{Lua functions}
%
%    \begin{macro}{\pdftexcmds@toks}
%    \begin{macrocode}
\begingroup\expandafter\expandafter\expandafter\endgroup
\expandafter\ifx\csname newtoks\endcsname\relax
  \toksdef\pdftexcmds@toks=0 %
\else
  \csname newtoks\endcsname\pdftexcmds@toks
\fi
%    \end{macrocode}
%    \end{macro}
%
%    \begin{macro}{\pdf@strcmp}
%    \begin{macrocode}
\long\def\pdf@strcmp#1#2{%
  \directlua0{%
    oberdiek.pdftexcmds.strcmp("\luaescapestring{#1}",%
        "\luaescapestring{#2}")%
  }%
}%
%    \end{macrocode}
%    \end{macro}
%    \begin{macro}{\pdf@escapehex}
%    \begin{macrocode}
\long\def\pdf@escapehex#1{%
  \directlua0{%
    oberdiek.pdftexcmds.escapehex("\luaescapestring{#1}", "byte")%
  }%
}%
%    \end{macrocode}
%    \end{macro}
%    \begin{macro}{\pdf@escapehexnative}
%    \begin{macrocode}
\long\def\pdf@escapehexnative#1{%
  \directlua0{%
    oberdiek.pdftexcmds.escapehex("\luaescapestring{#1}")%
  }%
}%
%    \end{macrocode}
%    \end{macro}
%    \begin{macro}{\pdf@unescapehex}
%    \begin{macrocode}
\def\pdf@unescapehex#1{%
  \the\expandafter\pdftexcmds@toks
  \directlua0{%
    oberdiek.pdftexcmds.toks="pdftexcmds@toks"%
    oberdiek.pdftexcmds.unescapehex("\luaescapestring{#1}", "byte")%
  }%
}%
%    \end{macrocode}
%    \end{macro}
%    \begin{macro}{\pdf@unescapehexnative}
%    \begin{macrocode}
\def\pdf@unescapehexnative#1{%
  \the\expandafter\pdftexcmds@toks
  \directlua0{%
    oberdiek.pdftexcmds.toks="pdftexcmds@toks"%
    oberdiek.pdftexcmds.unescapehex("\luaescapestring{#1}")%
  }%
}%
%    \end{macrocode}
%    \end{macro}
%    \begin{macro}{\pdf@escapestring}
%    \begin{macrocode}
\long\def\pdf@escapestring#1{%
  \directlua0{%
    oberdiek.pdftexcmds.escapestring("\luaescapestring{#1}", "byte")%
  }%
}
%    \end{macrocode}
%    \end{macro}
%    \begin{macro}{\pdf@escapename}
%    \begin{macrocode}
\long\def\pdf@escapename#1{%
  \directlua0{%
    oberdiek.pdftexcmds.escapename("\luaescapestring{#1}", "byte")%
  }%
}
%    \end{macrocode}
%    \end{macro}
%    \begin{macro}{\pdf@escapenamenative}
%    \begin{macrocode}
\long\def\pdf@escapenamenative#1{%
  \directlua0{%
    oberdiek.pdftexcmds.escapename("\luaescapestring{#1}")%
  }%
}
%    \end{macrocode}
%    \end{macro}
%    \begin{macro}{\pdf@filesize}
%    \begin{macrocode}
\def\pdf@filesize#1{%
  \directlua0{%
    oberdiek.pdftexcmds.filesize("\luaescapestring{#1}")%
  }%
}
%    \end{macrocode}
%    \end{macro}
%    \begin{macro}{\pdf@filemoddate}
%    \begin{macrocode}
\def\pdf@filemoddate#1{%
  \directlua0{%
    oberdiek.pdftexcmds.filemoddate("\luaescapestring{#1}")%
  }%
}
%    \end{macrocode}
%    \end{macro}
%    \begin{macro}{\pdf@filedump}
%    \begin{macrocode}
\def\pdf@filedump#1#2#3{%
  \directlua0{%
    oberdiek.pdftexcmds.filedump("\luaescapestring{\number#1}",%
        "\luaescapestring{\number#2}",%
        "\luaescapestring{#3}")%
  }%
}%
%    \end{macrocode}
%    \end{macro}
%    \begin{macro}{\pdf@mdfivesum}
%    \begin{macrocode}
\long\def\pdf@mdfivesum#1{%
  \directlua0{%
    oberdiek.pdftexcmds.mdfivesum("\luaescapestring{#1}", "byte")%
  }%
}%
%    \end{macrocode}
%    \end{macro}
%    \begin{macro}{\pdf@mdfivesumnative}
%    \begin{macrocode}
\long\def\pdf@mdfivesumnative#1{%
  \directlua0{%
    oberdiek.pdftexcmds.mdfivesum("\luaescapestring{#1}")%
  }%
}%
%    \end{macrocode}
%    \end{macro}
%    \begin{macro}{\pdf@filemdfivesum}
%    \begin{macrocode}
\def\pdf@filemdfivesum#1{%
  \directlua0{%
    oberdiek.pdftexcmds.filemdfivesum("\luaescapestring{#1}")%
  }%
}%
%    \end{macrocode}
%    \end{macro}
%
%    \begin{macro}{\pdf@shellescape}
%    \begin{macrocode}
\def\pdf@shellescape{%
  \directlua0{%
    oberdiek.pdftexcmds.shellescape()%
  }%
}
%    \end{macrocode}
%    \end{macro}
%
%    \begin{macro}{\pdf@system}
%    \begin{macrocode}
\def\pdf@system#1{%
  \directlua0{%
    oberdiek.pdftexcmds.system("\luaescapestring{#1}")%
  }%
}
%    \end{macrocode}
%    \end{macro}
%
%    \begin{macro}{\pdf@lastsystemstatus}
%    \begin{macrocode}
\def\pdf@lastsystemstatus{%
  \directlua0{%
    oberdiek.pdftexcmds.lastsystemstatus()%
  }%
}
%    \end{macrocode}
%    \end{macro}
%    \begin{macro}{\pdf@lastsystemexit}
%    \begin{macrocode}
\def\pdf@lastsystemexit{%
  \directlua0{%
    oberdiek.pdftexcmds.lastsystemexit()%
  }%
}
%    \end{macrocode}
%    \end{macro}
%
%    \begin{macro}{\pdf@pipe}
%    Check availability of |io.popen| first.
%    \begin{macrocode}
\ifnum0%
    \directlua0{%
      if io.popen then %
        tex.write("1")%
      end%
    }%
    =1 %
  \def\pdf@pipe#1{%
    \the\expandafter\pdftexcmds@toks
    \directlua0{%
      oberdiek.pdftexcmds.toks="pdftexcmds@toks"%
      oberdiek.pdftexcmds.pipe("\luaescapestring{#1}")%
    }%
  }%
\fi
%    \end{macrocode}
%    \end{macro}
%
%    \begin{macrocode}
\pdftexcmds@AtEnd
%</package>
%    \end{macrocode}
%
% \subsection{Lua module}
%
%    \begin{macrocode}
%<*lua>
%    \end{macrocode}
%
%    \begin{macrocode}
module("oberdiek.pdftexcmds", package.seeall)
local systemexitstatus
function strcmp(A, B)
  if A == B then
    tex.write("0")
  elseif A < B then
    tex.write("-1")
  else
    tex.write("1")
  end
end
local function utf8_to_byte(str)
  local i = 0
  local n = string.len(str)
  local t = {}
  while i < n do
    i = i + 1
    local a = string.byte(str, i)
    if a < 128 then
      table.insert(t, string.char(a))
    else
      if a >= 192 and i < n then
        i = i + 1
        local b = string.byte(str, i)
        if b < 128 or b >= 192 then
          i = i - 1
        elseif a == 194 then
          table.insert(t, string.char(b))
        elseif a == 195 then
          table.insert(t, string.char(b + 64))
        end
      end
    end
  end
  return table.concat(t)
end
function escapehex(str, mode)
  if mode == "byte" then
    str = utf8_to_byte(str)
  end
  tex.write((string.gsub(str, ".",
    function (ch)
      return string.format("%02X", string.byte(ch))
    end
  )))
end
%    \end{macrocode}
%    See procedure |unescapehex| in file \xfile{utils.c} of \pdfTeX.
%    Caution: |tex.write| ignores leading spaces.
%    \begin{macrocode}
function unescapehex(str, mode)
  local a = 0
  local first = true
  local result = {}
  for i = 1, string.len(str), 1 do
    local ch = string.byte(str, i)
    if ch >= 48 and ch <= 57 then
      ch = ch - 48
    elseif ch >= 65 and ch <= 70 then
      ch = ch - 55
    elseif ch >= 97 and ch <= 102 then
      ch = ch - 87
    else
      ch = nil
    end
    if ch then
      if first then
        a = ch * 16
        first = false
      else
        table.insert(result, a + ch)
        first = true
      end
    end
  end
  if not first then
    table.insert(result, a)
  end
  if mode == "byte" then
    local utf8 = {}
    for i, a in ipairs(result) do
      if a < 128 then
        table.insert(utf8, a)
      else
        if a < 192 then
          table.insert(utf8, 194)
          a = a - 128
        else
          table.insert(utf8, 195)
          a = a - 192
        end
        table.insert(utf8, a + 128)
      end
    end
    result = utf8
  end
  tex.settoks(toks, string.char(unpack(result)))
end
%    \end{macrocode}
%    See procedure |escapestring| in file \xfile{utils.c} of \pdfTeX.
%    \begin{macrocode}
function escapestring(str, mode)
  if mode == "byte" then
    str = utf8_to_byte(str)
  end
  tex.write((string.gsub(str, ".",
    function (ch)
      local b = string.byte(ch)
      if b < 33 or b > 126 then
        return string.format("\\%.3o", b)
      end
      if b == 40 or b == 41 or b == 92 then
        return "\\" .. ch
      end
%    \end{macrocode}
%    Lua 5.1 returns the match in case of return value |nil|.
%    \begin{macrocode}
      return nil
    end
  )))
end
%    \end{macrocode}
%    See procedure |escapename| in file \xfile{utils.c} of \pdfTeX.
%    \begin{macrocode}
function escapename(str, mode)
  if mode == "byte" then
    str = utf8_to_byte(str)
  end
  tex.write((string.gsub(str, ".",
    function (ch)
      local b = string.byte(ch)
      if b == 0 then
%    \end{macrocode}
%    In Lua 5.0 |nil| could be used for the empty string,
%    But |nil| returns the match in Lua 5.1, thus we use
%    the empty string explicitly.
%    \begin{macrocode}
        return ""
      end
      if b <= 32 or b >= 127
          or b == 35 or b == 37 or b == 40 or b == 41
          or b == 47 or b == 60 or b == 62 or b == 91
          or b == 93 or b == 123 or b == 125 then
        return string.format("#%.2X", b)
      else
%    \end{macrocode}
%    Lua 5.1 returns the match in case of return value |nil|.
%    \begin{macrocode}
        return nil
      end
    end
  )))
end
function filesize(filename)
  local foundfile = kpse.find_file(filename, "tex", true)
  if foundfile then
    local size = lfs.attributes(foundfile, "size")
    if size then
      tex.write(size)
    end
  end
end
%    \end{macrocode}
%    See procedure |makepdftime| in file \xfile{utils.c} of \pdfTeX.
%    \begin{macrocode}
function filemoddate(filename)
  local foundfile = kpse.find_file(filename, "tex", true)
  if foundfile then
    local date = lfs.attributes(foundfile, "modification")
    if date then
      local d = os.date("*t", date)
      if d.sec >= 60 then
        d.sec = 59
      end
      local u = os.date("!*t", date)
      local off = 60 * (d.hour - u.hour) + d.min - u.min
      if d.year ~= u.year then
        if d.year > u.year then
          off = off + 1440
        else
          off = off - 1440
        end
      elseif d.yday ~= u.yday then
        if d.yday > u.yday then
          off = off + 1440
        else
          off = off - 1440
        end
      end
      local timezone
      if off == 0 then
        timezone = "Z"
      else
        local hours = math.floor(off / 60)
        local mins = math.abs(off - hours * 60)
        timezone = string.format("%+03d'%02d'", hours, mins)
      end
      tex.write(string.format("D:%04d%02d%02d%02d%02d%02d%s",
          d.year, d.month, d.day, d.hour, d.min, d.sec, timezone))
    end
  end
end
function filedump(offset, length, filename)
  length = tonumber(length)
  if length and length > 0 then
    local foundfile = kpse.find_file(filename, "tex", true)
    if foundfile then
      offset = tonumber(offset)
      if not offset then
        offset = 0
      end
      local filehandle = io.open(foundfile, "r")
      if filehandle then
        if offset > 0 then
          filehandle:seek("set", offset)
        end
        local dump = filehandle:read(length)
        escapehex(dump)
      end
    end
  end
end
function mdfivesum(str, mode)
  if mode == "byte" then
    str = utf8_to_byte(str)
  end
  escapehex(md5.sum(str))
end
function filemdfivesum(filename)
  local foundfile = kpse.find_file(filename, "tex", true)
  if foundfile then
    local filehandle = io.open(foundfile, "r")
    if filehandle then
      local contents = filehandle:read("*a")
      escapehex(md5.sum(contents))
    end
  end
end
function shellescape()
  if os.execute then
    tex.write("1")
  else
    tex.write("0")
  end
end
function system(cmdline)
  systemexitstatus = nil
  texio.write_nl("log", "system(" .. cmdline .. ") ")
  if os.execute then
    texio.write("log", "executed.")
    systemexitstatus = os.execute(cmdline)
  else
    texio.write("log", "disabled.")
  end
end
function lastsystemstatus()
  local result = tonumber(systemexitstatus)
  if result then
    local x = math.floor(result / 256)
    tex.write(result - 256 * math.floor(result / 256))
  end
end
function lastsystemexit()
  local result = tonumber(systemexitstatus)
  if result then
    tex.write(math.floor(result / 256))
  end
end
function pipe(cmdline)
  local result
  systemexitstatus = nil
  texio.write_nl("log", "pipe(" .. cmdline ..") ")
  if io.popen then
    texio.write("log", "executed.")
    local handle = io.popen(cmdline, "r")
    if handle then
      result = handle:read("*a")
      handle:close()
    end
  else
    texio.write("log", "disabled.")
  end
  if result then
    tex.settoks(toks, result)
  else
    tex.settoks(toks, "")
  end
end
%    \end{macrocode}
%    \begin{macrocode}
%</lua>
%    \end{macrocode}
%
% \section{Test}
%
% \subsection{Catcode checks for loading}
%
%    \begin{macrocode}
%<*test1>
%    \end{macrocode}
%    \begin{macrocode}
\catcode`\{=1 %
\catcode`\}=2 %
\catcode`\#=6 %
\catcode`\@=11 %
\expandafter\ifx\csname count@\endcsname\relax
  \countdef\count@=255 %
\fi
\expandafter\ifx\csname @gobble\endcsname\relax
  \long\def\@gobble#1{}%
\fi
\expandafter\ifx\csname @firstofone\endcsname\relax
  \long\def\@firstofone#1{#1}%
\fi
\expandafter\ifx\csname loop\endcsname\relax
  \expandafter\@firstofone
\else
  \expandafter\@gobble
\fi
{%
  \def\loop#1\repeat{%
    \def\body{#1}%
    \iterate
  }%
  \def\iterate{%
    \body
      \let\next\iterate
    \else
      \let\next\relax
    \fi
    \next
  }%
  \let\repeat=\fi
}%
\def\RestoreCatcodes{}
\count@=0 %
\loop
  \edef\RestoreCatcodes{%
    \RestoreCatcodes
    \catcode\the\count@=\the\catcode\count@\relax
  }%
\ifnum\count@<255 %
  \advance\count@ 1 %
\repeat

\def\RangeCatcodeInvalid#1#2{%
  \count@=#1\relax
  \loop
    \catcode\count@=15 %
  \ifnum\count@<#2\relax
    \advance\count@ 1 %
  \repeat
}
\expandafter\ifx\csname LoadCommand\endcsname\relax
  \def\LoadCommand{\input pdftexcmds.sty\relax}%
\fi
\def\Test{%
  \RangeCatcodeInvalid{0}{47}%
  \RangeCatcodeInvalid{58}{64}%
  \RangeCatcodeInvalid{91}{96}%
  \RangeCatcodeInvalid{123}{255}%
  \catcode`\@=12 %
  \catcode`\\=0 %
  \catcode`\{=1 %
  \catcode`\}=2 %
  \catcode`\#=6 %
  \catcode`\[=12 %
  \catcode`\]=12 %
  \catcode`\%=14 %
  \catcode`\ =10 %
  \catcode13=5 %
  \LoadCommand
  \RestoreCatcodes
}
\Test
\csname @@end\endcsname
\end
%    \end{macrocode}
%    \begin{macrocode}
%</test1>
%    \end{macrocode}
%
% \section{Installation}
%
% \subsection{Download}
%
% \paragraph{Package.} This package is available on
% CTAN\footnote{\url{ftp://ftp.ctan.org/tex-archive/}}:
% \begin{description}
% \item[\CTAN{macros/latex/contrib/oberdiek/pdftexcmds.dtx}] The source file.
% \item[\CTAN{macros/latex/contrib/oberdiek/pdftexcmds.pdf}] Documentation.
% \end{description}
%
%
% \paragraph{Bundle.} All the packages of the bundle `oberdiek'
% are also available in a TDS compliant ZIP archive. There
% the packages are already unpacked and the documentation files
% are generated. The files and directories obey the TDS standard.
% \begin{description}
% \item[\CTAN{macros/latex/contrib/oberdiek/oberdiek-tds.zip}]
% \end{description}
% \emph{TDS} refers to the standard ``A Directory Structure
% for \TeX\ Files'' (\CTAN{tds/tds.pdf}). Directories
% with \xfile{texmf} in their name are usually organized this way.
%
% \subsection{Bundle installation}
%
% \paragraph{Unpacking.} Unpack the \xfile{oberdiek-tds.zip} in the
% TDS tree (also known as \xfile{texmf} tree) of your choice.
% Example (linux):
% \begin{quote}
%   |unzip oberdiek-tds.zip -d ~/texmf|
% \end{quote}
%
% \paragraph{Script installation.}
% Check the directory \xfile{TDS:scripts/oberdiek/} for
% scripts that need further installation steps.
% Package \xpackage{attachfile2} comes with the Perl script
% \xfile{pdfatfi.pl} that should be installed in such a way
% that it can be called as \texttt{pdfatfi}.
% Example (linux):
% \begin{quote}
%   |chmod +x scripts/oberdiek/pdfatfi.pl|\\
%   |cp scripts/oberdiek/pdfatfi.pl /usr/local/bin/|
% \end{quote}
%
% \subsection{Package installation}
%
% \paragraph{Unpacking.} The \xfile{.dtx} file is a self-extracting
% \docstrip\ archive. The files are extracted by running the
% \xfile{.dtx} through \plainTeX:
% \begin{quote}
%   \verb|tex pdftexcmds.dtx|
% \end{quote}
%
% \paragraph{TDS.} Now the different files must be moved into
% the different directories in your installation TDS tree
% (also known as \xfile{texmf} tree):
% \begin{quote}
% \def\t{^^A
% \begin{tabular}{@{}>{\ttfamily}l@{ $\rightarrow$ }>{\ttfamily}l@{}}
%   pdftexcmds.sty & tex/generic/oberdiek/pdftexcmds.sty\\
%   oberdiek.pdftexcmds.lua & scripts/oberdiek/oberdiek.pdftexcmds.lua\\
%   pdftexcmds.lua & scripts/oberdiek/pdftexcmds.lua\\
%   pdftexcmds.pdf & doc/latex/oberdiek/pdftexcmds.pdf\\
%   pdftexcmds.dtx & source/latex/oberdiek/pdftexcmds.dtx\\
% \end{tabular}^^A
% }^^A
% \sbox0{\t}^^A
% \ifdim\wd0>\linewidth
%   \begingroup
%     \advance\linewidth by\leftmargin
%     \advance\linewidth by\rightmargin
%   \edef\x{\endgroup
%     \def\noexpand\lw{\the\linewidth}^^A
%   }\x
%   \def\lwbox{^^A
%     \leavevmode
%     \hbox to \linewidth{^^A
%       \kern-\leftmargin\relax
%       \hss
%       \usebox0
%       \hss
%       \kern-\rightmargin\relax
%     }^^A
%   }^^A
%   \ifdim\wd0>\lw
%     \sbox0{\small\t}^^A
%     \ifdim\wd0>\linewidth
%       \ifdim\wd0>\lw
%         \sbox0{\footnotesize\t}^^A
%         \ifdim\wd0>\linewidth
%           \ifdim\wd0>\lw
%             \sbox0{\scriptsize\t}^^A
%             \ifdim\wd0>\linewidth
%               \ifdim\wd0>\lw
%                 \sbox0{\tiny\t}^^A
%                 \ifdim\wd0>\linewidth
%                   \lwbox
%                 \else
%                   \usebox0
%                 \fi
%               \else
%                 \lwbox
%               \fi
%             \else
%               \usebox0
%             \fi
%           \else
%             \lwbox
%           \fi
%         \else
%           \usebox0
%         \fi
%       \else
%         \lwbox
%       \fi
%     \else
%       \usebox0
%     \fi
%   \else
%     \lwbox
%   \fi
% \else
%   \usebox0
% \fi
% \end{quote}
% If you have a \xfile{docstrip.cfg} that configures and enables \docstrip's
% TDS installing feature, then some files can already be in the right
% place, see the documentation of \docstrip.
%
% \subsection{Refresh file name databases}
%
% If your \TeX~distribution
% (\teTeX, \mikTeX, \dots) relies on file name databases, you must refresh
% these. For example, \teTeX\ users run \verb|texhash| or
% \verb|mktexlsr|.
%
% \subsection{Some details for the interested}
%
% \paragraph{Attached source.}
%
% The PDF documentation on CTAN also includes the
% \xfile{.dtx} source file. It can be extracted by
% AcrobatReader 6 or higher. Another option is \textsf{pdftk},
% e.g. unpack the file into the current directory:
% \begin{quote}
%   \verb|pdftk pdftexcmds.pdf unpack_files output .|
% \end{quote}
%
% \paragraph{Unpacking with \LaTeX.}
% The \xfile{.dtx} chooses its action depending on the format:
% \begin{description}
% \item[\plainTeX:] Run \docstrip\ and extract the files.
% \item[\LaTeX:] Generate the documentation.
% \end{description}
% If you insist on using \LaTeX\ for \docstrip\ (really,
% \docstrip\ does not need \LaTeX), then inform the autodetect routine
% about your intention:
% \begin{quote}
%   \verb|latex \let\install=y\input{pdftexcmds.dtx}|
% \end{quote}
% Do not forget to quote the argument according to the demands
% of your shell.
%
% \paragraph{Generating the documentation.}
% You can use both the \xfile{.dtx} or the \xfile{.drv} to generate
% the documentation. The process can be configured by the
% configuration file \xfile{ltxdoc.cfg}. For instance, put this
% line into this file, if you want to have A4 as paper format:
% \begin{quote}
%   \verb|\PassOptionsToClass{a4paper}{article}|
% \end{quote}
% An example follows how to generate the
% documentation with pdf\LaTeX:
% \begin{quote}
%\begin{verbatim}
%pdflatex pdftexcmds.dtx
%makeindex -s gind.ist pdftexcmds.idx
%pdflatex pdftexcmds.dtx
%makeindex -s gind.ist pdftexcmds.idx
%pdflatex pdftexcmds.dtx
%\end{verbatim}
% \end{quote}
%
% \begin{History}
%   \begin{Version}{2007/11/11 v0.1}
%   \item
%     First version.
%   \end{Version}
%   \begin{Version}{2007/11/12 v0.2}
%   \item
%     Short description fixed.
%   \end{Version}
%   \begin{Version}{2007/12/12 v0.3}
%   \item
%     Organization of Lua code as module.
%   \end{Version}
% \end{History}
%
% \PrintIndex
%
% \Finale
\endinput
|
% \end{quote}
% Do not forget to quote the argument according to the demands
% of your shell.
%
% \paragraph{Generating the documentation.}
% You can use both the \xfile{.dtx} or the \xfile{.drv} to generate
% the documentation. The process can be configured by the
% configuration file \xfile{ltxdoc.cfg}. For instance, put this
% line into this file, if you want to have A4 as paper format:
% \begin{quote}
%   \verb|\PassOptionsToClass{a4paper}{article}|
% \end{quote}
% An example follows how to generate the
% documentation with pdf\LaTeX:
% \begin{quote}
%\begin{verbatim}
%pdflatex pdftexcmds.dtx
%makeindex -s gind.ist pdftexcmds.idx
%pdflatex pdftexcmds.dtx
%makeindex -s gind.ist pdftexcmds.idx
%pdflatex pdftexcmds.dtx
%\end{verbatim}
% \end{quote}
%
% \begin{History}
%   \begin{Version}{2007/11/11 v0.1}
%   \item
%     First version.
%   \end{Version}
%   \begin{Version}{2007/11/12 v0.2}
%   \item
%     Short description fixed.
%   \end{Version}
%   \begin{Version}{2007/12/12 v0.3}
%   \item
%     Organization of Lua code as module.
%   \end{Version}
% \end{History}
%
% \PrintIndex
%
% \Finale
\endinput
|
% \end{quote}
% Do not forget to quote the argument according to the demands
% of your shell.
%
% \paragraph{Generating the documentation.}
% You can use both the \xfile{.dtx} or the \xfile{.drv} to generate
% the documentation. The process can be configured by the
% configuration file \xfile{ltxdoc.cfg}. For instance, put this
% line into this file, if you want to have A4 as paper format:
% \begin{quote}
%   \verb|\PassOptionsToClass{a4paper}{article}|
% \end{quote}
% An example follows how to generate the
% documentation with pdf\LaTeX:
% \begin{quote}
%\begin{verbatim}
%pdflatex pdftexcmds.dtx
%makeindex -s gind.ist pdftexcmds.idx
%pdflatex pdftexcmds.dtx
%makeindex -s gind.ist pdftexcmds.idx
%pdflatex pdftexcmds.dtx
%\end{verbatim}
% \end{quote}
%
% \begin{History}
%   \begin{Version}{2007/11/11 v0.1}
%   \item
%     First version.
%   \end{Version}
%   \begin{Version}{2007/11/12 v0.2}
%   \item
%     Short description fixed.
%   \end{Version}
%   \begin{Version}{2007/12/12 v0.3}
%   \item
%     Organization of Lua code as module.
%   \end{Version}
% \end{History}
%
% \PrintIndex
%
% \Finale
\endinput

%        (quote the arguments according to the demands of your shell)
%
% Documentation:
%    (a) If pdftexcmds.drv is present:
%           latex pdftexcmds.drv
%    (b) Without pdftexcmds.drv:
%           latex pdftexcmds.dtx; ...
%    The class ltxdoc loads the configuration file ltxdoc.cfg
%    if available. Here you can specify further options, e.g.
%    use A4 as paper format:
%       \PassOptionsToClass{a4paper}{article}
%
%    Programm calls to get the documentation (example):
%       pdflatex pdftexcmds.dtx
%       makeindex -s gind.ist pdftexcmds.idx
%       pdflatex pdftexcmds.dtx
%       makeindex -s gind.ist pdftexcmds.idx
%       pdflatex pdftexcmds.dtx
%
% Installation:
%    TDS:tex/generic/oberdiek/pdftexcmds.sty
%    TDS:scripts/oberdiek/oberdiek.pdftexcmds.lua
%    TDS:scripts/oberdiek/pdftexcmds.lua
%    TDS:doc/latex/oberdiek/pdftexcmds.pdf
%    TDS:source/latex/oberdiek/pdftexcmds.dtx
%
%<*ignore>
\begingroup
  \def\x{LaTeX2e}%
\expandafter\endgroup
\ifcase 0\ifx\install y1\fi\expandafter
         \ifx\csname processbatchFile\endcsname\relax\else1\fi
         \ifx\fmtname\x\else 1\fi\relax
\else\csname fi\endcsname
%</ignore>
%<*install>
\input docstrip.tex
\Msg{************************************************************************}
\Msg{* Installation}
\Msg{* Package: pdftexcmds 2007/12/12 v0.3 LuaTeX support for pdfTeX utility functions (HO)}
\Msg{************************************************************************}

\keepsilent
\askforoverwritefalse

\let\MetaPrefix\relax
\preamble

This is a generated file.

Copyright (C) 2007 by Heiko Oberdiek <oberdiek@uni-freiburg.de>

This work may be distributed and/or modified under the
conditions of the LaTeX Project Public License, either
version 1.3 of this license or (at your option) any later
version. The latest version of this license is in
   http://www.latex-project.org/lppl.txt
and version 1.3 or later is part of all distributions of
LaTeX version 2005/12/01 or later.

This work has the LPPL maintenance status "maintained".

This Current Maintainer of this work is Heiko Oberdiek.

This work consists of the main source file pdftexcmds.dtx
and the derived files
   pdftexcmds.sty, pdftexcmds.pdf, pdftexcmds.ins, pdftexcmds.drv,
   oberdiek.pdftexcmds.lua, pdftexcmds.lua.

\endpreamble
\let\MetaPrefix\DoubleperCent

\generate{%
  \file{pdftexcmds.ins}{\from{pdftexcmds.dtx}{install}}%
  \file{pdftexcmds.drv}{\from{pdftexcmds.dtx}{driver}}%
  \usedir{tex/generic/oberdiek}%
  \file{pdftexcmds.sty}{\from{pdftexcmds.dtx}{package}}%
}
\def\MetaPrefix{-- }
\def\defaultpostamble{%
  \MetaPrefix^^J%
  \MetaPrefix\space End of File `\outFileName'.%
}
\def\currentpostamble{\defaultpostamble}%
\generate{%
  \usedir{scripts/oberdiek}%
  \file{oberdiek.pdftexcmds.lua}{\from{pdftexcmds.dtx}{lua}}%
  \file{pdftexcmds.lua}{\from{pdftexcmds.dtx}{lua}}%
}

\obeyspaces
\Msg{************************************************************************}
\Msg{*}
\Msg{* To finish the installation you have to move the following}
\Msg{* file into a directory searched by TeX:}
\Msg{*}
\Msg{*     pdftexcmds.sty}
\Msg{*}
\Msg{* And install the following script files:}
\Msg{*}
\Msg{*     oberdiek.pdftexcmds.lua, pdftexcmds.lua}
\Msg{*}
\Msg{* To produce the documentation run the file `pdftexcmds.drv'}
\Msg{* through LaTeX.}
\Msg{*}
\Msg{* Happy TeXing!}
\Msg{*}
\Msg{************************************************************************}

\endbatchfile
%</install>
%<*ignore>
\fi
%</ignore>
%<*driver>
\NeedsTeXFormat{LaTeX2e}
\ProvidesFile{pdftexcmds.drv}%
  [2007/12/12 v0.3 LuaTeX support for pdfTeX utility functions (HO)]%
\documentclass{ltxdoc}
\usepackage{holtxdoc}[2007/11/11]
\usepackage{paralist}
\begin{document}
  \DocInput{pdftexcmds.dtx}%
\end{document}
%</driver>
% \fi
%
% \CheckSum{552}
%
% \CharacterTable
%  {Upper-case    \A\B\C\D\E\F\G\H\I\J\K\L\M\N\O\P\Q\R\S\T\U\V\W\X\Y\Z
%   Lower-case    \a\b\c\d\e\f\g\h\i\j\k\l\m\n\o\p\q\r\s\t\u\v\w\x\y\z
%   Digits        \0\1\2\3\4\5\6\7\8\9
%   Exclamation   \!     Double quote  \"     Hash (number) \#
%   Dollar        \$     Percent       \%     Ampersand     \&
%   Acute accent  \'     Left paren    \(     Right paren   \)
%   Asterisk      \*     Plus          \+     Comma         \,
%   Minus         \-     Point         \.     Solidus       \/
%   Colon         \:     Semicolon     \;     Less than     \<
%   Equals        \=     Greater than  \>     Question mark \?
%   Commercial at \@     Left bracket  \[     Backslash     \\
%   Right bracket \]     Circumflex    \^     Underscore    \_
%   Grave accent  \`     Left brace    \{     Vertical bar  \|
%   Right brace   \}     Tilde         \~}
%
% \GetFileInfo{pdftexcmds.drv}
%
% \title{The \xpackage{pdftexcmds} package}
% \date{2007/12/12 v0.3}
% \author{Heiko Oberdiek\\\xemail{oberdiek@uni-freiburg.de}}
%
% \maketitle
%
% \begin{abstract}
% \LuaTeX\ provides most of the commands of \pdfTeX\ 1.40. However
% a number of utility functions are removed. This package tries to fill
% the gap and implements some of the missing primitive using Lua.
% \end{abstract}
%
% \tableofcontents
%
% \section{Documentation}
%
% Some primitives of \pdfTeX\ are not defined by \LuaTeX.
% This package implements macro based solutions using Lua code
% for the following missing \pdfTeX\ primitives;
% \begin{compactitem}
% \item \cs{pdfstrcmp}
% \item \cs{pdfunescapehex}
% \item \cs{pdfescapehex}
% \item \cs{pdfescapename}
% \item \cs{pdfescapestring}
% \item \cs{pdffilesize}
% \item \cs{pdffilemoddate}
% \item \cs{pdffiledump}
% \item \cs{pdfmdfivesum}
% \item |\immediate\write18|
% \end{compactitem}
% The original names of the primitives cannot be used:
% \begin{itemize}
% \item
% The syntax for their arguments cannot easily
% simulated by macros. The primitives using key words
% such as |file| (\cs{pdfmdfivesum}) or |offset| and |length|
% (\cs{pdffiledump}) and uses \meta{general text} for the other
% arguments. Using token registers assignments, \meta{general text} could
% be catched. However, the simulated primitives are expandable
% and register assignments would destroy this important property.
% (\meta{general text} allows something like |\expandafter\bgroup ...}|.)
% \item
% The original primitives can be expanded using one expansion step.
% The new macros need two expansion steps because of the additional
% macro expansion. Example:
% \begin{quote}
%   |\expandafter\foo\pdffilemoddate{file}|\\
%   vs.
%   |\expandafter\expandafter\expandafter\foo\pdf@filemoddate{file}|.
% \end{quote}
% \end{itemize}
%
% \LuaTeX\ isn't stable yet and thus the status of this package is
% \emph{experimental}. Feedback is welcome.
%
% \subsection{General principles}
%
% \begin{description}
% \item[Naming convention:]
%   Usually this package defines a macro |\pdf@|\meta{cmd} if
%   \pdfTeX\ provides |\pdf|\meta{cmd}.
% \item[Arguments:] The order of arguments in |\pdf@|\meta{cmd}
%   is the same as for the corresponding primitive of \pdfTeX.
%   The arguments are ordinary undelimited \TeX\ arguments,
%   no \meta{general text} and without additional keywords.
% \item[Expandibility:]
%   The macro |\pdf@|\meta{cmd} is expandable if the
%   corresponding \pdfTeX\ primitive has this property.
%   Exact two expansion steps are necessary (first is the macro
%   expansion).
% \item[Without \LuaTeX:]
%   The macros |\pdf@|\meta{cmd} are mapped to the commands
%   of \pdfTeX\ if they are available. Otherwise they are undefined.
% \end{description}
%
% \subsection{Macros}
%
% \begin{declcs}{pdf@strcmp} \M{stringA} \M{stringB}
% \end{declcs}
% Same as |\pdfstrcmp{|\meta{stringA}|}{|\meta{stringB}|}|.
%
% \begin{declcs}{pdf@unescapehex} \M{string}
% \end{declcs}
% Same as |\pdfunescapehex{|\meta{string}|}|.
% The argument is a byte string given in hexadecimal notation.
% The result are character tokens from 0 until 255 with
% catcode 12 and the space with catcode 10.
%
% \begin{declcs}{pdf@escapehex} \M{string}\\
%   \cs{pdf@escapestring} \M{string}\\
%   \cs{pdf@escapename} \M{string}
% \end{declcs}
% Same as the primitives of \pdfTeX. However \pdfTeX\ does not
% know about characters with codes 256 and larger. Thus the
% string is treated as byte string, characters with more than
% eight bits are ignored.
%
% \begin{declcs}{pdf@filesize} \M{filename}
% \end{declcs}
% Same as |\pdffilesize{|\meta{filename}|}|.
%
% \begin{declcs}{pdf@filemoddate} \M{filename}
% \end{declcs}
% Same as |\pdffilemoddate{|\meta{filename}|}|.
%
% \begin{declcs}{pdf@filedump} \M{offset} \M{length} \M{filename}
% \end{declcs}
% Same as |\pdffiledump offset| \meta{offset} |length| \meta{length}
% |{|\meta{filename}|}|. Both \meta{offset} and \meta{length} must
% not be empty, but must be a valid \TeX\ number.
%
% \begin{declcs}{pdf@mdfivesum} \M{string}
% \end{declcs}
% Same as |\pdfmdfivesum{|\meta{string}|}|. Keyword |file| is supported
% by macro \cs{pdf@filemdfivesum}.
%
% \begin{declcs}{pdf@filemdfivesum} \M{filename}
% \end{declcs}
% Same as |\pdfmdfivesum file{|\meta{filename}|}|.
%
% \begin{declcs}{pdf@shellescape}
% \end{declcs}
% Same as |\pdfshellescape|. It expands to |1| if external
% commands can be executed and |0| otherwise. In \pdfTeX\ external
% commands must be enabled first by command line option or
% configuration option. In \LuaTeX\ option |--safer| disables
% the execution of external commands.
%
% \begin{declcs}{pdf@system} \M{cmdline}
% \end{declcs}
% It is a wrapper for |\immediate\write18| in \pdfTeX\ or
% |os.execute| in \LuaTeX.
%
% In theory |os.execute|
% returns a status number. But its meaning is quite
% undefined. Are there some reliable properties?
% Does it make sense to provide an user interface to
% this status exit code?
%
% \subsubsection{Experimental}
%
% \begin{declcs}{pdf@unescapehexnative} \M{string}\\
%   \cs{pdf@escapehexnative} \M{string}\\
%   \cs{pdf@escapenamenative} \M{string}\\
%   \cs{pdf@mdfivesumnative} \M{string}
% \end{declcs}
% The variants without |native| in the macro name are supposed to
% be compatible with \pdfTeX. However characters with more than
% eight bits are not supported and are ignored. If \LuaTeX\ is
% running, then its UTF-8 coded strings are used. Thus the full
% unicode character range is supported. However the result
% differs from \pdfTeX\ for characters with eight or more bits.
%
% \begin{declcs}{pdf@pipe} \M{cmdline}
% \end{declcs}
% It calls \meta{cmdline} and returns the output of the external
% program in the usual manner as byte string (catcode 12, space with
% catcode 10). The Lua documentation says, that the used |io.popen|
% may not be available on all platforms. Then macro \cs{pdf@pipe}
% is undefined.
%
% \StopEventually{
% }
%
% \section{Implementation}
%
%    \begin{macrocode}
%<*package>
%    \end{macrocode}
%
% \subsection{Reload check and package identification}
%    Reload check, especially if the package is not used with \LaTeX.
%    \begin{macrocode}
\begingroup
  \catcode44 12 % ,
  \catcode45 12 % -
  \catcode46 12 % .
  \catcode58 12 % :
  \catcode64 11 % @
  \expandafter\let\expandafter\x\csname ver@pdftexcmds.sty\endcsname
  \ifcase 0%
    \ifx\x\relax % plain
    \else
      \ifx\x\empty % LaTeX
      \else
        1%
      \fi
    \fi
  \else
    \catcode35 6 % #
    \catcode123 1 % {
    \catcode125 2 % }
    \expandafter\ifx\csname PackageInfo\endcsname\relax
      \def\x#1#2{%
        \immediate\write-1{Package #1 Info: #2.}%
      }%
    \else
      \def\x#1#2{\PackageInfo{#1}{#2, stopped}}%
    \fi
    \x{pdftexcmds}{The package is already loaded}%
    \endgroup
    \expandafter\endinput
  \fi
\endgroup
%    \end{macrocode}
%    Package identification:
%    \begin{macrocode}
\begingroup
  \catcode35 6 % #
  \catcode40 12 % (
  \catcode41 12 % )
  \catcode44 12 % ,
  \catcode45 12 % -
  \catcode46 12 % .
  \catcode47 12 % /
  \catcode58 12 % :
  \catcode64 11 % @
  \catcode123 1 % {
  \catcode125 2 % }
  \expandafter\ifx\csname ProvidesPackage\endcsname\relax
    \def\x#1#2#3[#4]{\endgroup
      \immediate\write-1{Package: #3 #4}%
      \xdef#1{#4}%
    }%
  \else
    \def\x#1#2[#3]{\endgroup
      #2[{#3}]%
      \ifx#1\relax
        \xdef#1{#3}%
      \fi
    }%
  \fi
\expandafter\x\csname ver@pdftexcmds.sty\endcsname
\ProvidesPackage{pdftexcmds}%
  [2007/12/12 v0.3 LuaTeX support for pdfTeX utility functions (HO)]
%    \end{macrocode}
%
% \subsection{Catcodes}
%
%    \begin{macrocode}
\begingroup
  \catcode123 1 % {
  \catcode125 2 % }
  \def\x{\endgroup
    \expandafter\edef\csname pdftexcmds@AtEnd\endcsname{%
      \catcode35 \the\catcode35\relax
      \catcode64 \the\catcode64\relax
      \catcode123 \the\catcode123\relax
      \catcode125 \the\catcode125\relax
    }%
  }%
\x
\catcode35 6 % #
\catcode64 11 % @
\catcode123 1 % {
\catcode125 2 % }
\def\TMP@EnsureCode#1#2{%
  \edef\pdftexcmds@AtEnd{%
    \pdftexcmds@AtEnd
    \catcode#1 \the\catcode#1\relax
  }%
  \catcode#1 #2\relax
}
\TMP@EnsureCode{10}{12}% ^^J
\TMP@EnsureCode{33}{12}% !
\TMP@EnsureCode{34}{12}% "
\TMP@EnsureCode{39}{12}% '
\TMP@EnsureCode{40}{12}% (
\TMP@EnsureCode{41}{12}% )
\TMP@EnsureCode{42}{12}% *
\TMP@EnsureCode{43}{12}% +
\TMP@EnsureCode{44}{12}% ,
\TMP@EnsureCode{45}{12}% -
\TMP@EnsureCode{46}{12}% .
\TMP@EnsureCode{47}{12}% /
\TMP@EnsureCode{58}{12}% :
\TMP@EnsureCode{60}{12}% <
\TMP@EnsureCode{61}{12}% =
\TMP@EnsureCode{62}{12}% >
\TMP@EnsureCode{94}{7}% ^ (superscript)
\TMP@EnsureCode{95}{12}% _ (other)
\TMP@EnsureCode{126}{12}% ~ (other)
%    \end{macrocode}
%
% \subsection{Load package \xpackage{infwarerr}}
%
%    \begin{macrocode}
\begingroup\expandafter\expandafter\expandafter\endgroup
\expandafter\ifx\csname RequirePackage\endcsname\relax
  \input infwarerr.sty\relax
\else
  \RequirePackage{infwarerr}[2007/09/09]%
\fi
%    \end{macrocode}
%
% \subsection{Without \LuaTeX}
%
%    \begin{macrocode}
\begingroup\expandafter\expandafter\expandafter\endgroup
\expandafter\ifx\csname directlua\endcsname\relax
  \@PackageInfo{pdftexcmds}{LuaTeX not detected}%
  \def\pdftexcmds@nopdftex{%
    \@PackageInfoNoLine{pdftexcmds}{pdfTeX >= 1.30 not detected}%
    \let\pdftexcmds@nopdftex\relax
  }%
  \def\pdftexcmds@temp#1{%
    \begingroup\expandafter\expandafter\expandafter\endgroup
    \expandafter\ifx\csname pdf#1\endcsname\relax
      \pdftexcmds@nopdftex
    \else
      \expandafter\def\csname pdf@#1\expandafter\endcsname
      \expandafter##\expandafter{%
        \csname pdf#1\endcsname
      }%
    \fi
  }%
  \pdftexcmds@temp{strcmp}%
  \pdftexcmds@temp{escapehex}%
  \let\pdf@escapehexnative\pdf@escapehex
  \pdftexcmds@temp{unescapehex}%
  \let\pdf@unescapehexnative\pdf@unescapehex
  \pdftexcmds@temp{escapestring}%
  \pdftexcmds@temp{escapename}%
  \pdftexcmds@temp{filesize}%
  \pdftexcmds@temp{filemoddate}%
  \begingroup\expandafter\expandafter\expandafter\endgroup
  \expandafter\ifx\csname pdfshellescape\endcsname\relax
    \pdftexcmds@nopdftex
  \else
    \def\pdf@shellescape{%
      \pdfshellescape
    }%
  \fi
  \begingroup\expandafter\expandafter\expandafter\endgroup
  \expandafter\ifx\csname pdffiledump\endcsname\relax
    \pdftexcmds@nopdftex
  \else
    \def\pdf@filedump#1#2#3{%
      \pdffiledump offset#1 length#2{#3}%
    }%
  \fi
  \begingroup\expandafter\expandafter\expandafter\endgroup
  \expandafter\ifx\csname pdfmdfivesum\endcsname\relax
    \pdftexcmds@nopdftex
  \else
    \def\pdf@mdfivesum#{\pdfmdfivesum}%
    \let\pdf@mdfivesumnative\pdf@mdfivesum
    \def\pdf@filemdfivesum#{\pdfmdfivesum file}%
  \fi
  \def\pdf@system#{%
    \immediate\write18%
  }%
  \pdftexcmds@AtEnd
  \expandafter\endinput
\fi
%    \end{macrocode}
%
% \subsection{Load module}
%
%    \begin{macrocode}
\begingroup\expandafter\expandafter\expandafter\endgroup
\expandafter\ifx\csname RequirePackage\endcsname\relax
  \input luatex-loader.sty\relax
\else
  \RequirePackage{luatex-loader}[2007/12/12]%
\fi
\directlua0{%
  require("oberdiek.pdftexcmds")%
}
%    \end{macrocode}
%
% \subsection{Lua functions}
%
%    \begin{macro}{\pdftexcmds@toks}
%    \begin{macrocode}
\begingroup\expandafter\expandafter\expandafter\endgroup
\expandafter\ifx\csname newtoks\endcsname\relax
  \toksdef\pdftexcmds@toks=0 %
\else
  \csname newtoks\endcsname\pdftexcmds@toks
\fi
%    \end{macrocode}
%    \end{macro}
%
%    \begin{macro}{\pdf@strcmp}
%    \begin{macrocode}
\long\def\pdf@strcmp#1#2{%
  \directlua0{%
    oberdiek.pdftexcmds.strcmp("\luaescapestring{#1}",%
        "\luaescapestring{#2}")%
  }%
}%
%    \end{macrocode}
%    \end{macro}
%    \begin{macro}{\pdf@escapehex}
%    \begin{macrocode}
\long\def\pdf@escapehex#1{%
  \directlua0{%
    oberdiek.pdftexcmds.escapehex("\luaescapestring{#1}", "byte")%
  }%
}%
%    \end{macrocode}
%    \end{macro}
%    \begin{macro}{\pdf@escapehexnative}
%    \begin{macrocode}
\long\def\pdf@escapehexnative#1{%
  \directlua0{%
    oberdiek.pdftexcmds.escapehex("\luaescapestring{#1}")%
  }%
}%
%    \end{macrocode}
%    \end{macro}
%    \begin{macro}{\pdf@unescapehex}
%    \begin{macrocode}
\def\pdf@unescapehex#1{%
  \the\expandafter\pdftexcmds@toks
  \directlua0{%
    oberdiek.pdftexcmds.toks="pdftexcmds@toks"%
    oberdiek.pdftexcmds.unescapehex("\luaescapestring{#1}", "byte")%
  }%
}%
%    \end{macrocode}
%    \end{macro}
%    \begin{macro}{\pdf@unescapehexnative}
%    \begin{macrocode}
\def\pdf@unescapehexnative#1{%
  \the\expandafter\pdftexcmds@toks
  \directlua0{%
    oberdiek.pdftexcmds.toks="pdftexcmds@toks"%
    oberdiek.pdftexcmds.unescapehex("\luaescapestring{#1}")%
  }%
}%
%    \end{macrocode}
%    \end{macro}
%    \begin{macro}{\pdf@escapestring}
%    \begin{macrocode}
\long\def\pdf@escapestring#1{%
  \directlua0{%
    oberdiek.pdftexcmds.escapestring("\luaescapestring{#1}", "byte")%
  }%
}
%    \end{macrocode}
%    \end{macro}
%    \begin{macro}{\pdf@escapename}
%    \begin{macrocode}
\long\def\pdf@escapename#1{%
  \directlua0{%
    oberdiek.pdftexcmds.escapename("\luaescapestring{#1}", "byte")%
  }%
}
%    \end{macrocode}
%    \end{macro}
%    \begin{macro}{\pdf@escapenamenative}
%    \begin{macrocode}
\long\def\pdf@escapenamenative#1{%
  \directlua0{%
    oberdiek.pdftexcmds.escapename("\luaescapestring{#1}")%
  }%
}
%    \end{macrocode}
%    \end{macro}
%    \begin{macro}{\pdf@filesize}
%    \begin{macrocode}
\def\pdf@filesize#1{%
  \directlua0{%
    oberdiek.pdftexcmds.filesize("\luaescapestring{#1}")%
  }%
}
%    \end{macrocode}
%    \end{macro}
%    \begin{macro}{\pdf@filemoddate}
%    \begin{macrocode}
\def\pdf@filemoddate#1{%
  \directlua0{%
    oberdiek.pdftexcmds.filemoddate("\luaescapestring{#1}")%
  }%
}
%    \end{macrocode}
%    \end{macro}
%    \begin{macro}{\pdf@filedump}
%    \begin{macrocode}
\def\pdf@filedump#1#2#3{%
  \directlua0{%
    oberdiek.pdftexcmds.filedump("\luaescapestring{\number#1}",%
        "\luaescapestring{\number#2}",%
        "\luaescapestring{#3}")%
  }%
}%
%    \end{macrocode}
%    \end{macro}
%    \begin{macro}{\pdf@mdfivesum}
%    \begin{macrocode}
\long\def\pdf@mdfivesum#1{%
  \directlua0{%
    oberdiek.pdftexcmds.mdfivesum("\luaescapestring{#1}", "byte")%
  }%
}%
%    \end{macrocode}
%    \end{macro}
%    \begin{macro}{\pdf@mdfivesumnative}
%    \begin{macrocode}
\long\def\pdf@mdfivesumnative#1{%
  \directlua0{%
    oberdiek.pdftexcmds.mdfivesum("\luaescapestring{#1}")%
  }%
}%
%    \end{macrocode}
%    \end{macro}
%    \begin{macro}{\pdf@filemdfivesum}
%    \begin{macrocode}
\def\pdf@filemdfivesum#1{%
  \directlua0{%
    oberdiek.pdftexcmds.filemdfivesum("\luaescapestring{#1}")%
  }%
}%
%    \end{macrocode}
%    \end{macro}
%
%    \begin{macro}{\pdf@shellescape}
%    \begin{macrocode}
\def\pdf@shellescape{%
  \directlua0{%
    oberdiek.pdftexcmds.shellescape()%
  }%
}
%    \end{macrocode}
%    \end{macro}
%
%    \begin{macro}{\pdf@system}
%    \begin{macrocode}
\def\pdf@system#1{%
  \directlua0{%
    oberdiek.pdftexcmds.system("\luaescapestring{#1}")%
  }%
}
%    \end{macrocode}
%    \end{macro}
%
%    \begin{macro}{\pdf@lastsystemstatus}
%    \begin{macrocode}
\def\pdf@lastsystemstatus{%
  \directlua0{%
    oberdiek.pdftexcmds.lastsystemstatus()%
  }%
}
%    \end{macrocode}
%    \end{macro}
%    \begin{macro}{\pdf@lastsystemexit}
%    \begin{macrocode}
\def\pdf@lastsystemexit{%
  \directlua0{%
    oberdiek.pdftexcmds.lastsystemexit()%
  }%
}
%    \end{macrocode}
%    \end{macro}
%
%    \begin{macro}{\pdf@pipe}
%    Check availability of |io.popen| first.
%    \begin{macrocode}
\ifnum0%
    \directlua0{%
      if io.popen then %
        tex.write("1")%
      end%
    }%
    =1 %
  \def\pdf@pipe#1{%
    \the\expandafter\pdftexcmds@toks
    \directlua0{%
      oberdiek.pdftexcmds.toks="pdftexcmds@toks"%
      oberdiek.pdftexcmds.pipe("\luaescapestring{#1}")%
    }%
  }%
\fi
%    \end{macrocode}
%    \end{macro}
%
%    \begin{macrocode}
\pdftexcmds@AtEnd
%</package>
%    \end{macrocode}
%
% \subsection{Lua module}
%
%    \begin{macrocode}
%<*lua>
%    \end{macrocode}
%
%    \begin{macrocode}
module("oberdiek.pdftexcmds", package.seeall)
local systemexitstatus
function strcmp(A, B)
  if A == B then
    tex.write("0")
  elseif A < B then
    tex.write("-1")
  else
    tex.write("1")
  end
end
local function utf8_to_byte(str)
  local i = 0
  local n = string.len(str)
  local t = {}
  while i < n do
    i = i + 1
    local a = string.byte(str, i)
    if a < 128 then
      table.insert(t, string.char(a))
    else
      if a >= 192 and i < n then
        i = i + 1
        local b = string.byte(str, i)
        if b < 128 or b >= 192 then
          i = i - 1
        elseif a == 194 then
          table.insert(t, string.char(b))
        elseif a == 195 then
          table.insert(t, string.char(b + 64))
        end
      end
    end
  end
  return table.concat(t)
end
function escapehex(str, mode)
  if mode == "byte" then
    str = utf8_to_byte(str)
  end
  tex.write((string.gsub(str, ".",
    function (ch)
      return string.format("%02X", string.byte(ch))
    end
  )))
end
%    \end{macrocode}
%    See procedure |unescapehex| in file \xfile{utils.c} of \pdfTeX.
%    Caution: |tex.write| ignores leading spaces.
%    \begin{macrocode}
function unescapehex(str, mode)
  local a = 0
  local first = true
  local result = {}
  for i = 1, string.len(str), 1 do
    local ch = string.byte(str, i)
    if ch >= 48 and ch <= 57 then
      ch = ch - 48
    elseif ch >= 65 and ch <= 70 then
      ch = ch - 55
    elseif ch >= 97 and ch <= 102 then
      ch = ch - 87
    else
      ch = nil
    end
    if ch then
      if first then
        a = ch * 16
        first = false
      else
        table.insert(result, a + ch)
        first = true
      end
    end
  end
  if not first then
    table.insert(result, a)
  end
  if mode == "byte" then
    local utf8 = {}
    for i, a in ipairs(result) do
      if a < 128 then
        table.insert(utf8, a)
      else
        if a < 192 then
          table.insert(utf8, 194)
          a = a - 128
        else
          table.insert(utf8, 195)
          a = a - 192
        end
        table.insert(utf8, a + 128)
      end
    end
    result = utf8
  end
  tex.settoks(toks, string.char(unpack(result)))
end
%    \end{macrocode}
%    See procedure |escapestring| in file \xfile{utils.c} of \pdfTeX.
%    \begin{macrocode}
function escapestring(str, mode)
  if mode == "byte" then
    str = utf8_to_byte(str)
  end
  tex.write((string.gsub(str, ".",
    function (ch)
      local b = string.byte(ch)
      if b < 33 or b > 126 then
        return string.format("\\%.3o", b)
      end
      if b == 40 or b == 41 or b == 92 then
        return "\\" .. ch
      end
%    \end{macrocode}
%    Lua 5.1 returns the match in case of return value |nil|.
%    \begin{macrocode}
      return nil
    end
  )))
end
%    \end{macrocode}
%    See procedure |escapename| in file \xfile{utils.c} of \pdfTeX.
%    \begin{macrocode}
function escapename(str, mode)
  if mode == "byte" then
    str = utf8_to_byte(str)
  end
  tex.write((string.gsub(str, ".",
    function (ch)
      local b = string.byte(ch)
      if b == 0 then
%    \end{macrocode}
%    In Lua 5.0 |nil| could be used for the empty string,
%    But |nil| returns the match in Lua 5.1, thus we use
%    the empty string explicitly.
%    \begin{macrocode}
        return ""
      end
      if b <= 32 or b >= 127
          or b == 35 or b == 37 or b == 40 or b == 41
          or b == 47 or b == 60 or b == 62 or b == 91
          or b == 93 or b == 123 or b == 125 then
        return string.format("#%.2X", b)
      else
%    \end{macrocode}
%    Lua 5.1 returns the match in case of return value |nil|.
%    \begin{macrocode}
        return nil
      end
    end
  )))
end
function filesize(filename)
  local foundfile = kpse.find_file(filename, "tex", true)
  if foundfile then
    local size = lfs.attributes(foundfile, "size")
    if size then
      tex.write(size)
    end
  end
end
%    \end{macrocode}
%    See procedure |makepdftime| in file \xfile{utils.c} of \pdfTeX.
%    \begin{macrocode}
function filemoddate(filename)
  local foundfile = kpse.find_file(filename, "tex", true)
  if foundfile then
    local date = lfs.attributes(foundfile, "modification")
    if date then
      local d = os.date("*t", date)
      if d.sec >= 60 then
        d.sec = 59
      end
      local u = os.date("!*t", date)
      local off = 60 * (d.hour - u.hour) + d.min - u.min
      if d.year ~= u.year then
        if d.year > u.year then
          off = off + 1440
        else
          off = off - 1440
        end
      elseif d.yday ~= u.yday then
        if d.yday > u.yday then
          off = off + 1440
        else
          off = off - 1440
        end
      end
      local timezone
      if off == 0 then
        timezone = "Z"
      else
        local hours = math.floor(off / 60)
        local mins = math.abs(off - hours * 60)
        timezone = string.format("%+03d'%02d'", hours, mins)
      end
      tex.write(string.format("D:%04d%02d%02d%02d%02d%02d%s",
          d.year, d.month, d.day, d.hour, d.min, d.sec, timezone))
    end
  end
end
function filedump(offset, length, filename)
  length = tonumber(length)
  if length and length > 0 then
    local foundfile = kpse.find_file(filename, "tex", true)
    if foundfile then
      offset = tonumber(offset)
      if not offset then
        offset = 0
      end
      local filehandle = io.open(foundfile, "r")
      if filehandle then
        if offset > 0 then
          filehandle:seek("set", offset)
        end
        local dump = filehandle:read(length)
        escapehex(dump)
      end
    end
  end
end
function mdfivesum(str, mode)
  if mode == "byte" then
    str = utf8_to_byte(str)
  end
  escapehex(md5.sum(str))
end
function filemdfivesum(filename)
  local foundfile = kpse.find_file(filename, "tex", true)
  if foundfile then
    local filehandle = io.open(foundfile, "r")
    if filehandle then
      local contents = filehandle:read("*a")
      escapehex(md5.sum(contents))
    end
  end
end
function shellescape()
  if os.execute then
    tex.write("1")
  else
    tex.write("0")
  end
end
function system(cmdline)
  systemexitstatus = nil
  texio.write_nl("log", "system(" .. cmdline .. ") ")
  if os.execute then
    texio.write("log", "executed.")
    systemexitstatus = os.execute(cmdline)
  else
    texio.write("log", "disabled.")
  end
end
function lastsystemstatus()
  local result = tonumber(systemexitstatus)
  if result then
    local x = math.floor(result / 256)
    tex.write(result - 256 * math.floor(result / 256))
  end
end
function lastsystemexit()
  local result = tonumber(systemexitstatus)
  if result then
    tex.write(math.floor(result / 256))
  end
end
function pipe(cmdline)
  local result
  systemexitstatus = nil
  texio.write_nl("log", "pipe(" .. cmdline ..") ")
  if io.popen then
    texio.write("log", "executed.")
    local handle = io.popen(cmdline, "r")
    if handle then
      result = handle:read("*a")
      handle:close()
    end
  else
    texio.write("log", "disabled.")
  end
  if result then
    tex.settoks(toks, result)
  else
    tex.settoks(toks, "")
  end
end
%    \end{macrocode}
%    \begin{macrocode}
%</lua>
%    \end{macrocode}
%
% \section{Test}
%
% \subsection{Catcode checks for loading}
%
%    \begin{macrocode}
%<*test1>
%    \end{macrocode}
%    \begin{macrocode}
\catcode`\{=1 %
\catcode`\}=2 %
\catcode`\#=6 %
\catcode`\@=11 %
\expandafter\ifx\csname count@\endcsname\relax
  \countdef\count@=255 %
\fi
\expandafter\ifx\csname @gobble\endcsname\relax
  \long\def\@gobble#1{}%
\fi
\expandafter\ifx\csname @firstofone\endcsname\relax
  \long\def\@firstofone#1{#1}%
\fi
\expandafter\ifx\csname loop\endcsname\relax
  \expandafter\@firstofone
\else
  \expandafter\@gobble
\fi
{%
  \def\loop#1\repeat{%
    \def\body{#1}%
    \iterate
  }%
  \def\iterate{%
    \body
      \let\next\iterate
    \else
      \let\next\relax
    \fi
    \next
  }%
  \let\repeat=\fi
}%
\def\RestoreCatcodes{}
\count@=0 %
\loop
  \edef\RestoreCatcodes{%
    \RestoreCatcodes
    \catcode\the\count@=\the\catcode\count@\relax
  }%
\ifnum\count@<255 %
  \advance\count@ 1 %
\repeat

\def\RangeCatcodeInvalid#1#2{%
  \count@=#1\relax
  \loop
    \catcode\count@=15 %
  \ifnum\count@<#2\relax
    \advance\count@ 1 %
  \repeat
}
\expandafter\ifx\csname LoadCommand\endcsname\relax
  \def\LoadCommand{\input pdftexcmds.sty\relax}%
\fi
\def\Test{%
  \RangeCatcodeInvalid{0}{47}%
  \RangeCatcodeInvalid{58}{64}%
  \RangeCatcodeInvalid{91}{96}%
  \RangeCatcodeInvalid{123}{255}%
  \catcode`\@=12 %
  \catcode`\\=0 %
  \catcode`\{=1 %
  \catcode`\}=2 %
  \catcode`\#=6 %
  \catcode`\[=12 %
  \catcode`\]=12 %
  \catcode`\%=14 %
  \catcode`\ =10 %
  \catcode13=5 %
  \LoadCommand
  \RestoreCatcodes
}
\Test
\csname @@end\endcsname
\end
%    \end{macrocode}
%    \begin{macrocode}
%</test1>
%    \end{macrocode}
%
% \section{Installation}
%
% \subsection{Download}
%
% \paragraph{Package.} This package is available on
% CTAN\footnote{\url{ftp://ftp.ctan.org/tex-archive/}}:
% \begin{description}
% \item[\CTAN{macros/latex/contrib/oberdiek/pdftexcmds.dtx}] The source file.
% \item[\CTAN{macros/latex/contrib/oberdiek/pdftexcmds.pdf}] Documentation.
% \end{description}
%
%
% \paragraph{Bundle.} All the packages of the bundle `oberdiek'
% are also available in a TDS compliant ZIP archive. There
% the packages are already unpacked and the documentation files
% are generated. The files and directories obey the TDS standard.
% \begin{description}
% \item[\CTAN{macros/latex/contrib/oberdiek/oberdiek-tds.zip}]
% \end{description}
% \emph{TDS} refers to the standard ``A Directory Structure
% for \TeX\ Files'' (\CTAN{tds/tds.pdf}). Directories
% with \xfile{texmf} in their name are usually organized this way.
%
% \subsection{Bundle installation}
%
% \paragraph{Unpacking.} Unpack the \xfile{oberdiek-tds.zip} in the
% TDS tree (also known as \xfile{texmf} tree) of your choice.
% Example (linux):
% \begin{quote}
%   |unzip oberdiek-tds.zip -d ~/texmf|
% \end{quote}
%
% \paragraph{Script installation.}
% Check the directory \xfile{TDS:scripts/oberdiek/} for
% scripts that need further installation steps.
% Package \xpackage{attachfile2} comes with the Perl script
% \xfile{pdfatfi.pl} that should be installed in such a way
% that it can be called as \texttt{pdfatfi}.
% Example (linux):
% \begin{quote}
%   |chmod +x scripts/oberdiek/pdfatfi.pl|\\
%   |cp scripts/oberdiek/pdfatfi.pl /usr/local/bin/|
% \end{quote}
%
% \subsection{Package installation}
%
% \paragraph{Unpacking.} The \xfile{.dtx} file is a self-extracting
% \docstrip\ archive. The files are extracted by running the
% \xfile{.dtx} through \plainTeX:
% \begin{quote}
%   \verb|tex pdftexcmds.dtx|
% \end{quote}
%
% \paragraph{TDS.} Now the different files must be moved into
% the different directories in your installation TDS tree
% (also known as \xfile{texmf} tree):
% \begin{quote}
% \def\t{^^A
% \begin{tabular}{@{}>{\ttfamily}l@{ $\rightarrow$ }>{\ttfamily}l@{}}
%   pdftexcmds.sty & tex/generic/oberdiek/pdftexcmds.sty\\
%   oberdiek.pdftexcmds.lua & scripts/oberdiek/oberdiek.pdftexcmds.lua\\
%   pdftexcmds.lua & scripts/oberdiek/pdftexcmds.lua\\
%   pdftexcmds.pdf & doc/latex/oberdiek/pdftexcmds.pdf\\
%   pdftexcmds.dtx & source/latex/oberdiek/pdftexcmds.dtx\\
% \end{tabular}^^A
% }^^A
% \sbox0{\t}^^A
% \ifdim\wd0>\linewidth
%   \begingroup
%     \advance\linewidth by\leftmargin
%     \advance\linewidth by\rightmargin
%   \edef\x{\endgroup
%     \def\noexpand\lw{\the\linewidth}^^A
%   }\x
%   \def\lwbox{^^A
%     \leavevmode
%     \hbox to \linewidth{^^A
%       \kern-\leftmargin\relax
%       \hss
%       \usebox0
%       \hss
%       \kern-\rightmargin\relax
%     }^^A
%   }^^A
%   \ifdim\wd0>\lw
%     \sbox0{\small\t}^^A
%     \ifdim\wd0>\linewidth
%       \ifdim\wd0>\lw
%         \sbox0{\footnotesize\t}^^A
%         \ifdim\wd0>\linewidth
%           \ifdim\wd0>\lw
%             \sbox0{\scriptsize\t}^^A
%             \ifdim\wd0>\linewidth
%               \ifdim\wd0>\lw
%                 \sbox0{\tiny\t}^^A
%                 \ifdim\wd0>\linewidth
%                   \lwbox
%                 \else
%                   \usebox0
%                 \fi
%               \else
%                 \lwbox
%               \fi
%             \else
%               \usebox0
%             \fi
%           \else
%             \lwbox
%           \fi
%         \else
%           \usebox0
%         \fi
%       \else
%         \lwbox
%       \fi
%     \else
%       \usebox0
%     \fi
%   \else
%     \lwbox
%   \fi
% \else
%   \usebox0
% \fi
% \end{quote}
% If you have a \xfile{docstrip.cfg} that configures and enables \docstrip's
% TDS installing feature, then some files can already be in the right
% place, see the documentation of \docstrip.
%
% \subsection{Refresh file name databases}
%
% If your \TeX~distribution
% (\teTeX, \mikTeX, \dots) relies on file name databases, you must refresh
% these. For example, \teTeX\ users run \verb|texhash| or
% \verb|mktexlsr|.
%
% \subsection{Some details for the interested}
%
% \paragraph{Attached source.}
%
% The PDF documentation on CTAN also includes the
% \xfile{.dtx} source file. It can be extracted by
% AcrobatReader 6 or higher. Another option is \textsf{pdftk},
% e.g. unpack the file into the current directory:
% \begin{quote}
%   \verb|pdftk pdftexcmds.pdf unpack_files output .|
% \end{quote}
%
% \paragraph{Unpacking with \LaTeX.}
% The \xfile{.dtx} chooses its action depending on the format:
% \begin{description}
% \item[\plainTeX:] Run \docstrip\ and extract the files.
% \item[\LaTeX:] Generate the documentation.
% \end{description}
% If you insist on using \LaTeX\ for \docstrip\ (really,
% \docstrip\ does not need \LaTeX), then inform the autodetect routine
% about your intention:
% \begin{quote}
%   \verb|latex \let\install=y% \iffalse meta-comment
%
% Copyright (C) 2007 by Heiko Oberdiek <oberdiek@uni-freiburg.de>
%
% This work may be distributed and/or modified under the
% conditions of the LaTeX Project Public License, either
% version 1.3 of this license or (at your option) any later
% version. The latest version of this license is in
%    http://www.latex-project.org/lppl.txt
% and version 1.3 or later is part of all distributions of
% LaTeX version 2005/12/01 or later.
%
% This work has the LPPL maintenance status "maintained".
%
% This Current Maintainer of this work is Heiko Oberdiek.
%
% This work consists of the main source file pdftexcmds.dtx
% and the derived files
%    pdftexcmds.sty, pdftexcmds.pdf, pdftexcmds.ins, pdftexcmds.drv,
%    oberdiek.pdftexcmds.lua, pdftexcmds.lua.
%
% Distribution:
%    CTAN:macros/latex/contrib/oberdiek/pdftexcmds.dtx
%    CTAN:macros/latex/contrib/oberdiek/pdftexcmds.pdf
%
% Unpacking:
%    (a) If pdftexcmds.ins is present:
%           tex pdftexcmds.ins
%    (b) Without pdftexcmds.ins:
%           tex pdftexcmds.dtx
%    (c) If you insist on using LaTeX
%           latex \let\install=y% \iffalse meta-comment
%
% Copyright (C) 2007 by Heiko Oberdiek <oberdiek@uni-freiburg.de>
%
% This work may be distributed and/or modified under the
% conditions of the LaTeX Project Public License, either
% version 1.3 of this license or (at your option) any later
% version. The latest version of this license is in
%    http://www.latex-project.org/lppl.txt
% and version 1.3 or later is part of all distributions of
% LaTeX version 2005/12/01 or later.
%
% This work has the LPPL maintenance status "maintained".
%
% This Current Maintainer of this work is Heiko Oberdiek.
%
% This work consists of the main source file pdftexcmds.dtx
% and the derived files
%    pdftexcmds.sty, pdftexcmds.pdf, pdftexcmds.ins, pdftexcmds.drv,
%    oberdiek.pdftexcmds.lua, pdftexcmds.lua.
%
% Distribution:
%    CTAN:macros/latex/contrib/oberdiek/pdftexcmds.dtx
%    CTAN:macros/latex/contrib/oberdiek/pdftexcmds.pdf
%
% Unpacking:
%    (a) If pdftexcmds.ins is present:
%           tex pdftexcmds.ins
%    (b) Without pdftexcmds.ins:
%           tex pdftexcmds.dtx
%    (c) If you insist on using LaTeX
%           latex \let\install=y% \iffalse meta-comment
%
% Copyright (C) 2007 by Heiko Oberdiek <oberdiek@uni-freiburg.de>
%
% This work may be distributed and/or modified under the
% conditions of the LaTeX Project Public License, either
% version 1.3 of this license or (at your option) any later
% version. The latest version of this license is in
%    http://www.latex-project.org/lppl.txt
% and version 1.3 or later is part of all distributions of
% LaTeX version 2005/12/01 or later.
%
% This work has the LPPL maintenance status "maintained".
%
% This Current Maintainer of this work is Heiko Oberdiek.
%
% This work consists of the main source file pdftexcmds.dtx
% and the derived files
%    pdftexcmds.sty, pdftexcmds.pdf, pdftexcmds.ins, pdftexcmds.drv,
%    oberdiek.pdftexcmds.lua, pdftexcmds.lua.
%
% Distribution:
%    CTAN:macros/latex/contrib/oberdiek/pdftexcmds.dtx
%    CTAN:macros/latex/contrib/oberdiek/pdftexcmds.pdf
%
% Unpacking:
%    (a) If pdftexcmds.ins is present:
%           tex pdftexcmds.ins
%    (b) Without pdftexcmds.ins:
%           tex pdftexcmds.dtx
%    (c) If you insist on using LaTeX
%           latex \let\install=y\input{pdftexcmds.dtx}
%        (quote the arguments according to the demands of your shell)
%
% Documentation:
%    (a) If pdftexcmds.drv is present:
%           latex pdftexcmds.drv
%    (b) Without pdftexcmds.drv:
%           latex pdftexcmds.dtx; ...
%    The class ltxdoc loads the configuration file ltxdoc.cfg
%    if available. Here you can specify further options, e.g.
%    use A4 as paper format:
%       \PassOptionsToClass{a4paper}{article}
%
%    Programm calls to get the documentation (example):
%       pdflatex pdftexcmds.dtx
%       makeindex -s gind.ist pdftexcmds.idx
%       pdflatex pdftexcmds.dtx
%       makeindex -s gind.ist pdftexcmds.idx
%       pdflatex pdftexcmds.dtx
%
% Installation:
%    TDS:tex/generic/oberdiek/pdftexcmds.sty
%    TDS:scripts/oberdiek/oberdiek.pdftexcmds.lua
%    TDS:scripts/oberdiek/pdftexcmds.lua
%    TDS:doc/latex/oberdiek/pdftexcmds.pdf
%    TDS:source/latex/oberdiek/pdftexcmds.dtx
%
%<*ignore>
\begingroup
  \def\x{LaTeX2e}%
\expandafter\endgroup
\ifcase 0\ifx\install y1\fi\expandafter
         \ifx\csname processbatchFile\endcsname\relax\else1\fi
         \ifx\fmtname\x\else 1\fi\relax
\else\csname fi\endcsname
%</ignore>
%<*install>
\input docstrip.tex
\Msg{************************************************************************}
\Msg{* Installation}
\Msg{* Package: pdftexcmds 2007/12/12 v0.3 LuaTeX support for pdfTeX utility functions (HO)}
\Msg{************************************************************************}

\keepsilent
\askforoverwritefalse

\let\MetaPrefix\relax
\preamble

This is a generated file.

Copyright (C) 2007 by Heiko Oberdiek <oberdiek@uni-freiburg.de>

This work may be distributed and/or modified under the
conditions of the LaTeX Project Public License, either
version 1.3 of this license or (at your option) any later
version. The latest version of this license is in
   http://www.latex-project.org/lppl.txt
and version 1.3 or later is part of all distributions of
LaTeX version 2005/12/01 or later.

This work has the LPPL maintenance status "maintained".

This Current Maintainer of this work is Heiko Oberdiek.

This work consists of the main source file pdftexcmds.dtx
and the derived files
   pdftexcmds.sty, pdftexcmds.pdf, pdftexcmds.ins, pdftexcmds.drv,
   oberdiek.pdftexcmds.lua, pdftexcmds.lua.

\endpreamble
\let\MetaPrefix\DoubleperCent

\generate{%
  \file{pdftexcmds.ins}{\from{pdftexcmds.dtx}{install}}%
  \file{pdftexcmds.drv}{\from{pdftexcmds.dtx}{driver}}%
  \usedir{tex/generic/oberdiek}%
  \file{pdftexcmds.sty}{\from{pdftexcmds.dtx}{package}}%
}
\def\MetaPrefix{-- }
\def\defaultpostamble{%
  \MetaPrefix^^J%
  \MetaPrefix\space End of File `\outFileName'.%
}
\def\currentpostamble{\defaultpostamble}%
\generate{%
  \usedir{scripts/oberdiek}%
  \file{oberdiek.pdftexcmds.lua}{\from{pdftexcmds.dtx}{lua}}%
  \file{pdftexcmds.lua}{\from{pdftexcmds.dtx}{lua}}%
}

\obeyspaces
\Msg{************************************************************************}
\Msg{*}
\Msg{* To finish the installation you have to move the following}
\Msg{* file into a directory searched by TeX:}
\Msg{*}
\Msg{*     pdftexcmds.sty}
\Msg{*}
\Msg{* And install the following script files:}
\Msg{*}
\Msg{*     oberdiek.pdftexcmds.lua, pdftexcmds.lua}
\Msg{*}
\Msg{* To produce the documentation run the file `pdftexcmds.drv'}
\Msg{* through LaTeX.}
\Msg{*}
\Msg{* Happy TeXing!}
\Msg{*}
\Msg{************************************************************************}

\endbatchfile
%</install>
%<*ignore>
\fi
%</ignore>
%<*driver>
\NeedsTeXFormat{LaTeX2e}
\ProvidesFile{pdftexcmds.drv}%
  [2007/12/12 v0.3 LuaTeX support for pdfTeX utility functions (HO)]%
\documentclass{ltxdoc}
\usepackage{holtxdoc}[2007/11/11]
\usepackage{paralist}
\begin{document}
  \DocInput{pdftexcmds.dtx}%
\end{document}
%</driver>
% \fi
%
% \CheckSum{552}
%
% \CharacterTable
%  {Upper-case    \A\B\C\D\E\F\G\H\I\J\K\L\M\N\O\P\Q\R\S\T\U\V\W\X\Y\Z
%   Lower-case    \a\b\c\d\e\f\g\h\i\j\k\l\m\n\o\p\q\r\s\t\u\v\w\x\y\z
%   Digits        \0\1\2\3\4\5\6\7\8\9
%   Exclamation   \!     Double quote  \"     Hash (number) \#
%   Dollar        \$     Percent       \%     Ampersand     \&
%   Acute accent  \'     Left paren    \(     Right paren   \)
%   Asterisk      \*     Plus          \+     Comma         \,
%   Minus         \-     Point         \.     Solidus       \/
%   Colon         \:     Semicolon     \;     Less than     \<
%   Equals        \=     Greater than  \>     Question mark \?
%   Commercial at \@     Left bracket  \[     Backslash     \\
%   Right bracket \]     Circumflex    \^     Underscore    \_
%   Grave accent  \`     Left brace    \{     Vertical bar  \|
%   Right brace   \}     Tilde         \~}
%
% \GetFileInfo{pdftexcmds.drv}
%
% \title{The \xpackage{pdftexcmds} package}
% \date{2007/12/12 v0.3}
% \author{Heiko Oberdiek\\\xemail{oberdiek@uni-freiburg.de}}
%
% \maketitle
%
% \begin{abstract}
% \LuaTeX\ provides most of the commands of \pdfTeX\ 1.40. However
% a number of utility functions are removed. This package tries to fill
% the gap and implements some of the missing primitive using Lua.
% \end{abstract}
%
% \tableofcontents
%
% \section{Documentation}
%
% Some primitives of \pdfTeX\ are not defined by \LuaTeX.
% This package implements macro based solutions using Lua code
% for the following missing \pdfTeX\ primitives;
% \begin{compactitem}
% \item \cs{pdfstrcmp}
% \item \cs{pdfunescapehex}
% \item \cs{pdfescapehex}
% \item \cs{pdfescapename}
% \item \cs{pdfescapestring}
% \item \cs{pdffilesize}
% \item \cs{pdffilemoddate}
% \item \cs{pdffiledump}
% \item \cs{pdfmdfivesum}
% \item |\immediate\write18|
% \end{compactitem}
% The original names of the primitives cannot be used:
% \begin{itemize}
% \item
% The syntax for their arguments cannot easily
% simulated by macros. The primitives using key words
% such as |file| (\cs{pdfmdfivesum}) or |offset| and |length|
% (\cs{pdffiledump}) and uses \meta{general text} for the other
% arguments. Using token registers assignments, \meta{general text} could
% be catched. However, the simulated primitives are expandable
% and register assignments would destroy this important property.
% (\meta{general text} allows something like |\expandafter\bgroup ...}|.)
% \item
% The original primitives can be expanded using one expansion step.
% The new macros need two expansion steps because of the additional
% macro expansion. Example:
% \begin{quote}
%   |\expandafter\foo\pdffilemoddate{file}|\\
%   vs.
%   |\expandafter\expandafter\expandafter\foo\pdf@filemoddate{file}|.
% \end{quote}
% \end{itemize}
%
% \LuaTeX\ isn't stable yet and thus the status of this package is
% \emph{experimental}. Feedback is welcome.
%
% \subsection{General principles}
%
% \begin{description}
% \item[Naming convention:]
%   Usually this package defines a macro |\pdf@|\meta{cmd} if
%   \pdfTeX\ provides |\pdf|\meta{cmd}.
% \item[Arguments:] The order of arguments in |\pdf@|\meta{cmd}
%   is the same as for the corresponding primitive of \pdfTeX.
%   The arguments are ordinary undelimited \TeX\ arguments,
%   no \meta{general text} and without additional keywords.
% \item[Expandibility:]
%   The macro |\pdf@|\meta{cmd} is expandable if the
%   corresponding \pdfTeX\ primitive has this property.
%   Exact two expansion steps are necessary (first is the macro
%   expansion).
% \item[Without \LuaTeX:]
%   The macros |\pdf@|\meta{cmd} are mapped to the commands
%   of \pdfTeX\ if they are available. Otherwise they are undefined.
% \end{description}
%
% \subsection{Macros}
%
% \begin{declcs}{pdf@strcmp} \M{stringA} \M{stringB}
% \end{declcs}
% Same as |\pdfstrcmp{|\meta{stringA}|}{|\meta{stringB}|}|.
%
% \begin{declcs}{pdf@unescapehex} \M{string}
% \end{declcs}
% Same as |\pdfunescapehex{|\meta{string}|}|.
% The argument is a byte string given in hexadecimal notation.
% The result are character tokens from 0 until 255 with
% catcode 12 and the space with catcode 10.
%
% \begin{declcs}{pdf@escapehex} \M{string}\\
%   \cs{pdf@escapestring} \M{string}\\
%   \cs{pdf@escapename} \M{string}
% \end{declcs}
% Same as the primitives of \pdfTeX. However \pdfTeX\ does not
% know about characters with codes 256 and larger. Thus the
% string is treated as byte string, characters with more than
% eight bits are ignored.
%
% \begin{declcs}{pdf@filesize} \M{filename}
% \end{declcs}
% Same as |\pdffilesize{|\meta{filename}|}|.
%
% \begin{declcs}{pdf@filemoddate} \M{filename}
% \end{declcs}
% Same as |\pdffilemoddate{|\meta{filename}|}|.
%
% \begin{declcs}{pdf@filedump} \M{offset} \M{length} \M{filename}
% \end{declcs}
% Same as |\pdffiledump offset| \meta{offset} |length| \meta{length}
% |{|\meta{filename}|}|. Both \meta{offset} and \meta{length} must
% not be empty, but must be a valid \TeX\ number.
%
% \begin{declcs}{pdf@mdfivesum} \M{string}
% \end{declcs}
% Same as |\pdfmdfivesum{|\meta{string}|}|. Keyword |file| is supported
% by macro \cs{pdf@filemdfivesum}.
%
% \begin{declcs}{pdf@filemdfivesum} \M{filename}
% \end{declcs}
% Same as |\pdfmdfivesum file{|\meta{filename}|}|.
%
% \begin{declcs}{pdf@shellescape}
% \end{declcs}
% Same as |\pdfshellescape|. It expands to |1| if external
% commands can be executed and |0| otherwise. In \pdfTeX\ external
% commands must be enabled first by command line option or
% configuration option. In \LuaTeX\ option |--safer| disables
% the execution of external commands.
%
% \begin{declcs}{pdf@system} \M{cmdline}
% \end{declcs}
% It is a wrapper for |\immediate\write18| in \pdfTeX\ or
% |os.execute| in \LuaTeX.
%
% In theory |os.execute|
% returns a status number. But its meaning is quite
% undefined. Are there some reliable properties?
% Does it make sense to provide an user interface to
% this status exit code?
%
% \subsubsection{Experimental}
%
% \begin{declcs}{pdf@unescapehexnative} \M{string}\\
%   \cs{pdf@escapehexnative} \M{string}\\
%   \cs{pdf@escapenamenative} \M{string}\\
%   \cs{pdf@mdfivesumnative} \M{string}
% \end{declcs}
% The variants without |native| in the macro name are supposed to
% be compatible with \pdfTeX. However characters with more than
% eight bits are not supported and are ignored. If \LuaTeX\ is
% running, then its UTF-8 coded strings are used. Thus the full
% unicode character range is supported. However the result
% differs from \pdfTeX\ for characters with eight or more bits.
%
% \begin{declcs}{pdf@pipe} \M{cmdline}
% \end{declcs}
% It calls \meta{cmdline} and returns the output of the external
% program in the usual manner as byte string (catcode 12, space with
% catcode 10). The Lua documentation says, that the used |io.popen|
% may not be available on all platforms. Then macro \cs{pdf@pipe}
% is undefined.
%
% \StopEventually{
% }
%
% \section{Implementation}
%
%    \begin{macrocode}
%<*package>
%    \end{macrocode}
%
% \subsection{Reload check and package identification}
%    Reload check, especially if the package is not used with \LaTeX.
%    \begin{macrocode}
\begingroup
  \catcode44 12 % ,
  \catcode45 12 % -
  \catcode46 12 % .
  \catcode58 12 % :
  \catcode64 11 % @
  \expandafter\let\expandafter\x\csname ver@pdftexcmds.sty\endcsname
  \ifcase 0%
    \ifx\x\relax % plain
    \else
      \ifx\x\empty % LaTeX
      \else
        1%
      \fi
    \fi
  \else
    \catcode35 6 % #
    \catcode123 1 % {
    \catcode125 2 % }
    \expandafter\ifx\csname PackageInfo\endcsname\relax
      \def\x#1#2{%
        \immediate\write-1{Package #1 Info: #2.}%
      }%
    \else
      \def\x#1#2{\PackageInfo{#1}{#2, stopped}}%
    \fi
    \x{pdftexcmds}{The package is already loaded}%
    \endgroup
    \expandafter\endinput
  \fi
\endgroup
%    \end{macrocode}
%    Package identification:
%    \begin{macrocode}
\begingroup
  \catcode35 6 % #
  \catcode40 12 % (
  \catcode41 12 % )
  \catcode44 12 % ,
  \catcode45 12 % -
  \catcode46 12 % .
  \catcode47 12 % /
  \catcode58 12 % :
  \catcode64 11 % @
  \catcode123 1 % {
  \catcode125 2 % }
  \expandafter\ifx\csname ProvidesPackage\endcsname\relax
    \def\x#1#2#3[#4]{\endgroup
      \immediate\write-1{Package: #3 #4}%
      \xdef#1{#4}%
    }%
  \else
    \def\x#1#2[#3]{\endgroup
      #2[{#3}]%
      \ifx#1\relax
        \xdef#1{#3}%
      \fi
    }%
  \fi
\expandafter\x\csname ver@pdftexcmds.sty\endcsname
\ProvidesPackage{pdftexcmds}%
  [2007/12/12 v0.3 LuaTeX support for pdfTeX utility functions (HO)]
%    \end{macrocode}
%
% \subsection{Catcodes}
%
%    \begin{macrocode}
\begingroup
  \catcode123 1 % {
  \catcode125 2 % }
  \def\x{\endgroup
    \expandafter\edef\csname pdftexcmds@AtEnd\endcsname{%
      \catcode35 \the\catcode35\relax
      \catcode64 \the\catcode64\relax
      \catcode123 \the\catcode123\relax
      \catcode125 \the\catcode125\relax
    }%
  }%
\x
\catcode35 6 % #
\catcode64 11 % @
\catcode123 1 % {
\catcode125 2 % }
\def\TMP@EnsureCode#1#2{%
  \edef\pdftexcmds@AtEnd{%
    \pdftexcmds@AtEnd
    \catcode#1 \the\catcode#1\relax
  }%
  \catcode#1 #2\relax
}
\TMP@EnsureCode{10}{12}% ^^J
\TMP@EnsureCode{33}{12}% !
\TMP@EnsureCode{34}{12}% "
\TMP@EnsureCode{39}{12}% '
\TMP@EnsureCode{40}{12}% (
\TMP@EnsureCode{41}{12}% )
\TMP@EnsureCode{42}{12}% *
\TMP@EnsureCode{43}{12}% +
\TMP@EnsureCode{44}{12}% ,
\TMP@EnsureCode{45}{12}% -
\TMP@EnsureCode{46}{12}% .
\TMP@EnsureCode{47}{12}% /
\TMP@EnsureCode{58}{12}% :
\TMP@EnsureCode{60}{12}% <
\TMP@EnsureCode{61}{12}% =
\TMP@EnsureCode{62}{12}% >
\TMP@EnsureCode{94}{7}% ^ (superscript)
\TMP@EnsureCode{95}{12}% _ (other)
\TMP@EnsureCode{126}{12}% ~ (other)
%    \end{macrocode}
%
% \subsection{Load package \xpackage{infwarerr}}
%
%    \begin{macrocode}
\begingroup\expandafter\expandafter\expandafter\endgroup
\expandafter\ifx\csname RequirePackage\endcsname\relax
  \input infwarerr.sty\relax
\else
  \RequirePackage{infwarerr}[2007/09/09]%
\fi
%    \end{macrocode}
%
% \subsection{Without \LuaTeX}
%
%    \begin{macrocode}
\begingroup\expandafter\expandafter\expandafter\endgroup
\expandafter\ifx\csname directlua\endcsname\relax
  \@PackageInfo{pdftexcmds}{LuaTeX not detected}%
  \def\pdftexcmds@nopdftex{%
    \@PackageInfoNoLine{pdftexcmds}{pdfTeX >= 1.30 not detected}%
    \let\pdftexcmds@nopdftex\relax
  }%
  \def\pdftexcmds@temp#1{%
    \begingroup\expandafter\expandafter\expandafter\endgroup
    \expandafter\ifx\csname pdf#1\endcsname\relax
      \pdftexcmds@nopdftex
    \else
      \expandafter\def\csname pdf@#1\expandafter\endcsname
      \expandafter##\expandafter{%
        \csname pdf#1\endcsname
      }%
    \fi
  }%
  \pdftexcmds@temp{strcmp}%
  \pdftexcmds@temp{escapehex}%
  \let\pdf@escapehexnative\pdf@escapehex
  \pdftexcmds@temp{unescapehex}%
  \let\pdf@unescapehexnative\pdf@unescapehex
  \pdftexcmds@temp{escapestring}%
  \pdftexcmds@temp{escapename}%
  \pdftexcmds@temp{filesize}%
  \pdftexcmds@temp{filemoddate}%
  \begingroup\expandafter\expandafter\expandafter\endgroup
  \expandafter\ifx\csname pdfshellescape\endcsname\relax
    \pdftexcmds@nopdftex
  \else
    \def\pdf@shellescape{%
      \pdfshellescape
    }%
  \fi
  \begingroup\expandafter\expandafter\expandafter\endgroup
  \expandafter\ifx\csname pdffiledump\endcsname\relax
    \pdftexcmds@nopdftex
  \else
    \def\pdf@filedump#1#2#3{%
      \pdffiledump offset#1 length#2{#3}%
    }%
  \fi
  \begingroup\expandafter\expandafter\expandafter\endgroup
  \expandafter\ifx\csname pdfmdfivesum\endcsname\relax
    \pdftexcmds@nopdftex
  \else
    \def\pdf@mdfivesum#{\pdfmdfivesum}%
    \let\pdf@mdfivesumnative\pdf@mdfivesum
    \def\pdf@filemdfivesum#{\pdfmdfivesum file}%
  \fi
  \def\pdf@system#{%
    \immediate\write18%
  }%
  \pdftexcmds@AtEnd
  \expandafter\endinput
\fi
%    \end{macrocode}
%
% \subsection{Load module}
%
%    \begin{macrocode}
\begingroup\expandafter\expandafter\expandafter\endgroup
\expandafter\ifx\csname RequirePackage\endcsname\relax
  \input luatex-loader.sty\relax
\else
  \RequirePackage{luatex-loader}[2007/12/12]%
\fi
\directlua0{%
  require("oberdiek.pdftexcmds")%
}
%    \end{macrocode}
%
% \subsection{Lua functions}
%
%    \begin{macro}{\pdftexcmds@toks}
%    \begin{macrocode}
\begingroup\expandafter\expandafter\expandafter\endgroup
\expandafter\ifx\csname newtoks\endcsname\relax
  \toksdef\pdftexcmds@toks=0 %
\else
  \csname newtoks\endcsname\pdftexcmds@toks
\fi
%    \end{macrocode}
%    \end{macro}
%
%    \begin{macro}{\pdf@strcmp}
%    \begin{macrocode}
\long\def\pdf@strcmp#1#2{%
  \directlua0{%
    oberdiek.pdftexcmds.strcmp("\luaescapestring{#1}",%
        "\luaescapestring{#2}")%
  }%
}%
%    \end{macrocode}
%    \end{macro}
%    \begin{macro}{\pdf@escapehex}
%    \begin{macrocode}
\long\def\pdf@escapehex#1{%
  \directlua0{%
    oberdiek.pdftexcmds.escapehex("\luaescapestring{#1}", "byte")%
  }%
}%
%    \end{macrocode}
%    \end{macro}
%    \begin{macro}{\pdf@escapehexnative}
%    \begin{macrocode}
\long\def\pdf@escapehexnative#1{%
  \directlua0{%
    oberdiek.pdftexcmds.escapehex("\luaescapestring{#1}")%
  }%
}%
%    \end{macrocode}
%    \end{macro}
%    \begin{macro}{\pdf@unescapehex}
%    \begin{macrocode}
\def\pdf@unescapehex#1{%
  \the\expandafter\pdftexcmds@toks
  \directlua0{%
    oberdiek.pdftexcmds.toks="pdftexcmds@toks"%
    oberdiek.pdftexcmds.unescapehex("\luaescapestring{#1}", "byte")%
  }%
}%
%    \end{macrocode}
%    \end{macro}
%    \begin{macro}{\pdf@unescapehexnative}
%    \begin{macrocode}
\def\pdf@unescapehexnative#1{%
  \the\expandafter\pdftexcmds@toks
  \directlua0{%
    oberdiek.pdftexcmds.toks="pdftexcmds@toks"%
    oberdiek.pdftexcmds.unescapehex("\luaescapestring{#1}")%
  }%
}%
%    \end{macrocode}
%    \end{macro}
%    \begin{macro}{\pdf@escapestring}
%    \begin{macrocode}
\long\def\pdf@escapestring#1{%
  \directlua0{%
    oberdiek.pdftexcmds.escapestring("\luaescapestring{#1}", "byte")%
  }%
}
%    \end{macrocode}
%    \end{macro}
%    \begin{macro}{\pdf@escapename}
%    \begin{macrocode}
\long\def\pdf@escapename#1{%
  \directlua0{%
    oberdiek.pdftexcmds.escapename("\luaescapestring{#1}", "byte")%
  }%
}
%    \end{macrocode}
%    \end{macro}
%    \begin{macro}{\pdf@escapenamenative}
%    \begin{macrocode}
\long\def\pdf@escapenamenative#1{%
  \directlua0{%
    oberdiek.pdftexcmds.escapename("\luaescapestring{#1}")%
  }%
}
%    \end{macrocode}
%    \end{macro}
%    \begin{macro}{\pdf@filesize}
%    \begin{macrocode}
\def\pdf@filesize#1{%
  \directlua0{%
    oberdiek.pdftexcmds.filesize("\luaescapestring{#1}")%
  }%
}
%    \end{macrocode}
%    \end{macro}
%    \begin{macro}{\pdf@filemoddate}
%    \begin{macrocode}
\def\pdf@filemoddate#1{%
  \directlua0{%
    oberdiek.pdftexcmds.filemoddate("\luaescapestring{#1}")%
  }%
}
%    \end{macrocode}
%    \end{macro}
%    \begin{macro}{\pdf@filedump}
%    \begin{macrocode}
\def\pdf@filedump#1#2#3{%
  \directlua0{%
    oberdiek.pdftexcmds.filedump("\luaescapestring{\number#1}",%
        "\luaescapestring{\number#2}",%
        "\luaescapestring{#3}")%
  }%
}%
%    \end{macrocode}
%    \end{macro}
%    \begin{macro}{\pdf@mdfivesum}
%    \begin{macrocode}
\long\def\pdf@mdfivesum#1{%
  \directlua0{%
    oberdiek.pdftexcmds.mdfivesum("\luaescapestring{#1}", "byte")%
  }%
}%
%    \end{macrocode}
%    \end{macro}
%    \begin{macro}{\pdf@mdfivesumnative}
%    \begin{macrocode}
\long\def\pdf@mdfivesumnative#1{%
  \directlua0{%
    oberdiek.pdftexcmds.mdfivesum("\luaescapestring{#1}")%
  }%
}%
%    \end{macrocode}
%    \end{macro}
%    \begin{macro}{\pdf@filemdfivesum}
%    \begin{macrocode}
\def\pdf@filemdfivesum#1{%
  \directlua0{%
    oberdiek.pdftexcmds.filemdfivesum("\luaescapestring{#1}")%
  }%
}%
%    \end{macrocode}
%    \end{macro}
%
%    \begin{macro}{\pdf@shellescape}
%    \begin{macrocode}
\def\pdf@shellescape{%
  \directlua0{%
    oberdiek.pdftexcmds.shellescape()%
  }%
}
%    \end{macrocode}
%    \end{macro}
%
%    \begin{macro}{\pdf@system}
%    \begin{macrocode}
\def\pdf@system#1{%
  \directlua0{%
    oberdiek.pdftexcmds.system("\luaescapestring{#1}")%
  }%
}
%    \end{macrocode}
%    \end{macro}
%
%    \begin{macro}{\pdf@lastsystemstatus}
%    \begin{macrocode}
\def\pdf@lastsystemstatus{%
  \directlua0{%
    oberdiek.pdftexcmds.lastsystemstatus()%
  }%
}
%    \end{macrocode}
%    \end{macro}
%    \begin{macro}{\pdf@lastsystemexit}
%    \begin{macrocode}
\def\pdf@lastsystemexit{%
  \directlua0{%
    oberdiek.pdftexcmds.lastsystemexit()%
  }%
}
%    \end{macrocode}
%    \end{macro}
%
%    \begin{macro}{\pdf@pipe}
%    Check availability of |io.popen| first.
%    \begin{macrocode}
\ifnum0%
    \directlua0{%
      if io.popen then %
        tex.write("1")%
      end%
    }%
    =1 %
  \def\pdf@pipe#1{%
    \the\expandafter\pdftexcmds@toks
    \directlua0{%
      oberdiek.pdftexcmds.toks="pdftexcmds@toks"%
      oberdiek.pdftexcmds.pipe("\luaescapestring{#1}")%
    }%
  }%
\fi
%    \end{macrocode}
%    \end{macro}
%
%    \begin{macrocode}
\pdftexcmds@AtEnd
%</package>
%    \end{macrocode}
%
% \subsection{Lua module}
%
%    \begin{macrocode}
%<*lua>
%    \end{macrocode}
%
%    \begin{macrocode}
module("oberdiek.pdftexcmds", package.seeall)
local systemexitstatus
function strcmp(A, B)
  if A == B then
    tex.write("0")
  elseif A < B then
    tex.write("-1")
  else
    tex.write("1")
  end
end
local function utf8_to_byte(str)
  local i = 0
  local n = string.len(str)
  local t = {}
  while i < n do
    i = i + 1
    local a = string.byte(str, i)
    if a < 128 then
      table.insert(t, string.char(a))
    else
      if a >= 192 and i < n then
        i = i + 1
        local b = string.byte(str, i)
        if b < 128 or b >= 192 then
          i = i - 1
        elseif a == 194 then
          table.insert(t, string.char(b))
        elseif a == 195 then
          table.insert(t, string.char(b + 64))
        end
      end
    end
  end
  return table.concat(t)
end
function escapehex(str, mode)
  if mode == "byte" then
    str = utf8_to_byte(str)
  end
  tex.write((string.gsub(str, ".",
    function (ch)
      return string.format("%02X", string.byte(ch))
    end
  )))
end
%    \end{macrocode}
%    See procedure |unescapehex| in file \xfile{utils.c} of \pdfTeX.
%    Caution: |tex.write| ignores leading spaces.
%    \begin{macrocode}
function unescapehex(str, mode)
  local a = 0
  local first = true
  local result = {}
  for i = 1, string.len(str), 1 do
    local ch = string.byte(str, i)
    if ch >= 48 and ch <= 57 then
      ch = ch - 48
    elseif ch >= 65 and ch <= 70 then
      ch = ch - 55
    elseif ch >= 97 and ch <= 102 then
      ch = ch - 87
    else
      ch = nil
    end
    if ch then
      if first then
        a = ch * 16
        first = false
      else
        table.insert(result, a + ch)
        first = true
      end
    end
  end
  if not first then
    table.insert(result, a)
  end
  if mode == "byte" then
    local utf8 = {}
    for i, a in ipairs(result) do
      if a < 128 then
        table.insert(utf8, a)
      else
        if a < 192 then
          table.insert(utf8, 194)
          a = a - 128
        else
          table.insert(utf8, 195)
          a = a - 192
        end
        table.insert(utf8, a + 128)
      end
    end
    result = utf8
  end
  tex.settoks(toks, string.char(unpack(result)))
end
%    \end{macrocode}
%    See procedure |escapestring| in file \xfile{utils.c} of \pdfTeX.
%    \begin{macrocode}
function escapestring(str, mode)
  if mode == "byte" then
    str = utf8_to_byte(str)
  end
  tex.write((string.gsub(str, ".",
    function (ch)
      local b = string.byte(ch)
      if b < 33 or b > 126 then
        return string.format("\\%.3o", b)
      end
      if b == 40 or b == 41 or b == 92 then
        return "\\" .. ch
      end
%    \end{macrocode}
%    Lua 5.1 returns the match in case of return value |nil|.
%    \begin{macrocode}
      return nil
    end
  )))
end
%    \end{macrocode}
%    See procedure |escapename| in file \xfile{utils.c} of \pdfTeX.
%    \begin{macrocode}
function escapename(str, mode)
  if mode == "byte" then
    str = utf8_to_byte(str)
  end
  tex.write((string.gsub(str, ".",
    function (ch)
      local b = string.byte(ch)
      if b == 0 then
%    \end{macrocode}
%    In Lua 5.0 |nil| could be used for the empty string,
%    But |nil| returns the match in Lua 5.1, thus we use
%    the empty string explicitly.
%    \begin{macrocode}
        return ""
      end
      if b <= 32 or b >= 127
          or b == 35 or b == 37 or b == 40 or b == 41
          or b == 47 or b == 60 or b == 62 or b == 91
          or b == 93 or b == 123 or b == 125 then
        return string.format("#%.2X", b)
      else
%    \end{macrocode}
%    Lua 5.1 returns the match in case of return value |nil|.
%    \begin{macrocode}
        return nil
      end
    end
  )))
end
function filesize(filename)
  local foundfile = kpse.find_file(filename, "tex", true)
  if foundfile then
    local size = lfs.attributes(foundfile, "size")
    if size then
      tex.write(size)
    end
  end
end
%    \end{macrocode}
%    See procedure |makepdftime| in file \xfile{utils.c} of \pdfTeX.
%    \begin{macrocode}
function filemoddate(filename)
  local foundfile = kpse.find_file(filename, "tex", true)
  if foundfile then
    local date = lfs.attributes(foundfile, "modification")
    if date then
      local d = os.date("*t", date)
      if d.sec >= 60 then
        d.sec = 59
      end
      local u = os.date("!*t", date)
      local off = 60 * (d.hour - u.hour) + d.min - u.min
      if d.year ~= u.year then
        if d.year > u.year then
          off = off + 1440
        else
          off = off - 1440
        end
      elseif d.yday ~= u.yday then
        if d.yday > u.yday then
          off = off + 1440
        else
          off = off - 1440
        end
      end
      local timezone
      if off == 0 then
        timezone = "Z"
      else
        local hours = math.floor(off / 60)
        local mins = math.abs(off - hours * 60)
        timezone = string.format("%+03d'%02d'", hours, mins)
      end
      tex.write(string.format("D:%04d%02d%02d%02d%02d%02d%s",
          d.year, d.month, d.day, d.hour, d.min, d.sec, timezone))
    end
  end
end
function filedump(offset, length, filename)
  length = tonumber(length)
  if length and length > 0 then
    local foundfile = kpse.find_file(filename, "tex", true)
    if foundfile then
      offset = tonumber(offset)
      if not offset then
        offset = 0
      end
      local filehandle = io.open(foundfile, "r")
      if filehandle then
        if offset > 0 then
          filehandle:seek("set", offset)
        end
        local dump = filehandle:read(length)
        escapehex(dump)
      end
    end
  end
end
function mdfivesum(str, mode)
  if mode == "byte" then
    str = utf8_to_byte(str)
  end
  escapehex(md5.sum(str))
end
function filemdfivesum(filename)
  local foundfile = kpse.find_file(filename, "tex", true)
  if foundfile then
    local filehandle = io.open(foundfile, "r")
    if filehandle then
      local contents = filehandle:read("*a")
      escapehex(md5.sum(contents))
    end
  end
end
function shellescape()
  if os.execute then
    tex.write("1")
  else
    tex.write("0")
  end
end
function system(cmdline)
  systemexitstatus = nil
  texio.write_nl("log", "system(" .. cmdline .. ") ")
  if os.execute then
    texio.write("log", "executed.")
    systemexitstatus = os.execute(cmdline)
  else
    texio.write("log", "disabled.")
  end
end
function lastsystemstatus()
  local result = tonumber(systemexitstatus)
  if result then
    local x = math.floor(result / 256)
    tex.write(result - 256 * math.floor(result / 256))
  end
end
function lastsystemexit()
  local result = tonumber(systemexitstatus)
  if result then
    tex.write(math.floor(result / 256))
  end
end
function pipe(cmdline)
  local result
  systemexitstatus = nil
  texio.write_nl("log", "pipe(" .. cmdline ..") ")
  if io.popen then
    texio.write("log", "executed.")
    local handle = io.popen(cmdline, "r")
    if handle then
      result = handle:read("*a")
      handle:close()
    end
  else
    texio.write("log", "disabled.")
  end
  if result then
    tex.settoks(toks, result)
  else
    tex.settoks(toks, "")
  end
end
%    \end{macrocode}
%    \begin{macrocode}
%</lua>
%    \end{macrocode}
%
% \section{Test}
%
% \subsection{Catcode checks for loading}
%
%    \begin{macrocode}
%<*test1>
%    \end{macrocode}
%    \begin{macrocode}
\catcode`\{=1 %
\catcode`\}=2 %
\catcode`\#=6 %
\catcode`\@=11 %
\expandafter\ifx\csname count@\endcsname\relax
  \countdef\count@=255 %
\fi
\expandafter\ifx\csname @gobble\endcsname\relax
  \long\def\@gobble#1{}%
\fi
\expandafter\ifx\csname @firstofone\endcsname\relax
  \long\def\@firstofone#1{#1}%
\fi
\expandafter\ifx\csname loop\endcsname\relax
  \expandafter\@firstofone
\else
  \expandafter\@gobble
\fi
{%
  \def\loop#1\repeat{%
    \def\body{#1}%
    \iterate
  }%
  \def\iterate{%
    \body
      \let\next\iterate
    \else
      \let\next\relax
    \fi
    \next
  }%
  \let\repeat=\fi
}%
\def\RestoreCatcodes{}
\count@=0 %
\loop
  \edef\RestoreCatcodes{%
    \RestoreCatcodes
    \catcode\the\count@=\the\catcode\count@\relax
  }%
\ifnum\count@<255 %
  \advance\count@ 1 %
\repeat

\def\RangeCatcodeInvalid#1#2{%
  \count@=#1\relax
  \loop
    \catcode\count@=15 %
  \ifnum\count@<#2\relax
    \advance\count@ 1 %
  \repeat
}
\expandafter\ifx\csname LoadCommand\endcsname\relax
  \def\LoadCommand{\input pdftexcmds.sty\relax}%
\fi
\def\Test{%
  \RangeCatcodeInvalid{0}{47}%
  \RangeCatcodeInvalid{58}{64}%
  \RangeCatcodeInvalid{91}{96}%
  \RangeCatcodeInvalid{123}{255}%
  \catcode`\@=12 %
  \catcode`\\=0 %
  \catcode`\{=1 %
  \catcode`\}=2 %
  \catcode`\#=6 %
  \catcode`\[=12 %
  \catcode`\]=12 %
  \catcode`\%=14 %
  \catcode`\ =10 %
  \catcode13=5 %
  \LoadCommand
  \RestoreCatcodes
}
\Test
\csname @@end\endcsname
\end
%    \end{macrocode}
%    \begin{macrocode}
%</test1>
%    \end{macrocode}
%
% \section{Installation}
%
% \subsection{Download}
%
% \paragraph{Package.} This package is available on
% CTAN\footnote{\url{ftp://ftp.ctan.org/tex-archive/}}:
% \begin{description}
% \item[\CTAN{macros/latex/contrib/oberdiek/pdftexcmds.dtx}] The source file.
% \item[\CTAN{macros/latex/contrib/oberdiek/pdftexcmds.pdf}] Documentation.
% \end{description}
%
%
% \paragraph{Bundle.} All the packages of the bundle `oberdiek'
% are also available in a TDS compliant ZIP archive. There
% the packages are already unpacked and the documentation files
% are generated. The files and directories obey the TDS standard.
% \begin{description}
% \item[\CTAN{macros/latex/contrib/oberdiek/oberdiek-tds.zip}]
% \end{description}
% \emph{TDS} refers to the standard ``A Directory Structure
% for \TeX\ Files'' (\CTAN{tds/tds.pdf}). Directories
% with \xfile{texmf} in their name are usually organized this way.
%
% \subsection{Bundle installation}
%
% \paragraph{Unpacking.} Unpack the \xfile{oberdiek-tds.zip} in the
% TDS tree (also known as \xfile{texmf} tree) of your choice.
% Example (linux):
% \begin{quote}
%   |unzip oberdiek-tds.zip -d ~/texmf|
% \end{quote}
%
% \paragraph{Script installation.}
% Check the directory \xfile{TDS:scripts/oberdiek/} for
% scripts that need further installation steps.
% Package \xpackage{attachfile2} comes with the Perl script
% \xfile{pdfatfi.pl} that should be installed in such a way
% that it can be called as \texttt{pdfatfi}.
% Example (linux):
% \begin{quote}
%   |chmod +x scripts/oberdiek/pdfatfi.pl|\\
%   |cp scripts/oberdiek/pdfatfi.pl /usr/local/bin/|
% \end{quote}
%
% \subsection{Package installation}
%
% \paragraph{Unpacking.} The \xfile{.dtx} file is a self-extracting
% \docstrip\ archive. The files are extracted by running the
% \xfile{.dtx} through \plainTeX:
% \begin{quote}
%   \verb|tex pdftexcmds.dtx|
% \end{quote}
%
% \paragraph{TDS.} Now the different files must be moved into
% the different directories in your installation TDS tree
% (also known as \xfile{texmf} tree):
% \begin{quote}
% \def\t{^^A
% \begin{tabular}{@{}>{\ttfamily}l@{ $\rightarrow$ }>{\ttfamily}l@{}}
%   pdftexcmds.sty & tex/generic/oberdiek/pdftexcmds.sty\\
%   oberdiek.pdftexcmds.lua & scripts/oberdiek/oberdiek.pdftexcmds.lua\\
%   pdftexcmds.lua & scripts/oberdiek/pdftexcmds.lua\\
%   pdftexcmds.pdf & doc/latex/oberdiek/pdftexcmds.pdf\\
%   pdftexcmds.dtx & source/latex/oberdiek/pdftexcmds.dtx\\
% \end{tabular}^^A
% }^^A
% \sbox0{\t}^^A
% \ifdim\wd0>\linewidth
%   \begingroup
%     \advance\linewidth by\leftmargin
%     \advance\linewidth by\rightmargin
%   \edef\x{\endgroup
%     \def\noexpand\lw{\the\linewidth}^^A
%   }\x
%   \def\lwbox{^^A
%     \leavevmode
%     \hbox to \linewidth{^^A
%       \kern-\leftmargin\relax
%       \hss
%       \usebox0
%       \hss
%       \kern-\rightmargin\relax
%     }^^A
%   }^^A
%   \ifdim\wd0>\lw
%     \sbox0{\small\t}^^A
%     \ifdim\wd0>\linewidth
%       \ifdim\wd0>\lw
%         \sbox0{\footnotesize\t}^^A
%         \ifdim\wd0>\linewidth
%           \ifdim\wd0>\lw
%             \sbox0{\scriptsize\t}^^A
%             \ifdim\wd0>\linewidth
%               \ifdim\wd0>\lw
%                 \sbox0{\tiny\t}^^A
%                 \ifdim\wd0>\linewidth
%                   \lwbox
%                 \else
%                   \usebox0
%                 \fi
%               \else
%                 \lwbox
%               \fi
%             \else
%               \usebox0
%             \fi
%           \else
%             \lwbox
%           \fi
%         \else
%           \usebox0
%         \fi
%       \else
%         \lwbox
%       \fi
%     \else
%       \usebox0
%     \fi
%   \else
%     \lwbox
%   \fi
% \else
%   \usebox0
% \fi
% \end{quote}
% If you have a \xfile{docstrip.cfg} that configures and enables \docstrip's
% TDS installing feature, then some files can already be in the right
% place, see the documentation of \docstrip.
%
% \subsection{Refresh file name databases}
%
% If your \TeX~distribution
% (\teTeX, \mikTeX, \dots) relies on file name databases, you must refresh
% these. For example, \teTeX\ users run \verb|texhash| or
% \verb|mktexlsr|.
%
% \subsection{Some details for the interested}
%
% \paragraph{Attached source.}
%
% The PDF documentation on CTAN also includes the
% \xfile{.dtx} source file. It can be extracted by
% AcrobatReader 6 or higher. Another option is \textsf{pdftk},
% e.g. unpack the file into the current directory:
% \begin{quote}
%   \verb|pdftk pdftexcmds.pdf unpack_files output .|
% \end{quote}
%
% \paragraph{Unpacking with \LaTeX.}
% The \xfile{.dtx} chooses its action depending on the format:
% \begin{description}
% \item[\plainTeX:] Run \docstrip\ and extract the files.
% \item[\LaTeX:] Generate the documentation.
% \end{description}
% If you insist on using \LaTeX\ for \docstrip\ (really,
% \docstrip\ does not need \LaTeX), then inform the autodetect routine
% about your intention:
% \begin{quote}
%   \verb|latex \let\install=y\input{pdftexcmds.dtx}|
% \end{quote}
% Do not forget to quote the argument according to the demands
% of your shell.
%
% \paragraph{Generating the documentation.}
% You can use both the \xfile{.dtx} or the \xfile{.drv} to generate
% the documentation. The process can be configured by the
% configuration file \xfile{ltxdoc.cfg}. For instance, put this
% line into this file, if you want to have A4 as paper format:
% \begin{quote}
%   \verb|\PassOptionsToClass{a4paper}{article}|
% \end{quote}
% An example follows how to generate the
% documentation with pdf\LaTeX:
% \begin{quote}
%\begin{verbatim}
%pdflatex pdftexcmds.dtx
%makeindex -s gind.ist pdftexcmds.idx
%pdflatex pdftexcmds.dtx
%makeindex -s gind.ist pdftexcmds.idx
%pdflatex pdftexcmds.dtx
%\end{verbatim}
% \end{quote}
%
% \begin{History}
%   \begin{Version}{2007/11/11 v0.1}
%   \item
%     First version.
%   \end{Version}
%   \begin{Version}{2007/11/12 v0.2}
%   \item
%     Short description fixed.
%   \end{Version}
%   \begin{Version}{2007/12/12 v0.3}
%   \item
%     Organization of Lua code as module.
%   \end{Version}
% \end{History}
%
% \PrintIndex
%
% \Finale
\endinput

%        (quote the arguments according to the demands of your shell)
%
% Documentation:
%    (a) If pdftexcmds.drv is present:
%           latex pdftexcmds.drv
%    (b) Without pdftexcmds.drv:
%           latex pdftexcmds.dtx; ...
%    The class ltxdoc loads the configuration file ltxdoc.cfg
%    if available. Here you can specify further options, e.g.
%    use A4 as paper format:
%       \PassOptionsToClass{a4paper}{article}
%
%    Programm calls to get the documentation (example):
%       pdflatex pdftexcmds.dtx
%       makeindex -s gind.ist pdftexcmds.idx
%       pdflatex pdftexcmds.dtx
%       makeindex -s gind.ist pdftexcmds.idx
%       pdflatex pdftexcmds.dtx
%
% Installation:
%    TDS:tex/generic/oberdiek/pdftexcmds.sty
%    TDS:scripts/oberdiek/oberdiek.pdftexcmds.lua
%    TDS:scripts/oberdiek/pdftexcmds.lua
%    TDS:doc/latex/oberdiek/pdftexcmds.pdf
%    TDS:source/latex/oberdiek/pdftexcmds.dtx
%
%<*ignore>
\begingroup
  \def\x{LaTeX2e}%
\expandafter\endgroup
\ifcase 0\ifx\install y1\fi\expandafter
         \ifx\csname processbatchFile\endcsname\relax\else1\fi
         \ifx\fmtname\x\else 1\fi\relax
\else\csname fi\endcsname
%</ignore>
%<*install>
\input docstrip.tex
\Msg{************************************************************************}
\Msg{* Installation}
\Msg{* Package: pdftexcmds 2007/12/12 v0.3 LuaTeX support for pdfTeX utility functions (HO)}
\Msg{************************************************************************}

\keepsilent
\askforoverwritefalse

\let\MetaPrefix\relax
\preamble

This is a generated file.

Copyright (C) 2007 by Heiko Oberdiek <oberdiek@uni-freiburg.de>

This work may be distributed and/or modified under the
conditions of the LaTeX Project Public License, either
version 1.3 of this license or (at your option) any later
version. The latest version of this license is in
   http://www.latex-project.org/lppl.txt
and version 1.3 or later is part of all distributions of
LaTeX version 2005/12/01 or later.

This work has the LPPL maintenance status "maintained".

This Current Maintainer of this work is Heiko Oberdiek.

This work consists of the main source file pdftexcmds.dtx
and the derived files
   pdftexcmds.sty, pdftexcmds.pdf, pdftexcmds.ins, pdftexcmds.drv,
   oberdiek.pdftexcmds.lua, pdftexcmds.lua.

\endpreamble
\let\MetaPrefix\DoubleperCent

\generate{%
  \file{pdftexcmds.ins}{\from{pdftexcmds.dtx}{install}}%
  \file{pdftexcmds.drv}{\from{pdftexcmds.dtx}{driver}}%
  \usedir{tex/generic/oberdiek}%
  \file{pdftexcmds.sty}{\from{pdftexcmds.dtx}{package}}%
}
\def\MetaPrefix{-- }
\def\defaultpostamble{%
  \MetaPrefix^^J%
  \MetaPrefix\space End of File `\outFileName'.%
}
\def\currentpostamble{\defaultpostamble}%
\generate{%
  \usedir{scripts/oberdiek}%
  \file{oberdiek.pdftexcmds.lua}{\from{pdftexcmds.dtx}{lua}}%
  \file{pdftexcmds.lua}{\from{pdftexcmds.dtx}{lua}}%
}

\obeyspaces
\Msg{************************************************************************}
\Msg{*}
\Msg{* To finish the installation you have to move the following}
\Msg{* file into a directory searched by TeX:}
\Msg{*}
\Msg{*     pdftexcmds.sty}
\Msg{*}
\Msg{* And install the following script files:}
\Msg{*}
\Msg{*     oberdiek.pdftexcmds.lua, pdftexcmds.lua}
\Msg{*}
\Msg{* To produce the documentation run the file `pdftexcmds.drv'}
\Msg{* through LaTeX.}
\Msg{*}
\Msg{* Happy TeXing!}
\Msg{*}
\Msg{************************************************************************}

\endbatchfile
%</install>
%<*ignore>
\fi
%</ignore>
%<*driver>
\NeedsTeXFormat{LaTeX2e}
\ProvidesFile{pdftexcmds.drv}%
  [2007/12/12 v0.3 LuaTeX support for pdfTeX utility functions (HO)]%
\documentclass{ltxdoc}
\usepackage{holtxdoc}[2007/11/11]
\usepackage{paralist}
\begin{document}
  \DocInput{pdftexcmds.dtx}%
\end{document}
%</driver>
% \fi
%
% \CheckSum{552}
%
% \CharacterTable
%  {Upper-case    \A\B\C\D\E\F\G\H\I\J\K\L\M\N\O\P\Q\R\S\T\U\V\W\X\Y\Z
%   Lower-case    \a\b\c\d\e\f\g\h\i\j\k\l\m\n\o\p\q\r\s\t\u\v\w\x\y\z
%   Digits        \0\1\2\3\4\5\6\7\8\9
%   Exclamation   \!     Double quote  \"     Hash (number) \#
%   Dollar        \$     Percent       \%     Ampersand     \&
%   Acute accent  \'     Left paren    \(     Right paren   \)
%   Asterisk      \*     Plus          \+     Comma         \,
%   Minus         \-     Point         \.     Solidus       \/
%   Colon         \:     Semicolon     \;     Less than     \<
%   Equals        \=     Greater than  \>     Question mark \?
%   Commercial at \@     Left bracket  \[     Backslash     \\
%   Right bracket \]     Circumflex    \^     Underscore    \_
%   Grave accent  \`     Left brace    \{     Vertical bar  \|
%   Right brace   \}     Tilde         \~}
%
% \GetFileInfo{pdftexcmds.drv}
%
% \title{The \xpackage{pdftexcmds} package}
% \date{2007/12/12 v0.3}
% \author{Heiko Oberdiek\\\xemail{oberdiek@uni-freiburg.de}}
%
% \maketitle
%
% \begin{abstract}
% \LuaTeX\ provides most of the commands of \pdfTeX\ 1.40. However
% a number of utility functions are removed. This package tries to fill
% the gap and implements some of the missing primitive using Lua.
% \end{abstract}
%
% \tableofcontents
%
% \section{Documentation}
%
% Some primitives of \pdfTeX\ are not defined by \LuaTeX.
% This package implements macro based solutions using Lua code
% for the following missing \pdfTeX\ primitives;
% \begin{compactitem}
% \item \cs{pdfstrcmp}
% \item \cs{pdfunescapehex}
% \item \cs{pdfescapehex}
% \item \cs{pdfescapename}
% \item \cs{pdfescapestring}
% \item \cs{pdffilesize}
% \item \cs{pdffilemoddate}
% \item \cs{pdffiledump}
% \item \cs{pdfmdfivesum}
% \item |\immediate\write18|
% \end{compactitem}
% The original names of the primitives cannot be used:
% \begin{itemize}
% \item
% The syntax for their arguments cannot easily
% simulated by macros. The primitives using key words
% such as |file| (\cs{pdfmdfivesum}) or |offset| and |length|
% (\cs{pdffiledump}) and uses \meta{general text} for the other
% arguments. Using token registers assignments, \meta{general text} could
% be catched. However, the simulated primitives are expandable
% and register assignments would destroy this important property.
% (\meta{general text} allows something like |\expandafter\bgroup ...}|.)
% \item
% The original primitives can be expanded using one expansion step.
% The new macros need two expansion steps because of the additional
% macro expansion. Example:
% \begin{quote}
%   |\expandafter\foo\pdffilemoddate{file}|\\
%   vs.
%   |\expandafter\expandafter\expandafter\foo\pdf@filemoddate{file}|.
% \end{quote}
% \end{itemize}
%
% \LuaTeX\ isn't stable yet and thus the status of this package is
% \emph{experimental}. Feedback is welcome.
%
% \subsection{General principles}
%
% \begin{description}
% \item[Naming convention:]
%   Usually this package defines a macro |\pdf@|\meta{cmd} if
%   \pdfTeX\ provides |\pdf|\meta{cmd}.
% \item[Arguments:] The order of arguments in |\pdf@|\meta{cmd}
%   is the same as for the corresponding primitive of \pdfTeX.
%   The arguments are ordinary undelimited \TeX\ arguments,
%   no \meta{general text} and without additional keywords.
% \item[Expandibility:]
%   The macro |\pdf@|\meta{cmd} is expandable if the
%   corresponding \pdfTeX\ primitive has this property.
%   Exact two expansion steps are necessary (first is the macro
%   expansion).
% \item[Without \LuaTeX:]
%   The macros |\pdf@|\meta{cmd} are mapped to the commands
%   of \pdfTeX\ if they are available. Otherwise they are undefined.
% \end{description}
%
% \subsection{Macros}
%
% \begin{declcs}{pdf@strcmp} \M{stringA} \M{stringB}
% \end{declcs}
% Same as |\pdfstrcmp{|\meta{stringA}|}{|\meta{stringB}|}|.
%
% \begin{declcs}{pdf@unescapehex} \M{string}
% \end{declcs}
% Same as |\pdfunescapehex{|\meta{string}|}|.
% The argument is a byte string given in hexadecimal notation.
% The result are character tokens from 0 until 255 with
% catcode 12 and the space with catcode 10.
%
% \begin{declcs}{pdf@escapehex} \M{string}\\
%   \cs{pdf@escapestring} \M{string}\\
%   \cs{pdf@escapename} \M{string}
% \end{declcs}
% Same as the primitives of \pdfTeX. However \pdfTeX\ does not
% know about characters with codes 256 and larger. Thus the
% string is treated as byte string, characters with more than
% eight bits are ignored.
%
% \begin{declcs}{pdf@filesize} \M{filename}
% \end{declcs}
% Same as |\pdffilesize{|\meta{filename}|}|.
%
% \begin{declcs}{pdf@filemoddate} \M{filename}
% \end{declcs}
% Same as |\pdffilemoddate{|\meta{filename}|}|.
%
% \begin{declcs}{pdf@filedump} \M{offset} \M{length} \M{filename}
% \end{declcs}
% Same as |\pdffiledump offset| \meta{offset} |length| \meta{length}
% |{|\meta{filename}|}|. Both \meta{offset} and \meta{length} must
% not be empty, but must be a valid \TeX\ number.
%
% \begin{declcs}{pdf@mdfivesum} \M{string}
% \end{declcs}
% Same as |\pdfmdfivesum{|\meta{string}|}|. Keyword |file| is supported
% by macro \cs{pdf@filemdfivesum}.
%
% \begin{declcs}{pdf@filemdfivesum} \M{filename}
% \end{declcs}
% Same as |\pdfmdfivesum file{|\meta{filename}|}|.
%
% \begin{declcs}{pdf@shellescape}
% \end{declcs}
% Same as |\pdfshellescape|. It expands to |1| if external
% commands can be executed and |0| otherwise. In \pdfTeX\ external
% commands must be enabled first by command line option or
% configuration option. In \LuaTeX\ option |--safer| disables
% the execution of external commands.
%
% \begin{declcs}{pdf@system} \M{cmdline}
% \end{declcs}
% It is a wrapper for |\immediate\write18| in \pdfTeX\ or
% |os.execute| in \LuaTeX.
%
% In theory |os.execute|
% returns a status number. But its meaning is quite
% undefined. Are there some reliable properties?
% Does it make sense to provide an user interface to
% this status exit code?
%
% \subsubsection{Experimental}
%
% \begin{declcs}{pdf@unescapehexnative} \M{string}\\
%   \cs{pdf@escapehexnative} \M{string}\\
%   \cs{pdf@escapenamenative} \M{string}\\
%   \cs{pdf@mdfivesumnative} \M{string}
% \end{declcs}
% The variants without |native| in the macro name are supposed to
% be compatible with \pdfTeX. However characters with more than
% eight bits are not supported and are ignored. If \LuaTeX\ is
% running, then its UTF-8 coded strings are used. Thus the full
% unicode character range is supported. However the result
% differs from \pdfTeX\ for characters with eight or more bits.
%
% \begin{declcs}{pdf@pipe} \M{cmdline}
% \end{declcs}
% It calls \meta{cmdline} and returns the output of the external
% program in the usual manner as byte string (catcode 12, space with
% catcode 10). The Lua documentation says, that the used |io.popen|
% may not be available on all platforms. Then macro \cs{pdf@pipe}
% is undefined.
%
% \StopEventually{
% }
%
% \section{Implementation}
%
%    \begin{macrocode}
%<*package>
%    \end{macrocode}
%
% \subsection{Reload check and package identification}
%    Reload check, especially if the package is not used with \LaTeX.
%    \begin{macrocode}
\begingroup
  \catcode44 12 % ,
  \catcode45 12 % -
  \catcode46 12 % .
  \catcode58 12 % :
  \catcode64 11 % @
  \expandafter\let\expandafter\x\csname ver@pdftexcmds.sty\endcsname
  \ifcase 0%
    \ifx\x\relax % plain
    \else
      \ifx\x\empty % LaTeX
      \else
        1%
      \fi
    \fi
  \else
    \catcode35 6 % #
    \catcode123 1 % {
    \catcode125 2 % }
    \expandafter\ifx\csname PackageInfo\endcsname\relax
      \def\x#1#2{%
        \immediate\write-1{Package #1 Info: #2.}%
      }%
    \else
      \def\x#1#2{\PackageInfo{#1}{#2, stopped}}%
    \fi
    \x{pdftexcmds}{The package is already loaded}%
    \endgroup
    \expandafter\endinput
  \fi
\endgroup
%    \end{macrocode}
%    Package identification:
%    \begin{macrocode}
\begingroup
  \catcode35 6 % #
  \catcode40 12 % (
  \catcode41 12 % )
  \catcode44 12 % ,
  \catcode45 12 % -
  \catcode46 12 % .
  \catcode47 12 % /
  \catcode58 12 % :
  \catcode64 11 % @
  \catcode123 1 % {
  \catcode125 2 % }
  \expandafter\ifx\csname ProvidesPackage\endcsname\relax
    \def\x#1#2#3[#4]{\endgroup
      \immediate\write-1{Package: #3 #4}%
      \xdef#1{#4}%
    }%
  \else
    \def\x#1#2[#3]{\endgroup
      #2[{#3}]%
      \ifx#1\relax
        \xdef#1{#3}%
      \fi
    }%
  \fi
\expandafter\x\csname ver@pdftexcmds.sty\endcsname
\ProvidesPackage{pdftexcmds}%
  [2007/12/12 v0.3 LuaTeX support for pdfTeX utility functions (HO)]
%    \end{macrocode}
%
% \subsection{Catcodes}
%
%    \begin{macrocode}
\begingroup
  \catcode123 1 % {
  \catcode125 2 % }
  \def\x{\endgroup
    \expandafter\edef\csname pdftexcmds@AtEnd\endcsname{%
      \catcode35 \the\catcode35\relax
      \catcode64 \the\catcode64\relax
      \catcode123 \the\catcode123\relax
      \catcode125 \the\catcode125\relax
    }%
  }%
\x
\catcode35 6 % #
\catcode64 11 % @
\catcode123 1 % {
\catcode125 2 % }
\def\TMP@EnsureCode#1#2{%
  \edef\pdftexcmds@AtEnd{%
    \pdftexcmds@AtEnd
    \catcode#1 \the\catcode#1\relax
  }%
  \catcode#1 #2\relax
}
\TMP@EnsureCode{10}{12}% ^^J
\TMP@EnsureCode{33}{12}% !
\TMP@EnsureCode{34}{12}% "
\TMP@EnsureCode{39}{12}% '
\TMP@EnsureCode{40}{12}% (
\TMP@EnsureCode{41}{12}% )
\TMP@EnsureCode{42}{12}% *
\TMP@EnsureCode{43}{12}% +
\TMP@EnsureCode{44}{12}% ,
\TMP@EnsureCode{45}{12}% -
\TMP@EnsureCode{46}{12}% .
\TMP@EnsureCode{47}{12}% /
\TMP@EnsureCode{58}{12}% :
\TMP@EnsureCode{60}{12}% <
\TMP@EnsureCode{61}{12}% =
\TMP@EnsureCode{62}{12}% >
\TMP@EnsureCode{94}{7}% ^ (superscript)
\TMP@EnsureCode{95}{12}% _ (other)
\TMP@EnsureCode{126}{12}% ~ (other)
%    \end{macrocode}
%
% \subsection{Load package \xpackage{infwarerr}}
%
%    \begin{macrocode}
\begingroup\expandafter\expandafter\expandafter\endgroup
\expandafter\ifx\csname RequirePackage\endcsname\relax
  \input infwarerr.sty\relax
\else
  \RequirePackage{infwarerr}[2007/09/09]%
\fi
%    \end{macrocode}
%
% \subsection{Without \LuaTeX}
%
%    \begin{macrocode}
\begingroup\expandafter\expandafter\expandafter\endgroup
\expandafter\ifx\csname directlua\endcsname\relax
  \@PackageInfo{pdftexcmds}{LuaTeX not detected}%
  \def\pdftexcmds@nopdftex{%
    \@PackageInfoNoLine{pdftexcmds}{pdfTeX >= 1.30 not detected}%
    \let\pdftexcmds@nopdftex\relax
  }%
  \def\pdftexcmds@temp#1{%
    \begingroup\expandafter\expandafter\expandafter\endgroup
    \expandafter\ifx\csname pdf#1\endcsname\relax
      \pdftexcmds@nopdftex
    \else
      \expandafter\def\csname pdf@#1\expandafter\endcsname
      \expandafter##\expandafter{%
        \csname pdf#1\endcsname
      }%
    \fi
  }%
  \pdftexcmds@temp{strcmp}%
  \pdftexcmds@temp{escapehex}%
  \let\pdf@escapehexnative\pdf@escapehex
  \pdftexcmds@temp{unescapehex}%
  \let\pdf@unescapehexnative\pdf@unescapehex
  \pdftexcmds@temp{escapestring}%
  \pdftexcmds@temp{escapename}%
  \pdftexcmds@temp{filesize}%
  \pdftexcmds@temp{filemoddate}%
  \begingroup\expandafter\expandafter\expandafter\endgroup
  \expandafter\ifx\csname pdfshellescape\endcsname\relax
    \pdftexcmds@nopdftex
  \else
    \def\pdf@shellescape{%
      \pdfshellescape
    }%
  \fi
  \begingroup\expandafter\expandafter\expandafter\endgroup
  \expandafter\ifx\csname pdffiledump\endcsname\relax
    \pdftexcmds@nopdftex
  \else
    \def\pdf@filedump#1#2#3{%
      \pdffiledump offset#1 length#2{#3}%
    }%
  \fi
  \begingroup\expandafter\expandafter\expandafter\endgroup
  \expandafter\ifx\csname pdfmdfivesum\endcsname\relax
    \pdftexcmds@nopdftex
  \else
    \def\pdf@mdfivesum#{\pdfmdfivesum}%
    \let\pdf@mdfivesumnative\pdf@mdfivesum
    \def\pdf@filemdfivesum#{\pdfmdfivesum file}%
  \fi
  \def\pdf@system#{%
    \immediate\write18%
  }%
  \pdftexcmds@AtEnd
  \expandafter\endinput
\fi
%    \end{macrocode}
%
% \subsection{Load module}
%
%    \begin{macrocode}
\begingroup\expandafter\expandafter\expandafter\endgroup
\expandafter\ifx\csname RequirePackage\endcsname\relax
  \input luatex-loader.sty\relax
\else
  \RequirePackage{luatex-loader}[2007/12/12]%
\fi
\directlua0{%
  require("oberdiek.pdftexcmds")%
}
%    \end{macrocode}
%
% \subsection{Lua functions}
%
%    \begin{macro}{\pdftexcmds@toks}
%    \begin{macrocode}
\begingroup\expandafter\expandafter\expandafter\endgroup
\expandafter\ifx\csname newtoks\endcsname\relax
  \toksdef\pdftexcmds@toks=0 %
\else
  \csname newtoks\endcsname\pdftexcmds@toks
\fi
%    \end{macrocode}
%    \end{macro}
%
%    \begin{macro}{\pdf@strcmp}
%    \begin{macrocode}
\long\def\pdf@strcmp#1#2{%
  \directlua0{%
    oberdiek.pdftexcmds.strcmp("\luaescapestring{#1}",%
        "\luaescapestring{#2}")%
  }%
}%
%    \end{macrocode}
%    \end{macro}
%    \begin{macro}{\pdf@escapehex}
%    \begin{macrocode}
\long\def\pdf@escapehex#1{%
  \directlua0{%
    oberdiek.pdftexcmds.escapehex("\luaescapestring{#1}", "byte")%
  }%
}%
%    \end{macrocode}
%    \end{macro}
%    \begin{macro}{\pdf@escapehexnative}
%    \begin{macrocode}
\long\def\pdf@escapehexnative#1{%
  \directlua0{%
    oberdiek.pdftexcmds.escapehex("\luaescapestring{#1}")%
  }%
}%
%    \end{macrocode}
%    \end{macro}
%    \begin{macro}{\pdf@unescapehex}
%    \begin{macrocode}
\def\pdf@unescapehex#1{%
  \the\expandafter\pdftexcmds@toks
  \directlua0{%
    oberdiek.pdftexcmds.toks="pdftexcmds@toks"%
    oberdiek.pdftexcmds.unescapehex("\luaescapestring{#1}", "byte")%
  }%
}%
%    \end{macrocode}
%    \end{macro}
%    \begin{macro}{\pdf@unescapehexnative}
%    \begin{macrocode}
\def\pdf@unescapehexnative#1{%
  \the\expandafter\pdftexcmds@toks
  \directlua0{%
    oberdiek.pdftexcmds.toks="pdftexcmds@toks"%
    oberdiek.pdftexcmds.unescapehex("\luaescapestring{#1}")%
  }%
}%
%    \end{macrocode}
%    \end{macro}
%    \begin{macro}{\pdf@escapestring}
%    \begin{macrocode}
\long\def\pdf@escapestring#1{%
  \directlua0{%
    oberdiek.pdftexcmds.escapestring("\luaescapestring{#1}", "byte")%
  }%
}
%    \end{macrocode}
%    \end{macro}
%    \begin{macro}{\pdf@escapename}
%    \begin{macrocode}
\long\def\pdf@escapename#1{%
  \directlua0{%
    oberdiek.pdftexcmds.escapename("\luaescapestring{#1}", "byte")%
  }%
}
%    \end{macrocode}
%    \end{macro}
%    \begin{macro}{\pdf@escapenamenative}
%    \begin{macrocode}
\long\def\pdf@escapenamenative#1{%
  \directlua0{%
    oberdiek.pdftexcmds.escapename("\luaescapestring{#1}")%
  }%
}
%    \end{macrocode}
%    \end{macro}
%    \begin{macro}{\pdf@filesize}
%    \begin{macrocode}
\def\pdf@filesize#1{%
  \directlua0{%
    oberdiek.pdftexcmds.filesize("\luaescapestring{#1}")%
  }%
}
%    \end{macrocode}
%    \end{macro}
%    \begin{macro}{\pdf@filemoddate}
%    \begin{macrocode}
\def\pdf@filemoddate#1{%
  \directlua0{%
    oberdiek.pdftexcmds.filemoddate("\luaescapestring{#1}")%
  }%
}
%    \end{macrocode}
%    \end{macro}
%    \begin{macro}{\pdf@filedump}
%    \begin{macrocode}
\def\pdf@filedump#1#2#3{%
  \directlua0{%
    oberdiek.pdftexcmds.filedump("\luaescapestring{\number#1}",%
        "\luaescapestring{\number#2}",%
        "\luaescapestring{#3}")%
  }%
}%
%    \end{macrocode}
%    \end{macro}
%    \begin{macro}{\pdf@mdfivesum}
%    \begin{macrocode}
\long\def\pdf@mdfivesum#1{%
  \directlua0{%
    oberdiek.pdftexcmds.mdfivesum("\luaescapestring{#1}", "byte")%
  }%
}%
%    \end{macrocode}
%    \end{macro}
%    \begin{macro}{\pdf@mdfivesumnative}
%    \begin{macrocode}
\long\def\pdf@mdfivesumnative#1{%
  \directlua0{%
    oberdiek.pdftexcmds.mdfivesum("\luaescapestring{#1}")%
  }%
}%
%    \end{macrocode}
%    \end{macro}
%    \begin{macro}{\pdf@filemdfivesum}
%    \begin{macrocode}
\def\pdf@filemdfivesum#1{%
  \directlua0{%
    oberdiek.pdftexcmds.filemdfivesum("\luaescapestring{#1}")%
  }%
}%
%    \end{macrocode}
%    \end{macro}
%
%    \begin{macro}{\pdf@shellescape}
%    \begin{macrocode}
\def\pdf@shellescape{%
  \directlua0{%
    oberdiek.pdftexcmds.shellescape()%
  }%
}
%    \end{macrocode}
%    \end{macro}
%
%    \begin{macro}{\pdf@system}
%    \begin{macrocode}
\def\pdf@system#1{%
  \directlua0{%
    oberdiek.pdftexcmds.system("\luaescapestring{#1}")%
  }%
}
%    \end{macrocode}
%    \end{macro}
%
%    \begin{macro}{\pdf@lastsystemstatus}
%    \begin{macrocode}
\def\pdf@lastsystemstatus{%
  \directlua0{%
    oberdiek.pdftexcmds.lastsystemstatus()%
  }%
}
%    \end{macrocode}
%    \end{macro}
%    \begin{macro}{\pdf@lastsystemexit}
%    \begin{macrocode}
\def\pdf@lastsystemexit{%
  \directlua0{%
    oberdiek.pdftexcmds.lastsystemexit()%
  }%
}
%    \end{macrocode}
%    \end{macro}
%
%    \begin{macro}{\pdf@pipe}
%    Check availability of |io.popen| first.
%    \begin{macrocode}
\ifnum0%
    \directlua0{%
      if io.popen then %
        tex.write("1")%
      end%
    }%
    =1 %
  \def\pdf@pipe#1{%
    \the\expandafter\pdftexcmds@toks
    \directlua0{%
      oberdiek.pdftexcmds.toks="pdftexcmds@toks"%
      oberdiek.pdftexcmds.pipe("\luaescapestring{#1}")%
    }%
  }%
\fi
%    \end{macrocode}
%    \end{macro}
%
%    \begin{macrocode}
\pdftexcmds@AtEnd
%</package>
%    \end{macrocode}
%
% \subsection{Lua module}
%
%    \begin{macrocode}
%<*lua>
%    \end{macrocode}
%
%    \begin{macrocode}
module("oberdiek.pdftexcmds", package.seeall)
local systemexitstatus
function strcmp(A, B)
  if A == B then
    tex.write("0")
  elseif A < B then
    tex.write("-1")
  else
    tex.write("1")
  end
end
local function utf8_to_byte(str)
  local i = 0
  local n = string.len(str)
  local t = {}
  while i < n do
    i = i + 1
    local a = string.byte(str, i)
    if a < 128 then
      table.insert(t, string.char(a))
    else
      if a >= 192 and i < n then
        i = i + 1
        local b = string.byte(str, i)
        if b < 128 or b >= 192 then
          i = i - 1
        elseif a == 194 then
          table.insert(t, string.char(b))
        elseif a == 195 then
          table.insert(t, string.char(b + 64))
        end
      end
    end
  end
  return table.concat(t)
end
function escapehex(str, mode)
  if mode == "byte" then
    str = utf8_to_byte(str)
  end
  tex.write((string.gsub(str, ".",
    function (ch)
      return string.format("%02X", string.byte(ch))
    end
  )))
end
%    \end{macrocode}
%    See procedure |unescapehex| in file \xfile{utils.c} of \pdfTeX.
%    Caution: |tex.write| ignores leading spaces.
%    \begin{macrocode}
function unescapehex(str, mode)
  local a = 0
  local first = true
  local result = {}
  for i = 1, string.len(str), 1 do
    local ch = string.byte(str, i)
    if ch >= 48 and ch <= 57 then
      ch = ch - 48
    elseif ch >= 65 and ch <= 70 then
      ch = ch - 55
    elseif ch >= 97 and ch <= 102 then
      ch = ch - 87
    else
      ch = nil
    end
    if ch then
      if first then
        a = ch * 16
        first = false
      else
        table.insert(result, a + ch)
        first = true
      end
    end
  end
  if not first then
    table.insert(result, a)
  end
  if mode == "byte" then
    local utf8 = {}
    for i, a in ipairs(result) do
      if a < 128 then
        table.insert(utf8, a)
      else
        if a < 192 then
          table.insert(utf8, 194)
          a = a - 128
        else
          table.insert(utf8, 195)
          a = a - 192
        end
        table.insert(utf8, a + 128)
      end
    end
    result = utf8
  end
  tex.settoks(toks, string.char(unpack(result)))
end
%    \end{macrocode}
%    See procedure |escapestring| in file \xfile{utils.c} of \pdfTeX.
%    \begin{macrocode}
function escapestring(str, mode)
  if mode == "byte" then
    str = utf8_to_byte(str)
  end
  tex.write((string.gsub(str, ".",
    function (ch)
      local b = string.byte(ch)
      if b < 33 or b > 126 then
        return string.format("\\%.3o", b)
      end
      if b == 40 or b == 41 or b == 92 then
        return "\\" .. ch
      end
%    \end{macrocode}
%    Lua 5.1 returns the match in case of return value |nil|.
%    \begin{macrocode}
      return nil
    end
  )))
end
%    \end{macrocode}
%    See procedure |escapename| in file \xfile{utils.c} of \pdfTeX.
%    \begin{macrocode}
function escapename(str, mode)
  if mode == "byte" then
    str = utf8_to_byte(str)
  end
  tex.write((string.gsub(str, ".",
    function (ch)
      local b = string.byte(ch)
      if b == 0 then
%    \end{macrocode}
%    In Lua 5.0 |nil| could be used for the empty string,
%    But |nil| returns the match in Lua 5.1, thus we use
%    the empty string explicitly.
%    \begin{macrocode}
        return ""
      end
      if b <= 32 or b >= 127
          or b == 35 or b == 37 or b == 40 or b == 41
          or b == 47 or b == 60 or b == 62 or b == 91
          or b == 93 or b == 123 or b == 125 then
        return string.format("#%.2X", b)
      else
%    \end{macrocode}
%    Lua 5.1 returns the match in case of return value |nil|.
%    \begin{macrocode}
        return nil
      end
    end
  )))
end
function filesize(filename)
  local foundfile = kpse.find_file(filename, "tex", true)
  if foundfile then
    local size = lfs.attributes(foundfile, "size")
    if size then
      tex.write(size)
    end
  end
end
%    \end{macrocode}
%    See procedure |makepdftime| in file \xfile{utils.c} of \pdfTeX.
%    \begin{macrocode}
function filemoddate(filename)
  local foundfile = kpse.find_file(filename, "tex", true)
  if foundfile then
    local date = lfs.attributes(foundfile, "modification")
    if date then
      local d = os.date("*t", date)
      if d.sec >= 60 then
        d.sec = 59
      end
      local u = os.date("!*t", date)
      local off = 60 * (d.hour - u.hour) + d.min - u.min
      if d.year ~= u.year then
        if d.year > u.year then
          off = off + 1440
        else
          off = off - 1440
        end
      elseif d.yday ~= u.yday then
        if d.yday > u.yday then
          off = off + 1440
        else
          off = off - 1440
        end
      end
      local timezone
      if off == 0 then
        timezone = "Z"
      else
        local hours = math.floor(off / 60)
        local mins = math.abs(off - hours * 60)
        timezone = string.format("%+03d'%02d'", hours, mins)
      end
      tex.write(string.format("D:%04d%02d%02d%02d%02d%02d%s",
          d.year, d.month, d.day, d.hour, d.min, d.sec, timezone))
    end
  end
end
function filedump(offset, length, filename)
  length = tonumber(length)
  if length and length > 0 then
    local foundfile = kpse.find_file(filename, "tex", true)
    if foundfile then
      offset = tonumber(offset)
      if not offset then
        offset = 0
      end
      local filehandle = io.open(foundfile, "r")
      if filehandle then
        if offset > 0 then
          filehandle:seek("set", offset)
        end
        local dump = filehandle:read(length)
        escapehex(dump)
      end
    end
  end
end
function mdfivesum(str, mode)
  if mode == "byte" then
    str = utf8_to_byte(str)
  end
  escapehex(md5.sum(str))
end
function filemdfivesum(filename)
  local foundfile = kpse.find_file(filename, "tex", true)
  if foundfile then
    local filehandle = io.open(foundfile, "r")
    if filehandle then
      local contents = filehandle:read("*a")
      escapehex(md5.sum(contents))
    end
  end
end
function shellescape()
  if os.execute then
    tex.write("1")
  else
    tex.write("0")
  end
end
function system(cmdline)
  systemexitstatus = nil
  texio.write_nl("log", "system(" .. cmdline .. ") ")
  if os.execute then
    texio.write("log", "executed.")
    systemexitstatus = os.execute(cmdline)
  else
    texio.write("log", "disabled.")
  end
end
function lastsystemstatus()
  local result = tonumber(systemexitstatus)
  if result then
    local x = math.floor(result / 256)
    tex.write(result - 256 * math.floor(result / 256))
  end
end
function lastsystemexit()
  local result = tonumber(systemexitstatus)
  if result then
    tex.write(math.floor(result / 256))
  end
end
function pipe(cmdline)
  local result
  systemexitstatus = nil
  texio.write_nl("log", "pipe(" .. cmdline ..") ")
  if io.popen then
    texio.write("log", "executed.")
    local handle = io.popen(cmdline, "r")
    if handle then
      result = handle:read("*a")
      handle:close()
    end
  else
    texio.write("log", "disabled.")
  end
  if result then
    tex.settoks(toks, result)
  else
    tex.settoks(toks, "")
  end
end
%    \end{macrocode}
%    \begin{macrocode}
%</lua>
%    \end{macrocode}
%
% \section{Test}
%
% \subsection{Catcode checks for loading}
%
%    \begin{macrocode}
%<*test1>
%    \end{macrocode}
%    \begin{macrocode}
\catcode`\{=1 %
\catcode`\}=2 %
\catcode`\#=6 %
\catcode`\@=11 %
\expandafter\ifx\csname count@\endcsname\relax
  \countdef\count@=255 %
\fi
\expandafter\ifx\csname @gobble\endcsname\relax
  \long\def\@gobble#1{}%
\fi
\expandafter\ifx\csname @firstofone\endcsname\relax
  \long\def\@firstofone#1{#1}%
\fi
\expandafter\ifx\csname loop\endcsname\relax
  \expandafter\@firstofone
\else
  \expandafter\@gobble
\fi
{%
  \def\loop#1\repeat{%
    \def\body{#1}%
    \iterate
  }%
  \def\iterate{%
    \body
      \let\next\iterate
    \else
      \let\next\relax
    \fi
    \next
  }%
  \let\repeat=\fi
}%
\def\RestoreCatcodes{}
\count@=0 %
\loop
  \edef\RestoreCatcodes{%
    \RestoreCatcodes
    \catcode\the\count@=\the\catcode\count@\relax
  }%
\ifnum\count@<255 %
  \advance\count@ 1 %
\repeat

\def\RangeCatcodeInvalid#1#2{%
  \count@=#1\relax
  \loop
    \catcode\count@=15 %
  \ifnum\count@<#2\relax
    \advance\count@ 1 %
  \repeat
}
\expandafter\ifx\csname LoadCommand\endcsname\relax
  \def\LoadCommand{\input pdftexcmds.sty\relax}%
\fi
\def\Test{%
  \RangeCatcodeInvalid{0}{47}%
  \RangeCatcodeInvalid{58}{64}%
  \RangeCatcodeInvalid{91}{96}%
  \RangeCatcodeInvalid{123}{255}%
  \catcode`\@=12 %
  \catcode`\\=0 %
  \catcode`\{=1 %
  \catcode`\}=2 %
  \catcode`\#=6 %
  \catcode`\[=12 %
  \catcode`\]=12 %
  \catcode`\%=14 %
  \catcode`\ =10 %
  \catcode13=5 %
  \LoadCommand
  \RestoreCatcodes
}
\Test
\csname @@end\endcsname
\end
%    \end{macrocode}
%    \begin{macrocode}
%</test1>
%    \end{macrocode}
%
% \section{Installation}
%
% \subsection{Download}
%
% \paragraph{Package.} This package is available on
% CTAN\footnote{\url{ftp://ftp.ctan.org/tex-archive/}}:
% \begin{description}
% \item[\CTAN{macros/latex/contrib/oberdiek/pdftexcmds.dtx}] The source file.
% \item[\CTAN{macros/latex/contrib/oberdiek/pdftexcmds.pdf}] Documentation.
% \end{description}
%
%
% \paragraph{Bundle.} All the packages of the bundle `oberdiek'
% are also available in a TDS compliant ZIP archive. There
% the packages are already unpacked and the documentation files
% are generated. The files and directories obey the TDS standard.
% \begin{description}
% \item[\CTAN{macros/latex/contrib/oberdiek/oberdiek-tds.zip}]
% \end{description}
% \emph{TDS} refers to the standard ``A Directory Structure
% for \TeX\ Files'' (\CTAN{tds/tds.pdf}). Directories
% with \xfile{texmf} in their name are usually organized this way.
%
% \subsection{Bundle installation}
%
% \paragraph{Unpacking.} Unpack the \xfile{oberdiek-tds.zip} in the
% TDS tree (also known as \xfile{texmf} tree) of your choice.
% Example (linux):
% \begin{quote}
%   |unzip oberdiek-tds.zip -d ~/texmf|
% \end{quote}
%
% \paragraph{Script installation.}
% Check the directory \xfile{TDS:scripts/oberdiek/} for
% scripts that need further installation steps.
% Package \xpackage{attachfile2} comes with the Perl script
% \xfile{pdfatfi.pl} that should be installed in such a way
% that it can be called as \texttt{pdfatfi}.
% Example (linux):
% \begin{quote}
%   |chmod +x scripts/oberdiek/pdfatfi.pl|\\
%   |cp scripts/oberdiek/pdfatfi.pl /usr/local/bin/|
% \end{quote}
%
% \subsection{Package installation}
%
% \paragraph{Unpacking.} The \xfile{.dtx} file is a self-extracting
% \docstrip\ archive. The files are extracted by running the
% \xfile{.dtx} through \plainTeX:
% \begin{quote}
%   \verb|tex pdftexcmds.dtx|
% \end{quote}
%
% \paragraph{TDS.} Now the different files must be moved into
% the different directories in your installation TDS tree
% (also known as \xfile{texmf} tree):
% \begin{quote}
% \def\t{^^A
% \begin{tabular}{@{}>{\ttfamily}l@{ $\rightarrow$ }>{\ttfamily}l@{}}
%   pdftexcmds.sty & tex/generic/oberdiek/pdftexcmds.sty\\
%   oberdiek.pdftexcmds.lua & scripts/oberdiek/oberdiek.pdftexcmds.lua\\
%   pdftexcmds.lua & scripts/oberdiek/pdftexcmds.lua\\
%   pdftexcmds.pdf & doc/latex/oberdiek/pdftexcmds.pdf\\
%   pdftexcmds.dtx & source/latex/oberdiek/pdftexcmds.dtx\\
% \end{tabular}^^A
% }^^A
% \sbox0{\t}^^A
% \ifdim\wd0>\linewidth
%   \begingroup
%     \advance\linewidth by\leftmargin
%     \advance\linewidth by\rightmargin
%   \edef\x{\endgroup
%     \def\noexpand\lw{\the\linewidth}^^A
%   }\x
%   \def\lwbox{^^A
%     \leavevmode
%     \hbox to \linewidth{^^A
%       \kern-\leftmargin\relax
%       \hss
%       \usebox0
%       \hss
%       \kern-\rightmargin\relax
%     }^^A
%   }^^A
%   \ifdim\wd0>\lw
%     \sbox0{\small\t}^^A
%     \ifdim\wd0>\linewidth
%       \ifdim\wd0>\lw
%         \sbox0{\footnotesize\t}^^A
%         \ifdim\wd0>\linewidth
%           \ifdim\wd0>\lw
%             \sbox0{\scriptsize\t}^^A
%             \ifdim\wd0>\linewidth
%               \ifdim\wd0>\lw
%                 \sbox0{\tiny\t}^^A
%                 \ifdim\wd0>\linewidth
%                   \lwbox
%                 \else
%                   \usebox0
%                 \fi
%               \else
%                 \lwbox
%               \fi
%             \else
%               \usebox0
%             \fi
%           \else
%             \lwbox
%           \fi
%         \else
%           \usebox0
%         \fi
%       \else
%         \lwbox
%       \fi
%     \else
%       \usebox0
%     \fi
%   \else
%     \lwbox
%   \fi
% \else
%   \usebox0
% \fi
% \end{quote}
% If you have a \xfile{docstrip.cfg} that configures and enables \docstrip's
% TDS installing feature, then some files can already be in the right
% place, see the documentation of \docstrip.
%
% \subsection{Refresh file name databases}
%
% If your \TeX~distribution
% (\teTeX, \mikTeX, \dots) relies on file name databases, you must refresh
% these. For example, \teTeX\ users run \verb|texhash| or
% \verb|mktexlsr|.
%
% \subsection{Some details for the interested}
%
% \paragraph{Attached source.}
%
% The PDF documentation on CTAN also includes the
% \xfile{.dtx} source file. It can be extracted by
% AcrobatReader 6 or higher. Another option is \textsf{pdftk},
% e.g. unpack the file into the current directory:
% \begin{quote}
%   \verb|pdftk pdftexcmds.pdf unpack_files output .|
% \end{quote}
%
% \paragraph{Unpacking with \LaTeX.}
% The \xfile{.dtx} chooses its action depending on the format:
% \begin{description}
% \item[\plainTeX:] Run \docstrip\ and extract the files.
% \item[\LaTeX:] Generate the documentation.
% \end{description}
% If you insist on using \LaTeX\ for \docstrip\ (really,
% \docstrip\ does not need \LaTeX), then inform the autodetect routine
% about your intention:
% \begin{quote}
%   \verb|latex \let\install=y% \iffalse meta-comment
%
% Copyright (C) 2007 by Heiko Oberdiek <oberdiek@uni-freiburg.de>
%
% This work may be distributed and/or modified under the
% conditions of the LaTeX Project Public License, either
% version 1.3 of this license or (at your option) any later
% version. The latest version of this license is in
%    http://www.latex-project.org/lppl.txt
% and version 1.3 or later is part of all distributions of
% LaTeX version 2005/12/01 or later.
%
% This work has the LPPL maintenance status "maintained".
%
% This Current Maintainer of this work is Heiko Oberdiek.
%
% This work consists of the main source file pdftexcmds.dtx
% and the derived files
%    pdftexcmds.sty, pdftexcmds.pdf, pdftexcmds.ins, pdftexcmds.drv,
%    oberdiek.pdftexcmds.lua, pdftexcmds.lua.
%
% Distribution:
%    CTAN:macros/latex/contrib/oberdiek/pdftexcmds.dtx
%    CTAN:macros/latex/contrib/oberdiek/pdftexcmds.pdf
%
% Unpacking:
%    (a) If pdftexcmds.ins is present:
%           tex pdftexcmds.ins
%    (b) Without pdftexcmds.ins:
%           tex pdftexcmds.dtx
%    (c) If you insist on using LaTeX
%           latex \let\install=y\input{pdftexcmds.dtx}
%        (quote the arguments according to the demands of your shell)
%
% Documentation:
%    (a) If pdftexcmds.drv is present:
%           latex pdftexcmds.drv
%    (b) Without pdftexcmds.drv:
%           latex pdftexcmds.dtx; ...
%    The class ltxdoc loads the configuration file ltxdoc.cfg
%    if available. Here you can specify further options, e.g.
%    use A4 as paper format:
%       \PassOptionsToClass{a4paper}{article}
%
%    Programm calls to get the documentation (example):
%       pdflatex pdftexcmds.dtx
%       makeindex -s gind.ist pdftexcmds.idx
%       pdflatex pdftexcmds.dtx
%       makeindex -s gind.ist pdftexcmds.idx
%       pdflatex pdftexcmds.dtx
%
% Installation:
%    TDS:tex/generic/oberdiek/pdftexcmds.sty
%    TDS:scripts/oberdiek/oberdiek.pdftexcmds.lua
%    TDS:scripts/oberdiek/pdftexcmds.lua
%    TDS:doc/latex/oberdiek/pdftexcmds.pdf
%    TDS:source/latex/oberdiek/pdftexcmds.dtx
%
%<*ignore>
\begingroup
  \def\x{LaTeX2e}%
\expandafter\endgroup
\ifcase 0\ifx\install y1\fi\expandafter
         \ifx\csname processbatchFile\endcsname\relax\else1\fi
         \ifx\fmtname\x\else 1\fi\relax
\else\csname fi\endcsname
%</ignore>
%<*install>
\input docstrip.tex
\Msg{************************************************************************}
\Msg{* Installation}
\Msg{* Package: pdftexcmds 2007/12/12 v0.3 LuaTeX support for pdfTeX utility functions (HO)}
\Msg{************************************************************************}

\keepsilent
\askforoverwritefalse

\let\MetaPrefix\relax
\preamble

This is a generated file.

Copyright (C) 2007 by Heiko Oberdiek <oberdiek@uni-freiburg.de>

This work may be distributed and/or modified under the
conditions of the LaTeX Project Public License, either
version 1.3 of this license or (at your option) any later
version. The latest version of this license is in
   http://www.latex-project.org/lppl.txt
and version 1.3 or later is part of all distributions of
LaTeX version 2005/12/01 or later.

This work has the LPPL maintenance status "maintained".

This Current Maintainer of this work is Heiko Oberdiek.

This work consists of the main source file pdftexcmds.dtx
and the derived files
   pdftexcmds.sty, pdftexcmds.pdf, pdftexcmds.ins, pdftexcmds.drv,
   oberdiek.pdftexcmds.lua, pdftexcmds.lua.

\endpreamble
\let\MetaPrefix\DoubleperCent

\generate{%
  \file{pdftexcmds.ins}{\from{pdftexcmds.dtx}{install}}%
  \file{pdftexcmds.drv}{\from{pdftexcmds.dtx}{driver}}%
  \usedir{tex/generic/oberdiek}%
  \file{pdftexcmds.sty}{\from{pdftexcmds.dtx}{package}}%
}
\def\MetaPrefix{-- }
\def\defaultpostamble{%
  \MetaPrefix^^J%
  \MetaPrefix\space End of File `\outFileName'.%
}
\def\currentpostamble{\defaultpostamble}%
\generate{%
  \usedir{scripts/oberdiek}%
  \file{oberdiek.pdftexcmds.lua}{\from{pdftexcmds.dtx}{lua}}%
  \file{pdftexcmds.lua}{\from{pdftexcmds.dtx}{lua}}%
}

\obeyspaces
\Msg{************************************************************************}
\Msg{*}
\Msg{* To finish the installation you have to move the following}
\Msg{* file into a directory searched by TeX:}
\Msg{*}
\Msg{*     pdftexcmds.sty}
\Msg{*}
\Msg{* And install the following script files:}
\Msg{*}
\Msg{*     oberdiek.pdftexcmds.lua, pdftexcmds.lua}
\Msg{*}
\Msg{* To produce the documentation run the file `pdftexcmds.drv'}
\Msg{* through LaTeX.}
\Msg{*}
\Msg{* Happy TeXing!}
\Msg{*}
\Msg{************************************************************************}

\endbatchfile
%</install>
%<*ignore>
\fi
%</ignore>
%<*driver>
\NeedsTeXFormat{LaTeX2e}
\ProvidesFile{pdftexcmds.drv}%
  [2007/12/12 v0.3 LuaTeX support for pdfTeX utility functions (HO)]%
\documentclass{ltxdoc}
\usepackage{holtxdoc}[2007/11/11]
\usepackage{paralist}
\begin{document}
  \DocInput{pdftexcmds.dtx}%
\end{document}
%</driver>
% \fi
%
% \CheckSum{552}
%
% \CharacterTable
%  {Upper-case    \A\B\C\D\E\F\G\H\I\J\K\L\M\N\O\P\Q\R\S\T\U\V\W\X\Y\Z
%   Lower-case    \a\b\c\d\e\f\g\h\i\j\k\l\m\n\o\p\q\r\s\t\u\v\w\x\y\z
%   Digits        \0\1\2\3\4\5\6\7\8\9
%   Exclamation   \!     Double quote  \"     Hash (number) \#
%   Dollar        \$     Percent       \%     Ampersand     \&
%   Acute accent  \'     Left paren    \(     Right paren   \)
%   Asterisk      \*     Plus          \+     Comma         \,
%   Minus         \-     Point         \.     Solidus       \/
%   Colon         \:     Semicolon     \;     Less than     \<
%   Equals        \=     Greater than  \>     Question mark \?
%   Commercial at \@     Left bracket  \[     Backslash     \\
%   Right bracket \]     Circumflex    \^     Underscore    \_
%   Grave accent  \`     Left brace    \{     Vertical bar  \|
%   Right brace   \}     Tilde         \~}
%
% \GetFileInfo{pdftexcmds.drv}
%
% \title{The \xpackage{pdftexcmds} package}
% \date{2007/12/12 v0.3}
% \author{Heiko Oberdiek\\\xemail{oberdiek@uni-freiburg.de}}
%
% \maketitle
%
% \begin{abstract}
% \LuaTeX\ provides most of the commands of \pdfTeX\ 1.40. However
% a number of utility functions are removed. This package tries to fill
% the gap and implements some of the missing primitive using Lua.
% \end{abstract}
%
% \tableofcontents
%
% \section{Documentation}
%
% Some primitives of \pdfTeX\ are not defined by \LuaTeX.
% This package implements macro based solutions using Lua code
% for the following missing \pdfTeX\ primitives;
% \begin{compactitem}
% \item \cs{pdfstrcmp}
% \item \cs{pdfunescapehex}
% \item \cs{pdfescapehex}
% \item \cs{pdfescapename}
% \item \cs{pdfescapestring}
% \item \cs{pdffilesize}
% \item \cs{pdffilemoddate}
% \item \cs{pdffiledump}
% \item \cs{pdfmdfivesum}
% \item |\immediate\write18|
% \end{compactitem}
% The original names of the primitives cannot be used:
% \begin{itemize}
% \item
% The syntax for their arguments cannot easily
% simulated by macros. The primitives using key words
% such as |file| (\cs{pdfmdfivesum}) or |offset| and |length|
% (\cs{pdffiledump}) and uses \meta{general text} for the other
% arguments. Using token registers assignments, \meta{general text} could
% be catched. However, the simulated primitives are expandable
% and register assignments would destroy this important property.
% (\meta{general text} allows something like |\expandafter\bgroup ...}|.)
% \item
% The original primitives can be expanded using one expansion step.
% The new macros need two expansion steps because of the additional
% macro expansion. Example:
% \begin{quote}
%   |\expandafter\foo\pdffilemoddate{file}|\\
%   vs.
%   |\expandafter\expandafter\expandafter\foo\pdf@filemoddate{file}|.
% \end{quote}
% \end{itemize}
%
% \LuaTeX\ isn't stable yet and thus the status of this package is
% \emph{experimental}. Feedback is welcome.
%
% \subsection{General principles}
%
% \begin{description}
% \item[Naming convention:]
%   Usually this package defines a macro |\pdf@|\meta{cmd} if
%   \pdfTeX\ provides |\pdf|\meta{cmd}.
% \item[Arguments:] The order of arguments in |\pdf@|\meta{cmd}
%   is the same as for the corresponding primitive of \pdfTeX.
%   The arguments are ordinary undelimited \TeX\ arguments,
%   no \meta{general text} and without additional keywords.
% \item[Expandibility:]
%   The macro |\pdf@|\meta{cmd} is expandable if the
%   corresponding \pdfTeX\ primitive has this property.
%   Exact two expansion steps are necessary (first is the macro
%   expansion).
% \item[Without \LuaTeX:]
%   The macros |\pdf@|\meta{cmd} are mapped to the commands
%   of \pdfTeX\ if they are available. Otherwise they are undefined.
% \end{description}
%
% \subsection{Macros}
%
% \begin{declcs}{pdf@strcmp} \M{stringA} \M{stringB}
% \end{declcs}
% Same as |\pdfstrcmp{|\meta{stringA}|}{|\meta{stringB}|}|.
%
% \begin{declcs}{pdf@unescapehex} \M{string}
% \end{declcs}
% Same as |\pdfunescapehex{|\meta{string}|}|.
% The argument is a byte string given in hexadecimal notation.
% The result are character tokens from 0 until 255 with
% catcode 12 and the space with catcode 10.
%
% \begin{declcs}{pdf@escapehex} \M{string}\\
%   \cs{pdf@escapestring} \M{string}\\
%   \cs{pdf@escapename} \M{string}
% \end{declcs}
% Same as the primitives of \pdfTeX. However \pdfTeX\ does not
% know about characters with codes 256 and larger. Thus the
% string is treated as byte string, characters with more than
% eight bits are ignored.
%
% \begin{declcs}{pdf@filesize} \M{filename}
% \end{declcs}
% Same as |\pdffilesize{|\meta{filename}|}|.
%
% \begin{declcs}{pdf@filemoddate} \M{filename}
% \end{declcs}
% Same as |\pdffilemoddate{|\meta{filename}|}|.
%
% \begin{declcs}{pdf@filedump} \M{offset} \M{length} \M{filename}
% \end{declcs}
% Same as |\pdffiledump offset| \meta{offset} |length| \meta{length}
% |{|\meta{filename}|}|. Both \meta{offset} and \meta{length} must
% not be empty, but must be a valid \TeX\ number.
%
% \begin{declcs}{pdf@mdfivesum} \M{string}
% \end{declcs}
% Same as |\pdfmdfivesum{|\meta{string}|}|. Keyword |file| is supported
% by macro \cs{pdf@filemdfivesum}.
%
% \begin{declcs}{pdf@filemdfivesum} \M{filename}
% \end{declcs}
% Same as |\pdfmdfivesum file{|\meta{filename}|}|.
%
% \begin{declcs}{pdf@shellescape}
% \end{declcs}
% Same as |\pdfshellescape|. It expands to |1| if external
% commands can be executed and |0| otherwise. In \pdfTeX\ external
% commands must be enabled first by command line option or
% configuration option. In \LuaTeX\ option |--safer| disables
% the execution of external commands.
%
% \begin{declcs}{pdf@system} \M{cmdline}
% \end{declcs}
% It is a wrapper for |\immediate\write18| in \pdfTeX\ or
% |os.execute| in \LuaTeX.
%
% In theory |os.execute|
% returns a status number. But its meaning is quite
% undefined. Are there some reliable properties?
% Does it make sense to provide an user interface to
% this status exit code?
%
% \subsubsection{Experimental}
%
% \begin{declcs}{pdf@unescapehexnative} \M{string}\\
%   \cs{pdf@escapehexnative} \M{string}\\
%   \cs{pdf@escapenamenative} \M{string}\\
%   \cs{pdf@mdfivesumnative} \M{string}
% \end{declcs}
% The variants without |native| in the macro name are supposed to
% be compatible with \pdfTeX. However characters with more than
% eight bits are not supported and are ignored. If \LuaTeX\ is
% running, then its UTF-8 coded strings are used. Thus the full
% unicode character range is supported. However the result
% differs from \pdfTeX\ for characters with eight or more bits.
%
% \begin{declcs}{pdf@pipe} \M{cmdline}
% \end{declcs}
% It calls \meta{cmdline} and returns the output of the external
% program in the usual manner as byte string (catcode 12, space with
% catcode 10). The Lua documentation says, that the used |io.popen|
% may not be available on all platforms. Then macro \cs{pdf@pipe}
% is undefined.
%
% \StopEventually{
% }
%
% \section{Implementation}
%
%    \begin{macrocode}
%<*package>
%    \end{macrocode}
%
% \subsection{Reload check and package identification}
%    Reload check, especially if the package is not used with \LaTeX.
%    \begin{macrocode}
\begingroup
  \catcode44 12 % ,
  \catcode45 12 % -
  \catcode46 12 % .
  \catcode58 12 % :
  \catcode64 11 % @
  \expandafter\let\expandafter\x\csname ver@pdftexcmds.sty\endcsname
  \ifcase 0%
    \ifx\x\relax % plain
    \else
      \ifx\x\empty % LaTeX
      \else
        1%
      \fi
    \fi
  \else
    \catcode35 6 % #
    \catcode123 1 % {
    \catcode125 2 % }
    \expandafter\ifx\csname PackageInfo\endcsname\relax
      \def\x#1#2{%
        \immediate\write-1{Package #1 Info: #2.}%
      }%
    \else
      \def\x#1#2{\PackageInfo{#1}{#2, stopped}}%
    \fi
    \x{pdftexcmds}{The package is already loaded}%
    \endgroup
    \expandafter\endinput
  \fi
\endgroup
%    \end{macrocode}
%    Package identification:
%    \begin{macrocode}
\begingroup
  \catcode35 6 % #
  \catcode40 12 % (
  \catcode41 12 % )
  \catcode44 12 % ,
  \catcode45 12 % -
  \catcode46 12 % .
  \catcode47 12 % /
  \catcode58 12 % :
  \catcode64 11 % @
  \catcode123 1 % {
  \catcode125 2 % }
  \expandafter\ifx\csname ProvidesPackage\endcsname\relax
    \def\x#1#2#3[#4]{\endgroup
      \immediate\write-1{Package: #3 #4}%
      \xdef#1{#4}%
    }%
  \else
    \def\x#1#2[#3]{\endgroup
      #2[{#3}]%
      \ifx#1\relax
        \xdef#1{#3}%
      \fi
    }%
  \fi
\expandafter\x\csname ver@pdftexcmds.sty\endcsname
\ProvidesPackage{pdftexcmds}%
  [2007/12/12 v0.3 LuaTeX support for pdfTeX utility functions (HO)]
%    \end{macrocode}
%
% \subsection{Catcodes}
%
%    \begin{macrocode}
\begingroup
  \catcode123 1 % {
  \catcode125 2 % }
  \def\x{\endgroup
    \expandafter\edef\csname pdftexcmds@AtEnd\endcsname{%
      \catcode35 \the\catcode35\relax
      \catcode64 \the\catcode64\relax
      \catcode123 \the\catcode123\relax
      \catcode125 \the\catcode125\relax
    }%
  }%
\x
\catcode35 6 % #
\catcode64 11 % @
\catcode123 1 % {
\catcode125 2 % }
\def\TMP@EnsureCode#1#2{%
  \edef\pdftexcmds@AtEnd{%
    \pdftexcmds@AtEnd
    \catcode#1 \the\catcode#1\relax
  }%
  \catcode#1 #2\relax
}
\TMP@EnsureCode{10}{12}% ^^J
\TMP@EnsureCode{33}{12}% !
\TMP@EnsureCode{34}{12}% "
\TMP@EnsureCode{39}{12}% '
\TMP@EnsureCode{40}{12}% (
\TMP@EnsureCode{41}{12}% )
\TMP@EnsureCode{42}{12}% *
\TMP@EnsureCode{43}{12}% +
\TMP@EnsureCode{44}{12}% ,
\TMP@EnsureCode{45}{12}% -
\TMP@EnsureCode{46}{12}% .
\TMP@EnsureCode{47}{12}% /
\TMP@EnsureCode{58}{12}% :
\TMP@EnsureCode{60}{12}% <
\TMP@EnsureCode{61}{12}% =
\TMP@EnsureCode{62}{12}% >
\TMP@EnsureCode{94}{7}% ^ (superscript)
\TMP@EnsureCode{95}{12}% _ (other)
\TMP@EnsureCode{126}{12}% ~ (other)
%    \end{macrocode}
%
% \subsection{Load package \xpackage{infwarerr}}
%
%    \begin{macrocode}
\begingroup\expandafter\expandafter\expandafter\endgroup
\expandafter\ifx\csname RequirePackage\endcsname\relax
  \input infwarerr.sty\relax
\else
  \RequirePackage{infwarerr}[2007/09/09]%
\fi
%    \end{macrocode}
%
% \subsection{Without \LuaTeX}
%
%    \begin{macrocode}
\begingroup\expandafter\expandafter\expandafter\endgroup
\expandafter\ifx\csname directlua\endcsname\relax
  \@PackageInfo{pdftexcmds}{LuaTeX not detected}%
  \def\pdftexcmds@nopdftex{%
    \@PackageInfoNoLine{pdftexcmds}{pdfTeX >= 1.30 not detected}%
    \let\pdftexcmds@nopdftex\relax
  }%
  \def\pdftexcmds@temp#1{%
    \begingroup\expandafter\expandafter\expandafter\endgroup
    \expandafter\ifx\csname pdf#1\endcsname\relax
      \pdftexcmds@nopdftex
    \else
      \expandafter\def\csname pdf@#1\expandafter\endcsname
      \expandafter##\expandafter{%
        \csname pdf#1\endcsname
      }%
    \fi
  }%
  \pdftexcmds@temp{strcmp}%
  \pdftexcmds@temp{escapehex}%
  \let\pdf@escapehexnative\pdf@escapehex
  \pdftexcmds@temp{unescapehex}%
  \let\pdf@unescapehexnative\pdf@unescapehex
  \pdftexcmds@temp{escapestring}%
  \pdftexcmds@temp{escapename}%
  \pdftexcmds@temp{filesize}%
  \pdftexcmds@temp{filemoddate}%
  \begingroup\expandafter\expandafter\expandafter\endgroup
  \expandafter\ifx\csname pdfshellescape\endcsname\relax
    \pdftexcmds@nopdftex
  \else
    \def\pdf@shellescape{%
      \pdfshellescape
    }%
  \fi
  \begingroup\expandafter\expandafter\expandafter\endgroup
  \expandafter\ifx\csname pdffiledump\endcsname\relax
    \pdftexcmds@nopdftex
  \else
    \def\pdf@filedump#1#2#3{%
      \pdffiledump offset#1 length#2{#3}%
    }%
  \fi
  \begingroup\expandafter\expandafter\expandafter\endgroup
  \expandafter\ifx\csname pdfmdfivesum\endcsname\relax
    \pdftexcmds@nopdftex
  \else
    \def\pdf@mdfivesum#{\pdfmdfivesum}%
    \let\pdf@mdfivesumnative\pdf@mdfivesum
    \def\pdf@filemdfivesum#{\pdfmdfivesum file}%
  \fi
  \def\pdf@system#{%
    \immediate\write18%
  }%
  \pdftexcmds@AtEnd
  \expandafter\endinput
\fi
%    \end{macrocode}
%
% \subsection{Load module}
%
%    \begin{macrocode}
\begingroup\expandafter\expandafter\expandafter\endgroup
\expandafter\ifx\csname RequirePackage\endcsname\relax
  \input luatex-loader.sty\relax
\else
  \RequirePackage{luatex-loader}[2007/12/12]%
\fi
\directlua0{%
  require("oberdiek.pdftexcmds")%
}
%    \end{macrocode}
%
% \subsection{Lua functions}
%
%    \begin{macro}{\pdftexcmds@toks}
%    \begin{macrocode}
\begingroup\expandafter\expandafter\expandafter\endgroup
\expandafter\ifx\csname newtoks\endcsname\relax
  \toksdef\pdftexcmds@toks=0 %
\else
  \csname newtoks\endcsname\pdftexcmds@toks
\fi
%    \end{macrocode}
%    \end{macro}
%
%    \begin{macro}{\pdf@strcmp}
%    \begin{macrocode}
\long\def\pdf@strcmp#1#2{%
  \directlua0{%
    oberdiek.pdftexcmds.strcmp("\luaescapestring{#1}",%
        "\luaescapestring{#2}")%
  }%
}%
%    \end{macrocode}
%    \end{macro}
%    \begin{macro}{\pdf@escapehex}
%    \begin{macrocode}
\long\def\pdf@escapehex#1{%
  \directlua0{%
    oberdiek.pdftexcmds.escapehex("\luaescapestring{#1}", "byte")%
  }%
}%
%    \end{macrocode}
%    \end{macro}
%    \begin{macro}{\pdf@escapehexnative}
%    \begin{macrocode}
\long\def\pdf@escapehexnative#1{%
  \directlua0{%
    oberdiek.pdftexcmds.escapehex("\luaescapestring{#1}")%
  }%
}%
%    \end{macrocode}
%    \end{macro}
%    \begin{macro}{\pdf@unescapehex}
%    \begin{macrocode}
\def\pdf@unescapehex#1{%
  \the\expandafter\pdftexcmds@toks
  \directlua0{%
    oberdiek.pdftexcmds.toks="pdftexcmds@toks"%
    oberdiek.pdftexcmds.unescapehex("\luaescapestring{#1}", "byte")%
  }%
}%
%    \end{macrocode}
%    \end{macro}
%    \begin{macro}{\pdf@unescapehexnative}
%    \begin{macrocode}
\def\pdf@unescapehexnative#1{%
  \the\expandafter\pdftexcmds@toks
  \directlua0{%
    oberdiek.pdftexcmds.toks="pdftexcmds@toks"%
    oberdiek.pdftexcmds.unescapehex("\luaescapestring{#1}")%
  }%
}%
%    \end{macrocode}
%    \end{macro}
%    \begin{macro}{\pdf@escapestring}
%    \begin{macrocode}
\long\def\pdf@escapestring#1{%
  \directlua0{%
    oberdiek.pdftexcmds.escapestring("\luaescapestring{#1}", "byte")%
  }%
}
%    \end{macrocode}
%    \end{macro}
%    \begin{macro}{\pdf@escapename}
%    \begin{macrocode}
\long\def\pdf@escapename#1{%
  \directlua0{%
    oberdiek.pdftexcmds.escapename("\luaescapestring{#1}", "byte")%
  }%
}
%    \end{macrocode}
%    \end{macro}
%    \begin{macro}{\pdf@escapenamenative}
%    \begin{macrocode}
\long\def\pdf@escapenamenative#1{%
  \directlua0{%
    oberdiek.pdftexcmds.escapename("\luaescapestring{#1}")%
  }%
}
%    \end{macrocode}
%    \end{macro}
%    \begin{macro}{\pdf@filesize}
%    \begin{macrocode}
\def\pdf@filesize#1{%
  \directlua0{%
    oberdiek.pdftexcmds.filesize("\luaescapestring{#1}")%
  }%
}
%    \end{macrocode}
%    \end{macro}
%    \begin{macro}{\pdf@filemoddate}
%    \begin{macrocode}
\def\pdf@filemoddate#1{%
  \directlua0{%
    oberdiek.pdftexcmds.filemoddate("\luaescapestring{#1}")%
  }%
}
%    \end{macrocode}
%    \end{macro}
%    \begin{macro}{\pdf@filedump}
%    \begin{macrocode}
\def\pdf@filedump#1#2#3{%
  \directlua0{%
    oberdiek.pdftexcmds.filedump("\luaescapestring{\number#1}",%
        "\luaescapestring{\number#2}",%
        "\luaescapestring{#3}")%
  }%
}%
%    \end{macrocode}
%    \end{macro}
%    \begin{macro}{\pdf@mdfivesum}
%    \begin{macrocode}
\long\def\pdf@mdfivesum#1{%
  \directlua0{%
    oberdiek.pdftexcmds.mdfivesum("\luaescapestring{#1}", "byte")%
  }%
}%
%    \end{macrocode}
%    \end{macro}
%    \begin{macro}{\pdf@mdfivesumnative}
%    \begin{macrocode}
\long\def\pdf@mdfivesumnative#1{%
  \directlua0{%
    oberdiek.pdftexcmds.mdfivesum("\luaescapestring{#1}")%
  }%
}%
%    \end{macrocode}
%    \end{macro}
%    \begin{macro}{\pdf@filemdfivesum}
%    \begin{macrocode}
\def\pdf@filemdfivesum#1{%
  \directlua0{%
    oberdiek.pdftexcmds.filemdfivesum("\luaescapestring{#1}")%
  }%
}%
%    \end{macrocode}
%    \end{macro}
%
%    \begin{macro}{\pdf@shellescape}
%    \begin{macrocode}
\def\pdf@shellescape{%
  \directlua0{%
    oberdiek.pdftexcmds.shellescape()%
  }%
}
%    \end{macrocode}
%    \end{macro}
%
%    \begin{macro}{\pdf@system}
%    \begin{macrocode}
\def\pdf@system#1{%
  \directlua0{%
    oberdiek.pdftexcmds.system("\luaescapestring{#1}")%
  }%
}
%    \end{macrocode}
%    \end{macro}
%
%    \begin{macro}{\pdf@lastsystemstatus}
%    \begin{macrocode}
\def\pdf@lastsystemstatus{%
  \directlua0{%
    oberdiek.pdftexcmds.lastsystemstatus()%
  }%
}
%    \end{macrocode}
%    \end{macro}
%    \begin{macro}{\pdf@lastsystemexit}
%    \begin{macrocode}
\def\pdf@lastsystemexit{%
  \directlua0{%
    oberdiek.pdftexcmds.lastsystemexit()%
  }%
}
%    \end{macrocode}
%    \end{macro}
%
%    \begin{macro}{\pdf@pipe}
%    Check availability of |io.popen| first.
%    \begin{macrocode}
\ifnum0%
    \directlua0{%
      if io.popen then %
        tex.write("1")%
      end%
    }%
    =1 %
  \def\pdf@pipe#1{%
    \the\expandafter\pdftexcmds@toks
    \directlua0{%
      oberdiek.pdftexcmds.toks="pdftexcmds@toks"%
      oberdiek.pdftexcmds.pipe("\luaescapestring{#1}")%
    }%
  }%
\fi
%    \end{macrocode}
%    \end{macro}
%
%    \begin{macrocode}
\pdftexcmds@AtEnd
%</package>
%    \end{macrocode}
%
% \subsection{Lua module}
%
%    \begin{macrocode}
%<*lua>
%    \end{macrocode}
%
%    \begin{macrocode}
module("oberdiek.pdftexcmds", package.seeall)
local systemexitstatus
function strcmp(A, B)
  if A == B then
    tex.write("0")
  elseif A < B then
    tex.write("-1")
  else
    tex.write("1")
  end
end
local function utf8_to_byte(str)
  local i = 0
  local n = string.len(str)
  local t = {}
  while i < n do
    i = i + 1
    local a = string.byte(str, i)
    if a < 128 then
      table.insert(t, string.char(a))
    else
      if a >= 192 and i < n then
        i = i + 1
        local b = string.byte(str, i)
        if b < 128 or b >= 192 then
          i = i - 1
        elseif a == 194 then
          table.insert(t, string.char(b))
        elseif a == 195 then
          table.insert(t, string.char(b + 64))
        end
      end
    end
  end
  return table.concat(t)
end
function escapehex(str, mode)
  if mode == "byte" then
    str = utf8_to_byte(str)
  end
  tex.write((string.gsub(str, ".",
    function (ch)
      return string.format("%02X", string.byte(ch))
    end
  )))
end
%    \end{macrocode}
%    See procedure |unescapehex| in file \xfile{utils.c} of \pdfTeX.
%    Caution: |tex.write| ignores leading spaces.
%    \begin{macrocode}
function unescapehex(str, mode)
  local a = 0
  local first = true
  local result = {}
  for i = 1, string.len(str), 1 do
    local ch = string.byte(str, i)
    if ch >= 48 and ch <= 57 then
      ch = ch - 48
    elseif ch >= 65 and ch <= 70 then
      ch = ch - 55
    elseif ch >= 97 and ch <= 102 then
      ch = ch - 87
    else
      ch = nil
    end
    if ch then
      if first then
        a = ch * 16
        first = false
      else
        table.insert(result, a + ch)
        first = true
      end
    end
  end
  if not first then
    table.insert(result, a)
  end
  if mode == "byte" then
    local utf8 = {}
    for i, a in ipairs(result) do
      if a < 128 then
        table.insert(utf8, a)
      else
        if a < 192 then
          table.insert(utf8, 194)
          a = a - 128
        else
          table.insert(utf8, 195)
          a = a - 192
        end
        table.insert(utf8, a + 128)
      end
    end
    result = utf8
  end
  tex.settoks(toks, string.char(unpack(result)))
end
%    \end{macrocode}
%    See procedure |escapestring| in file \xfile{utils.c} of \pdfTeX.
%    \begin{macrocode}
function escapestring(str, mode)
  if mode == "byte" then
    str = utf8_to_byte(str)
  end
  tex.write((string.gsub(str, ".",
    function (ch)
      local b = string.byte(ch)
      if b < 33 or b > 126 then
        return string.format("\\%.3o", b)
      end
      if b == 40 or b == 41 or b == 92 then
        return "\\" .. ch
      end
%    \end{macrocode}
%    Lua 5.1 returns the match in case of return value |nil|.
%    \begin{macrocode}
      return nil
    end
  )))
end
%    \end{macrocode}
%    See procedure |escapename| in file \xfile{utils.c} of \pdfTeX.
%    \begin{macrocode}
function escapename(str, mode)
  if mode == "byte" then
    str = utf8_to_byte(str)
  end
  tex.write((string.gsub(str, ".",
    function (ch)
      local b = string.byte(ch)
      if b == 0 then
%    \end{macrocode}
%    In Lua 5.0 |nil| could be used for the empty string,
%    But |nil| returns the match in Lua 5.1, thus we use
%    the empty string explicitly.
%    \begin{macrocode}
        return ""
      end
      if b <= 32 or b >= 127
          or b == 35 or b == 37 or b == 40 or b == 41
          or b == 47 or b == 60 or b == 62 or b == 91
          or b == 93 or b == 123 or b == 125 then
        return string.format("#%.2X", b)
      else
%    \end{macrocode}
%    Lua 5.1 returns the match in case of return value |nil|.
%    \begin{macrocode}
        return nil
      end
    end
  )))
end
function filesize(filename)
  local foundfile = kpse.find_file(filename, "tex", true)
  if foundfile then
    local size = lfs.attributes(foundfile, "size")
    if size then
      tex.write(size)
    end
  end
end
%    \end{macrocode}
%    See procedure |makepdftime| in file \xfile{utils.c} of \pdfTeX.
%    \begin{macrocode}
function filemoddate(filename)
  local foundfile = kpse.find_file(filename, "tex", true)
  if foundfile then
    local date = lfs.attributes(foundfile, "modification")
    if date then
      local d = os.date("*t", date)
      if d.sec >= 60 then
        d.sec = 59
      end
      local u = os.date("!*t", date)
      local off = 60 * (d.hour - u.hour) + d.min - u.min
      if d.year ~= u.year then
        if d.year > u.year then
          off = off + 1440
        else
          off = off - 1440
        end
      elseif d.yday ~= u.yday then
        if d.yday > u.yday then
          off = off + 1440
        else
          off = off - 1440
        end
      end
      local timezone
      if off == 0 then
        timezone = "Z"
      else
        local hours = math.floor(off / 60)
        local mins = math.abs(off - hours * 60)
        timezone = string.format("%+03d'%02d'", hours, mins)
      end
      tex.write(string.format("D:%04d%02d%02d%02d%02d%02d%s",
          d.year, d.month, d.day, d.hour, d.min, d.sec, timezone))
    end
  end
end
function filedump(offset, length, filename)
  length = tonumber(length)
  if length and length > 0 then
    local foundfile = kpse.find_file(filename, "tex", true)
    if foundfile then
      offset = tonumber(offset)
      if not offset then
        offset = 0
      end
      local filehandle = io.open(foundfile, "r")
      if filehandle then
        if offset > 0 then
          filehandle:seek("set", offset)
        end
        local dump = filehandle:read(length)
        escapehex(dump)
      end
    end
  end
end
function mdfivesum(str, mode)
  if mode == "byte" then
    str = utf8_to_byte(str)
  end
  escapehex(md5.sum(str))
end
function filemdfivesum(filename)
  local foundfile = kpse.find_file(filename, "tex", true)
  if foundfile then
    local filehandle = io.open(foundfile, "r")
    if filehandle then
      local contents = filehandle:read("*a")
      escapehex(md5.sum(contents))
    end
  end
end
function shellescape()
  if os.execute then
    tex.write("1")
  else
    tex.write("0")
  end
end
function system(cmdline)
  systemexitstatus = nil
  texio.write_nl("log", "system(" .. cmdline .. ") ")
  if os.execute then
    texio.write("log", "executed.")
    systemexitstatus = os.execute(cmdline)
  else
    texio.write("log", "disabled.")
  end
end
function lastsystemstatus()
  local result = tonumber(systemexitstatus)
  if result then
    local x = math.floor(result / 256)
    tex.write(result - 256 * math.floor(result / 256))
  end
end
function lastsystemexit()
  local result = tonumber(systemexitstatus)
  if result then
    tex.write(math.floor(result / 256))
  end
end
function pipe(cmdline)
  local result
  systemexitstatus = nil
  texio.write_nl("log", "pipe(" .. cmdline ..") ")
  if io.popen then
    texio.write("log", "executed.")
    local handle = io.popen(cmdline, "r")
    if handle then
      result = handle:read("*a")
      handle:close()
    end
  else
    texio.write("log", "disabled.")
  end
  if result then
    tex.settoks(toks, result)
  else
    tex.settoks(toks, "")
  end
end
%    \end{macrocode}
%    \begin{macrocode}
%</lua>
%    \end{macrocode}
%
% \section{Test}
%
% \subsection{Catcode checks for loading}
%
%    \begin{macrocode}
%<*test1>
%    \end{macrocode}
%    \begin{macrocode}
\catcode`\{=1 %
\catcode`\}=2 %
\catcode`\#=6 %
\catcode`\@=11 %
\expandafter\ifx\csname count@\endcsname\relax
  \countdef\count@=255 %
\fi
\expandafter\ifx\csname @gobble\endcsname\relax
  \long\def\@gobble#1{}%
\fi
\expandafter\ifx\csname @firstofone\endcsname\relax
  \long\def\@firstofone#1{#1}%
\fi
\expandafter\ifx\csname loop\endcsname\relax
  \expandafter\@firstofone
\else
  \expandafter\@gobble
\fi
{%
  \def\loop#1\repeat{%
    \def\body{#1}%
    \iterate
  }%
  \def\iterate{%
    \body
      \let\next\iterate
    \else
      \let\next\relax
    \fi
    \next
  }%
  \let\repeat=\fi
}%
\def\RestoreCatcodes{}
\count@=0 %
\loop
  \edef\RestoreCatcodes{%
    \RestoreCatcodes
    \catcode\the\count@=\the\catcode\count@\relax
  }%
\ifnum\count@<255 %
  \advance\count@ 1 %
\repeat

\def\RangeCatcodeInvalid#1#2{%
  \count@=#1\relax
  \loop
    \catcode\count@=15 %
  \ifnum\count@<#2\relax
    \advance\count@ 1 %
  \repeat
}
\expandafter\ifx\csname LoadCommand\endcsname\relax
  \def\LoadCommand{\input pdftexcmds.sty\relax}%
\fi
\def\Test{%
  \RangeCatcodeInvalid{0}{47}%
  \RangeCatcodeInvalid{58}{64}%
  \RangeCatcodeInvalid{91}{96}%
  \RangeCatcodeInvalid{123}{255}%
  \catcode`\@=12 %
  \catcode`\\=0 %
  \catcode`\{=1 %
  \catcode`\}=2 %
  \catcode`\#=6 %
  \catcode`\[=12 %
  \catcode`\]=12 %
  \catcode`\%=14 %
  \catcode`\ =10 %
  \catcode13=5 %
  \LoadCommand
  \RestoreCatcodes
}
\Test
\csname @@end\endcsname
\end
%    \end{macrocode}
%    \begin{macrocode}
%</test1>
%    \end{macrocode}
%
% \section{Installation}
%
% \subsection{Download}
%
% \paragraph{Package.} This package is available on
% CTAN\footnote{\url{ftp://ftp.ctan.org/tex-archive/}}:
% \begin{description}
% \item[\CTAN{macros/latex/contrib/oberdiek/pdftexcmds.dtx}] The source file.
% \item[\CTAN{macros/latex/contrib/oberdiek/pdftexcmds.pdf}] Documentation.
% \end{description}
%
%
% \paragraph{Bundle.} All the packages of the bundle `oberdiek'
% are also available in a TDS compliant ZIP archive. There
% the packages are already unpacked and the documentation files
% are generated. The files and directories obey the TDS standard.
% \begin{description}
% \item[\CTAN{macros/latex/contrib/oberdiek/oberdiek-tds.zip}]
% \end{description}
% \emph{TDS} refers to the standard ``A Directory Structure
% for \TeX\ Files'' (\CTAN{tds/tds.pdf}). Directories
% with \xfile{texmf} in their name are usually organized this way.
%
% \subsection{Bundle installation}
%
% \paragraph{Unpacking.} Unpack the \xfile{oberdiek-tds.zip} in the
% TDS tree (also known as \xfile{texmf} tree) of your choice.
% Example (linux):
% \begin{quote}
%   |unzip oberdiek-tds.zip -d ~/texmf|
% \end{quote}
%
% \paragraph{Script installation.}
% Check the directory \xfile{TDS:scripts/oberdiek/} for
% scripts that need further installation steps.
% Package \xpackage{attachfile2} comes with the Perl script
% \xfile{pdfatfi.pl} that should be installed in such a way
% that it can be called as \texttt{pdfatfi}.
% Example (linux):
% \begin{quote}
%   |chmod +x scripts/oberdiek/pdfatfi.pl|\\
%   |cp scripts/oberdiek/pdfatfi.pl /usr/local/bin/|
% \end{quote}
%
% \subsection{Package installation}
%
% \paragraph{Unpacking.} The \xfile{.dtx} file is a self-extracting
% \docstrip\ archive. The files are extracted by running the
% \xfile{.dtx} through \plainTeX:
% \begin{quote}
%   \verb|tex pdftexcmds.dtx|
% \end{quote}
%
% \paragraph{TDS.} Now the different files must be moved into
% the different directories in your installation TDS tree
% (also known as \xfile{texmf} tree):
% \begin{quote}
% \def\t{^^A
% \begin{tabular}{@{}>{\ttfamily}l@{ $\rightarrow$ }>{\ttfamily}l@{}}
%   pdftexcmds.sty & tex/generic/oberdiek/pdftexcmds.sty\\
%   oberdiek.pdftexcmds.lua & scripts/oberdiek/oberdiek.pdftexcmds.lua\\
%   pdftexcmds.lua & scripts/oberdiek/pdftexcmds.lua\\
%   pdftexcmds.pdf & doc/latex/oberdiek/pdftexcmds.pdf\\
%   pdftexcmds.dtx & source/latex/oberdiek/pdftexcmds.dtx\\
% \end{tabular}^^A
% }^^A
% \sbox0{\t}^^A
% \ifdim\wd0>\linewidth
%   \begingroup
%     \advance\linewidth by\leftmargin
%     \advance\linewidth by\rightmargin
%   \edef\x{\endgroup
%     \def\noexpand\lw{\the\linewidth}^^A
%   }\x
%   \def\lwbox{^^A
%     \leavevmode
%     \hbox to \linewidth{^^A
%       \kern-\leftmargin\relax
%       \hss
%       \usebox0
%       \hss
%       \kern-\rightmargin\relax
%     }^^A
%   }^^A
%   \ifdim\wd0>\lw
%     \sbox0{\small\t}^^A
%     \ifdim\wd0>\linewidth
%       \ifdim\wd0>\lw
%         \sbox0{\footnotesize\t}^^A
%         \ifdim\wd0>\linewidth
%           \ifdim\wd0>\lw
%             \sbox0{\scriptsize\t}^^A
%             \ifdim\wd0>\linewidth
%               \ifdim\wd0>\lw
%                 \sbox0{\tiny\t}^^A
%                 \ifdim\wd0>\linewidth
%                   \lwbox
%                 \else
%                   \usebox0
%                 \fi
%               \else
%                 \lwbox
%               \fi
%             \else
%               \usebox0
%             \fi
%           \else
%             \lwbox
%           \fi
%         \else
%           \usebox0
%         \fi
%       \else
%         \lwbox
%       \fi
%     \else
%       \usebox0
%     \fi
%   \else
%     \lwbox
%   \fi
% \else
%   \usebox0
% \fi
% \end{quote}
% If you have a \xfile{docstrip.cfg} that configures and enables \docstrip's
% TDS installing feature, then some files can already be in the right
% place, see the documentation of \docstrip.
%
% \subsection{Refresh file name databases}
%
% If your \TeX~distribution
% (\teTeX, \mikTeX, \dots) relies on file name databases, you must refresh
% these. For example, \teTeX\ users run \verb|texhash| or
% \verb|mktexlsr|.
%
% \subsection{Some details for the interested}
%
% \paragraph{Attached source.}
%
% The PDF documentation on CTAN also includes the
% \xfile{.dtx} source file. It can be extracted by
% AcrobatReader 6 or higher. Another option is \textsf{pdftk},
% e.g. unpack the file into the current directory:
% \begin{quote}
%   \verb|pdftk pdftexcmds.pdf unpack_files output .|
% \end{quote}
%
% \paragraph{Unpacking with \LaTeX.}
% The \xfile{.dtx} chooses its action depending on the format:
% \begin{description}
% \item[\plainTeX:] Run \docstrip\ and extract the files.
% \item[\LaTeX:] Generate the documentation.
% \end{description}
% If you insist on using \LaTeX\ for \docstrip\ (really,
% \docstrip\ does not need \LaTeX), then inform the autodetect routine
% about your intention:
% \begin{quote}
%   \verb|latex \let\install=y\input{pdftexcmds.dtx}|
% \end{quote}
% Do not forget to quote the argument according to the demands
% of your shell.
%
% \paragraph{Generating the documentation.}
% You can use both the \xfile{.dtx} or the \xfile{.drv} to generate
% the documentation. The process can be configured by the
% configuration file \xfile{ltxdoc.cfg}. For instance, put this
% line into this file, if you want to have A4 as paper format:
% \begin{quote}
%   \verb|\PassOptionsToClass{a4paper}{article}|
% \end{quote}
% An example follows how to generate the
% documentation with pdf\LaTeX:
% \begin{quote}
%\begin{verbatim}
%pdflatex pdftexcmds.dtx
%makeindex -s gind.ist pdftexcmds.idx
%pdflatex pdftexcmds.dtx
%makeindex -s gind.ist pdftexcmds.idx
%pdflatex pdftexcmds.dtx
%\end{verbatim}
% \end{quote}
%
% \begin{History}
%   \begin{Version}{2007/11/11 v0.1}
%   \item
%     First version.
%   \end{Version}
%   \begin{Version}{2007/11/12 v0.2}
%   \item
%     Short description fixed.
%   \end{Version}
%   \begin{Version}{2007/12/12 v0.3}
%   \item
%     Organization of Lua code as module.
%   \end{Version}
% \end{History}
%
% \PrintIndex
%
% \Finale
\endinput
|
% \end{quote}
% Do not forget to quote the argument according to the demands
% of your shell.
%
% \paragraph{Generating the documentation.}
% You can use both the \xfile{.dtx} or the \xfile{.drv} to generate
% the documentation. The process can be configured by the
% configuration file \xfile{ltxdoc.cfg}. For instance, put this
% line into this file, if you want to have A4 as paper format:
% \begin{quote}
%   \verb|\PassOptionsToClass{a4paper}{article}|
% \end{quote}
% An example follows how to generate the
% documentation with pdf\LaTeX:
% \begin{quote}
%\begin{verbatim}
%pdflatex pdftexcmds.dtx
%makeindex -s gind.ist pdftexcmds.idx
%pdflatex pdftexcmds.dtx
%makeindex -s gind.ist pdftexcmds.idx
%pdflatex pdftexcmds.dtx
%\end{verbatim}
% \end{quote}
%
% \begin{History}
%   \begin{Version}{2007/11/11 v0.1}
%   \item
%     First version.
%   \end{Version}
%   \begin{Version}{2007/11/12 v0.2}
%   \item
%     Short description fixed.
%   \end{Version}
%   \begin{Version}{2007/12/12 v0.3}
%   \item
%     Organization of Lua code as module.
%   \end{Version}
% \end{History}
%
% \PrintIndex
%
% \Finale
\endinput

%        (quote the arguments according to the demands of your shell)
%
% Documentation:
%    (a) If pdftexcmds.drv is present:
%           latex pdftexcmds.drv
%    (b) Without pdftexcmds.drv:
%           latex pdftexcmds.dtx; ...
%    The class ltxdoc loads the configuration file ltxdoc.cfg
%    if available. Here you can specify further options, e.g.
%    use A4 as paper format:
%       \PassOptionsToClass{a4paper}{article}
%
%    Programm calls to get the documentation (example):
%       pdflatex pdftexcmds.dtx
%       makeindex -s gind.ist pdftexcmds.idx
%       pdflatex pdftexcmds.dtx
%       makeindex -s gind.ist pdftexcmds.idx
%       pdflatex pdftexcmds.dtx
%
% Installation:
%    TDS:tex/generic/oberdiek/pdftexcmds.sty
%    TDS:scripts/oberdiek/oberdiek.pdftexcmds.lua
%    TDS:scripts/oberdiek/pdftexcmds.lua
%    TDS:doc/latex/oberdiek/pdftexcmds.pdf
%    TDS:source/latex/oberdiek/pdftexcmds.dtx
%
%<*ignore>
\begingroup
  \def\x{LaTeX2e}%
\expandafter\endgroup
\ifcase 0\ifx\install y1\fi\expandafter
         \ifx\csname processbatchFile\endcsname\relax\else1\fi
         \ifx\fmtname\x\else 1\fi\relax
\else\csname fi\endcsname
%</ignore>
%<*install>
\input docstrip.tex
\Msg{************************************************************************}
\Msg{* Installation}
\Msg{* Package: pdftexcmds 2007/12/12 v0.3 LuaTeX support for pdfTeX utility functions (HO)}
\Msg{************************************************************************}

\keepsilent
\askforoverwritefalse

\let\MetaPrefix\relax
\preamble

This is a generated file.

Copyright (C) 2007 by Heiko Oberdiek <oberdiek@uni-freiburg.de>

This work may be distributed and/or modified under the
conditions of the LaTeX Project Public License, either
version 1.3 of this license or (at your option) any later
version. The latest version of this license is in
   http://www.latex-project.org/lppl.txt
and version 1.3 or later is part of all distributions of
LaTeX version 2005/12/01 or later.

This work has the LPPL maintenance status "maintained".

This Current Maintainer of this work is Heiko Oberdiek.

This work consists of the main source file pdftexcmds.dtx
and the derived files
   pdftexcmds.sty, pdftexcmds.pdf, pdftexcmds.ins, pdftexcmds.drv,
   oberdiek.pdftexcmds.lua, pdftexcmds.lua.

\endpreamble
\let\MetaPrefix\DoubleperCent

\generate{%
  \file{pdftexcmds.ins}{\from{pdftexcmds.dtx}{install}}%
  \file{pdftexcmds.drv}{\from{pdftexcmds.dtx}{driver}}%
  \usedir{tex/generic/oberdiek}%
  \file{pdftexcmds.sty}{\from{pdftexcmds.dtx}{package}}%
}
\def\MetaPrefix{-- }
\def\defaultpostamble{%
  \MetaPrefix^^J%
  \MetaPrefix\space End of File `\outFileName'.%
}
\def\currentpostamble{\defaultpostamble}%
\generate{%
  \usedir{scripts/oberdiek}%
  \file{oberdiek.pdftexcmds.lua}{\from{pdftexcmds.dtx}{lua}}%
  \file{pdftexcmds.lua}{\from{pdftexcmds.dtx}{lua}}%
}

\obeyspaces
\Msg{************************************************************************}
\Msg{*}
\Msg{* To finish the installation you have to move the following}
\Msg{* file into a directory searched by TeX:}
\Msg{*}
\Msg{*     pdftexcmds.sty}
\Msg{*}
\Msg{* And install the following script files:}
\Msg{*}
\Msg{*     oberdiek.pdftexcmds.lua, pdftexcmds.lua}
\Msg{*}
\Msg{* To produce the documentation run the file `pdftexcmds.drv'}
\Msg{* through LaTeX.}
\Msg{*}
\Msg{* Happy TeXing!}
\Msg{*}
\Msg{************************************************************************}

\endbatchfile
%</install>
%<*ignore>
\fi
%</ignore>
%<*driver>
\NeedsTeXFormat{LaTeX2e}
\ProvidesFile{pdftexcmds.drv}%
  [2007/12/12 v0.3 LuaTeX support for pdfTeX utility functions (HO)]%
\documentclass{ltxdoc}
\usepackage{holtxdoc}[2007/11/11]
\usepackage{paralist}
\begin{document}
  \DocInput{pdftexcmds.dtx}%
\end{document}
%</driver>
% \fi
%
% \CheckSum{552}
%
% \CharacterTable
%  {Upper-case    \A\B\C\D\E\F\G\H\I\J\K\L\M\N\O\P\Q\R\S\T\U\V\W\X\Y\Z
%   Lower-case    \a\b\c\d\e\f\g\h\i\j\k\l\m\n\o\p\q\r\s\t\u\v\w\x\y\z
%   Digits        \0\1\2\3\4\5\6\7\8\9
%   Exclamation   \!     Double quote  \"     Hash (number) \#
%   Dollar        \$     Percent       \%     Ampersand     \&
%   Acute accent  \'     Left paren    \(     Right paren   \)
%   Asterisk      \*     Plus          \+     Comma         \,
%   Minus         \-     Point         \.     Solidus       \/
%   Colon         \:     Semicolon     \;     Less than     \<
%   Equals        \=     Greater than  \>     Question mark \?
%   Commercial at \@     Left bracket  \[     Backslash     \\
%   Right bracket \]     Circumflex    \^     Underscore    \_
%   Grave accent  \`     Left brace    \{     Vertical bar  \|
%   Right brace   \}     Tilde         \~}
%
% \GetFileInfo{pdftexcmds.drv}
%
% \title{The \xpackage{pdftexcmds} package}
% \date{2007/12/12 v0.3}
% \author{Heiko Oberdiek\\\xemail{oberdiek@uni-freiburg.de}}
%
% \maketitle
%
% \begin{abstract}
% \LuaTeX\ provides most of the commands of \pdfTeX\ 1.40. However
% a number of utility functions are removed. This package tries to fill
% the gap and implements some of the missing primitive using Lua.
% \end{abstract}
%
% \tableofcontents
%
% \section{Documentation}
%
% Some primitives of \pdfTeX\ are not defined by \LuaTeX.
% This package implements macro based solutions using Lua code
% for the following missing \pdfTeX\ primitives;
% \begin{compactitem}
% \item \cs{pdfstrcmp}
% \item \cs{pdfunescapehex}
% \item \cs{pdfescapehex}
% \item \cs{pdfescapename}
% \item \cs{pdfescapestring}
% \item \cs{pdffilesize}
% \item \cs{pdffilemoddate}
% \item \cs{pdffiledump}
% \item \cs{pdfmdfivesum}
% \item |\immediate\write18|
% \end{compactitem}
% The original names of the primitives cannot be used:
% \begin{itemize}
% \item
% The syntax for their arguments cannot easily
% simulated by macros. The primitives using key words
% such as |file| (\cs{pdfmdfivesum}) or |offset| and |length|
% (\cs{pdffiledump}) and uses \meta{general text} for the other
% arguments. Using token registers assignments, \meta{general text} could
% be catched. However, the simulated primitives are expandable
% and register assignments would destroy this important property.
% (\meta{general text} allows something like |\expandafter\bgroup ...}|.)
% \item
% The original primitives can be expanded using one expansion step.
% The new macros need two expansion steps because of the additional
% macro expansion. Example:
% \begin{quote}
%   |\expandafter\foo\pdffilemoddate{file}|\\
%   vs.
%   |\expandafter\expandafter\expandafter\foo\pdf@filemoddate{file}|.
% \end{quote}
% \end{itemize}
%
% \LuaTeX\ isn't stable yet and thus the status of this package is
% \emph{experimental}. Feedback is welcome.
%
% \subsection{General principles}
%
% \begin{description}
% \item[Naming convention:]
%   Usually this package defines a macro |\pdf@|\meta{cmd} if
%   \pdfTeX\ provides |\pdf|\meta{cmd}.
% \item[Arguments:] The order of arguments in |\pdf@|\meta{cmd}
%   is the same as for the corresponding primitive of \pdfTeX.
%   The arguments are ordinary undelimited \TeX\ arguments,
%   no \meta{general text} and without additional keywords.
% \item[Expandibility:]
%   The macro |\pdf@|\meta{cmd} is expandable if the
%   corresponding \pdfTeX\ primitive has this property.
%   Exact two expansion steps are necessary (first is the macro
%   expansion).
% \item[Without \LuaTeX:]
%   The macros |\pdf@|\meta{cmd} are mapped to the commands
%   of \pdfTeX\ if they are available. Otherwise they are undefined.
% \end{description}
%
% \subsection{Macros}
%
% \begin{declcs}{pdf@strcmp} \M{stringA} \M{stringB}
% \end{declcs}
% Same as |\pdfstrcmp{|\meta{stringA}|}{|\meta{stringB}|}|.
%
% \begin{declcs}{pdf@unescapehex} \M{string}
% \end{declcs}
% Same as |\pdfunescapehex{|\meta{string}|}|.
% The argument is a byte string given in hexadecimal notation.
% The result are character tokens from 0 until 255 with
% catcode 12 and the space with catcode 10.
%
% \begin{declcs}{pdf@escapehex} \M{string}\\
%   \cs{pdf@escapestring} \M{string}\\
%   \cs{pdf@escapename} \M{string}
% \end{declcs}
% Same as the primitives of \pdfTeX. However \pdfTeX\ does not
% know about characters with codes 256 and larger. Thus the
% string is treated as byte string, characters with more than
% eight bits are ignored.
%
% \begin{declcs}{pdf@filesize} \M{filename}
% \end{declcs}
% Same as |\pdffilesize{|\meta{filename}|}|.
%
% \begin{declcs}{pdf@filemoddate} \M{filename}
% \end{declcs}
% Same as |\pdffilemoddate{|\meta{filename}|}|.
%
% \begin{declcs}{pdf@filedump} \M{offset} \M{length} \M{filename}
% \end{declcs}
% Same as |\pdffiledump offset| \meta{offset} |length| \meta{length}
% |{|\meta{filename}|}|. Both \meta{offset} and \meta{length} must
% not be empty, but must be a valid \TeX\ number.
%
% \begin{declcs}{pdf@mdfivesum} \M{string}
% \end{declcs}
% Same as |\pdfmdfivesum{|\meta{string}|}|. Keyword |file| is supported
% by macro \cs{pdf@filemdfivesum}.
%
% \begin{declcs}{pdf@filemdfivesum} \M{filename}
% \end{declcs}
% Same as |\pdfmdfivesum file{|\meta{filename}|}|.
%
% \begin{declcs}{pdf@shellescape}
% \end{declcs}
% Same as |\pdfshellescape|. It expands to |1| if external
% commands can be executed and |0| otherwise. In \pdfTeX\ external
% commands must be enabled first by command line option or
% configuration option. In \LuaTeX\ option |--safer| disables
% the execution of external commands.
%
% \begin{declcs}{pdf@system} \M{cmdline}
% \end{declcs}
% It is a wrapper for |\immediate\write18| in \pdfTeX\ or
% |os.execute| in \LuaTeX.
%
% In theory |os.execute|
% returns a status number. But its meaning is quite
% undefined. Are there some reliable properties?
% Does it make sense to provide an user interface to
% this status exit code?
%
% \subsubsection{Experimental}
%
% \begin{declcs}{pdf@unescapehexnative} \M{string}\\
%   \cs{pdf@escapehexnative} \M{string}\\
%   \cs{pdf@escapenamenative} \M{string}\\
%   \cs{pdf@mdfivesumnative} \M{string}
% \end{declcs}
% The variants without |native| in the macro name are supposed to
% be compatible with \pdfTeX. However characters with more than
% eight bits are not supported and are ignored. If \LuaTeX\ is
% running, then its UTF-8 coded strings are used. Thus the full
% unicode character range is supported. However the result
% differs from \pdfTeX\ for characters with eight or more bits.
%
% \begin{declcs}{pdf@pipe} \M{cmdline}
% \end{declcs}
% It calls \meta{cmdline} and returns the output of the external
% program in the usual manner as byte string (catcode 12, space with
% catcode 10). The Lua documentation says, that the used |io.popen|
% may not be available on all platforms. Then macro \cs{pdf@pipe}
% is undefined.
%
% \StopEventually{
% }
%
% \section{Implementation}
%
%    \begin{macrocode}
%<*package>
%    \end{macrocode}
%
% \subsection{Reload check and package identification}
%    Reload check, especially if the package is not used with \LaTeX.
%    \begin{macrocode}
\begingroup
  \catcode44 12 % ,
  \catcode45 12 % -
  \catcode46 12 % .
  \catcode58 12 % :
  \catcode64 11 % @
  \expandafter\let\expandafter\x\csname ver@pdftexcmds.sty\endcsname
  \ifcase 0%
    \ifx\x\relax % plain
    \else
      \ifx\x\empty % LaTeX
      \else
        1%
      \fi
    \fi
  \else
    \catcode35 6 % #
    \catcode123 1 % {
    \catcode125 2 % }
    \expandafter\ifx\csname PackageInfo\endcsname\relax
      \def\x#1#2{%
        \immediate\write-1{Package #1 Info: #2.}%
      }%
    \else
      \def\x#1#2{\PackageInfo{#1}{#2, stopped}}%
    \fi
    \x{pdftexcmds}{The package is already loaded}%
    \endgroup
    \expandafter\endinput
  \fi
\endgroup
%    \end{macrocode}
%    Package identification:
%    \begin{macrocode}
\begingroup
  \catcode35 6 % #
  \catcode40 12 % (
  \catcode41 12 % )
  \catcode44 12 % ,
  \catcode45 12 % -
  \catcode46 12 % .
  \catcode47 12 % /
  \catcode58 12 % :
  \catcode64 11 % @
  \catcode123 1 % {
  \catcode125 2 % }
  \expandafter\ifx\csname ProvidesPackage\endcsname\relax
    \def\x#1#2#3[#4]{\endgroup
      \immediate\write-1{Package: #3 #4}%
      \xdef#1{#4}%
    }%
  \else
    \def\x#1#2[#3]{\endgroup
      #2[{#3}]%
      \ifx#1\relax
        \xdef#1{#3}%
      \fi
    }%
  \fi
\expandafter\x\csname ver@pdftexcmds.sty\endcsname
\ProvidesPackage{pdftexcmds}%
  [2007/12/12 v0.3 LuaTeX support for pdfTeX utility functions (HO)]
%    \end{macrocode}
%
% \subsection{Catcodes}
%
%    \begin{macrocode}
\begingroup
  \catcode123 1 % {
  \catcode125 2 % }
  \def\x{\endgroup
    \expandafter\edef\csname pdftexcmds@AtEnd\endcsname{%
      \catcode35 \the\catcode35\relax
      \catcode64 \the\catcode64\relax
      \catcode123 \the\catcode123\relax
      \catcode125 \the\catcode125\relax
    }%
  }%
\x
\catcode35 6 % #
\catcode64 11 % @
\catcode123 1 % {
\catcode125 2 % }
\def\TMP@EnsureCode#1#2{%
  \edef\pdftexcmds@AtEnd{%
    \pdftexcmds@AtEnd
    \catcode#1 \the\catcode#1\relax
  }%
  \catcode#1 #2\relax
}
\TMP@EnsureCode{10}{12}% ^^J
\TMP@EnsureCode{33}{12}% !
\TMP@EnsureCode{34}{12}% "
\TMP@EnsureCode{39}{12}% '
\TMP@EnsureCode{40}{12}% (
\TMP@EnsureCode{41}{12}% )
\TMP@EnsureCode{42}{12}% *
\TMP@EnsureCode{43}{12}% +
\TMP@EnsureCode{44}{12}% ,
\TMP@EnsureCode{45}{12}% -
\TMP@EnsureCode{46}{12}% .
\TMP@EnsureCode{47}{12}% /
\TMP@EnsureCode{58}{12}% :
\TMP@EnsureCode{60}{12}% <
\TMP@EnsureCode{61}{12}% =
\TMP@EnsureCode{62}{12}% >
\TMP@EnsureCode{94}{7}% ^ (superscript)
\TMP@EnsureCode{95}{12}% _ (other)
\TMP@EnsureCode{126}{12}% ~ (other)
%    \end{macrocode}
%
% \subsection{Load package \xpackage{infwarerr}}
%
%    \begin{macrocode}
\begingroup\expandafter\expandafter\expandafter\endgroup
\expandafter\ifx\csname RequirePackage\endcsname\relax
  \input infwarerr.sty\relax
\else
  \RequirePackage{infwarerr}[2007/09/09]%
\fi
%    \end{macrocode}
%
% \subsection{Without \LuaTeX}
%
%    \begin{macrocode}
\begingroup\expandafter\expandafter\expandafter\endgroup
\expandafter\ifx\csname directlua\endcsname\relax
  \@PackageInfo{pdftexcmds}{LuaTeX not detected}%
  \def\pdftexcmds@nopdftex{%
    \@PackageInfoNoLine{pdftexcmds}{pdfTeX >= 1.30 not detected}%
    \let\pdftexcmds@nopdftex\relax
  }%
  \def\pdftexcmds@temp#1{%
    \begingroup\expandafter\expandafter\expandafter\endgroup
    \expandafter\ifx\csname pdf#1\endcsname\relax
      \pdftexcmds@nopdftex
    \else
      \expandafter\def\csname pdf@#1\expandafter\endcsname
      \expandafter##\expandafter{%
        \csname pdf#1\endcsname
      }%
    \fi
  }%
  \pdftexcmds@temp{strcmp}%
  \pdftexcmds@temp{escapehex}%
  \let\pdf@escapehexnative\pdf@escapehex
  \pdftexcmds@temp{unescapehex}%
  \let\pdf@unescapehexnative\pdf@unescapehex
  \pdftexcmds@temp{escapestring}%
  \pdftexcmds@temp{escapename}%
  \pdftexcmds@temp{filesize}%
  \pdftexcmds@temp{filemoddate}%
  \begingroup\expandafter\expandafter\expandafter\endgroup
  \expandafter\ifx\csname pdfshellescape\endcsname\relax
    \pdftexcmds@nopdftex
  \else
    \def\pdf@shellescape{%
      \pdfshellescape
    }%
  \fi
  \begingroup\expandafter\expandafter\expandafter\endgroup
  \expandafter\ifx\csname pdffiledump\endcsname\relax
    \pdftexcmds@nopdftex
  \else
    \def\pdf@filedump#1#2#3{%
      \pdffiledump offset#1 length#2{#3}%
    }%
  \fi
  \begingroup\expandafter\expandafter\expandafter\endgroup
  \expandafter\ifx\csname pdfmdfivesum\endcsname\relax
    \pdftexcmds@nopdftex
  \else
    \def\pdf@mdfivesum#{\pdfmdfivesum}%
    \let\pdf@mdfivesumnative\pdf@mdfivesum
    \def\pdf@filemdfivesum#{\pdfmdfivesum file}%
  \fi
  \def\pdf@system#{%
    \immediate\write18%
  }%
  \pdftexcmds@AtEnd
  \expandafter\endinput
\fi
%    \end{macrocode}
%
% \subsection{Load module}
%
%    \begin{macrocode}
\begingroup\expandafter\expandafter\expandafter\endgroup
\expandafter\ifx\csname RequirePackage\endcsname\relax
  \input luatex-loader.sty\relax
\else
  \RequirePackage{luatex-loader}[2007/12/12]%
\fi
\directlua0{%
  require("oberdiek.pdftexcmds")%
}
%    \end{macrocode}
%
% \subsection{Lua functions}
%
%    \begin{macro}{\pdftexcmds@toks}
%    \begin{macrocode}
\begingroup\expandafter\expandafter\expandafter\endgroup
\expandafter\ifx\csname newtoks\endcsname\relax
  \toksdef\pdftexcmds@toks=0 %
\else
  \csname newtoks\endcsname\pdftexcmds@toks
\fi
%    \end{macrocode}
%    \end{macro}
%
%    \begin{macro}{\pdf@strcmp}
%    \begin{macrocode}
\long\def\pdf@strcmp#1#2{%
  \directlua0{%
    oberdiek.pdftexcmds.strcmp("\luaescapestring{#1}",%
        "\luaescapestring{#2}")%
  }%
}%
%    \end{macrocode}
%    \end{macro}
%    \begin{macro}{\pdf@escapehex}
%    \begin{macrocode}
\long\def\pdf@escapehex#1{%
  \directlua0{%
    oberdiek.pdftexcmds.escapehex("\luaescapestring{#1}", "byte")%
  }%
}%
%    \end{macrocode}
%    \end{macro}
%    \begin{macro}{\pdf@escapehexnative}
%    \begin{macrocode}
\long\def\pdf@escapehexnative#1{%
  \directlua0{%
    oberdiek.pdftexcmds.escapehex("\luaescapestring{#1}")%
  }%
}%
%    \end{macrocode}
%    \end{macro}
%    \begin{macro}{\pdf@unescapehex}
%    \begin{macrocode}
\def\pdf@unescapehex#1{%
  \the\expandafter\pdftexcmds@toks
  \directlua0{%
    oberdiek.pdftexcmds.toks="pdftexcmds@toks"%
    oberdiek.pdftexcmds.unescapehex("\luaescapestring{#1}", "byte")%
  }%
}%
%    \end{macrocode}
%    \end{macro}
%    \begin{macro}{\pdf@unescapehexnative}
%    \begin{macrocode}
\def\pdf@unescapehexnative#1{%
  \the\expandafter\pdftexcmds@toks
  \directlua0{%
    oberdiek.pdftexcmds.toks="pdftexcmds@toks"%
    oberdiek.pdftexcmds.unescapehex("\luaescapestring{#1}")%
  }%
}%
%    \end{macrocode}
%    \end{macro}
%    \begin{macro}{\pdf@escapestring}
%    \begin{macrocode}
\long\def\pdf@escapestring#1{%
  \directlua0{%
    oberdiek.pdftexcmds.escapestring("\luaescapestring{#1}", "byte")%
  }%
}
%    \end{macrocode}
%    \end{macro}
%    \begin{macro}{\pdf@escapename}
%    \begin{macrocode}
\long\def\pdf@escapename#1{%
  \directlua0{%
    oberdiek.pdftexcmds.escapename("\luaescapestring{#1}", "byte")%
  }%
}
%    \end{macrocode}
%    \end{macro}
%    \begin{macro}{\pdf@escapenamenative}
%    \begin{macrocode}
\long\def\pdf@escapenamenative#1{%
  \directlua0{%
    oberdiek.pdftexcmds.escapename("\luaescapestring{#1}")%
  }%
}
%    \end{macrocode}
%    \end{macro}
%    \begin{macro}{\pdf@filesize}
%    \begin{macrocode}
\def\pdf@filesize#1{%
  \directlua0{%
    oberdiek.pdftexcmds.filesize("\luaescapestring{#1}")%
  }%
}
%    \end{macrocode}
%    \end{macro}
%    \begin{macro}{\pdf@filemoddate}
%    \begin{macrocode}
\def\pdf@filemoddate#1{%
  \directlua0{%
    oberdiek.pdftexcmds.filemoddate("\luaescapestring{#1}")%
  }%
}
%    \end{macrocode}
%    \end{macro}
%    \begin{macro}{\pdf@filedump}
%    \begin{macrocode}
\def\pdf@filedump#1#2#3{%
  \directlua0{%
    oberdiek.pdftexcmds.filedump("\luaescapestring{\number#1}",%
        "\luaescapestring{\number#2}",%
        "\luaescapestring{#3}")%
  }%
}%
%    \end{macrocode}
%    \end{macro}
%    \begin{macro}{\pdf@mdfivesum}
%    \begin{macrocode}
\long\def\pdf@mdfivesum#1{%
  \directlua0{%
    oberdiek.pdftexcmds.mdfivesum("\luaescapestring{#1}", "byte")%
  }%
}%
%    \end{macrocode}
%    \end{macro}
%    \begin{macro}{\pdf@mdfivesumnative}
%    \begin{macrocode}
\long\def\pdf@mdfivesumnative#1{%
  \directlua0{%
    oberdiek.pdftexcmds.mdfivesum("\luaescapestring{#1}")%
  }%
}%
%    \end{macrocode}
%    \end{macro}
%    \begin{macro}{\pdf@filemdfivesum}
%    \begin{macrocode}
\def\pdf@filemdfivesum#1{%
  \directlua0{%
    oberdiek.pdftexcmds.filemdfivesum("\luaescapestring{#1}")%
  }%
}%
%    \end{macrocode}
%    \end{macro}
%
%    \begin{macro}{\pdf@shellescape}
%    \begin{macrocode}
\def\pdf@shellescape{%
  \directlua0{%
    oberdiek.pdftexcmds.shellescape()%
  }%
}
%    \end{macrocode}
%    \end{macro}
%
%    \begin{macro}{\pdf@system}
%    \begin{macrocode}
\def\pdf@system#1{%
  \directlua0{%
    oberdiek.pdftexcmds.system("\luaescapestring{#1}")%
  }%
}
%    \end{macrocode}
%    \end{macro}
%
%    \begin{macro}{\pdf@lastsystemstatus}
%    \begin{macrocode}
\def\pdf@lastsystemstatus{%
  \directlua0{%
    oberdiek.pdftexcmds.lastsystemstatus()%
  }%
}
%    \end{macrocode}
%    \end{macro}
%    \begin{macro}{\pdf@lastsystemexit}
%    \begin{macrocode}
\def\pdf@lastsystemexit{%
  \directlua0{%
    oberdiek.pdftexcmds.lastsystemexit()%
  }%
}
%    \end{macrocode}
%    \end{macro}
%
%    \begin{macro}{\pdf@pipe}
%    Check availability of |io.popen| first.
%    \begin{macrocode}
\ifnum0%
    \directlua0{%
      if io.popen then %
        tex.write("1")%
      end%
    }%
    =1 %
  \def\pdf@pipe#1{%
    \the\expandafter\pdftexcmds@toks
    \directlua0{%
      oberdiek.pdftexcmds.toks="pdftexcmds@toks"%
      oberdiek.pdftexcmds.pipe("\luaescapestring{#1}")%
    }%
  }%
\fi
%    \end{macrocode}
%    \end{macro}
%
%    \begin{macrocode}
\pdftexcmds@AtEnd
%</package>
%    \end{macrocode}
%
% \subsection{Lua module}
%
%    \begin{macrocode}
%<*lua>
%    \end{macrocode}
%
%    \begin{macrocode}
module("oberdiek.pdftexcmds", package.seeall)
local systemexitstatus
function strcmp(A, B)
  if A == B then
    tex.write("0")
  elseif A < B then
    tex.write("-1")
  else
    tex.write("1")
  end
end
local function utf8_to_byte(str)
  local i = 0
  local n = string.len(str)
  local t = {}
  while i < n do
    i = i + 1
    local a = string.byte(str, i)
    if a < 128 then
      table.insert(t, string.char(a))
    else
      if a >= 192 and i < n then
        i = i + 1
        local b = string.byte(str, i)
        if b < 128 or b >= 192 then
          i = i - 1
        elseif a == 194 then
          table.insert(t, string.char(b))
        elseif a == 195 then
          table.insert(t, string.char(b + 64))
        end
      end
    end
  end
  return table.concat(t)
end
function escapehex(str, mode)
  if mode == "byte" then
    str = utf8_to_byte(str)
  end
  tex.write((string.gsub(str, ".",
    function (ch)
      return string.format("%02X", string.byte(ch))
    end
  )))
end
%    \end{macrocode}
%    See procedure |unescapehex| in file \xfile{utils.c} of \pdfTeX.
%    Caution: |tex.write| ignores leading spaces.
%    \begin{macrocode}
function unescapehex(str, mode)
  local a = 0
  local first = true
  local result = {}
  for i = 1, string.len(str), 1 do
    local ch = string.byte(str, i)
    if ch >= 48 and ch <= 57 then
      ch = ch - 48
    elseif ch >= 65 and ch <= 70 then
      ch = ch - 55
    elseif ch >= 97 and ch <= 102 then
      ch = ch - 87
    else
      ch = nil
    end
    if ch then
      if first then
        a = ch * 16
        first = false
      else
        table.insert(result, a + ch)
        first = true
      end
    end
  end
  if not first then
    table.insert(result, a)
  end
  if mode == "byte" then
    local utf8 = {}
    for i, a in ipairs(result) do
      if a < 128 then
        table.insert(utf8, a)
      else
        if a < 192 then
          table.insert(utf8, 194)
          a = a - 128
        else
          table.insert(utf8, 195)
          a = a - 192
        end
        table.insert(utf8, a + 128)
      end
    end
    result = utf8
  end
  tex.settoks(toks, string.char(unpack(result)))
end
%    \end{macrocode}
%    See procedure |escapestring| in file \xfile{utils.c} of \pdfTeX.
%    \begin{macrocode}
function escapestring(str, mode)
  if mode == "byte" then
    str = utf8_to_byte(str)
  end
  tex.write((string.gsub(str, ".",
    function (ch)
      local b = string.byte(ch)
      if b < 33 or b > 126 then
        return string.format("\\%.3o", b)
      end
      if b == 40 or b == 41 or b == 92 then
        return "\\" .. ch
      end
%    \end{macrocode}
%    Lua 5.1 returns the match in case of return value |nil|.
%    \begin{macrocode}
      return nil
    end
  )))
end
%    \end{macrocode}
%    See procedure |escapename| in file \xfile{utils.c} of \pdfTeX.
%    \begin{macrocode}
function escapename(str, mode)
  if mode == "byte" then
    str = utf8_to_byte(str)
  end
  tex.write((string.gsub(str, ".",
    function (ch)
      local b = string.byte(ch)
      if b == 0 then
%    \end{macrocode}
%    In Lua 5.0 |nil| could be used for the empty string,
%    But |nil| returns the match in Lua 5.1, thus we use
%    the empty string explicitly.
%    \begin{macrocode}
        return ""
      end
      if b <= 32 or b >= 127
          or b == 35 or b == 37 or b == 40 or b == 41
          or b == 47 or b == 60 or b == 62 or b == 91
          or b == 93 or b == 123 or b == 125 then
        return string.format("#%.2X", b)
      else
%    \end{macrocode}
%    Lua 5.1 returns the match in case of return value |nil|.
%    \begin{macrocode}
        return nil
      end
    end
  )))
end
function filesize(filename)
  local foundfile = kpse.find_file(filename, "tex", true)
  if foundfile then
    local size = lfs.attributes(foundfile, "size")
    if size then
      tex.write(size)
    end
  end
end
%    \end{macrocode}
%    See procedure |makepdftime| in file \xfile{utils.c} of \pdfTeX.
%    \begin{macrocode}
function filemoddate(filename)
  local foundfile = kpse.find_file(filename, "tex", true)
  if foundfile then
    local date = lfs.attributes(foundfile, "modification")
    if date then
      local d = os.date("*t", date)
      if d.sec >= 60 then
        d.sec = 59
      end
      local u = os.date("!*t", date)
      local off = 60 * (d.hour - u.hour) + d.min - u.min
      if d.year ~= u.year then
        if d.year > u.year then
          off = off + 1440
        else
          off = off - 1440
        end
      elseif d.yday ~= u.yday then
        if d.yday > u.yday then
          off = off + 1440
        else
          off = off - 1440
        end
      end
      local timezone
      if off == 0 then
        timezone = "Z"
      else
        local hours = math.floor(off / 60)
        local mins = math.abs(off - hours * 60)
        timezone = string.format("%+03d'%02d'", hours, mins)
      end
      tex.write(string.format("D:%04d%02d%02d%02d%02d%02d%s",
          d.year, d.month, d.day, d.hour, d.min, d.sec, timezone))
    end
  end
end
function filedump(offset, length, filename)
  length = tonumber(length)
  if length and length > 0 then
    local foundfile = kpse.find_file(filename, "tex", true)
    if foundfile then
      offset = tonumber(offset)
      if not offset then
        offset = 0
      end
      local filehandle = io.open(foundfile, "r")
      if filehandle then
        if offset > 0 then
          filehandle:seek("set", offset)
        end
        local dump = filehandle:read(length)
        escapehex(dump)
      end
    end
  end
end
function mdfivesum(str, mode)
  if mode == "byte" then
    str = utf8_to_byte(str)
  end
  escapehex(md5.sum(str))
end
function filemdfivesum(filename)
  local foundfile = kpse.find_file(filename, "tex", true)
  if foundfile then
    local filehandle = io.open(foundfile, "r")
    if filehandle then
      local contents = filehandle:read("*a")
      escapehex(md5.sum(contents))
    end
  end
end
function shellescape()
  if os.execute then
    tex.write("1")
  else
    tex.write("0")
  end
end
function system(cmdline)
  systemexitstatus = nil
  texio.write_nl("log", "system(" .. cmdline .. ") ")
  if os.execute then
    texio.write("log", "executed.")
    systemexitstatus = os.execute(cmdline)
  else
    texio.write("log", "disabled.")
  end
end
function lastsystemstatus()
  local result = tonumber(systemexitstatus)
  if result then
    local x = math.floor(result / 256)
    tex.write(result - 256 * math.floor(result / 256))
  end
end
function lastsystemexit()
  local result = tonumber(systemexitstatus)
  if result then
    tex.write(math.floor(result / 256))
  end
end
function pipe(cmdline)
  local result
  systemexitstatus = nil
  texio.write_nl("log", "pipe(" .. cmdline ..") ")
  if io.popen then
    texio.write("log", "executed.")
    local handle = io.popen(cmdline, "r")
    if handle then
      result = handle:read("*a")
      handle:close()
    end
  else
    texio.write("log", "disabled.")
  end
  if result then
    tex.settoks(toks, result)
  else
    tex.settoks(toks, "")
  end
end
%    \end{macrocode}
%    \begin{macrocode}
%</lua>
%    \end{macrocode}
%
% \section{Test}
%
% \subsection{Catcode checks for loading}
%
%    \begin{macrocode}
%<*test1>
%    \end{macrocode}
%    \begin{macrocode}
\catcode`\{=1 %
\catcode`\}=2 %
\catcode`\#=6 %
\catcode`\@=11 %
\expandafter\ifx\csname count@\endcsname\relax
  \countdef\count@=255 %
\fi
\expandafter\ifx\csname @gobble\endcsname\relax
  \long\def\@gobble#1{}%
\fi
\expandafter\ifx\csname @firstofone\endcsname\relax
  \long\def\@firstofone#1{#1}%
\fi
\expandafter\ifx\csname loop\endcsname\relax
  \expandafter\@firstofone
\else
  \expandafter\@gobble
\fi
{%
  \def\loop#1\repeat{%
    \def\body{#1}%
    \iterate
  }%
  \def\iterate{%
    \body
      \let\next\iterate
    \else
      \let\next\relax
    \fi
    \next
  }%
  \let\repeat=\fi
}%
\def\RestoreCatcodes{}
\count@=0 %
\loop
  \edef\RestoreCatcodes{%
    \RestoreCatcodes
    \catcode\the\count@=\the\catcode\count@\relax
  }%
\ifnum\count@<255 %
  \advance\count@ 1 %
\repeat

\def\RangeCatcodeInvalid#1#2{%
  \count@=#1\relax
  \loop
    \catcode\count@=15 %
  \ifnum\count@<#2\relax
    \advance\count@ 1 %
  \repeat
}
\expandafter\ifx\csname LoadCommand\endcsname\relax
  \def\LoadCommand{\input pdftexcmds.sty\relax}%
\fi
\def\Test{%
  \RangeCatcodeInvalid{0}{47}%
  \RangeCatcodeInvalid{58}{64}%
  \RangeCatcodeInvalid{91}{96}%
  \RangeCatcodeInvalid{123}{255}%
  \catcode`\@=12 %
  \catcode`\\=0 %
  \catcode`\{=1 %
  \catcode`\}=2 %
  \catcode`\#=6 %
  \catcode`\[=12 %
  \catcode`\]=12 %
  \catcode`\%=14 %
  \catcode`\ =10 %
  \catcode13=5 %
  \LoadCommand
  \RestoreCatcodes
}
\Test
\csname @@end\endcsname
\end
%    \end{macrocode}
%    \begin{macrocode}
%</test1>
%    \end{macrocode}
%
% \section{Installation}
%
% \subsection{Download}
%
% \paragraph{Package.} This package is available on
% CTAN\footnote{\url{ftp://ftp.ctan.org/tex-archive/}}:
% \begin{description}
% \item[\CTAN{macros/latex/contrib/oberdiek/pdftexcmds.dtx}] The source file.
% \item[\CTAN{macros/latex/contrib/oberdiek/pdftexcmds.pdf}] Documentation.
% \end{description}
%
%
% \paragraph{Bundle.} All the packages of the bundle `oberdiek'
% are also available in a TDS compliant ZIP archive. There
% the packages are already unpacked and the documentation files
% are generated. The files and directories obey the TDS standard.
% \begin{description}
% \item[\CTAN{macros/latex/contrib/oberdiek/oberdiek-tds.zip}]
% \end{description}
% \emph{TDS} refers to the standard ``A Directory Structure
% for \TeX\ Files'' (\CTAN{tds/tds.pdf}). Directories
% with \xfile{texmf} in their name are usually organized this way.
%
% \subsection{Bundle installation}
%
% \paragraph{Unpacking.} Unpack the \xfile{oberdiek-tds.zip} in the
% TDS tree (also known as \xfile{texmf} tree) of your choice.
% Example (linux):
% \begin{quote}
%   |unzip oberdiek-tds.zip -d ~/texmf|
% \end{quote}
%
% \paragraph{Script installation.}
% Check the directory \xfile{TDS:scripts/oberdiek/} for
% scripts that need further installation steps.
% Package \xpackage{attachfile2} comes with the Perl script
% \xfile{pdfatfi.pl} that should be installed in such a way
% that it can be called as \texttt{pdfatfi}.
% Example (linux):
% \begin{quote}
%   |chmod +x scripts/oberdiek/pdfatfi.pl|\\
%   |cp scripts/oberdiek/pdfatfi.pl /usr/local/bin/|
% \end{quote}
%
% \subsection{Package installation}
%
% \paragraph{Unpacking.} The \xfile{.dtx} file is a self-extracting
% \docstrip\ archive. The files are extracted by running the
% \xfile{.dtx} through \plainTeX:
% \begin{quote}
%   \verb|tex pdftexcmds.dtx|
% \end{quote}
%
% \paragraph{TDS.} Now the different files must be moved into
% the different directories in your installation TDS tree
% (also known as \xfile{texmf} tree):
% \begin{quote}
% \def\t{^^A
% \begin{tabular}{@{}>{\ttfamily}l@{ $\rightarrow$ }>{\ttfamily}l@{}}
%   pdftexcmds.sty & tex/generic/oberdiek/pdftexcmds.sty\\
%   oberdiek.pdftexcmds.lua & scripts/oberdiek/oberdiek.pdftexcmds.lua\\
%   pdftexcmds.lua & scripts/oberdiek/pdftexcmds.lua\\
%   pdftexcmds.pdf & doc/latex/oberdiek/pdftexcmds.pdf\\
%   pdftexcmds.dtx & source/latex/oberdiek/pdftexcmds.dtx\\
% \end{tabular}^^A
% }^^A
% \sbox0{\t}^^A
% \ifdim\wd0>\linewidth
%   \begingroup
%     \advance\linewidth by\leftmargin
%     \advance\linewidth by\rightmargin
%   \edef\x{\endgroup
%     \def\noexpand\lw{\the\linewidth}^^A
%   }\x
%   \def\lwbox{^^A
%     \leavevmode
%     \hbox to \linewidth{^^A
%       \kern-\leftmargin\relax
%       \hss
%       \usebox0
%       \hss
%       \kern-\rightmargin\relax
%     }^^A
%   }^^A
%   \ifdim\wd0>\lw
%     \sbox0{\small\t}^^A
%     \ifdim\wd0>\linewidth
%       \ifdim\wd0>\lw
%         \sbox0{\footnotesize\t}^^A
%         \ifdim\wd0>\linewidth
%           \ifdim\wd0>\lw
%             \sbox0{\scriptsize\t}^^A
%             \ifdim\wd0>\linewidth
%               \ifdim\wd0>\lw
%                 \sbox0{\tiny\t}^^A
%                 \ifdim\wd0>\linewidth
%                   \lwbox
%                 \else
%                   \usebox0
%                 \fi
%               \else
%                 \lwbox
%               \fi
%             \else
%               \usebox0
%             \fi
%           \else
%             \lwbox
%           \fi
%         \else
%           \usebox0
%         \fi
%       \else
%         \lwbox
%       \fi
%     \else
%       \usebox0
%     \fi
%   \else
%     \lwbox
%   \fi
% \else
%   \usebox0
% \fi
% \end{quote}
% If you have a \xfile{docstrip.cfg} that configures and enables \docstrip's
% TDS installing feature, then some files can already be in the right
% place, see the documentation of \docstrip.
%
% \subsection{Refresh file name databases}
%
% If your \TeX~distribution
% (\teTeX, \mikTeX, \dots) relies on file name databases, you must refresh
% these. For example, \teTeX\ users run \verb|texhash| or
% \verb|mktexlsr|.
%
% \subsection{Some details for the interested}
%
% \paragraph{Attached source.}
%
% The PDF documentation on CTAN also includes the
% \xfile{.dtx} source file. It can be extracted by
% AcrobatReader 6 or higher. Another option is \textsf{pdftk},
% e.g. unpack the file into the current directory:
% \begin{quote}
%   \verb|pdftk pdftexcmds.pdf unpack_files output .|
% \end{quote}
%
% \paragraph{Unpacking with \LaTeX.}
% The \xfile{.dtx} chooses its action depending on the format:
% \begin{description}
% \item[\plainTeX:] Run \docstrip\ and extract the files.
% \item[\LaTeX:] Generate the documentation.
% \end{description}
% If you insist on using \LaTeX\ for \docstrip\ (really,
% \docstrip\ does not need \LaTeX), then inform the autodetect routine
% about your intention:
% \begin{quote}
%   \verb|latex \let\install=y% \iffalse meta-comment
%
% Copyright (C) 2007 by Heiko Oberdiek <oberdiek@uni-freiburg.de>
%
% This work may be distributed and/or modified under the
% conditions of the LaTeX Project Public License, either
% version 1.3 of this license or (at your option) any later
% version. The latest version of this license is in
%    http://www.latex-project.org/lppl.txt
% and version 1.3 or later is part of all distributions of
% LaTeX version 2005/12/01 or later.
%
% This work has the LPPL maintenance status "maintained".
%
% This Current Maintainer of this work is Heiko Oberdiek.
%
% This work consists of the main source file pdftexcmds.dtx
% and the derived files
%    pdftexcmds.sty, pdftexcmds.pdf, pdftexcmds.ins, pdftexcmds.drv,
%    oberdiek.pdftexcmds.lua, pdftexcmds.lua.
%
% Distribution:
%    CTAN:macros/latex/contrib/oberdiek/pdftexcmds.dtx
%    CTAN:macros/latex/contrib/oberdiek/pdftexcmds.pdf
%
% Unpacking:
%    (a) If pdftexcmds.ins is present:
%           tex pdftexcmds.ins
%    (b) Without pdftexcmds.ins:
%           tex pdftexcmds.dtx
%    (c) If you insist on using LaTeX
%           latex \let\install=y% \iffalse meta-comment
%
% Copyright (C) 2007 by Heiko Oberdiek <oberdiek@uni-freiburg.de>
%
% This work may be distributed and/or modified under the
% conditions of the LaTeX Project Public License, either
% version 1.3 of this license or (at your option) any later
% version. The latest version of this license is in
%    http://www.latex-project.org/lppl.txt
% and version 1.3 or later is part of all distributions of
% LaTeX version 2005/12/01 or later.
%
% This work has the LPPL maintenance status "maintained".
%
% This Current Maintainer of this work is Heiko Oberdiek.
%
% This work consists of the main source file pdftexcmds.dtx
% and the derived files
%    pdftexcmds.sty, pdftexcmds.pdf, pdftexcmds.ins, pdftexcmds.drv,
%    oberdiek.pdftexcmds.lua, pdftexcmds.lua.
%
% Distribution:
%    CTAN:macros/latex/contrib/oberdiek/pdftexcmds.dtx
%    CTAN:macros/latex/contrib/oberdiek/pdftexcmds.pdf
%
% Unpacking:
%    (a) If pdftexcmds.ins is present:
%           tex pdftexcmds.ins
%    (b) Without pdftexcmds.ins:
%           tex pdftexcmds.dtx
%    (c) If you insist on using LaTeX
%           latex \let\install=y\input{pdftexcmds.dtx}
%        (quote the arguments according to the demands of your shell)
%
% Documentation:
%    (a) If pdftexcmds.drv is present:
%           latex pdftexcmds.drv
%    (b) Without pdftexcmds.drv:
%           latex pdftexcmds.dtx; ...
%    The class ltxdoc loads the configuration file ltxdoc.cfg
%    if available. Here you can specify further options, e.g.
%    use A4 as paper format:
%       \PassOptionsToClass{a4paper}{article}
%
%    Programm calls to get the documentation (example):
%       pdflatex pdftexcmds.dtx
%       makeindex -s gind.ist pdftexcmds.idx
%       pdflatex pdftexcmds.dtx
%       makeindex -s gind.ist pdftexcmds.idx
%       pdflatex pdftexcmds.dtx
%
% Installation:
%    TDS:tex/generic/oberdiek/pdftexcmds.sty
%    TDS:scripts/oberdiek/oberdiek.pdftexcmds.lua
%    TDS:scripts/oberdiek/pdftexcmds.lua
%    TDS:doc/latex/oberdiek/pdftexcmds.pdf
%    TDS:source/latex/oberdiek/pdftexcmds.dtx
%
%<*ignore>
\begingroup
  \def\x{LaTeX2e}%
\expandafter\endgroup
\ifcase 0\ifx\install y1\fi\expandafter
         \ifx\csname processbatchFile\endcsname\relax\else1\fi
         \ifx\fmtname\x\else 1\fi\relax
\else\csname fi\endcsname
%</ignore>
%<*install>
\input docstrip.tex
\Msg{************************************************************************}
\Msg{* Installation}
\Msg{* Package: pdftexcmds 2007/12/12 v0.3 LuaTeX support for pdfTeX utility functions (HO)}
\Msg{************************************************************************}

\keepsilent
\askforoverwritefalse

\let\MetaPrefix\relax
\preamble

This is a generated file.

Copyright (C) 2007 by Heiko Oberdiek <oberdiek@uni-freiburg.de>

This work may be distributed and/or modified under the
conditions of the LaTeX Project Public License, either
version 1.3 of this license or (at your option) any later
version. The latest version of this license is in
   http://www.latex-project.org/lppl.txt
and version 1.3 or later is part of all distributions of
LaTeX version 2005/12/01 or later.

This work has the LPPL maintenance status "maintained".

This Current Maintainer of this work is Heiko Oberdiek.

This work consists of the main source file pdftexcmds.dtx
and the derived files
   pdftexcmds.sty, pdftexcmds.pdf, pdftexcmds.ins, pdftexcmds.drv,
   oberdiek.pdftexcmds.lua, pdftexcmds.lua.

\endpreamble
\let\MetaPrefix\DoubleperCent

\generate{%
  \file{pdftexcmds.ins}{\from{pdftexcmds.dtx}{install}}%
  \file{pdftexcmds.drv}{\from{pdftexcmds.dtx}{driver}}%
  \usedir{tex/generic/oberdiek}%
  \file{pdftexcmds.sty}{\from{pdftexcmds.dtx}{package}}%
}
\def\MetaPrefix{-- }
\def\defaultpostamble{%
  \MetaPrefix^^J%
  \MetaPrefix\space End of File `\outFileName'.%
}
\def\currentpostamble{\defaultpostamble}%
\generate{%
  \usedir{scripts/oberdiek}%
  \file{oberdiek.pdftexcmds.lua}{\from{pdftexcmds.dtx}{lua}}%
  \file{pdftexcmds.lua}{\from{pdftexcmds.dtx}{lua}}%
}

\obeyspaces
\Msg{************************************************************************}
\Msg{*}
\Msg{* To finish the installation you have to move the following}
\Msg{* file into a directory searched by TeX:}
\Msg{*}
\Msg{*     pdftexcmds.sty}
\Msg{*}
\Msg{* And install the following script files:}
\Msg{*}
\Msg{*     oberdiek.pdftexcmds.lua, pdftexcmds.lua}
\Msg{*}
\Msg{* To produce the documentation run the file `pdftexcmds.drv'}
\Msg{* through LaTeX.}
\Msg{*}
\Msg{* Happy TeXing!}
\Msg{*}
\Msg{************************************************************************}

\endbatchfile
%</install>
%<*ignore>
\fi
%</ignore>
%<*driver>
\NeedsTeXFormat{LaTeX2e}
\ProvidesFile{pdftexcmds.drv}%
  [2007/12/12 v0.3 LuaTeX support for pdfTeX utility functions (HO)]%
\documentclass{ltxdoc}
\usepackage{holtxdoc}[2007/11/11]
\usepackage{paralist}
\begin{document}
  \DocInput{pdftexcmds.dtx}%
\end{document}
%</driver>
% \fi
%
% \CheckSum{552}
%
% \CharacterTable
%  {Upper-case    \A\B\C\D\E\F\G\H\I\J\K\L\M\N\O\P\Q\R\S\T\U\V\W\X\Y\Z
%   Lower-case    \a\b\c\d\e\f\g\h\i\j\k\l\m\n\o\p\q\r\s\t\u\v\w\x\y\z
%   Digits        \0\1\2\3\4\5\6\7\8\9
%   Exclamation   \!     Double quote  \"     Hash (number) \#
%   Dollar        \$     Percent       \%     Ampersand     \&
%   Acute accent  \'     Left paren    \(     Right paren   \)
%   Asterisk      \*     Plus          \+     Comma         \,
%   Minus         \-     Point         \.     Solidus       \/
%   Colon         \:     Semicolon     \;     Less than     \<
%   Equals        \=     Greater than  \>     Question mark \?
%   Commercial at \@     Left bracket  \[     Backslash     \\
%   Right bracket \]     Circumflex    \^     Underscore    \_
%   Grave accent  \`     Left brace    \{     Vertical bar  \|
%   Right brace   \}     Tilde         \~}
%
% \GetFileInfo{pdftexcmds.drv}
%
% \title{The \xpackage{pdftexcmds} package}
% \date{2007/12/12 v0.3}
% \author{Heiko Oberdiek\\\xemail{oberdiek@uni-freiburg.de}}
%
% \maketitle
%
% \begin{abstract}
% \LuaTeX\ provides most of the commands of \pdfTeX\ 1.40. However
% a number of utility functions are removed. This package tries to fill
% the gap and implements some of the missing primitive using Lua.
% \end{abstract}
%
% \tableofcontents
%
% \section{Documentation}
%
% Some primitives of \pdfTeX\ are not defined by \LuaTeX.
% This package implements macro based solutions using Lua code
% for the following missing \pdfTeX\ primitives;
% \begin{compactitem}
% \item \cs{pdfstrcmp}
% \item \cs{pdfunescapehex}
% \item \cs{pdfescapehex}
% \item \cs{pdfescapename}
% \item \cs{pdfescapestring}
% \item \cs{pdffilesize}
% \item \cs{pdffilemoddate}
% \item \cs{pdffiledump}
% \item \cs{pdfmdfivesum}
% \item |\immediate\write18|
% \end{compactitem}
% The original names of the primitives cannot be used:
% \begin{itemize}
% \item
% The syntax for their arguments cannot easily
% simulated by macros. The primitives using key words
% such as |file| (\cs{pdfmdfivesum}) or |offset| and |length|
% (\cs{pdffiledump}) and uses \meta{general text} for the other
% arguments. Using token registers assignments, \meta{general text} could
% be catched. However, the simulated primitives are expandable
% and register assignments would destroy this important property.
% (\meta{general text} allows something like |\expandafter\bgroup ...}|.)
% \item
% The original primitives can be expanded using one expansion step.
% The new macros need two expansion steps because of the additional
% macro expansion. Example:
% \begin{quote}
%   |\expandafter\foo\pdffilemoddate{file}|\\
%   vs.
%   |\expandafter\expandafter\expandafter\foo\pdf@filemoddate{file}|.
% \end{quote}
% \end{itemize}
%
% \LuaTeX\ isn't stable yet and thus the status of this package is
% \emph{experimental}. Feedback is welcome.
%
% \subsection{General principles}
%
% \begin{description}
% \item[Naming convention:]
%   Usually this package defines a macro |\pdf@|\meta{cmd} if
%   \pdfTeX\ provides |\pdf|\meta{cmd}.
% \item[Arguments:] The order of arguments in |\pdf@|\meta{cmd}
%   is the same as for the corresponding primitive of \pdfTeX.
%   The arguments are ordinary undelimited \TeX\ arguments,
%   no \meta{general text} and without additional keywords.
% \item[Expandibility:]
%   The macro |\pdf@|\meta{cmd} is expandable if the
%   corresponding \pdfTeX\ primitive has this property.
%   Exact two expansion steps are necessary (first is the macro
%   expansion).
% \item[Without \LuaTeX:]
%   The macros |\pdf@|\meta{cmd} are mapped to the commands
%   of \pdfTeX\ if they are available. Otherwise they are undefined.
% \end{description}
%
% \subsection{Macros}
%
% \begin{declcs}{pdf@strcmp} \M{stringA} \M{stringB}
% \end{declcs}
% Same as |\pdfstrcmp{|\meta{stringA}|}{|\meta{stringB}|}|.
%
% \begin{declcs}{pdf@unescapehex} \M{string}
% \end{declcs}
% Same as |\pdfunescapehex{|\meta{string}|}|.
% The argument is a byte string given in hexadecimal notation.
% The result are character tokens from 0 until 255 with
% catcode 12 and the space with catcode 10.
%
% \begin{declcs}{pdf@escapehex} \M{string}\\
%   \cs{pdf@escapestring} \M{string}\\
%   \cs{pdf@escapename} \M{string}
% \end{declcs}
% Same as the primitives of \pdfTeX. However \pdfTeX\ does not
% know about characters with codes 256 and larger. Thus the
% string is treated as byte string, characters with more than
% eight bits are ignored.
%
% \begin{declcs}{pdf@filesize} \M{filename}
% \end{declcs}
% Same as |\pdffilesize{|\meta{filename}|}|.
%
% \begin{declcs}{pdf@filemoddate} \M{filename}
% \end{declcs}
% Same as |\pdffilemoddate{|\meta{filename}|}|.
%
% \begin{declcs}{pdf@filedump} \M{offset} \M{length} \M{filename}
% \end{declcs}
% Same as |\pdffiledump offset| \meta{offset} |length| \meta{length}
% |{|\meta{filename}|}|. Both \meta{offset} and \meta{length} must
% not be empty, but must be a valid \TeX\ number.
%
% \begin{declcs}{pdf@mdfivesum} \M{string}
% \end{declcs}
% Same as |\pdfmdfivesum{|\meta{string}|}|. Keyword |file| is supported
% by macro \cs{pdf@filemdfivesum}.
%
% \begin{declcs}{pdf@filemdfivesum} \M{filename}
% \end{declcs}
% Same as |\pdfmdfivesum file{|\meta{filename}|}|.
%
% \begin{declcs}{pdf@shellescape}
% \end{declcs}
% Same as |\pdfshellescape|. It expands to |1| if external
% commands can be executed and |0| otherwise. In \pdfTeX\ external
% commands must be enabled first by command line option or
% configuration option. In \LuaTeX\ option |--safer| disables
% the execution of external commands.
%
% \begin{declcs}{pdf@system} \M{cmdline}
% \end{declcs}
% It is a wrapper for |\immediate\write18| in \pdfTeX\ or
% |os.execute| in \LuaTeX.
%
% In theory |os.execute|
% returns a status number. But its meaning is quite
% undefined. Are there some reliable properties?
% Does it make sense to provide an user interface to
% this status exit code?
%
% \subsubsection{Experimental}
%
% \begin{declcs}{pdf@unescapehexnative} \M{string}\\
%   \cs{pdf@escapehexnative} \M{string}\\
%   \cs{pdf@escapenamenative} \M{string}\\
%   \cs{pdf@mdfivesumnative} \M{string}
% \end{declcs}
% The variants without |native| in the macro name are supposed to
% be compatible with \pdfTeX. However characters with more than
% eight bits are not supported and are ignored. If \LuaTeX\ is
% running, then its UTF-8 coded strings are used. Thus the full
% unicode character range is supported. However the result
% differs from \pdfTeX\ for characters with eight or more bits.
%
% \begin{declcs}{pdf@pipe} \M{cmdline}
% \end{declcs}
% It calls \meta{cmdline} and returns the output of the external
% program in the usual manner as byte string (catcode 12, space with
% catcode 10). The Lua documentation says, that the used |io.popen|
% may not be available on all platforms. Then macro \cs{pdf@pipe}
% is undefined.
%
% \StopEventually{
% }
%
% \section{Implementation}
%
%    \begin{macrocode}
%<*package>
%    \end{macrocode}
%
% \subsection{Reload check and package identification}
%    Reload check, especially if the package is not used with \LaTeX.
%    \begin{macrocode}
\begingroup
  \catcode44 12 % ,
  \catcode45 12 % -
  \catcode46 12 % .
  \catcode58 12 % :
  \catcode64 11 % @
  \expandafter\let\expandafter\x\csname ver@pdftexcmds.sty\endcsname
  \ifcase 0%
    \ifx\x\relax % plain
    \else
      \ifx\x\empty % LaTeX
      \else
        1%
      \fi
    \fi
  \else
    \catcode35 6 % #
    \catcode123 1 % {
    \catcode125 2 % }
    \expandafter\ifx\csname PackageInfo\endcsname\relax
      \def\x#1#2{%
        \immediate\write-1{Package #1 Info: #2.}%
      }%
    \else
      \def\x#1#2{\PackageInfo{#1}{#2, stopped}}%
    \fi
    \x{pdftexcmds}{The package is already loaded}%
    \endgroup
    \expandafter\endinput
  \fi
\endgroup
%    \end{macrocode}
%    Package identification:
%    \begin{macrocode}
\begingroup
  \catcode35 6 % #
  \catcode40 12 % (
  \catcode41 12 % )
  \catcode44 12 % ,
  \catcode45 12 % -
  \catcode46 12 % .
  \catcode47 12 % /
  \catcode58 12 % :
  \catcode64 11 % @
  \catcode123 1 % {
  \catcode125 2 % }
  \expandafter\ifx\csname ProvidesPackage\endcsname\relax
    \def\x#1#2#3[#4]{\endgroup
      \immediate\write-1{Package: #3 #4}%
      \xdef#1{#4}%
    }%
  \else
    \def\x#1#2[#3]{\endgroup
      #2[{#3}]%
      \ifx#1\relax
        \xdef#1{#3}%
      \fi
    }%
  \fi
\expandafter\x\csname ver@pdftexcmds.sty\endcsname
\ProvidesPackage{pdftexcmds}%
  [2007/12/12 v0.3 LuaTeX support for pdfTeX utility functions (HO)]
%    \end{macrocode}
%
% \subsection{Catcodes}
%
%    \begin{macrocode}
\begingroup
  \catcode123 1 % {
  \catcode125 2 % }
  \def\x{\endgroup
    \expandafter\edef\csname pdftexcmds@AtEnd\endcsname{%
      \catcode35 \the\catcode35\relax
      \catcode64 \the\catcode64\relax
      \catcode123 \the\catcode123\relax
      \catcode125 \the\catcode125\relax
    }%
  }%
\x
\catcode35 6 % #
\catcode64 11 % @
\catcode123 1 % {
\catcode125 2 % }
\def\TMP@EnsureCode#1#2{%
  \edef\pdftexcmds@AtEnd{%
    \pdftexcmds@AtEnd
    \catcode#1 \the\catcode#1\relax
  }%
  \catcode#1 #2\relax
}
\TMP@EnsureCode{10}{12}% ^^J
\TMP@EnsureCode{33}{12}% !
\TMP@EnsureCode{34}{12}% "
\TMP@EnsureCode{39}{12}% '
\TMP@EnsureCode{40}{12}% (
\TMP@EnsureCode{41}{12}% )
\TMP@EnsureCode{42}{12}% *
\TMP@EnsureCode{43}{12}% +
\TMP@EnsureCode{44}{12}% ,
\TMP@EnsureCode{45}{12}% -
\TMP@EnsureCode{46}{12}% .
\TMP@EnsureCode{47}{12}% /
\TMP@EnsureCode{58}{12}% :
\TMP@EnsureCode{60}{12}% <
\TMP@EnsureCode{61}{12}% =
\TMP@EnsureCode{62}{12}% >
\TMP@EnsureCode{94}{7}% ^ (superscript)
\TMP@EnsureCode{95}{12}% _ (other)
\TMP@EnsureCode{126}{12}% ~ (other)
%    \end{macrocode}
%
% \subsection{Load package \xpackage{infwarerr}}
%
%    \begin{macrocode}
\begingroup\expandafter\expandafter\expandafter\endgroup
\expandafter\ifx\csname RequirePackage\endcsname\relax
  \input infwarerr.sty\relax
\else
  \RequirePackage{infwarerr}[2007/09/09]%
\fi
%    \end{macrocode}
%
% \subsection{Without \LuaTeX}
%
%    \begin{macrocode}
\begingroup\expandafter\expandafter\expandafter\endgroup
\expandafter\ifx\csname directlua\endcsname\relax
  \@PackageInfo{pdftexcmds}{LuaTeX not detected}%
  \def\pdftexcmds@nopdftex{%
    \@PackageInfoNoLine{pdftexcmds}{pdfTeX >= 1.30 not detected}%
    \let\pdftexcmds@nopdftex\relax
  }%
  \def\pdftexcmds@temp#1{%
    \begingroup\expandafter\expandafter\expandafter\endgroup
    \expandafter\ifx\csname pdf#1\endcsname\relax
      \pdftexcmds@nopdftex
    \else
      \expandafter\def\csname pdf@#1\expandafter\endcsname
      \expandafter##\expandafter{%
        \csname pdf#1\endcsname
      }%
    \fi
  }%
  \pdftexcmds@temp{strcmp}%
  \pdftexcmds@temp{escapehex}%
  \let\pdf@escapehexnative\pdf@escapehex
  \pdftexcmds@temp{unescapehex}%
  \let\pdf@unescapehexnative\pdf@unescapehex
  \pdftexcmds@temp{escapestring}%
  \pdftexcmds@temp{escapename}%
  \pdftexcmds@temp{filesize}%
  \pdftexcmds@temp{filemoddate}%
  \begingroup\expandafter\expandafter\expandafter\endgroup
  \expandafter\ifx\csname pdfshellescape\endcsname\relax
    \pdftexcmds@nopdftex
  \else
    \def\pdf@shellescape{%
      \pdfshellescape
    }%
  \fi
  \begingroup\expandafter\expandafter\expandafter\endgroup
  \expandafter\ifx\csname pdffiledump\endcsname\relax
    \pdftexcmds@nopdftex
  \else
    \def\pdf@filedump#1#2#3{%
      \pdffiledump offset#1 length#2{#3}%
    }%
  \fi
  \begingroup\expandafter\expandafter\expandafter\endgroup
  \expandafter\ifx\csname pdfmdfivesum\endcsname\relax
    \pdftexcmds@nopdftex
  \else
    \def\pdf@mdfivesum#{\pdfmdfivesum}%
    \let\pdf@mdfivesumnative\pdf@mdfivesum
    \def\pdf@filemdfivesum#{\pdfmdfivesum file}%
  \fi
  \def\pdf@system#{%
    \immediate\write18%
  }%
  \pdftexcmds@AtEnd
  \expandafter\endinput
\fi
%    \end{macrocode}
%
% \subsection{Load module}
%
%    \begin{macrocode}
\begingroup\expandafter\expandafter\expandafter\endgroup
\expandafter\ifx\csname RequirePackage\endcsname\relax
  \input luatex-loader.sty\relax
\else
  \RequirePackage{luatex-loader}[2007/12/12]%
\fi
\directlua0{%
  require("oberdiek.pdftexcmds")%
}
%    \end{macrocode}
%
% \subsection{Lua functions}
%
%    \begin{macro}{\pdftexcmds@toks}
%    \begin{macrocode}
\begingroup\expandafter\expandafter\expandafter\endgroup
\expandafter\ifx\csname newtoks\endcsname\relax
  \toksdef\pdftexcmds@toks=0 %
\else
  \csname newtoks\endcsname\pdftexcmds@toks
\fi
%    \end{macrocode}
%    \end{macro}
%
%    \begin{macro}{\pdf@strcmp}
%    \begin{macrocode}
\long\def\pdf@strcmp#1#2{%
  \directlua0{%
    oberdiek.pdftexcmds.strcmp("\luaescapestring{#1}",%
        "\luaescapestring{#2}")%
  }%
}%
%    \end{macrocode}
%    \end{macro}
%    \begin{macro}{\pdf@escapehex}
%    \begin{macrocode}
\long\def\pdf@escapehex#1{%
  \directlua0{%
    oberdiek.pdftexcmds.escapehex("\luaescapestring{#1}", "byte")%
  }%
}%
%    \end{macrocode}
%    \end{macro}
%    \begin{macro}{\pdf@escapehexnative}
%    \begin{macrocode}
\long\def\pdf@escapehexnative#1{%
  \directlua0{%
    oberdiek.pdftexcmds.escapehex("\luaescapestring{#1}")%
  }%
}%
%    \end{macrocode}
%    \end{macro}
%    \begin{macro}{\pdf@unescapehex}
%    \begin{macrocode}
\def\pdf@unescapehex#1{%
  \the\expandafter\pdftexcmds@toks
  \directlua0{%
    oberdiek.pdftexcmds.toks="pdftexcmds@toks"%
    oberdiek.pdftexcmds.unescapehex("\luaescapestring{#1}", "byte")%
  }%
}%
%    \end{macrocode}
%    \end{macro}
%    \begin{macro}{\pdf@unescapehexnative}
%    \begin{macrocode}
\def\pdf@unescapehexnative#1{%
  \the\expandafter\pdftexcmds@toks
  \directlua0{%
    oberdiek.pdftexcmds.toks="pdftexcmds@toks"%
    oberdiek.pdftexcmds.unescapehex("\luaescapestring{#1}")%
  }%
}%
%    \end{macrocode}
%    \end{macro}
%    \begin{macro}{\pdf@escapestring}
%    \begin{macrocode}
\long\def\pdf@escapestring#1{%
  \directlua0{%
    oberdiek.pdftexcmds.escapestring("\luaescapestring{#1}", "byte")%
  }%
}
%    \end{macrocode}
%    \end{macro}
%    \begin{macro}{\pdf@escapename}
%    \begin{macrocode}
\long\def\pdf@escapename#1{%
  \directlua0{%
    oberdiek.pdftexcmds.escapename("\luaescapestring{#1}", "byte")%
  }%
}
%    \end{macrocode}
%    \end{macro}
%    \begin{macro}{\pdf@escapenamenative}
%    \begin{macrocode}
\long\def\pdf@escapenamenative#1{%
  \directlua0{%
    oberdiek.pdftexcmds.escapename("\luaescapestring{#1}")%
  }%
}
%    \end{macrocode}
%    \end{macro}
%    \begin{macro}{\pdf@filesize}
%    \begin{macrocode}
\def\pdf@filesize#1{%
  \directlua0{%
    oberdiek.pdftexcmds.filesize("\luaescapestring{#1}")%
  }%
}
%    \end{macrocode}
%    \end{macro}
%    \begin{macro}{\pdf@filemoddate}
%    \begin{macrocode}
\def\pdf@filemoddate#1{%
  \directlua0{%
    oberdiek.pdftexcmds.filemoddate("\luaescapestring{#1}")%
  }%
}
%    \end{macrocode}
%    \end{macro}
%    \begin{macro}{\pdf@filedump}
%    \begin{macrocode}
\def\pdf@filedump#1#2#3{%
  \directlua0{%
    oberdiek.pdftexcmds.filedump("\luaescapestring{\number#1}",%
        "\luaescapestring{\number#2}",%
        "\luaescapestring{#3}")%
  }%
}%
%    \end{macrocode}
%    \end{macro}
%    \begin{macro}{\pdf@mdfivesum}
%    \begin{macrocode}
\long\def\pdf@mdfivesum#1{%
  \directlua0{%
    oberdiek.pdftexcmds.mdfivesum("\luaescapestring{#1}", "byte")%
  }%
}%
%    \end{macrocode}
%    \end{macro}
%    \begin{macro}{\pdf@mdfivesumnative}
%    \begin{macrocode}
\long\def\pdf@mdfivesumnative#1{%
  \directlua0{%
    oberdiek.pdftexcmds.mdfivesum("\luaescapestring{#1}")%
  }%
}%
%    \end{macrocode}
%    \end{macro}
%    \begin{macro}{\pdf@filemdfivesum}
%    \begin{macrocode}
\def\pdf@filemdfivesum#1{%
  \directlua0{%
    oberdiek.pdftexcmds.filemdfivesum("\luaescapestring{#1}")%
  }%
}%
%    \end{macrocode}
%    \end{macro}
%
%    \begin{macro}{\pdf@shellescape}
%    \begin{macrocode}
\def\pdf@shellescape{%
  \directlua0{%
    oberdiek.pdftexcmds.shellescape()%
  }%
}
%    \end{macrocode}
%    \end{macro}
%
%    \begin{macro}{\pdf@system}
%    \begin{macrocode}
\def\pdf@system#1{%
  \directlua0{%
    oberdiek.pdftexcmds.system("\luaescapestring{#1}")%
  }%
}
%    \end{macrocode}
%    \end{macro}
%
%    \begin{macro}{\pdf@lastsystemstatus}
%    \begin{macrocode}
\def\pdf@lastsystemstatus{%
  \directlua0{%
    oberdiek.pdftexcmds.lastsystemstatus()%
  }%
}
%    \end{macrocode}
%    \end{macro}
%    \begin{macro}{\pdf@lastsystemexit}
%    \begin{macrocode}
\def\pdf@lastsystemexit{%
  \directlua0{%
    oberdiek.pdftexcmds.lastsystemexit()%
  }%
}
%    \end{macrocode}
%    \end{macro}
%
%    \begin{macro}{\pdf@pipe}
%    Check availability of |io.popen| first.
%    \begin{macrocode}
\ifnum0%
    \directlua0{%
      if io.popen then %
        tex.write("1")%
      end%
    }%
    =1 %
  \def\pdf@pipe#1{%
    \the\expandafter\pdftexcmds@toks
    \directlua0{%
      oberdiek.pdftexcmds.toks="pdftexcmds@toks"%
      oberdiek.pdftexcmds.pipe("\luaescapestring{#1}")%
    }%
  }%
\fi
%    \end{macrocode}
%    \end{macro}
%
%    \begin{macrocode}
\pdftexcmds@AtEnd
%</package>
%    \end{macrocode}
%
% \subsection{Lua module}
%
%    \begin{macrocode}
%<*lua>
%    \end{macrocode}
%
%    \begin{macrocode}
module("oberdiek.pdftexcmds", package.seeall)
local systemexitstatus
function strcmp(A, B)
  if A == B then
    tex.write("0")
  elseif A < B then
    tex.write("-1")
  else
    tex.write("1")
  end
end
local function utf8_to_byte(str)
  local i = 0
  local n = string.len(str)
  local t = {}
  while i < n do
    i = i + 1
    local a = string.byte(str, i)
    if a < 128 then
      table.insert(t, string.char(a))
    else
      if a >= 192 and i < n then
        i = i + 1
        local b = string.byte(str, i)
        if b < 128 or b >= 192 then
          i = i - 1
        elseif a == 194 then
          table.insert(t, string.char(b))
        elseif a == 195 then
          table.insert(t, string.char(b + 64))
        end
      end
    end
  end
  return table.concat(t)
end
function escapehex(str, mode)
  if mode == "byte" then
    str = utf8_to_byte(str)
  end
  tex.write((string.gsub(str, ".",
    function (ch)
      return string.format("%02X", string.byte(ch))
    end
  )))
end
%    \end{macrocode}
%    See procedure |unescapehex| in file \xfile{utils.c} of \pdfTeX.
%    Caution: |tex.write| ignores leading spaces.
%    \begin{macrocode}
function unescapehex(str, mode)
  local a = 0
  local first = true
  local result = {}
  for i = 1, string.len(str), 1 do
    local ch = string.byte(str, i)
    if ch >= 48 and ch <= 57 then
      ch = ch - 48
    elseif ch >= 65 and ch <= 70 then
      ch = ch - 55
    elseif ch >= 97 and ch <= 102 then
      ch = ch - 87
    else
      ch = nil
    end
    if ch then
      if first then
        a = ch * 16
        first = false
      else
        table.insert(result, a + ch)
        first = true
      end
    end
  end
  if not first then
    table.insert(result, a)
  end
  if mode == "byte" then
    local utf8 = {}
    for i, a in ipairs(result) do
      if a < 128 then
        table.insert(utf8, a)
      else
        if a < 192 then
          table.insert(utf8, 194)
          a = a - 128
        else
          table.insert(utf8, 195)
          a = a - 192
        end
        table.insert(utf8, a + 128)
      end
    end
    result = utf8
  end
  tex.settoks(toks, string.char(unpack(result)))
end
%    \end{macrocode}
%    See procedure |escapestring| in file \xfile{utils.c} of \pdfTeX.
%    \begin{macrocode}
function escapestring(str, mode)
  if mode == "byte" then
    str = utf8_to_byte(str)
  end
  tex.write((string.gsub(str, ".",
    function (ch)
      local b = string.byte(ch)
      if b < 33 or b > 126 then
        return string.format("\\%.3o", b)
      end
      if b == 40 or b == 41 or b == 92 then
        return "\\" .. ch
      end
%    \end{macrocode}
%    Lua 5.1 returns the match in case of return value |nil|.
%    \begin{macrocode}
      return nil
    end
  )))
end
%    \end{macrocode}
%    See procedure |escapename| in file \xfile{utils.c} of \pdfTeX.
%    \begin{macrocode}
function escapename(str, mode)
  if mode == "byte" then
    str = utf8_to_byte(str)
  end
  tex.write((string.gsub(str, ".",
    function (ch)
      local b = string.byte(ch)
      if b == 0 then
%    \end{macrocode}
%    In Lua 5.0 |nil| could be used for the empty string,
%    But |nil| returns the match in Lua 5.1, thus we use
%    the empty string explicitly.
%    \begin{macrocode}
        return ""
      end
      if b <= 32 or b >= 127
          or b == 35 or b == 37 or b == 40 or b == 41
          or b == 47 or b == 60 or b == 62 or b == 91
          or b == 93 or b == 123 or b == 125 then
        return string.format("#%.2X", b)
      else
%    \end{macrocode}
%    Lua 5.1 returns the match in case of return value |nil|.
%    \begin{macrocode}
        return nil
      end
    end
  )))
end
function filesize(filename)
  local foundfile = kpse.find_file(filename, "tex", true)
  if foundfile then
    local size = lfs.attributes(foundfile, "size")
    if size then
      tex.write(size)
    end
  end
end
%    \end{macrocode}
%    See procedure |makepdftime| in file \xfile{utils.c} of \pdfTeX.
%    \begin{macrocode}
function filemoddate(filename)
  local foundfile = kpse.find_file(filename, "tex", true)
  if foundfile then
    local date = lfs.attributes(foundfile, "modification")
    if date then
      local d = os.date("*t", date)
      if d.sec >= 60 then
        d.sec = 59
      end
      local u = os.date("!*t", date)
      local off = 60 * (d.hour - u.hour) + d.min - u.min
      if d.year ~= u.year then
        if d.year > u.year then
          off = off + 1440
        else
          off = off - 1440
        end
      elseif d.yday ~= u.yday then
        if d.yday > u.yday then
          off = off + 1440
        else
          off = off - 1440
        end
      end
      local timezone
      if off == 0 then
        timezone = "Z"
      else
        local hours = math.floor(off / 60)
        local mins = math.abs(off - hours * 60)
        timezone = string.format("%+03d'%02d'", hours, mins)
      end
      tex.write(string.format("D:%04d%02d%02d%02d%02d%02d%s",
          d.year, d.month, d.day, d.hour, d.min, d.sec, timezone))
    end
  end
end
function filedump(offset, length, filename)
  length = tonumber(length)
  if length and length > 0 then
    local foundfile = kpse.find_file(filename, "tex", true)
    if foundfile then
      offset = tonumber(offset)
      if not offset then
        offset = 0
      end
      local filehandle = io.open(foundfile, "r")
      if filehandle then
        if offset > 0 then
          filehandle:seek("set", offset)
        end
        local dump = filehandle:read(length)
        escapehex(dump)
      end
    end
  end
end
function mdfivesum(str, mode)
  if mode == "byte" then
    str = utf8_to_byte(str)
  end
  escapehex(md5.sum(str))
end
function filemdfivesum(filename)
  local foundfile = kpse.find_file(filename, "tex", true)
  if foundfile then
    local filehandle = io.open(foundfile, "r")
    if filehandle then
      local contents = filehandle:read("*a")
      escapehex(md5.sum(contents))
    end
  end
end
function shellescape()
  if os.execute then
    tex.write("1")
  else
    tex.write("0")
  end
end
function system(cmdline)
  systemexitstatus = nil
  texio.write_nl("log", "system(" .. cmdline .. ") ")
  if os.execute then
    texio.write("log", "executed.")
    systemexitstatus = os.execute(cmdline)
  else
    texio.write("log", "disabled.")
  end
end
function lastsystemstatus()
  local result = tonumber(systemexitstatus)
  if result then
    local x = math.floor(result / 256)
    tex.write(result - 256 * math.floor(result / 256))
  end
end
function lastsystemexit()
  local result = tonumber(systemexitstatus)
  if result then
    tex.write(math.floor(result / 256))
  end
end
function pipe(cmdline)
  local result
  systemexitstatus = nil
  texio.write_nl("log", "pipe(" .. cmdline ..") ")
  if io.popen then
    texio.write("log", "executed.")
    local handle = io.popen(cmdline, "r")
    if handle then
      result = handle:read("*a")
      handle:close()
    end
  else
    texio.write("log", "disabled.")
  end
  if result then
    tex.settoks(toks, result)
  else
    tex.settoks(toks, "")
  end
end
%    \end{macrocode}
%    \begin{macrocode}
%</lua>
%    \end{macrocode}
%
% \section{Test}
%
% \subsection{Catcode checks for loading}
%
%    \begin{macrocode}
%<*test1>
%    \end{macrocode}
%    \begin{macrocode}
\catcode`\{=1 %
\catcode`\}=2 %
\catcode`\#=6 %
\catcode`\@=11 %
\expandafter\ifx\csname count@\endcsname\relax
  \countdef\count@=255 %
\fi
\expandafter\ifx\csname @gobble\endcsname\relax
  \long\def\@gobble#1{}%
\fi
\expandafter\ifx\csname @firstofone\endcsname\relax
  \long\def\@firstofone#1{#1}%
\fi
\expandafter\ifx\csname loop\endcsname\relax
  \expandafter\@firstofone
\else
  \expandafter\@gobble
\fi
{%
  \def\loop#1\repeat{%
    \def\body{#1}%
    \iterate
  }%
  \def\iterate{%
    \body
      \let\next\iterate
    \else
      \let\next\relax
    \fi
    \next
  }%
  \let\repeat=\fi
}%
\def\RestoreCatcodes{}
\count@=0 %
\loop
  \edef\RestoreCatcodes{%
    \RestoreCatcodes
    \catcode\the\count@=\the\catcode\count@\relax
  }%
\ifnum\count@<255 %
  \advance\count@ 1 %
\repeat

\def\RangeCatcodeInvalid#1#2{%
  \count@=#1\relax
  \loop
    \catcode\count@=15 %
  \ifnum\count@<#2\relax
    \advance\count@ 1 %
  \repeat
}
\expandafter\ifx\csname LoadCommand\endcsname\relax
  \def\LoadCommand{\input pdftexcmds.sty\relax}%
\fi
\def\Test{%
  \RangeCatcodeInvalid{0}{47}%
  \RangeCatcodeInvalid{58}{64}%
  \RangeCatcodeInvalid{91}{96}%
  \RangeCatcodeInvalid{123}{255}%
  \catcode`\@=12 %
  \catcode`\\=0 %
  \catcode`\{=1 %
  \catcode`\}=2 %
  \catcode`\#=6 %
  \catcode`\[=12 %
  \catcode`\]=12 %
  \catcode`\%=14 %
  \catcode`\ =10 %
  \catcode13=5 %
  \LoadCommand
  \RestoreCatcodes
}
\Test
\csname @@end\endcsname
\end
%    \end{macrocode}
%    \begin{macrocode}
%</test1>
%    \end{macrocode}
%
% \section{Installation}
%
% \subsection{Download}
%
% \paragraph{Package.} This package is available on
% CTAN\footnote{\url{ftp://ftp.ctan.org/tex-archive/}}:
% \begin{description}
% \item[\CTAN{macros/latex/contrib/oberdiek/pdftexcmds.dtx}] The source file.
% \item[\CTAN{macros/latex/contrib/oberdiek/pdftexcmds.pdf}] Documentation.
% \end{description}
%
%
% \paragraph{Bundle.} All the packages of the bundle `oberdiek'
% are also available in a TDS compliant ZIP archive. There
% the packages are already unpacked and the documentation files
% are generated. The files and directories obey the TDS standard.
% \begin{description}
% \item[\CTAN{macros/latex/contrib/oberdiek/oberdiek-tds.zip}]
% \end{description}
% \emph{TDS} refers to the standard ``A Directory Structure
% for \TeX\ Files'' (\CTAN{tds/tds.pdf}). Directories
% with \xfile{texmf} in their name are usually organized this way.
%
% \subsection{Bundle installation}
%
% \paragraph{Unpacking.} Unpack the \xfile{oberdiek-tds.zip} in the
% TDS tree (also known as \xfile{texmf} tree) of your choice.
% Example (linux):
% \begin{quote}
%   |unzip oberdiek-tds.zip -d ~/texmf|
% \end{quote}
%
% \paragraph{Script installation.}
% Check the directory \xfile{TDS:scripts/oberdiek/} for
% scripts that need further installation steps.
% Package \xpackage{attachfile2} comes with the Perl script
% \xfile{pdfatfi.pl} that should be installed in such a way
% that it can be called as \texttt{pdfatfi}.
% Example (linux):
% \begin{quote}
%   |chmod +x scripts/oberdiek/pdfatfi.pl|\\
%   |cp scripts/oberdiek/pdfatfi.pl /usr/local/bin/|
% \end{quote}
%
% \subsection{Package installation}
%
% \paragraph{Unpacking.} The \xfile{.dtx} file is a self-extracting
% \docstrip\ archive. The files are extracted by running the
% \xfile{.dtx} through \plainTeX:
% \begin{quote}
%   \verb|tex pdftexcmds.dtx|
% \end{quote}
%
% \paragraph{TDS.} Now the different files must be moved into
% the different directories in your installation TDS tree
% (also known as \xfile{texmf} tree):
% \begin{quote}
% \def\t{^^A
% \begin{tabular}{@{}>{\ttfamily}l@{ $\rightarrow$ }>{\ttfamily}l@{}}
%   pdftexcmds.sty & tex/generic/oberdiek/pdftexcmds.sty\\
%   oberdiek.pdftexcmds.lua & scripts/oberdiek/oberdiek.pdftexcmds.lua\\
%   pdftexcmds.lua & scripts/oberdiek/pdftexcmds.lua\\
%   pdftexcmds.pdf & doc/latex/oberdiek/pdftexcmds.pdf\\
%   pdftexcmds.dtx & source/latex/oberdiek/pdftexcmds.dtx\\
% \end{tabular}^^A
% }^^A
% \sbox0{\t}^^A
% \ifdim\wd0>\linewidth
%   \begingroup
%     \advance\linewidth by\leftmargin
%     \advance\linewidth by\rightmargin
%   \edef\x{\endgroup
%     \def\noexpand\lw{\the\linewidth}^^A
%   }\x
%   \def\lwbox{^^A
%     \leavevmode
%     \hbox to \linewidth{^^A
%       \kern-\leftmargin\relax
%       \hss
%       \usebox0
%       \hss
%       \kern-\rightmargin\relax
%     }^^A
%   }^^A
%   \ifdim\wd0>\lw
%     \sbox0{\small\t}^^A
%     \ifdim\wd0>\linewidth
%       \ifdim\wd0>\lw
%         \sbox0{\footnotesize\t}^^A
%         \ifdim\wd0>\linewidth
%           \ifdim\wd0>\lw
%             \sbox0{\scriptsize\t}^^A
%             \ifdim\wd0>\linewidth
%               \ifdim\wd0>\lw
%                 \sbox0{\tiny\t}^^A
%                 \ifdim\wd0>\linewidth
%                   \lwbox
%                 \else
%                   \usebox0
%                 \fi
%               \else
%                 \lwbox
%               \fi
%             \else
%               \usebox0
%             \fi
%           \else
%             \lwbox
%           \fi
%         \else
%           \usebox0
%         \fi
%       \else
%         \lwbox
%       \fi
%     \else
%       \usebox0
%     \fi
%   \else
%     \lwbox
%   \fi
% \else
%   \usebox0
% \fi
% \end{quote}
% If you have a \xfile{docstrip.cfg} that configures and enables \docstrip's
% TDS installing feature, then some files can already be in the right
% place, see the documentation of \docstrip.
%
% \subsection{Refresh file name databases}
%
% If your \TeX~distribution
% (\teTeX, \mikTeX, \dots) relies on file name databases, you must refresh
% these. For example, \teTeX\ users run \verb|texhash| or
% \verb|mktexlsr|.
%
% \subsection{Some details for the interested}
%
% \paragraph{Attached source.}
%
% The PDF documentation on CTAN also includes the
% \xfile{.dtx} source file. It can be extracted by
% AcrobatReader 6 or higher. Another option is \textsf{pdftk},
% e.g. unpack the file into the current directory:
% \begin{quote}
%   \verb|pdftk pdftexcmds.pdf unpack_files output .|
% \end{quote}
%
% \paragraph{Unpacking with \LaTeX.}
% The \xfile{.dtx} chooses its action depending on the format:
% \begin{description}
% \item[\plainTeX:] Run \docstrip\ and extract the files.
% \item[\LaTeX:] Generate the documentation.
% \end{description}
% If you insist on using \LaTeX\ for \docstrip\ (really,
% \docstrip\ does not need \LaTeX), then inform the autodetect routine
% about your intention:
% \begin{quote}
%   \verb|latex \let\install=y\input{pdftexcmds.dtx}|
% \end{quote}
% Do not forget to quote the argument according to the demands
% of your shell.
%
% \paragraph{Generating the documentation.}
% You can use both the \xfile{.dtx} or the \xfile{.drv} to generate
% the documentation. The process can be configured by the
% configuration file \xfile{ltxdoc.cfg}. For instance, put this
% line into this file, if you want to have A4 as paper format:
% \begin{quote}
%   \verb|\PassOptionsToClass{a4paper}{article}|
% \end{quote}
% An example follows how to generate the
% documentation with pdf\LaTeX:
% \begin{quote}
%\begin{verbatim}
%pdflatex pdftexcmds.dtx
%makeindex -s gind.ist pdftexcmds.idx
%pdflatex pdftexcmds.dtx
%makeindex -s gind.ist pdftexcmds.idx
%pdflatex pdftexcmds.dtx
%\end{verbatim}
% \end{quote}
%
% \begin{History}
%   \begin{Version}{2007/11/11 v0.1}
%   \item
%     First version.
%   \end{Version}
%   \begin{Version}{2007/11/12 v0.2}
%   \item
%     Short description fixed.
%   \end{Version}
%   \begin{Version}{2007/12/12 v0.3}
%   \item
%     Organization of Lua code as module.
%   \end{Version}
% \end{History}
%
% \PrintIndex
%
% \Finale
\endinput

%        (quote the arguments according to the demands of your shell)
%
% Documentation:
%    (a) If pdftexcmds.drv is present:
%           latex pdftexcmds.drv
%    (b) Without pdftexcmds.drv:
%           latex pdftexcmds.dtx; ...
%    The class ltxdoc loads the configuration file ltxdoc.cfg
%    if available. Here you can specify further options, e.g.
%    use A4 as paper format:
%       \PassOptionsToClass{a4paper}{article}
%
%    Programm calls to get the documentation (example):
%       pdflatex pdftexcmds.dtx
%       makeindex -s gind.ist pdftexcmds.idx
%       pdflatex pdftexcmds.dtx
%       makeindex -s gind.ist pdftexcmds.idx
%       pdflatex pdftexcmds.dtx
%
% Installation:
%    TDS:tex/generic/oberdiek/pdftexcmds.sty
%    TDS:scripts/oberdiek/oberdiek.pdftexcmds.lua
%    TDS:scripts/oberdiek/pdftexcmds.lua
%    TDS:doc/latex/oberdiek/pdftexcmds.pdf
%    TDS:source/latex/oberdiek/pdftexcmds.dtx
%
%<*ignore>
\begingroup
  \def\x{LaTeX2e}%
\expandafter\endgroup
\ifcase 0\ifx\install y1\fi\expandafter
         \ifx\csname processbatchFile\endcsname\relax\else1\fi
         \ifx\fmtname\x\else 1\fi\relax
\else\csname fi\endcsname
%</ignore>
%<*install>
\input docstrip.tex
\Msg{************************************************************************}
\Msg{* Installation}
\Msg{* Package: pdftexcmds 2007/12/12 v0.3 LuaTeX support for pdfTeX utility functions (HO)}
\Msg{************************************************************************}

\keepsilent
\askforoverwritefalse

\let\MetaPrefix\relax
\preamble

This is a generated file.

Copyright (C) 2007 by Heiko Oberdiek <oberdiek@uni-freiburg.de>

This work may be distributed and/or modified under the
conditions of the LaTeX Project Public License, either
version 1.3 of this license or (at your option) any later
version. The latest version of this license is in
   http://www.latex-project.org/lppl.txt
and version 1.3 or later is part of all distributions of
LaTeX version 2005/12/01 or later.

This work has the LPPL maintenance status "maintained".

This Current Maintainer of this work is Heiko Oberdiek.

This work consists of the main source file pdftexcmds.dtx
and the derived files
   pdftexcmds.sty, pdftexcmds.pdf, pdftexcmds.ins, pdftexcmds.drv,
   oberdiek.pdftexcmds.lua, pdftexcmds.lua.

\endpreamble
\let\MetaPrefix\DoubleperCent

\generate{%
  \file{pdftexcmds.ins}{\from{pdftexcmds.dtx}{install}}%
  \file{pdftexcmds.drv}{\from{pdftexcmds.dtx}{driver}}%
  \usedir{tex/generic/oberdiek}%
  \file{pdftexcmds.sty}{\from{pdftexcmds.dtx}{package}}%
}
\def\MetaPrefix{-- }
\def\defaultpostamble{%
  \MetaPrefix^^J%
  \MetaPrefix\space End of File `\outFileName'.%
}
\def\currentpostamble{\defaultpostamble}%
\generate{%
  \usedir{scripts/oberdiek}%
  \file{oberdiek.pdftexcmds.lua}{\from{pdftexcmds.dtx}{lua}}%
  \file{pdftexcmds.lua}{\from{pdftexcmds.dtx}{lua}}%
}

\obeyspaces
\Msg{************************************************************************}
\Msg{*}
\Msg{* To finish the installation you have to move the following}
\Msg{* file into a directory searched by TeX:}
\Msg{*}
\Msg{*     pdftexcmds.sty}
\Msg{*}
\Msg{* And install the following script files:}
\Msg{*}
\Msg{*     oberdiek.pdftexcmds.lua, pdftexcmds.lua}
\Msg{*}
\Msg{* To produce the documentation run the file `pdftexcmds.drv'}
\Msg{* through LaTeX.}
\Msg{*}
\Msg{* Happy TeXing!}
\Msg{*}
\Msg{************************************************************************}

\endbatchfile
%</install>
%<*ignore>
\fi
%</ignore>
%<*driver>
\NeedsTeXFormat{LaTeX2e}
\ProvidesFile{pdftexcmds.drv}%
  [2007/12/12 v0.3 LuaTeX support for pdfTeX utility functions (HO)]%
\documentclass{ltxdoc}
\usepackage{holtxdoc}[2007/11/11]
\usepackage{paralist}
\begin{document}
  \DocInput{pdftexcmds.dtx}%
\end{document}
%</driver>
% \fi
%
% \CheckSum{552}
%
% \CharacterTable
%  {Upper-case    \A\B\C\D\E\F\G\H\I\J\K\L\M\N\O\P\Q\R\S\T\U\V\W\X\Y\Z
%   Lower-case    \a\b\c\d\e\f\g\h\i\j\k\l\m\n\o\p\q\r\s\t\u\v\w\x\y\z
%   Digits        \0\1\2\3\4\5\6\7\8\9
%   Exclamation   \!     Double quote  \"     Hash (number) \#
%   Dollar        \$     Percent       \%     Ampersand     \&
%   Acute accent  \'     Left paren    \(     Right paren   \)
%   Asterisk      \*     Plus          \+     Comma         \,
%   Minus         \-     Point         \.     Solidus       \/
%   Colon         \:     Semicolon     \;     Less than     \<
%   Equals        \=     Greater than  \>     Question mark \?
%   Commercial at \@     Left bracket  \[     Backslash     \\
%   Right bracket \]     Circumflex    \^     Underscore    \_
%   Grave accent  \`     Left brace    \{     Vertical bar  \|
%   Right brace   \}     Tilde         \~}
%
% \GetFileInfo{pdftexcmds.drv}
%
% \title{The \xpackage{pdftexcmds} package}
% \date{2007/12/12 v0.3}
% \author{Heiko Oberdiek\\\xemail{oberdiek@uni-freiburg.de}}
%
% \maketitle
%
% \begin{abstract}
% \LuaTeX\ provides most of the commands of \pdfTeX\ 1.40. However
% a number of utility functions are removed. This package tries to fill
% the gap and implements some of the missing primitive using Lua.
% \end{abstract}
%
% \tableofcontents
%
% \section{Documentation}
%
% Some primitives of \pdfTeX\ are not defined by \LuaTeX.
% This package implements macro based solutions using Lua code
% for the following missing \pdfTeX\ primitives;
% \begin{compactitem}
% \item \cs{pdfstrcmp}
% \item \cs{pdfunescapehex}
% \item \cs{pdfescapehex}
% \item \cs{pdfescapename}
% \item \cs{pdfescapestring}
% \item \cs{pdffilesize}
% \item \cs{pdffilemoddate}
% \item \cs{pdffiledump}
% \item \cs{pdfmdfivesum}
% \item |\immediate\write18|
% \end{compactitem}
% The original names of the primitives cannot be used:
% \begin{itemize}
% \item
% The syntax for their arguments cannot easily
% simulated by macros. The primitives using key words
% such as |file| (\cs{pdfmdfivesum}) or |offset| and |length|
% (\cs{pdffiledump}) and uses \meta{general text} for the other
% arguments. Using token registers assignments, \meta{general text} could
% be catched. However, the simulated primitives are expandable
% and register assignments would destroy this important property.
% (\meta{general text} allows something like |\expandafter\bgroup ...}|.)
% \item
% The original primitives can be expanded using one expansion step.
% The new macros need two expansion steps because of the additional
% macro expansion. Example:
% \begin{quote}
%   |\expandafter\foo\pdffilemoddate{file}|\\
%   vs.
%   |\expandafter\expandafter\expandafter\foo\pdf@filemoddate{file}|.
% \end{quote}
% \end{itemize}
%
% \LuaTeX\ isn't stable yet and thus the status of this package is
% \emph{experimental}. Feedback is welcome.
%
% \subsection{General principles}
%
% \begin{description}
% \item[Naming convention:]
%   Usually this package defines a macro |\pdf@|\meta{cmd} if
%   \pdfTeX\ provides |\pdf|\meta{cmd}.
% \item[Arguments:] The order of arguments in |\pdf@|\meta{cmd}
%   is the same as for the corresponding primitive of \pdfTeX.
%   The arguments are ordinary undelimited \TeX\ arguments,
%   no \meta{general text} and without additional keywords.
% \item[Expandibility:]
%   The macro |\pdf@|\meta{cmd} is expandable if the
%   corresponding \pdfTeX\ primitive has this property.
%   Exact two expansion steps are necessary (first is the macro
%   expansion).
% \item[Without \LuaTeX:]
%   The macros |\pdf@|\meta{cmd} are mapped to the commands
%   of \pdfTeX\ if they are available. Otherwise they are undefined.
% \end{description}
%
% \subsection{Macros}
%
% \begin{declcs}{pdf@strcmp} \M{stringA} \M{stringB}
% \end{declcs}
% Same as |\pdfstrcmp{|\meta{stringA}|}{|\meta{stringB}|}|.
%
% \begin{declcs}{pdf@unescapehex} \M{string}
% \end{declcs}
% Same as |\pdfunescapehex{|\meta{string}|}|.
% The argument is a byte string given in hexadecimal notation.
% The result are character tokens from 0 until 255 with
% catcode 12 and the space with catcode 10.
%
% \begin{declcs}{pdf@escapehex} \M{string}\\
%   \cs{pdf@escapestring} \M{string}\\
%   \cs{pdf@escapename} \M{string}
% \end{declcs}
% Same as the primitives of \pdfTeX. However \pdfTeX\ does not
% know about characters with codes 256 and larger. Thus the
% string is treated as byte string, characters with more than
% eight bits are ignored.
%
% \begin{declcs}{pdf@filesize} \M{filename}
% \end{declcs}
% Same as |\pdffilesize{|\meta{filename}|}|.
%
% \begin{declcs}{pdf@filemoddate} \M{filename}
% \end{declcs}
% Same as |\pdffilemoddate{|\meta{filename}|}|.
%
% \begin{declcs}{pdf@filedump} \M{offset} \M{length} \M{filename}
% \end{declcs}
% Same as |\pdffiledump offset| \meta{offset} |length| \meta{length}
% |{|\meta{filename}|}|. Both \meta{offset} and \meta{length} must
% not be empty, but must be a valid \TeX\ number.
%
% \begin{declcs}{pdf@mdfivesum} \M{string}
% \end{declcs}
% Same as |\pdfmdfivesum{|\meta{string}|}|. Keyword |file| is supported
% by macro \cs{pdf@filemdfivesum}.
%
% \begin{declcs}{pdf@filemdfivesum} \M{filename}
% \end{declcs}
% Same as |\pdfmdfivesum file{|\meta{filename}|}|.
%
% \begin{declcs}{pdf@shellescape}
% \end{declcs}
% Same as |\pdfshellescape|. It expands to |1| if external
% commands can be executed and |0| otherwise. In \pdfTeX\ external
% commands must be enabled first by command line option or
% configuration option. In \LuaTeX\ option |--safer| disables
% the execution of external commands.
%
% \begin{declcs}{pdf@system} \M{cmdline}
% \end{declcs}
% It is a wrapper for |\immediate\write18| in \pdfTeX\ or
% |os.execute| in \LuaTeX.
%
% In theory |os.execute|
% returns a status number. But its meaning is quite
% undefined. Are there some reliable properties?
% Does it make sense to provide an user interface to
% this status exit code?
%
% \subsubsection{Experimental}
%
% \begin{declcs}{pdf@unescapehexnative} \M{string}\\
%   \cs{pdf@escapehexnative} \M{string}\\
%   \cs{pdf@escapenamenative} \M{string}\\
%   \cs{pdf@mdfivesumnative} \M{string}
% \end{declcs}
% The variants without |native| in the macro name are supposed to
% be compatible with \pdfTeX. However characters with more than
% eight bits are not supported and are ignored. If \LuaTeX\ is
% running, then its UTF-8 coded strings are used. Thus the full
% unicode character range is supported. However the result
% differs from \pdfTeX\ for characters with eight or more bits.
%
% \begin{declcs}{pdf@pipe} \M{cmdline}
% \end{declcs}
% It calls \meta{cmdline} and returns the output of the external
% program in the usual manner as byte string (catcode 12, space with
% catcode 10). The Lua documentation says, that the used |io.popen|
% may not be available on all platforms. Then macro \cs{pdf@pipe}
% is undefined.
%
% \StopEventually{
% }
%
% \section{Implementation}
%
%    \begin{macrocode}
%<*package>
%    \end{macrocode}
%
% \subsection{Reload check and package identification}
%    Reload check, especially if the package is not used with \LaTeX.
%    \begin{macrocode}
\begingroup
  \catcode44 12 % ,
  \catcode45 12 % -
  \catcode46 12 % .
  \catcode58 12 % :
  \catcode64 11 % @
  \expandafter\let\expandafter\x\csname ver@pdftexcmds.sty\endcsname
  \ifcase 0%
    \ifx\x\relax % plain
    \else
      \ifx\x\empty % LaTeX
      \else
        1%
      \fi
    \fi
  \else
    \catcode35 6 % #
    \catcode123 1 % {
    \catcode125 2 % }
    \expandafter\ifx\csname PackageInfo\endcsname\relax
      \def\x#1#2{%
        \immediate\write-1{Package #1 Info: #2.}%
      }%
    \else
      \def\x#1#2{\PackageInfo{#1}{#2, stopped}}%
    \fi
    \x{pdftexcmds}{The package is already loaded}%
    \endgroup
    \expandafter\endinput
  \fi
\endgroup
%    \end{macrocode}
%    Package identification:
%    \begin{macrocode}
\begingroup
  \catcode35 6 % #
  \catcode40 12 % (
  \catcode41 12 % )
  \catcode44 12 % ,
  \catcode45 12 % -
  \catcode46 12 % .
  \catcode47 12 % /
  \catcode58 12 % :
  \catcode64 11 % @
  \catcode123 1 % {
  \catcode125 2 % }
  \expandafter\ifx\csname ProvidesPackage\endcsname\relax
    \def\x#1#2#3[#4]{\endgroup
      \immediate\write-1{Package: #3 #4}%
      \xdef#1{#4}%
    }%
  \else
    \def\x#1#2[#3]{\endgroup
      #2[{#3}]%
      \ifx#1\relax
        \xdef#1{#3}%
      \fi
    }%
  \fi
\expandafter\x\csname ver@pdftexcmds.sty\endcsname
\ProvidesPackage{pdftexcmds}%
  [2007/12/12 v0.3 LuaTeX support for pdfTeX utility functions (HO)]
%    \end{macrocode}
%
% \subsection{Catcodes}
%
%    \begin{macrocode}
\begingroup
  \catcode123 1 % {
  \catcode125 2 % }
  \def\x{\endgroup
    \expandafter\edef\csname pdftexcmds@AtEnd\endcsname{%
      \catcode35 \the\catcode35\relax
      \catcode64 \the\catcode64\relax
      \catcode123 \the\catcode123\relax
      \catcode125 \the\catcode125\relax
    }%
  }%
\x
\catcode35 6 % #
\catcode64 11 % @
\catcode123 1 % {
\catcode125 2 % }
\def\TMP@EnsureCode#1#2{%
  \edef\pdftexcmds@AtEnd{%
    \pdftexcmds@AtEnd
    \catcode#1 \the\catcode#1\relax
  }%
  \catcode#1 #2\relax
}
\TMP@EnsureCode{10}{12}% ^^J
\TMP@EnsureCode{33}{12}% !
\TMP@EnsureCode{34}{12}% "
\TMP@EnsureCode{39}{12}% '
\TMP@EnsureCode{40}{12}% (
\TMP@EnsureCode{41}{12}% )
\TMP@EnsureCode{42}{12}% *
\TMP@EnsureCode{43}{12}% +
\TMP@EnsureCode{44}{12}% ,
\TMP@EnsureCode{45}{12}% -
\TMP@EnsureCode{46}{12}% .
\TMP@EnsureCode{47}{12}% /
\TMP@EnsureCode{58}{12}% :
\TMP@EnsureCode{60}{12}% <
\TMP@EnsureCode{61}{12}% =
\TMP@EnsureCode{62}{12}% >
\TMP@EnsureCode{94}{7}% ^ (superscript)
\TMP@EnsureCode{95}{12}% _ (other)
\TMP@EnsureCode{126}{12}% ~ (other)
%    \end{macrocode}
%
% \subsection{Load package \xpackage{infwarerr}}
%
%    \begin{macrocode}
\begingroup\expandafter\expandafter\expandafter\endgroup
\expandafter\ifx\csname RequirePackage\endcsname\relax
  \input infwarerr.sty\relax
\else
  \RequirePackage{infwarerr}[2007/09/09]%
\fi
%    \end{macrocode}
%
% \subsection{Without \LuaTeX}
%
%    \begin{macrocode}
\begingroup\expandafter\expandafter\expandafter\endgroup
\expandafter\ifx\csname directlua\endcsname\relax
  \@PackageInfo{pdftexcmds}{LuaTeX not detected}%
  \def\pdftexcmds@nopdftex{%
    \@PackageInfoNoLine{pdftexcmds}{pdfTeX >= 1.30 not detected}%
    \let\pdftexcmds@nopdftex\relax
  }%
  \def\pdftexcmds@temp#1{%
    \begingroup\expandafter\expandafter\expandafter\endgroup
    \expandafter\ifx\csname pdf#1\endcsname\relax
      \pdftexcmds@nopdftex
    \else
      \expandafter\def\csname pdf@#1\expandafter\endcsname
      \expandafter##\expandafter{%
        \csname pdf#1\endcsname
      }%
    \fi
  }%
  \pdftexcmds@temp{strcmp}%
  \pdftexcmds@temp{escapehex}%
  \let\pdf@escapehexnative\pdf@escapehex
  \pdftexcmds@temp{unescapehex}%
  \let\pdf@unescapehexnative\pdf@unescapehex
  \pdftexcmds@temp{escapestring}%
  \pdftexcmds@temp{escapename}%
  \pdftexcmds@temp{filesize}%
  \pdftexcmds@temp{filemoddate}%
  \begingroup\expandafter\expandafter\expandafter\endgroup
  \expandafter\ifx\csname pdfshellescape\endcsname\relax
    \pdftexcmds@nopdftex
  \else
    \def\pdf@shellescape{%
      \pdfshellescape
    }%
  \fi
  \begingroup\expandafter\expandafter\expandafter\endgroup
  \expandafter\ifx\csname pdffiledump\endcsname\relax
    \pdftexcmds@nopdftex
  \else
    \def\pdf@filedump#1#2#3{%
      \pdffiledump offset#1 length#2{#3}%
    }%
  \fi
  \begingroup\expandafter\expandafter\expandafter\endgroup
  \expandafter\ifx\csname pdfmdfivesum\endcsname\relax
    \pdftexcmds@nopdftex
  \else
    \def\pdf@mdfivesum#{\pdfmdfivesum}%
    \let\pdf@mdfivesumnative\pdf@mdfivesum
    \def\pdf@filemdfivesum#{\pdfmdfivesum file}%
  \fi
  \def\pdf@system#{%
    \immediate\write18%
  }%
  \pdftexcmds@AtEnd
  \expandafter\endinput
\fi
%    \end{macrocode}
%
% \subsection{Load module}
%
%    \begin{macrocode}
\begingroup\expandafter\expandafter\expandafter\endgroup
\expandafter\ifx\csname RequirePackage\endcsname\relax
  \input luatex-loader.sty\relax
\else
  \RequirePackage{luatex-loader}[2007/12/12]%
\fi
\directlua0{%
  require("oberdiek.pdftexcmds")%
}
%    \end{macrocode}
%
% \subsection{Lua functions}
%
%    \begin{macro}{\pdftexcmds@toks}
%    \begin{macrocode}
\begingroup\expandafter\expandafter\expandafter\endgroup
\expandafter\ifx\csname newtoks\endcsname\relax
  \toksdef\pdftexcmds@toks=0 %
\else
  \csname newtoks\endcsname\pdftexcmds@toks
\fi
%    \end{macrocode}
%    \end{macro}
%
%    \begin{macro}{\pdf@strcmp}
%    \begin{macrocode}
\long\def\pdf@strcmp#1#2{%
  \directlua0{%
    oberdiek.pdftexcmds.strcmp("\luaescapestring{#1}",%
        "\luaescapestring{#2}")%
  }%
}%
%    \end{macrocode}
%    \end{macro}
%    \begin{macro}{\pdf@escapehex}
%    \begin{macrocode}
\long\def\pdf@escapehex#1{%
  \directlua0{%
    oberdiek.pdftexcmds.escapehex("\luaescapestring{#1}", "byte")%
  }%
}%
%    \end{macrocode}
%    \end{macro}
%    \begin{macro}{\pdf@escapehexnative}
%    \begin{macrocode}
\long\def\pdf@escapehexnative#1{%
  \directlua0{%
    oberdiek.pdftexcmds.escapehex("\luaescapestring{#1}")%
  }%
}%
%    \end{macrocode}
%    \end{macro}
%    \begin{macro}{\pdf@unescapehex}
%    \begin{macrocode}
\def\pdf@unescapehex#1{%
  \the\expandafter\pdftexcmds@toks
  \directlua0{%
    oberdiek.pdftexcmds.toks="pdftexcmds@toks"%
    oberdiek.pdftexcmds.unescapehex("\luaescapestring{#1}", "byte")%
  }%
}%
%    \end{macrocode}
%    \end{macro}
%    \begin{macro}{\pdf@unescapehexnative}
%    \begin{macrocode}
\def\pdf@unescapehexnative#1{%
  \the\expandafter\pdftexcmds@toks
  \directlua0{%
    oberdiek.pdftexcmds.toks="pdftexcmds@toks"%
    oberdiek.pdftexcmds.unescapehex("\luaescapestring{#1}")%
  }%
}%
%    \end{macrocode}
%    \end{macro}
%    \begin{macro}{\pdf@escapestring}
%    \begin{macrocode}
\long\def\pdf@escapestring#1{%
  \directlua0{%
    oberdiek.pdftexcmds.escapestring("\luaescapestring{#1}", "byte")%
  }%
}
%    \end{macrocode}
%    \end{macro}
%    \begin{macro}{\pdf@escapename}
%    \begin{macrocode}
\long\def\pdf@escapename#1{%
  \directlua0{%
    oberdiek.pdftexcmds.escapename("\luaescapestring{#1}", "byte")%
  }%
}
%    \end{macrocode}
%    \end{macro}
%    \begin{macro}{\pdf@escapenamenative}
%    \begin{macrocode}
\long\def\pdf@escapenamenative#1{%
  \directlua0{%
    oberdiek.pdftexcmds.escapename("\luaescapestring{#1}")%
  }%
}
%    \end{macrocode}
%    \end{macro}
%    \begin{macro}{\pdf@filesize}
%    \begin{macrocode}
\def\pdf@filesize#1{%
  \directlua0{%
    oberdiek.pdftexcmds.filesize("\luaescapestring{#1}")%
  }%
}
%    \end{macrocode}
%    \end{macro}
%    \begin{macro}{\pdf@filemoddate}
%    \begin{macrocode}
\def\pdf@filemoddate#1{%
  \directlua0{%
    oberdiek.pdftexcmds.filemoddate("\luaescapestring{#1}")%
  }%
}
%    \end{macrocode}
%    \end{macro}
%    \begin{macro}{\pdf@filedump}
%    \begin{macrocode}
\def\pdf@filedump#1#2#3{%
  \directlua0{%
    oberdiek.pdftexcmds.filedump("\luaescapestring{\number#1}",%
        "\luaescapestring{\number#2}",%
        "\luaescapestring{#3}")%
  }%
}%
%    \end{macrocode}
%    \end{macro}
%    \begin{macro}{\pdf@mdfivesum}
%    \begin{macrocode}
\long\def\pdf@mdfivesum#1{%
  \directlua0{%
    oberdiek.pdftexcmds.mdfivesum("\luaescapestring{#1}", "byte")%
  }%
}%
%    \end{macrocode}
%    \end{macro}
%    \begin{macro}{\pdf@mdfivesumnative}
%    \begin{macrocode}
\long\def\pdf@mdfivesumnative#1{%
  \directlua0{%
    oberdiek.pdftexcmds.mdfivesum("\luaescapestring{#1}")%
  }%
}%
%    \end{macrocode}
%    \end{macro}
%    \begin{macro}{\pdf@filemdfivesum}
%    \begin{macrocode}
\def\pdf@filemdfivesum#1{%
  \directlua0{%
    oberdiek.pdftexcmds.filemdfivesum("\luaescapestring{#1}")%
  }%
}%
%    \end{macrocode}
%    \end{macro}
%
%    \begin{macro}{\pdf@shellescape}
%    \begin{macrocode}
\def\pdf@shellescape{%
  \directlua0{%
    oberdiek.pdftexcmds.shellescape()%
  }%
}
%    \end{macrocode}
%    \end{macro}
%
%    \begin{macro}{\pdf@system}
%    \begin{macrocode}
\def\pdf@system#1{%
  \directlua0{%
    oberdiek.pdftexcmds.system("\luaescapestring{#1}")%
  }%
}
%    \end{macrocode}
%    \end{macro}
%
%    \begin{macro}{\pdf@lastsystemstatus}
%    \begin{macrocode}
\def\pdf@lastsystemstatus{%
  \directlua0{%
    oberdiek.pdftexcmds.lastsystemstatus()%
  }%
}
%    \end{macrocode}
%    \end{macro}
%    \begin{macro}{\pdf@lastsystemexit}
%    \begin{macrocode}
\def\pdf@lastsystemexit{%
  \directlua0{%
    oberdiek.pdftexcmds.lastsystemexit()%
  }%
}
%    \end{macrocode}
%    \end{macro}
%
%    \begin{macro}{\pdf@pipe}
%    Check availability of |io.popen| first.
%    \begin{macrocode}
\ifnum0%
    \directlua0{%
      if io.popen then %
        tex.write("1")%
      end%
    }%
    =1 %
  \def\pdf@pipe#1{%
    \the\expandafter\pdftexcmds@toks
    \directlua0{%
      oberdiek.pdftexcmds.toks="pdftexcmds@toks"%
      oberdiek.pdftexcmds.pipe("\luaescapestring{#1}")%
    }%
  }%
\fi
%    \end{macrocode}
%    \end{macro}
%
%    \begin{macrocode}
\pdftexcmds@AtEnd
%</package>
%    \end{macrocode}
%
% \subsection{Lua module}
%
%    \begin{macrocode}
%<*lua>
%    \end{macrocode}
%
%    \begin{macrocode}
module("oberdiek.pdftexcmds", package.seeall)
local systemexitstatus
function strcmp(A, B)
  if A == B then
    tex.write("0")
  elseif A < B then
    tex.write("-1")
  else
    tex.write("1")
  end
end
local function utf8_to_byte(str)
  local i = 0
  local n = string.len(str)
  local t = {}
  while i < n do
    i = i + 1
    local a = string.byte(str, i)
    if a < 128 then
      table.insert(t, string.char(a))
    else
      if a >= 192 and i < n then
        i = i + 1
        local b = string.byte(str, i)
        if b < 128 or b >= 192 then
          i = i - 1
        elseif a == 194 then
          table.insert(t, string.char(b))
        elseif a == 195 then
          table.insert(t, string.char(b + 64))
        end
      end
    end
  end
  return table.concat(t)
end
function escapehex(str, mode)
  if mode == "byte" then
    str = utf8_to_byte(str)
  end
  tex.write((string.gsub(str, ".",
    function (ch)
      return string.format("%02X", string.byte(ch))
    end
  )))
end
%    \end{macrocode}
%    See procedure |unescapehex| in file \xfile{utils.c} of \pdfTeX.
%    Caution: |tex.write| ignores leading spaces.
%    \begin{macrocode}
function unescapehex(str, mode)
  local a = 0
  local first = true
  local result = {}
  for i = 1, string.len(str), 1 do
    local ch = string.byte(str, i)
    if ch >= 48 and ch <= 57 then
      ch = ch - 48
    elseif ch >= 65 and ch <= 70 then
      ch = ch - 55
    elseif ch >= 97 and ch <= 102 then
      ch = ch - 87
    else
      ch = nil
    end
    if ch then
      if first then
        a = ch * 16
        first = false
      else
        table.insert(result, a + ch)
        first = true
      end
    end
  end
  if not first then
    table.insert(result, a)
  end
  if mode == "byte" then
    local utf8 = {}
    for i, a in ipairs(result) do
      if a < 128 then
        table.insert(utf8, a)
      else
        if a < 192 then
          table.insert(utf8, 194)
          a = a - 128
        else
          table.insert(utf8, 195)
          a = a - 192
        end
        table.insert(utf8, a + 128)
      end
    end
    result = utf8
  end
  tex.settoks(toks, string.char(unpack(result)))
end
%    \end{macrocode}
%    See procedure |escapestring| in file \xfile{utils.c} of \pdfTeX.
%    \begin{macrocode}
function escapestring(str, mode)
  if mode == "byte" then
    str = utf8_to_byte(str)
  end
  tex.write((string.gsub(str, ".",
    function (ch)
      local b = string.byte(ch)
      if b < 33 or b > 126 then
        return string.format("\\%.3o", b)
      end
      if b == 40 or b == 41 or b == 92 then
        return "\\" .. ch
      end
%    \end{macrocode}
%    Lua 5.1 returns the match in case of return value |nil|.
%    \begin{macrocode}
      return nil
    end
  )))
end
%    \end{macrocode}
%    See procedure |escapename| in file \xfile{utils.c} of \pdfTeX.
%    \begin{macrocode}
function escapename(str, mode)
  if mode == "byte" then
    str = utf8_to_byte(str)
  end
  tex.write((string.gsub(str, ".",
    function (ch)
      local b = string.byte(ch)
      if b == 0 then
%    \end{macrocode}
%    In Lua 5.0 |nil| could be used for the empty string,
%    But |nil| returns the match in Lua 5.1, thus we use
%    the empty string explicitly.
%    \begin{macrocode}
        return ""
      end
      if b <= 32 or b >= 127
          or b == 35 or b == 37 or b == 40 or b == 41
          or b == 47 or b == 60 or b == 62 or b == 91
          or b == 93 or b == 123 or b == 125 then
        return string.format("#%.2X", b)
      else
%    \end{macrocode}
%    Lua 5.1 returns the match in case of return value |nil|.
%    \begin{macrocode}
        return nil
      end
    end
  )))
end
function filesize(filename)
  local foundfile = kpse.find_file(filename, "tex", true)
  if foundfile then
    local size = lfs.attributes(foundfile, "size")
    if size then
      tex.write(size)
    end
  end
end
%    \end{macrocode}
%    See procedure |makepdftime| in file \xfile{utils.c} of \pdfTeX.
%    \begin{macrocode}
function filemoddate(filename)
  local foundfile = kpse.find_file(filename, "tex", true)
  if foundfile then
    local date = lfs.attributes(foundfile, "modification")
    if date then
      local d = os.date("*t", date)
      if d.sec >= 60 then
        d.sec = 59
      end
      local u = os.date("!*t", date)
      local off = 60 * (d.hour - u.hour) + d.min - u.min
      if d.year ~= u.year then
        if d.year > u.year then
          off = off + 1440
        else
          off = off - 1440
        end
      elseif d.yday ~= u.yday then
        if d.yday > u.yday then
          off = off + 1440
        else
          off = off - 1440
        end
      end
      local timezone
      if off == 0 then
        timezone = "Z"
      else
        local hours = math.floor(off / 60)
        local mins = math.abs(off - hours * 60)
        timezone = string.format("%+03d'%02d'", hours, mins)
      end
      tex.write(string.format("D:%04d%02d%02d%02d%02d%02d%s",
          d.year, d.month, d.day, d.hour, d.min, d.sec, timezone))
    end
  end
end
function filedump(offset, length, filename)
  length = tonumber(length)
  if length and length > 0 then
    local foundfile = kpse.find_file(filename, "tex", true)
    if foundfile then
      offset = tonumber(offset)
      if not offset then
        offset = 0
      end
      local filehandle = io.open(foundfile, "r")
      if filehandle then
        if offset > 0 then
          filehandle:seek("set", offset)
        end
        local dump = filehandle:read(length)
        escapehex(dump)
      end
    end
  end
end
function mdfivesum(str, mode)
  if mode == "byte" then
    str = utf8_to_byte(str)
  end
  escapehex(md5.sum(str))
end
function filemdfivesum(filename)
  local foundfile = kpse.find_file(filename, "tex", true)
  if foundfile then
    local filehandle = io.open(foundfile, "r")
    if filehandle then
      local contents = filehandle:read("*a")
      escapehex(md5.sum(contents))
    end
  end
end
function shellescape()
  if os.execute then
    tex.write("1")
  else
    tex.write("0")
  end
end
function system(cmdline)
  systemexitstatus = nil
  texio.write_nl("log", "system(" .. cmdline .. ") ")
  if os.execute then
    texio.write("log", "executed.")
    systemexitstatus = os.execute(cmdline)
  else
    texio.write("log", "disabled.")
  end
end
function lastsystemstatus()
  local result = tonumber(systemexitstatus)
  if result then
    local x = math.floor(result / 256)
    tex.write(result - 256 * math.floor(result / 256))
  end
end
function lastsystemexit()
  local result = tonumber(systemexitstatus)
  if result then
    tex.write(math.floor(result / 256))
  end
end
function pipe(cmdline)
  local result
  systemexitstatus = nil
  texio.write_nl("log", "pipe(" .. cmdline ..") ")
  if io.popen then
    texio.write("log", "executed.")
    local handle = io.popen(cmdline, "r")
    if handle then
      result = handle:read("*a")
      handle:close()
    end
  else
    texio.write("log", "disabled.")
  end
  if result then
    tex.settoks(toks, result)
  else
    tex.settoks(toks, "")
  end
end
%    \end{macrocode}
%    \begin{macrocode}
%</lua>
%    \end{macrocode}
%
% \section{Test}
%
% \subsection{Catcode checks for loading}
%
%    \begin{macrocode}
%<*test1>
%    \end{macrocode}
%    \begin{macrocode}
\catcode`\{=1 %
\catcode`\}=2 %
\catcode`\#=6 %
\catcode`\@=11 %
\expandafter\ifx\csname count@\endcsname\relax
  \countdef\count@=255 %
\fi
\expandafter\ifx\csname @gobble\endcsname\relax
  \long\def\@gobble#1{}%
\fi
\expandafter\ifx\csname @firstofone\endcsname\relax
  \long\def\@firstofone#1{#1}%
\fi
\expandafter\ifx\csname loop\endcsname\relax
  \expandafter\@firstofone
\else
  \expandafter\@gobble
\fi
{%
  \def\loop#1\repeat{%
    \def\body{#1}%
    \iterate
  }%
  \def\iterate{%
    \body
      \let\next\iterate
    \else
      \let\next\relax
    \fi
    \next
  }%
  \let\repeat=\fi
}%
\def\RestoreCatcodes{}
\count@=0 %
\loop
  \edef\RestoreCatcodes{%
    \RestoreCatcodes
    \catcode\the\count@=\the\catcode\count@\relax
  }%
\ifnum\count@<255 %
  \advance\count@ 1 %
\repeat

\def\RangeCatcodeInvalid#1#2{%
  \count@=#1\relax
  \loop
    \catcode\count@=15 %
  \ifnum\count@<#2\relax
    \advance\count@ 1 %
  \repeat
}
\expandafter\ifx\csname LoadCommand\endcsname\relax
  \def\LoadCommand{\input pdftexcmds.sty\relax}%
\fi
\def\Test{%
  \RangeCatcodeInvalid{0}{47}%
  \RangeCatcodeInvalid{58}{64}%
  \RangeCatcodeInvalid{91}{96}%
  \RangeCatcodeInvalid{123}{255}%
  \catcode`\@=12 %
  \catcode`\\=0 %
  \catcode`\{=1 %
  \catcode`\}=2 %
  \catcode`\#=6 %
  \catcode`\[=12 %
  \catcode`\]=12 %
  \catcode`\%=14 %
  \catcode`\ =10 %
  \catcode13=5 %
  \LoadCommand
  \RestoreCatcodes
}
\Test
\csname @@end\endcsname
\end
%    \end{macrocode}
%    \begin{macrocode}
%</test1>
%    \end{macrocode}
%
% \section{Installation}
%
% \subsection{Download}
%
% \paragraph{Package.} This package is available on
% CTAN\footnote{\url{ftp://ftp.ctan.org/tex-archive/}}:
% \begin{description}
% \item[\CTAN{macros/latex/contrib/oberdiek/pdftexcmds.dtx}] The source file.
% \item[\CTAN{macros/latex/contrib/oberdiek/pdftexcmds.pdf}] Documentation.
% \end{description}
%
%
% \paragraph{Bundle.} All the packages of the bundle `oberdiek'
% are also available in a TDS compliant ZIP archive. There
% the packages are already unpacked and the documentation files
% are generated. The files and directories obey the TDS standard.
% \begin{description}
% \item[\CTAN{macros/latex/contrib/oberdiek/oberdiek-tds.zip}]
% \end{description}
% \emph{TDS} refers to the standard ``A Directory Structure
% for \TeX\ Files'' (\CTAN{tds/tds.pdf}). Directories
% with \xfile{texmf} in their name are usually organized this way.
%
% \subsection{Bundle installation}
%
% \paragraph{Unpacking.} Unpack the \xfile{oberdiek-tds.zip} in the
% TDS tree (also known as \xfile{texmf} tree) of your choice.
% Example (linux):
% \begin{quote}
%   |unzip oberdiek-tds.zip -d ~/texmf|
% \end{quote}
%
% \paragraph{Script installation.}
% Check the directory \xfile{TDS:scripts/oberdiek/} for
% scripts that need further installation steps.
% Package \xpackage{attachfile2} comes with the Perl script
% \xfile{pdfatfi.pl} that should be installed in such a way
% that it can be called as \texttt{pdfatfi}.
% Example (linux):
% \begin{quote}
%   |chmod +x scripts/oberdiek/pdfatfi.pl|\\
%   |cp scripts/oberdiek/pdfatfi.pl /usr/local/bin/|
% \end{quote}
%
% \subsection{Package installation}
%
% \paragraph{Unpacking.} The \xfile{.dtx} file is a self-extracting
% \docstrip\ archive. The files are extracted by running the
% \xfile{.dtx} through \plainTeX:
% \begin{quote}
%   \verb|tex pdftexcmds.dtx|
% \end{quote}
%
% \paragraph{TDS.} Now the different files must be moved into
% the different directories in your installation TDS tree
% (also known as \xfile{texmf} tree):
% \begin{quote}
% \def\t{^^A
% \begin{tabular}{@{}>{\ttfamily}l@{ $\rightarrow$ }>{\ttfamily}l@{}}
%   pdftexcmds.sty & tex/generic/oberdiek/pdftexcmds.sty\\
%   oberdiek.pdftexcmds.lua & scripts/oberdiek/oberdiek.pdftexcmds.lua\\
%   pdftexcmds.lua & scripts/oberdiek/pdftexcmds.lua\\
%   pdftexcmds.pdf & doc/latex/oberdiek/pdftexcmds.pdf\\
%   pdftexcmds.dtx & source/latex/oberdiek/pdftexcmds.dtx\\
% \end{tabular}^^A
% }^^A
% \sbox0{\t}^^A
% \ifdim\wd0>\linewidth
%   \begingroup
%     \advance\linewidth by\leftmargin
%     \advance\linewidth by\rightmargin
%   \edef\x{\endgroup
%     \def\noexpand\lw{\the\linewidth}^^A
%   }\x
%   \def\lwbox{^^A
%     \leavevmode
%     \hbox to \linewidth{^^A
%       \kern-\leftmargin\relax
%       \hss
%       \usebox0
%       \hss
%       \kern-\rightmargin\relax
%     }^^A
%   }^^A
%   \ifdim\wd0>\lw
%     \sbox0{\small\t}^^A
%     \ifdim\wd0>\linewidth
%       \ifdim\wd0>\lw
%         \sbox0{\footnotesize\t}^^A
%         \ifdim\wd0>\linewidth
%           \ifdim\wd0>\lw
%             \sbox0{\scriptsize\t}^^A
%             \ifdim\wd0>\linewidth
%               \ifdim\wd0>\lw
%                 \sbox0{\tiny\t}^^A
%                 \ifdim\wd0>\linewidth
%                   \lwbox
%                 \else
%                   \usebox0
%                 \fi
%               \else
%                 \lwbox
%               \fi
%             \else
%               \usebox0
%             \fi
%           \else
%             \lwbox
%           \fi
%         \else
%           \usebox0
%         \fi
%       \else
%         \lwbox
%       \fi
%     \else
%       \usebox0
%     \fi
%   \else
%     \lwbox
%   \fi
% \else
%   \usebox0
% \fi
% \end{quote}
% If you have a \xfile{docstrip.cfg} that configures and enables \docstrip's
% TDS installing feature, then some files can already be in the right
% place, see the documentation of \docstrip.
%
% \subsection{Refresh file name databases}
%
% If your \TeX~distribution
% (\teTeX, \mikTeX, \dots) relies on file name databases, you must refresh
% these. For example, \teTeX\ users run \verb|texhash| or
% \verb|mktexlsr|.
%
% \subsection{Some details for the interested}
%
% \paragraph{Attached source.}
%
% The PDF documentation on CTAN also includes the
% \xfile{.dtx} source file. It can be extracted by
% AcrobatReader 6 or higher. Another option is \textsf{pdftk},
% e.g. unpack the file into the current directory:
% \begin{quote}
%   \verb|pdftk pdftexcmds.pdf unpack_files output .|
% \end{quote}
%
% \paragraph{Unpacking with \LaTeX.}
% The \xfile{.dtx} chooses its action depending on the format:
% \begin{description}
% \item[\plainTeX:] Run \docstrip\ and extract the files.
% \item[\LaTeX:] Generate the documentation.
% \end{description}
% If you insist on using \LaTeX\ for \docstrip\ (really,
% \docstrip\ does not need \LaTeX), then inform the autodetect routine
% about your intention:
% \begin{quote}
%   \verb|latex \let\install=y% \iffalse meta-comment
%
% Copyright (C) 2007 by Heiko Oberdiek <oberdiek@uni-freiburg.de>
%
% This work may be distributed and/or modified under the
% conditions of the LaTeX Project Public License, either
% version 1.3 of this license or (at your option) any later
% version. The latest version of this license is in
%    http://www.latex-project.org/lppl.txt
% and version 1.3 or later is part of all distributions of
% LaTeX version 2005/12/01 or later.
%
% This work has the LPPL maintenance status "maintained".
%
% This Current Maintainer of this work is Heiko Oberdiek.
%
% This work consists of the main source file pdftexcmds.dtx
% and the derived files
%    pdftexcmds.sty, pdftexcmds.pdf, pdftexcmds.ins, pdftexcmds.drv,
%    oberdiek.pdftexcmds.lua, pdftexcmds.lua.
%
% Distribution:
%    CTAN:macros/latex/contrib/oberdiek/pdftexcmds.dtx
%    CTAN:macros/latex/contrib/oberdiek/pdftexcmds.pdf
%
% Unpacking:
%    (a) If pdftexcmds.ins is present:
%           tex pdftexcmds.ins
%    (b) Without pdftexcmds.ins:
%           tex pdftexcmds.dtx
%    (c) If you insist on using LaTeX
%           latex \let\install=y\input{pdftexcmds.dtx}
%        (quote the arguments according to the demands of your shell)
%
% Documentation:
%    (a) If pdftexcmds.drv is present:
%           latex pdftexcmds.drv
%    (b) Without pdftexcmds.drv:
%           latex pdftexcmds.dtx; ...
%    The class ltxdoc loads the configuration file ltxdoc.cfg
%    if available. Here you can specify further options, e.g.
%    use A4 as paper format:
%       \PassOptionsToClass{a4paper}{article}
%
%    Programm calls to get the documentation (example):
%       pdflatex pdftexcmds.dtx
%       makeindex -s gind.ist pdftexcmds.idx
%       pdflatex pdftexcmds.dtx
%       makeindex -s gind.ist pdftexcmds.idx
%       pdflatex pdftexcmds.dtx
%
% Installation:
%    TDS:tex/generic/oberdiek/pdftexcmds.sty
%    TDS:scripts/oberdiek/oberdiek.pdftexcmds.lua
%    TDS:scripts/oberdiek/pdftexcmds.lua
%    TDS:doc/latex/oberdiek/pdftexcmds.pdf
%    TDS:source/latex/oberdiek/pdftexcmds.dtx
%
%<*ignore>
\begingroup
  \def\x{LaTeX2e}%
\expandafter\endgroup
\ifcase 0\ifx\install y1\fi\expandafter
         \ifx\csname processbatchFile\endcsname\relax\else1\fi
         \ifx\fmtname\x\else 1\fi\relax
\else\csname fi\endcsname
%</ignore>
%<*install>
\input docstrip.tex
\Msg{************************************************************************}
\Msg{* Installation}
\Msg{* Package: pdftexcmds 2007/12/12 v0.3 LuaTeX support for pdfTeX utility functions (HO)}
\Msg{************************************************************************}

\keepsilent
\askforoverwritefalse

\let\MetaPrefix\relax
\preamble

This is a generated file.

Copyright (C) 2007 by Heiko Oberdiek <oberdiek@uni-freiburg.de>

This work may be distributed and/or modified under the
conditions of the LaTeX Project Public License, either
version 1.3 of this license or (at your option) any later
version. The latest version of this license is in
   http://www.latex-project.org/lppl.txt
and version 1.3 or later is part of all distributions of
LaTeX version 2005/12/01 or later.

This work has the LPPL maintenance status "maintained".

This Current Maintainer of this work is Heiko Oberdiek.

This work consists of the main source file pdftexcmds.dtx
and the derived files
   pdftexcmds.sty, pdftexcmds.pdf, pdftexcmds.ins, pdftexcmds.drv,
   oberdiek.pdftexcmds.lua, pdftexcmds.lua.

\endpreamble
\let\MetaPrefix\DoubleperCent

\generate{%
  \file{pdftexcmds.ins}{\from{pdftexcmds.dtx}{install}}%
  \file{pdftexcmds.drv}{\from{pdftexcmds.dtx}{driver}}%
  \usedir{tex/generic/oberdiek}%
  \file{pdftexcmds.sty}{\from{pdftexcmds.dtx}{package}}%
}
\def\MetaPrefix{-- }
\def\defaultpostamble{%
  \MetaPrefix^^J%
  \MetaPrefix\space End of File `\outFileName'.%
}
\def\currentpostamble{\defaultpostamble}%
\generate{%
  \usedir{scripts/oberdiek}%
  \file{oberdiek.pdftexcmds.lua}{\from{pdftexcmds.dtx}{lua}}%
  \file{pdftexcmds.lua}{\from{pdftexcmds.dtx}{lua}}%
}

\obeyspaces
\Msg{************************************************************************}
\Msg{*}
\Msg{* To finish the installation you have to move the following}
\Msg{* file into a directory searched by TeX:}
\Msg{*}
\Msg{*     pdftexcmds.sty}
\Msg{*}
\Msg{* And install the following script files:}
\Msg{*}
\Msg{*     oberdiek.pdftexcmds.lua, pdftexcmds.lua}
\Msg{*}
\Msg{* To produce the documentation run the file `pdftexcmds.drv'}
\Msg{* through LaTeX.}
\Msg{*}
\Msg{* Happy TeXing!}
\Msg{*}
\Msg{************************************************************************}

\endbatchfile
%</install>
%<*ignore>
\fi
%</ignore>
%<*driver>
\NeedsTeXFormat{LaTeX2e}
\ProvidesFile{pdftexcmds.drv}%
  [2007/12/12 v0.3 LuaTeX support for pdfTeX utility functions (HO)]%
\documentclass{ltxdoc}
\usepackage{holtxdoc}[2007/11/11]
\usepackage{paralist}
\begin{document}
  \DocInput{pdftexcmds.dtx}%
\end{document}
%</driver>
% \fi
%
% \CheckSum{552}
%
% \CharacterTable
%  {Upper-case    \A\B\C\D\E\F\G\H\I\J\K\L\M\N\O\P\Q\R\S\T\U\V\W\X\Y\Z
%   Lower-case    \a\b\c\d\e\f\g\h\i\j\k\l\m\n\o\p\q\r\s\t\u\v\w\x\y\z
%   Digits        \0\1\2\3\4\5\6\7\8\9
%   Exclamation   \!     Double quote  \"     Hash (number) \#
%   Dollar        \$     Percent       \%     Ampersand     \&
%   Acute accent  \'     Left paren    \(     Right paren   \)
%   Asterisk      \*     Plus          \+     Comma         \,
%   Minus         \-     Point         \.     Solidus       \/
%   Colon         \:     Semicolon     \;     Less than     \<
%   Equals        \=     Greater than  \>     Question mark \?
%   Commercial at \@     Left bracket  \[     Backslash     \\
%   Right bracket \]     Circumflex    \^     Underscore    \_
%   Grave accent  \`     Left brace    \{     Vertical bar  \|
%   Right brace   \}     Tilde         \~}
%
% \GetFileInfo{pdftexcmds.drv}
%
% \title{The \xpackage{pdftexcmds} package}
% \date{2007/12/12 v0.3}
% \author{Heiko Oberdiek\\\xemail{oberdiek@uni-freiburg.de}}
%
% \maketitle
%
% \begin{abstract}
% \LuaTeX\ provides most of the commands of \pdfTeX\ 1.40. However
% a number of utility functions are removed. This package tries to fill
% the gap and implements some of the missing primitive using Lua.
% \end{abstract}
%
% \tableofcontents
%
% \section{Documentation}
%
% Some primitives of \pdfTeX\ are not defined by \LuaTeX.
% This package implements macro based solutions using Lua code
% for the following missing \pdfTeX\ primitives;
% \begin{compactitem}
% \item \cs{pdfstrcmp}
% \item \cs{pdfunescapehex}
% \item \cs{pdfescapehex}
% \item \cs{pdfescapename}
% \item \cs{pdfescapestring}
% \item \cs{pdffilesize}
% \item \cs{pdffilemoddate}
% \item \cs{pdffiledump}
% \item \cs{pdfmdfivesum}
% \item |\immediate\write18|
% \end{compactitem}
% The original names of the primitives cannot be used:
% \begin{itemize}
% \item
% The syntax for their arguments cannot easily
% simulated by macros. The primitives using key words
% such as |file| (\cs{pdfmdfivesum}) or |offset| and |length|
% (\cs{pdffiledump}) and uses \meta{general text} for the other
% arguments. Using token registers assignments, \meta{general text} could
% be catched. However, the simulated primitives are expandable
% and register assignments would destroy this important property.
% (\meta{general text} allows something like |\expandafter\bgroup ...}|.)
% \item
% The original primitives can be expanded using one expansion step.
% The new macros need two expansion steps because of the additional
% macro expansion. Example:
% \begin{quote}
%   |\expandafter\foo\pdffilemoddate{file}|\\
%   vs.
%   |\expandafter\expandafter\expandafter\foo\pdf@filemoddate{file}|.
% \end{quote}
% \end{itemize}
%
% \LuaTeX\ isn't stable yet and thus the status of this package is
% \emph{experimental}. Feedback is welcome.
%
% \subsection{General principles}
%
% \begin{description}
% \item[Naming convention:]
%   Usually this package defines a macro |\pdf@|\meta{cmd} if
%   \pdfTeX\ provides |\pdf|\meta{cmd}.
% \item[Arguments:] The order of arguments in |\pdf@|\meta{cmd}
%   is the same as for the corresponding primitive of \pdfTeX.
%   The arguments are ordinary undelimited \TeX\ arguments,
%   no \meta{general text} and without additional keywords.
% \item[Expandibility:]
%   The macro |\pdf@|\meta{cmd} is expandable if the
%   corresponding \pdfTeX\ primitive has this property.
%   Exact two expansion steps are necessary (first is the macro
%   expansion).
% \item[Without \LuaTeX:]
%   The macros |\pdf@|\meta{cmd} are mapped to the commands
%   of \pdfTeX\ if they are available. Otherwise they are undefined.
% \end{description}
%
% \subsection{Macros}
%
% \begin{declcs}{pdf@strcmp} \M{stringA} \M{stringB}
% \end{declcs}
% Same as |\pdfstrcmp{|\meta{stringA}|}{|\meta{stringB}|}|.
%
% \begin{declcs}{pdf@unescapehex} \M{string}
% \end{declcs}
% Same as |\pdfunescapehex{|\meta{string}|}|.
% The argument is a byte string given in hexadecimal notation.
% The result are character tokens from 0 until 255 with
% catcode 12 and the space with catcode 10.
%
% \begin{declcs}{pdf@escapehex} \M{string}\\
%   \cs{pdf@escapestring} \M{string}\\
%   \cs{pdf@escapename} \M{string}
% \end{declcs}
% Same as the primitives of \pdfTeX. However \pdfTeX\ does not
% know about characters with codes 256 and larger. Thus the
% string is treated as byte string, characters with more than
% eight bits are ignored.
%
% \begin{declcs}{pdf@filesize} \M{filename}
% \end{declcs}
% Same as |\pdffilesize{|\meta{filename}|}|.
%
% \begin{declcs}{pdf@filemoddate} \M{filename}
% \end{declcs}
% Same as |\pdffilemoddate{|\meta{filename}|}|.
%
% \begin{declcs}{pdf@filedump} \M{offset} \M{length} \M{filename}
% \end{declcs}
% Same as |\pdffiledump offset| \meta{offset} |length| \meta{length}
% |{|\meta{filename}|}|. Both \meta{offset} and \meta{length} must
% not be empty, but must be a valid \TeX\ number.
%
% \begin{declcs}{pdf@mdfivesum} \M{string}
% \end{declcs}
% Same as |\pdfmdfivesum{|\meta{string}|}|. Keyword |file| is supported
% by macro \cs{pdf@filemdfivesum}.
%
% \begin{declcs}{pdf@filemdfivesum} \M{filename}
% \end{declcs}
% Same as |\pdfmdfivesum file{|\meta{filename}|}|.
%
% \begin{declcs}{pdf@shellescape}
% \end{declcs}
% Same as |\pdfshellescape|. It expands to |1| if external
% commands can be executed and |0| otherwise. In \pdfTeX\ external
% commands must be enabled first by command line option or
% configuration option. In \LuaTeX\ option |--safer| disables
% the execution of external commands.
%
% \begin{declcs}{pdf@system} \M{cmdline}
% \end{declcs}
% It is a wrapper for |\immediate\write18| in \pdfTeX\ or
% |os.execute| in \LuaTeX.
%
% In theory |os.execute|
% returns a status number. But its meaning is quite
% undefined. Are there some reliable properties?
% Does it make sense to provide an user interface to
% this status exit code?
%
% \subsubsection{Experimental}
%
% \begin{declcs}{pdf@unescapehexnative} \M{string}\\
%   \cs{pdf@escapehexnative} \M{string}\\
%   \cs{pdf@escapenamenative} \M{string}\\
%   \cs{pdf@mdfivesumnative} \M{string}
% \end{declcs}
% The variants without |native| in the macro name are supposed to
% be compatible with \pdfTeX. However characters with more than
% eight bits are not supported and are ignored. If \LuaTeX\ is
% running, then its UTF-8 coded strings are used. Thus the full
% unicode character range is supported. However the result
% differs from \pdfTeX\ for characters with eight or more bits.
%
% \begin{declcs}{pdf@pipe} \M{cmdline}
% \end{declcs}
% It calls \meta{cmdline} and returns the output of the external
% program in the usual manner as byte string (catcode 12, space with
% catcode 10). The Lua documentation says, that the used |io.popen|
% may not be available on all platforms. Then macro \cs{pdf@pipe}
% is undefined.
%
% \StopEventually{
% }
%
% \section{Implementation}
%
%    \begin{macrocode}
%<*package>
%    \end{macrocode}
%
% \subsection{Reload check and package identification}
%    Reload check, especially if the package is not used with \LaTeX.
%    \begin{macrocode}
\begingroup
  \catcode44 12 % ,
  \catcode45 12 % -
  \catcode46 12 % .
  \catcode58 12 % :
  \catcode64 11 % @
  \expandafter\let\expandafter\x\csname ver@pdftexcmds.sty\endcsname
  \ifcase 0%
    \ifx\x\relax % plain
    \else
      \ifx\x\empty % LaTeX
      \else
        1%
      \fi
    \fi
  \else
    \catcode35 6 % #
    \catcode123 1 % {
    \catcode125 2 % }
    \expandafter\ifx\csname PackageInfo\endcsname\relax
      \def\x#1#2{%
        \immediate\write-1{Package #1 Info: #2.}%
      }%
    \else
      \def\x#1#2{\PackageInfo{#1}{#2, stopped}}%
    \fi
    \x{pdftexcmds}{The package is already loaded}%
    \endgroup
    \expandafter\endinput
  \fi
\endgroup
%    \end{macrocode}
%    Package identification:
%    \begin{macrocode}
\begingroup
  \catcode35 6 % #
  \catcode40 12 % (
  \catcode41 12 % )
  \catcode44 12 % ,
  \catcode45 12 % -
  \catcode46 12 % .
  \catcode47 12 % /
  \catcode58 12 % :
  \catcode64 11 % @
  \catcode123 1 % {
  \catcode125 2 % }
  \expandafter\ifx\csname ProvidesPackage\endcsname\relax
    \def\x#1#2#3[#4]{\endgroup
      \immediate\write-1{Package: #3 #4}%
      \xdef#1{#4}%
    }%
  \else
    \def\x#1#2[#3]{\endgroup
      #2[{#3}]%
      \ifx#1\relax
        \xdef#1{#3}%
      \fi
    }%
  \fi
\expandafter\x\csname ver@pdftexcmds.sty\endcsname
\ProvidesPackage{pdftexcmds}%
  [2007/12/12 v0.3 LuaTeX support for pdfTeX utility functions (HO)]
%    \end{macrocode}
%
% \subsection{Catcodes}
%
%    \begin{macrocode}
\begingroup
  \catcode123 1 % {
  \catcode125 2 % }
  \def\x{\endgroup
    \expandafter\edef\csname pdftexcmds@AtEnd\endcsname{%
      \catcode35 \the\catcode35\relax
      \catcode64 \the\catcode64\relax
      \catcode123 \the\catcode123\relax
      \catcode125 \the\catcode125\relax
    }%
  }%
\x
\catcode35 6 % #
\catcode64 11 % @
\catcode123 1 % {
\catcode125 2 % }
\def\TMP@EnsureCode#1#2{%
  \edef\pdftexcmds@AtEnd{%
    \pdftexcmds@AtEnd
    \catcode#1 \the\catcode#1\relax
  }%
  \catcode#1 #2\relax
}
\TMP@EnsureCode{10}{12}% ^^J
\TMP@EnsureCode{33}{12}% !
\TMP@EnsureCode{34}{12}% "
\TMP@EnsureCode{39}{12}% '
\TMP@EnsureCode{40}{12}% (
\TMP@EnsureCode{41}{12}% )
\TMP@EnsureCode{42}{12}% *
\TMP@EnsureCode{43}{12}% +
\TMP@EnsureCode{44}{12}% ,
\TMP@EnsureCode{45}{12}% -
\TMP@EnsureCode{46}{12}% .
\TMP@EnsureCode{47}{12}% /
\TMP@EnsureCode{58}{12}% :
\TMP@EnsureCode{60}{12}% <
\TMP@EnsureCode{61}{12}% =
\TMP@EnsureCode{62}{12}% >
\TMP@EnsureCode{94}{7}% ^ (superscript)
\TMP@EnsureCode{95}{12}% _ (other)
\TMP@EnsureCode{126}{12}% ~ (other)
%    \end{macrocode}
%
% \subsection{Load package \xpackage{infwarerr}}
%
%    \begin{macrocode}
\begingroup\expandafter\expandafter\expandafter\endgroup
\expandafter\ifx\csname RequirePackage\endcsname\relax
  \input infwarerr.sty\relax
\else
  \RequirePackage{infwarerr}[2007/09/09]%
\fi
%    \end{macrocode}
%
% \subsection{Without \LuaTeX}
%
%    \begin{macrocode}
\begingroup\expandafter\expandafter\expandafter\endgroup
\expandafter\ifx\csname directlua\endcsname\relax
  \@PackageInfo{pdftexcmds}{LuaTeX not detected}%
  \def\pdftexcmds@nopdftex{%
    \@PackageInfoNoLine{pdftexcmds}{pdfTeX >= 1.30 not detected}%
    \let\pdftexcmds@nopdftex\relax
  }%
  \def\pdftexcmds@temp#1{%
    \begingroup\expandafter\expandafter\expandafter\endgroup
    \expandafter\ifx\csname pdf#1\endcsname\relax
      \pdftexcmds@nopdftex
    \else
      \expandafter\def\csname pdf@#1\expandafter\endcsname
      \expandafter##\expandafter{%
        \csname pdf#1\endcsname
      }%
    \fi
  }%
  \pdftexcmds@temp{strcmp}%
  \pdftexcmds@temp{escapehex}%
  \let\pdf@escapehexnative\pdf@escapehex
  \pdftexcmds@temp{unescapehex}%
  \let\pdf@unescapehexnative\pdf@unescapehex
  \pdftexcmds@temp{escapestring}%
  \pdftexcmds@temp{escapename}%
  \pdftexcmds@temp{filesize}%
  \pdftexcmds@temp{filemoddate}%
  \begingroup\expandafter\expandafter\expandafter\endgroup
  \expandafter\ifx\csname pdfshellescape\endcsname\relax
    \pdftexcmds@nopdftex
  \else
    \def\pdf@shellescape{%
      \pdfshellescape
    }%
  \fi
  \begingroup\expandafter\expandafter\expandafter\endgroup
  \expandafter\ifx\csname pdffiledump\endcsname\relax
    \pdftexcmds@nopdftex
  \else
    \def\pdf@filedump#1#2#3{%
      \pdffiledump offset#1 length#2{#3}%
    }%
  \fi
  \begingroup\expandafter\expandafter\expandafter\endgroup
  \expandafter\ifx\csname pdfmdfivesum\endcsname\relax
    \pdftexcmds@nopdftex
  \else
    \def\pdf@mdfivesum#{\pdfmdfivesum}%
    \let\pdf@mdfivesumnative\pdf@mdfivesum
    \def\pdf@filemdfivesum#{\pdfmdfivesum file}%
  \fi
  \def\pdf@system#{%
    \immediate\write18%
  }%
  \pdftexcmds@AtEnd
  \expandafter\endinput
\fi
%    \end{macrocode}
%
% \subsection{Load module}
%
%    \begin{macrocode}
\begingroup\expandafter\expandafter\expandafter\endgroup
\expandafter\ifx\csname RequirePackage\endcsname\relax
  \input luatex-loader.sty\relax
\else
  \RequirePackage{luatex-loader}[2007/12/12]%
\fi
\directlua0{%
  require("oberdiek.pdftexcmds")%
}
%    \end{macrocode}
%
% \subsection{Lua functions}
%
%    \begin{macro}{\pdftexcmds@toks}
%    \begin{macrocode}
\begingroup\expandafter\expandafter\expandafter\endgroup
\expandafter\ifx\csname newtoks\endcsname\relax
  \toksdef\pdftexcmds@toks=0 %
\else
  \csname newtoks\endcsname\pdftexcmds@toks
\fi
%    \end{macrocode}
%    \end{macro}
%
%    \begin{macro}{\pdf@strcmp}
%    \begin{macrocode}
\long\def\pdf@strcmp#1#2{%
  \directlua0{%
    oberdiek.pdftexcmds.strcmp("\luaescapestring{#1}",%
        "\luaescapestring{#2}")%
  }%
}%
%    \end{macrocode}
%    \end{macro}
%    \begin{macro}{\pdf@escapehex}
%    \begin{macrocode}
\long\def\pdf@escapehex#1{%
  \directlua0{%
    oberdiek.pdftexcmds.escapehex("\luaescapestring{#1}", "byte")%
  }%
}%
%    \end{macrocode}
%    \end{macro}
%    \begin{macro}{\pdf@escapehexnative}
%    \begin{macrocode}
\long\def\pdf@escapehexnative#1{%
  \directlua0{%
    oberdiek.pdftexcmds.escapehex("\luaescapestring{#1}")%
  }%
}%
%    \end{macrocode}
%    \end{macro}
%    \begin{macro}{\pdf@unescapehex}
%    \begin{macrocode}
\def\pdf@unescapehex#1{%
  \the\expandafter\pdftexcmds@toks
  \directlua0{%
    oberdiek.pdftexcmds.toks="pdftexcmds@toks"%
    oberdiek.pdftexcmds.unescapehex("\luaescapestring{#1}", "byte")%
  }%
}%
%    \end{macrocode}
%    \end{macro}
%    \begin{macro}{\pdf@unescapehexnative}
%    \begin{macrocode}
\def\pdf@unescapehexnative#1{%
  \the\expandafter\pdftexcmds@toks
  \directlua0{%
    oberdiek.pdftexcmds.toks="pdftexcmds@toks"%
    oberdiek.pdftexcmds.unescapehex("\luaescapestring{#1}")%
  }%
}%
%    \end{macrocode}
%    \end{macro}
%    \begin{macro}{\pdf@escapestring}
%    \begin{macrocode}
\long\def\pdf@escapestring#1{%
  \directlua0{%
    oberdiek.pdftexcmds.escapestring("\luaescapestring{#1}", "byte")%
  }%
}
%    \end{macrocode}
%    \end{macro}
%    \begin{macro}{\pdf@escapename}
%    \begin{macrocode}
\long\def\pdf@escapename#1{%
  \directlua0{%
    oberdiek.pdftexcmds.escapename("\luaescapestring{#1}", "byte")%
  }%
}
%    \end{macrocode}
%    \end{macro}
%    \begin{macro}{\pdf@escapenamenative}
%    \begin{macrocode}
\long\def\pdf@escapenamenative#1{%
  \directlua0{%
    oberdiek.pdftexcmds.escapename("\luaescapestring{#1}")%
  }%
}
%    \end{macrocode}
%    \end{macro}
%    \begin{macro}{\pdf@filesize}
%    \begin{macrocode}
\def\pdf@filesize#1{%
  \directlua0{%
    oberdiek.pdftexcmds.filesize("\luaescapestring{#1}")%
  }%
}
%    \end{macrocode}
%    \end{macro}
%    \begin{macro}{\pdf@filemoddate}
%    \begin{macrocode}
\def\pdf@filemoddate#1{%
  \directlua0{%
    oberdiek.pdftexcmds.filemoddate("\luaescapestring{#1}")%
  }%
}
%    \end{macrocode}
%    \end{macro}
%    \begin{macro}{\pdf@filedump}
%    \begin{macrocode}
\def\pdf@filedump#1#2#3{%
  \directlua0{%
    oberdiek.pdftexcmds.filedump("\luaescapestring{\number#1}",%
        "\luaescapestring{\number#2}",%
        "\luaescapestring{#3}")%
  }%
}%
%    \end{macrocode}
%    \end{macro}
%    \begin{macro}{\pdf@mdfivesum}
%    \begin{macrocode}
\long\def\pdf@mdfivesum#1{%
  \directlua0{%
    oberdiek.pdftexcmds.mdfivesum("\luaescapestring{#1}", "byte")%
  }%
}%
%    \end{macrocode}
%    \end{macro}
%    \begin{macro}{\pdf@mdfivesumnative}
%    \begin{macrocode}
\long\def\pdf@mdfivesumnative#1{%
  \directlua0{%
    oberdiek.pdftexcmds.mdfivesum("\luaescapestring{#1}")%
  }%
}%
%    \end{macrocode}
%    \end{macro}
%    \begin{macro}{\pdf@filemdfivesum}
%    \begin{macrocode}
\def\pdf@filemdfivesum#1{%
  \directlua0{%
    oberdiek.pdftexcmds.filemdfivesum("\luaescapestring{#1}")%
  }%
}%
%    \end{macrocode}
%    \end{macro}
%
%    \begin{macro}{\pdf@shellescape}
%    \begin{macrocode}
\def\pdf@shellescape{%
  \directlua0{%
    oberdiek.pdftexcmds.shellescape()%
  }%
}
%    \end{macrocode}
%    \end{macro}
%
%    \begin{macro}{\pdf@system}
%    \begin{macrocode}
\def\pdf@system#1{%
  \directlua0{%
    oberdiek.pdftexcmds.system("\luaescapestring{#1}")%
  }%
}
%    \end{macrocode}
%    \end{macro}
%
%    \begin{macro}{\pdf@lastsystemstatus}
%    \begin{macrocode}
\def\pdf@lastsystemstatus{%
  \directlua0{%
    oberdiek.pdftexcmds.lastsystemstatus()%
  }%
}
%    \end{macrocode}
%    \end{macro}
%    \begin{macro}{\pdf@lastsystemexit}
%    \begin{macrocode}
\def\pdf@lastsystemexit{%
  \directlua0{%
    oberdiek.pdftexcmds.lastsystemexit()%
  }%
}
%    \end{macrocode}
%    \end{macro}
%
%    \begin{macro}{\pdf@pipe}
%    Check availability of |io.popen| first.
%    \begin{macrocode}
\ifnum0%
    \directlua0{%
      if io.popen then %
        tex.write("1")%
      end%
    }%
    =1 %
  \def\pdf@pipe#1{%
    \the\expandafter\pdftexcmds@toks
    \directlua0{%
      oberdiek.pdftexcmds.toks="pdftexcmds@toks"%
      oberdiek.pdftexcmds.pipe("\luaescapestring{#1}")%
    }%
  }%
\fi
%    \end{macrocode}
%    \end{macro}
%
%    \begin{macrocode}
\pdftexcmds@AtEnd
%</package>
%    \end{macrocode}
%
% \subsection{Lua module}
%
%    \begin{macrocode}
%<*lua>
%    \end{macrocode}
%
%    \begin{macrocode}
module("oberdiek.pdftexcmds", package.seeall)
local systemexitstatus
function strcmp(A, B)
  if A == B then
    tex.write("0")
  elseif A < B then
    tex.write("-1")
  else
    tex.write("1")
  end
end
local function utf8_to_byte(str)
  local i = 0
  local n = string.len(str)
  local t = {}
  while i < n do
    i = i + 1
    local a = string.byte(str, i)
    if a < 128 then
      table.insert(t, string.char(a))
    else
      if a >= 192 and i < n then
        i = i + 1
        local b = string.byte(str, i)
        if b < 128 or b >= 192 then
          i = i - 1
        elseif a == 194 then
          table.insert(t, string.char(b))
        elseif a == 195 then
          table.insert(t, string.char(b + 64))
        end
      end
    end
  end
  return table.concat(t)
end
function escapehex(str, mode)
  if mode == "byte" then
    str = utf8_to_byte(str)
  end
  tex.write((string.gsub(str, ".",
    function (ch)
      return string.format("%02X", string.byte(ch))
    end
  )))
end
%    \end{macrocode}
%    See procedure |unescapehex| in file \xfile{utils.c} of \pdfTeX.
%    Caution: |tex.write| ignores leading spaces.
%    \begin{macrocode}
function unescapehex(str, mode)
  local a = 0
  local first = true
  local result = {}
  for i = 1, string.len(str), 1 do
    local ch = string.byte(str, i)
    if ch >= 48 and ch <= 57 then
      ch = ch - 48
    elseif ch >= 65 and ch <= 70 then
      ch = ch - 55
    elseif ch >= 97 and ch <= 102 then
      ch = ch - 87
    else
      ch = nil
    end
    if ch then
      if first then
        a = ch * 16
        first = false
      else
        table.insert(result, a + ch)
        first = true
      end
    end
  end
  if not first then
    table.insert(result, a)
  end
  if mode == "byte" then
    local utf8 = {}
    for i, a in ipairs(result) do
      if a < 128 then
        table.insert(utf8, a)
      else
        if a < 192 then
          table.insert(utf8, 194)
          a = a - 128
        else
          table.insert(utf8, 195)
          a = a - 192
        end
        table.insert(utf8, a + 128)
      end
    end
    result = utf8
  end
  tex.settoks(toks, string.char(unpack(result)))
end
%    \end{macrocode}
%    See procedure |escapestring| in file \xfile{utils.c} of \pdfTeX.
%    \begin{macrocode}
function escapestring(str, mode)
  if mode == "byte" then
    str = utf8_to_byte(str)
  end
  tex.write((string.gsub(str, ".",
    function (ch)
      local b = string.byte(ch)
      if b < 33 or b > 126 then
        return string.format("\\%.3o", b)
      end
      if b == 40 or b == 41 or b == 92 then
        return "\\" .. ch
      end
%    \end{macrocode}
%    Lua 5.1 returns the match in case of return value |nil|.
%    \begin{macrocode}
      return nil
    end
  )))
end
%    \end{macrocode}
%    See procedure |escapename| in file \xfile{utils.c} of \pdfTeX.
%    \begin{macrocode}
function escapename(str, mode)
  if mode == "byte" then
    str = utf8_to_byte(str)
  end
  tex.write((string.gsub(str, ".",
    function (ch)
      local b = string.byte(ch)
      if b == 0 then
%    \end{macrocode}
%    In Lua 5.0 |nil| could be used for the empty string,
%    But |nil| returns the match in Lua 5.1, thus we use
%    the empty string explicitly.
%    \begin{macrocode}
        return ""
      end
      if b <= 32 or b >= 127
          or b == 35 or b == 37 or b == 40 or b == 41
          or b == 47 or b == 60 or b == 62 or b == 91
          or b == 93 or b == 123 or b == 125 then
        return string.format("#%.2X", b)
      else
%    \end{macrocode}
%    Lua 5.1 returns the match in case of return value |nil|.
%    \begin{macrocode}
        return nil
      end
    end
  )))
end
function filesize(filename)
  local foundfile = kpse.find_file(filename, "tex", true)
  if foundfile then
    local size = lfs.attributes(foundfile, "size")
    if size then
      tex.write(size)
    end
  end
end
%    \end{macrocode}
%    See procedure |makepdftime| in file \xfile{utils.c} of \pdfTeX.
%    \begin{macrocode}
function filemoddate(filename)
  local foundfile = kpse.find_file(filename, "tex", true)
  if foundfile then
    local date = lfs.attributes(foundfile, "modification")
    if date then
      local d = os.date("*t", date)
      if d.sec >= 60 then
        d.sec = 59
      end
      local u = os.date("!*t", date)
      local off = 60 * (d.hour - u.hour) + d.min - u.min
      if d.year ~= u.year then
        if d.year > u.year then
          off = off + 1440
        else
          off = off - 1440
        end
      elseif d.yday ~= u.yday then
        if d.yday > u.yday then
          off = off + 1440
        else
          off = off - 1440
        end
      end
      local timezone
      if off == 0 then
        timezone = "Z"
      else
        local hours = math.floor(off / 60)
        local mins = math.abs(off - hours * 60)
        timezone = string.format("%+03d'%02d'", hours, mins)
      end
      tex.write(string.format("D:%04d%02d%02d%02d%02d%02d%s",
          d.year, d.month, d.day, d.hour, d.min, d.sec, timezone))
    end
  end
end
function filedump(offset, length, filename)
  length = tonumber(length)
  if length and length > 0 then
    local foundfile = kpse.find_file(filename, "tex", true)
    if foundfile then
      offset = tonumber(offset)
      if not offset then
        offset = 0
      end
      local filehandle = io.open(foundfile, "r")
      if filehandle then
        if offset > 0 then
          filehandle:seek("set", offset)
        end
        local dump = filehandle:read(length)
        escapehex(dump)
      end
    end
  end
end
function mdfivesum(str, mode)
  if mode == "byte" then
    str = utf8_to_byte(str)
  end
  escapehex(md5.sum(str))
end
function filemdfivesum(filename)
  local foundfile = kpse.find_file(filename, "tex", true)
  if foundfile then
    local filehandle = io.open(foundfile, "r")
    if filehandle then
      local contents = filehandle:read("*a")
      escapehex(md5.sum(contents))
    end
  end
end
function shellescape()
  if os.execute then
    tex.write("1")
  else
    tex.write("0")
  end
end
function system(cmdline)
  systemexitstatus = nil
  texio.write_nl("log", "system(" .. cmdline .. ") ")
  if os.execute then
    texio.write("log", "executed.")
    systemexitstatus = os.execute(cmdline)
  else
    texio.write("log", "disabled.")
  end
end
function lastsystemstatus()
  local result = tonumber(systemexitstatus)
  if result then
    local x = math.floor(result / 256)
    tex.write(result - 256 * math.floor(result / 256))
  end
end
function lastsystemexit()
  local result = tonumber(systemexitstatus)
  if result then
    tex.write(math.floor(result / 256))
  end
end
function pipe(cmdline)
  local result
  systemexitstatus = nil
  texio.write_nl("log", "pipe(" .. cmdline ..") ")
  if io.popen then
    texio.write("log", "executed.")
    local handle = io.popen(cmdline, "r")
    if handle then
      result = handle:read("*a")
      handle:close()
    end
  else
    texio.write("log", "disabled.")
  end
  if result then
    tex.settoks(toks, result)
  else
    tex.settoks(toks, "")
  end
end
%    \end{macrocode}
%    \begin{macrocode}
%</lua>
%    \end{macrocode}
%
% \section{Test}
%
% \subsection{Catcode checks for loading}
%
%    \begin{macrocode}
%<*test1>
%    \end{macrocode}
%    \begin{macrocode}
\catcode`\{=1 %
\catcode`\}=2 %
\catcode`\#=6 %
\catcode`\@=11 %
\expandafter\ifx\csname count@\endcsname\relax
  \countdef\count@=255 %
\fi
\expandafter\ifx\csname @gobble\endcsname\relax
  \long\def\@gobble#1{}%
\fi
\expandafter\ifx\csname @firstofone\endcsname\relax
  \long\def\@firstofone#1{#1}%
\fi
\expandafter\ifx\csname loop\endcsname\relax
  \expandafter\@firstofone
\else
  \expandafter\@gobble
\fi
{%
  \def\loop#1\repeat{%
    \def\body{#1}%
    \iterate
  }%
  \def\iterate{%
    \body
      \let\next\iterate
    \else
      \let\next\relax
    \fi
    \next
  }%
  \let\repeat=\fi
}%
\def\RestoreCatcodes{}
\count@=0 %
\loop
  \edef\RestoreCatcodes{%
    \RestoreCatcodes
    \catcode\the\count@=\the\catcode\count@\relax
  }%
\ifnum\count@<255 %
  \advance\count@ 1 %
\repeat

\def\RangeCatcodeInvalid#1#2{%
  \count@=#1\relax
  \loop
    \catcode\count@=15 %
  \ifnum\count@<#2\relax
    \advance\count@ 1 %
  \repeat
}
\expandafter\ifx\csname LoadCommand\endcsname\relax
  \def\LoadCommand{\input pdftexcmds.sty\relax}%
\fi
\def\Test{%
  \RangeCatcodeInvalid{0}{47}%
  \RangeCatcodeInvalid{58}{64}%
  \RangeCatcodeInvalid{91}{96}%
  \RangeCatcodeInvalid{123}{255}%
  \catcode`\@=12 %
  \catcode`\\=0 %
  \catcode`\{=1 %
  \catcode`\}=2 %
  \catcode`\#=6 %
  \catcode`\[=12 %
  \catcode`\]=12 %
  \catcode`\%=14 %
  \catcode`\ =10 %
  \catcode13=5 %
  \LoadCommand
  \RestoreCatcodes
}
\Test
\csname @@end\endcsname
\end
%    \end{macrocode}
%    \begin{macrocode}
%</test1>
%    \end{macrocode}
%
% \section{Installation}
%
% \subsection{Download}
%
% \paragraph{Package.} This package is available on
% CTAN\footnote{\url{ftp://ftp.ctan.org/tex-archive/}}:
% \begin{description}
% \item[\CTAN{macros/latex/contrib/oberdiek/pdftexcmds.dtx}] The source file.
% \item[\CTAN{macros/latex/contrib/oberdiek/pdftexcmds.pdf}] Documentation.
% \end{description}
%
%
% \paragraph{Bundle.} All the packages of the bundle `oberdiek'
% are also available in a TDS compliant ZIP archive. There
% the packages are already unpacked and the documentation files
% are generated. The files and directories obey the TDS standard.
% \begin{description}
% \item[\CTAN{macros/latex/contrib/oberdiek/oberdiek-tds.zip}]
% \end{description}
% \emph{TDS} refers to the standard ``A Directory Structure
% for \TeX\ Files'' (\CTAN{tds/tds.pdf}). Directories
% with \xfile{texmf} in their name are usually organized this way.
%
% \subsection{Bundle installation}
%
% \paragraph{Unpacking.} Unpack the \xfile{oberdiek-tds.zip} in the
% TDS tree (also known as \xfile{texmf} tree) of your choice.
% Example (linux):
% \begin{quote}
%   |unzip oberdiek-tds.zip -d ~/texmf|
% \end{quote}
%
% \paragraph{Script installation.}
% Check the directory \xfile{TDS:scripts/oberdiek/} for
% scripts that need further installation steps.
% Package \xpackage{attachfile2} comes with the Perl script
% \xfile{pdfatfi.pl} that should be installed in such a way
% that it can be called as \texttt{pdfatfi}.
% Example (linux):
% \begin{quote}
%   |chmod +x scripts/oberdiek/pdfatfi.pl|\\
%   |cp scripts/oberdiek/pdfatfi.pl /usr/local/bin/|
% \end{quote}
%
% \subsection{Package installation}
%
% \paragraph{Unpacking.} The \xfile{.dtx} file is a self-extracting
% \docstrip\ archive. The files are extracted by running the
% \xfile{.dtx} through \plainTeX:
% \begin{quote}
%   \verb|tex pdftexcmds.dtx|
% \end{quote}
%
% \paragraph{TDS.} Now the different files must be moved into
% the different directories in your installation TDS tree
% (also known as \xfile{texmf} tree):
% \begin{quote}
% \def\t{^^A
% \begin{tabular}{@{}>{\ttfamily}l@{ $\rightarrow$ }>{\ttfamily}l@{}}
%   pdftexcmds.sty & tex/generic/oberdiek/pdftexcmds.sty\\
%   oberdiek.pdftexcmds.lua & scripts/oberdiek/oberdiek.pdftexcmds.lua\\
%   pdftexcmds.lua & scripts/oberdiek/pdftexcmds.lua\\
%   pdftexcmds.pdf & doc/latex/oberdiek/pdftexcmds.pdf\\
%   pdftexcmds.dtx & source/latex/oberdiek/pdftexcmds.dtx\\
% \end{tabular}^^A
% }^^A
% \sbox0{\t}^^A
% \ifdim\wd0>\linewidth
%   \begingroup
%     \advance\linewidth by\leftmargin
%     \advance\linewidth by\rightmargin
%   \edef\x{\endgroup
%     \def\noexpand\lw{\the\linewidth}^^A
%   }\x
%   \def\lwbox{^^A
%     \leavevmode
%     \hbox to \linewidth{^^A
%       \kern-\leftmargin\relax
%       \hss
%       \usebox0
%       \hss
%       \kern-\rightmargin\relax
%     }^^A
%   }^^A
%   \ifdim\wd0>\lw
%     \sbox0{\small\t}^^A
%     \ifdim\wd0>\linewidth
%       \ifdim\wd0>\lw
%         \sbox0{\footnotesize\t}^^A
%         \ifdim\wd0>\linewidth
%           \ifdim\wd0>\lw
%             \sbox0{\scriptsize\t}^^A
%             \ifdim\wd0>\linewidth
%               \ifdim\wd0>\lw
%                 \sbox0{\tiny\t}^^A
%                 \ifdim\wd0>\linewidth
%                   \lwbox
%                 \else
%                   \usebox0
%                 \fi
%               \else
%                 \lwbox
%               \fi
%             \else
%               \usebox0
%             \fi
%           \else
%             \lwbox
%           \fi
%         \else
%           \usebox0
%         \fi
%       \else
%         \lwbox
%       \fi
%     \else
%       \usebox0
%     \fi
%   \else
%     \lwbox
%   \fi
% \else
%   \usebox0
% \fi
% \end{quote}
% If you have a \xfile{docstrip.cfg} that configures and enables \docstrip's
% TDS installing feature, then some files can already be in the right
% place, see the documentation of \docstrip.
%
% \subsection{Refresh file name databases}
%
% If your \TeX~distribution
% (\teTeX, \mikTeX, \dots) relies on file name databases, you must refresh
% these. For example, \teTeX\ users run \verb|texhash| or
% \verb|mktexlsr|.
%
% \subsection{Some details for the interested}
%
% \paragraph{Attached source.}
%
% The PDF documentation on CTAN also includes the
% \xfile{.dtx} source file. It can be extracted by
% AcrobatReader 6 or higher. Another option is \textsf{pdftk},
% e.g. unpack the file into the current directory:
% \begin{quote}
%   \verb|pdftk pdftexcmds.pdf unpack_files output .|
% \end{quote}
%
% \paragraph{Unpacking with \LaTeX.}
% The \xfile{.dtx} chooses its action depending on the format:
% \begin{description}
% \item[\plainTeX:] Run \docstrip\ and extract the files.
% \item[\LaTeX:] Generate the documentation.
% \end{description}
% If you insist on using \LaTeX\ for \docstrip\ (really,
% \docstrip\ does not need \LaTeX), then inform the autodetect routine
% about your intention:
% \begin{quote}
%   \verb|latex \let\install=y\input{pdftexcmds.dtx}|
% \end{quote}
% Do not forget to quote the argument according to the demands
% of your shell.
%
% \paragraph{Generating the documentation.}
% You can use both the \xfile{.dtx} or the \xfile{.drv} to generate
% the documentation. The process can be configured by the
% configuration file \xfile{ltxdoc.cfg}. For instance, put this
% line into this file, if you want to have A4 as paper format:
% \begin{quote}
%   \verb|\PassOptionsToClass{a4paper}{article}|
% \end{quote}
% An example follows how to generate the
% documentation with pdf\LaTeX:
% \begin{quote}
%\begin{verbatim}
%pdflatex pdftexcmds.dtx
%makeindex -s gind.ist pdftexcmds.idx
%pdflatex pdftexcmds.dtx
%makeindex -s gind.ist pdftexcmds.idx
%pdflatex pdftexcmds.dtx
%\end{verbatim}
% \end{quote}
%
% \begin{History}
%   \begin{Version}{2007/11/11 v0.1}
%   \item
%     First version.
%   \end{Version}
%   \begin{Version}{2007/11/12 v0.2}
%   \item
%     Short description fixed.
%   \end{Version}
%   \begin{Version}{2007/12/12 v0.3}
%   \item
%     Organization of Lua code as module.
%   \end{Version}
% \end{History}
%
% \PrintIndex
%
% \Finale
\endinput
|
% \end{quote}
% Do not forget to quote the argument according to the demands
% of your shell.
%
% \paragraph{Generating the documentation.}
% You can use both the \xfile{.dtx} or the \xfile{.drv} to generate
% the documentation. The process can be configured by the
% configuration file \xfile{ltxdoc.cfg}. For instance, put this
% line into this file, if you want to have A4 as paper format:
% \begin{quote}
%   \verb|\PassOptionsToClass{a4paper}{article}|
% \end{quote}
% An example follows how to generate the
% documentation with pdf\LaTeX:
% \begin{quote}
%\begin{verbatim}
%pdflatex pdftexcmds.dtx
%makeindex -s gind.ist pdftexcmds.idx
%pdflatex pdftexcmds.dtx
%makeindex -s gind.ist pdftexcmds.idx
%pdflatex pdftexcmds.dtx
%\end{verbatim}
% \end{quote}
%
% \begin{History}
%   \begin{Version}{2007/11/11 v0.1}
%   \item
%     First version.
%   \end{Version}
%   \begin{Version}{2007/11/12 v0.2}
%   \item
%     Short description fixed.
%   \end{Version}
%   \begin{Version}{2007/12/12 v0.3}
%   \item
%     Organization of Lua code as module.
%   \end{Version}
% \end{History}
%
% \PrintIndex
%
% \Finale
\endinput
|
% \end{quote}
% Do not forget to quote the argument according to the demands
% of your shell.
%
% \paragraph{Generating the documentation.}
% You can use both the \xfile{.dtx} or the \xfile{.drv} to generate
% the documentation. The process can be configured by the
% configuration file \xfile{ltxdoc.cfg}. For instance, put this
% line into this file, if you want to have A4 as paper format:
% \begin{quote}
%   \verb|\PassOptionsToClass{a4paper}{article}|
% \end{quote}
% An example follows how to generate the
% documentation with pdf\LaTeX:
% \begin{quote}
%\begin{verbatim}
%pdflatex pdftexcmds.dtx
%makeindex -s gind.ist pdftexcmds.idx
%pdflatex pdftexcmds.dtx
%makeindex -s gind.ist pdftexcmds.idx
%pdflatex pdftexcmds.dtx
%\end{verbatim}
% \end{quote}
%
% \begin{History}
%   \begin{Version}{2007/11/11 v0.1}
%   \item
%     First version.
%   \end{Version}
%   \begin{Version}{2007/11/12 v0.2}
%   \item
%     Short description fixed.
%   \end{Version}
%   \begin{Version}{2007/12/12 v0.3}
%   \item
%     Organization of Lua code as module.
%   \end{Version}
% \end{History}
%
% \PrintIndex
%
% \Finale
\endinput
|
% \end{quote}
% Do not forget to quote the argument according to the demands
% of your shell.
%
% \paragraph{Generating the documentation.}
% You can use both the \xfile{.dtx} or the \xfile{.drv} to generate
% the documentation. The process can be configured by the
% configuration file \xfile{ltxdoc.cfg}. For instance, put this
% line into this file, if you want to have A4 as paper format:
% \begin{quote}
%   \verb|\PassOptionsToClass{a4paper}{article}|
% \end{quote}
% An example follows how to generate the
% documentation with pdf\LaTeX:
% \begin{quote}
%\begin{verbatim}
%pdflatex pdftexcmds.dtx
%makeindex -s gind.ist pdftexcmds.idx
%pdflatex pdftexcmds.dtx
%makeindex -s gind.ist pdftexcmds.idx
%pdflatex pdftexcmds.dtx
%\end{verbatim}
% \end{quote}
%
% \begin{History}
%   \begin{Version}{2007/11/11 v0.1}
%   \item
%     First version.
%   \end{Version}
%   \begin{Version}{2007/11/12 v0.2}
%   \item
%     Short description fixed.
%   \end{Version}
%   \begin{Version}{2007/12/12 v0.3}
%   \item
%     Organization of Lua code as module.
%   \end{Version}
% \end{History}
%
% \PrintIndex
%
% \Finale
\endinput
