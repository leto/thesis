\chapter{CHAPTER FOUR: RESULTS} \label{chapter_4}

In this thesis, we apply a recently developed technique to
comprehensively categorize all possible families of solitary wave
solutions in two models of topical interest.

The models considered are:
\begin{itemize}
\item the Generalized Pochhammer-Chree Equations, which  govern the propagation of longitudinal waves in elastic rods,

and

\item a generalized microstructure PDE.
\end{itemize}

Limited analytic results exist for the occurrence of one family of  solitary
wave solutions for the Microstructure equation and results using a Hamiltonian
formulation have recently been found in the Pochhammer-Chree equations \cite{LiZhang}. Since, as
mentioned above, solitary wave solutions often play a central role in the
long-time evolution of an initial disturbance, we consider such solutions of
both models here (via the normal form approach) within the framework of
reversible systems theory.

Besides confirming the existence of the known family of solitary waves for each
model, of the form
\begin{equation} 
A\left(z\right) = \ell \space \mathrm{sech}^2\left(k z\right)
\end{equation}
we find a continuum of delocalized solitary waves (or homoclinics to
small-amplitude periodic orbits).  On isolated curves in the relevant parameter
region, the delocalized waves reduce to genuine embedded solitons. 
These solitary waves are called delocalized because they have exponentially
small oscillations as $|z|\rightarrow\infty$ and so are not localized in space.
This is often referred to as a soliton in a "sea of radiation."

These curves are defined by the behavior of the four eigenvalues of the characteristic
equation $ \lambda^4 - q \lambda^2 - \epsilon =  0$. Specifically, the 
multiplicity of the eigenvalues change as the parameters vary across these curves.
Thus, these curves define separatrices between vastly different dynamics in 
the traveling wave ODE as well as the original PDE.

One may easily verify that $\lim_{z\rightarrow\pm\infty} A(z) = 0$, therefore
$A(z)$ compromises a homoclinic orbit, since it connects the fixed point $0$ to
itself. The importance of homoclinic orbits in the traveling wave ODE is that
they correspond to soliton pulse solutions of the original PDE \cite{IA}. Iooss
\& P\'erou\`eme have proved that solutions in the traveling wave ODE persist in
the original system \cite{IP} for reversible 1:1 resonance vector fields.  For
the microstructure equation, the new family of solutions occur in regions of
parameter space distinct from the known solitary wave solutions and are thus
entirely new. 

%42
