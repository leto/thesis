\chapter{CHAPTER ONE: INTRODUCTION} \label{chapter_1}

Solitary wave solutions of nonlinear models have become increasingly
important, both as possible information carriers,
as well as organizing centers for the solution dynamics in regimes
where the initial conditions naturally break into stable pulses or
pulse-trains.

The Korteweg \& de Vries (KdV) equation $ u_t + u u_x + \delta^2 u_{xxx} = 0$
was the first nonlinear equation found to admit solitons, first derived in 1895
to describe weakly nonlinear long water waves. Some particular solutions were
known at this time, but general solution method was known.  It was not until
seventy years later until any further progress was made.  Since the numerical
"re-discovery" of solitons in the KdV equation \cite{ZK} in 1965 there has been
intense research in equations that admit soliton solutions.  An analytic soliton
solution to the KdV equation was found in 1967 \cite{GGKM} by quite unique means
and at the time it was not clear whether the method was generally applicable.
%The applications of the KdV equation are ubiquitous because it is a canonical
%equation which describes weakly nonlinear long waves. 

A general principle for associating nonlinear evolution equations with the
eigenvalues of linear operators was discovered in 1968 \cite{Lax}.  Soon after,
solitons were found in an even more fundamental and canonical system, the
nonlinear Schr\"odinger equations $ i \Psi_t + \Psi_{xx} \pm \Psi\|\Psi\|^2 = 0
$ \cite{ZS}. These equations arise in diverse areas because they are canonical
equations governing the modulation of the amplitude $\Psi$ of a weakly nonlinear
wave packets \cite{DJ}.  This led to a comprehensive theory, an extension of
Fourier analysis for nonlinear systems, called the Inverse Scattering Transform
\cite{AKNS}.

These standard techniques for investigating solitary waves of integrable
nonlinear PDEs do not carry over to the non-integrable models which are of
increasing relevance in modern applications. By non-integrable we mean equations
for which an Inverse Scattering Transform does not exist.

Other techniques which have been devised, such as variational ones, and
exponential asymptotics methods, each yield results in certain regimes of the
systems parameters.

In this thesis, we apply a recently developed technique to
comprehensively categorize all possible families of solitary wave
solutions in two models of topical interest.

The models considered are:
\begin{itemize}
\item the Generalized Pochhammer-Chree Equations, which  govern the
propagation of longitudinal waves in elastic rods,
\begin{equation}\label{eq:GPC1}
\left( u - u_{xx} \right)_{tt} - \left( a_1 u + a_2 u^2 + a_3 u^3 \right)_{xx} =0  
\end{equation}
and
\begin{equation}  \label{eq:GPC2} 
\left( u - u_{xx} \right)_{tt} - \left( a_1 u + a_3 u^3 + a_5 u^5 \right)_{xx} =0
\end{equation}

\item a generalized microstructure PDE.
\begin{equation}\label{eq:MS}
v_{tt} - b v_{xx} - \frac{\mu}{2} \left( v^2 \right)_{xx} - \delta \left( \beta v_{tt} - \gamma v_{xx}\right)_{xx} = 0 
\end{equation}
\end{itemize}

The phase space of the traveling-wave equation will be studied, specifically the
homoclinic orbits, which correspond to solitary wave solutions in the original
PDE. A homoclinic orbit is defined as any orbit which connects a fixed point to
itself \cite{Strogatz}.

Limited analytic results exist for the occurrence of one family
of  solitary wave solutions for each of these equations. Since, as
mentioned above, solitary wave solutions often play a central role in
the long-time evolution of an initial disturbance, we consider
such solutions of both models here (via the normal form approach)
within the framework of reversible systems theory. Recently, an
alternative approach using a Hamiltonian formulation has also
been used to analyze the traveling wave ODE \cite{LiZhang}.

Besides confirming the existence of the known family of solitary waves for each
model, we find a continuum of delocalized solitary waves (or homoclinics to
small-amplitude periodic orbits).  On isolated curves in the relevant parameter
region, the delocalized waves reduce to genuine embedded solitons.  These curves
are determined from the regions of different eigenvalue configurations in the
characteristic equation of the traveling wave ODE. An example of this
would be an eigenvalue of multiplicity two splitting into two simple eigenvalues,
or two simple eigenvalues coalescing into an eigenvalue of multiplicity two
as a parameter is varied.

For the generalized Microstructure equation, the new family of solutions occur
in regions of parameter space distinct from the known solitary wave solutions
and are thus entirely new.

Directions for future work, including the dynamics of each family of
solitary waves using exponential asymptotics techniques, are also mentioned.


