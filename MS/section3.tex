\section{Normal form near $C_0$: solitary-wave solutions}

Using \eqref{eq:linode}, the curve $C_0$, corresponding to $\lambda = 0,0,\pm \tilde{ \lambda } $, is given by
\begin{equation}
C_0: { p=0, q > 0 }
\end{equation}

Using \eqref{eq:qdef} implies
\begin{equation}
 \frac{c^2 - b}{\delta\left(\beta c^2 - \gamma\right)} > 0
\end{equation}

Denoting $\phi$ by $y_1$, equation \eqref{eq:ode} may be written as the system
\begin{subequations}\label{eq:system}
\begin{eqnarray}
\frac{d y_1 }{d z} &=& y_2 \\
\frac{d y_2 }{d z} &=& y_3 \\
\frac{d y_3 }{d z} &=& y_4 \\
\frac{d y_4 }{d z} &=& q y_3 - p y_1 - \left(\Delta_1 y_2^2 + b \Delta_1 y_1 y_3 \right)
\end{eqnarray}
\end{subequations}

We wish to rewrite this as a first order reversible system in order to invoke the relevant theory \cite{IA}. 
To that end, defining  $Y=\left<y_1,y_2,y_3,y_4\right>^T$, equation \eqref{eq:system} may be written 

\begin{equation}\label{eq:bilinear}
\frac{ dY }{ dz } = L_{pq} Y - F_2(Y,Y) \end{equation}

where 
\begin{equation}
L_{pq} = \left( 
\begin{array}{cccc}
0&1&0&0\\
q/3&0&1&0\\
0&q/3&0&1\\
q^2 - p &0&q/3&0 \end{array} \right) \end{equation}

Since $p=0$ for \eqref{eq:MS}, we have 
\begin{equation} \label{eq:bilinear2}
 \frac{ dY }{ dz } = L_{0q} Y - F_2(Y,Y) 
\end{equation}
where 
\begin{equation}\label{eq:nonlinear}
F_2(Y,Y) = \left<0,0,0,\Delta_1 y_2^2 + b \Delta_1 y_1 y_3 \right>^T
\end{equation}

Next we calculate the normal form of \eqref{eq:bilinear2} near $C_0$. The procedure is
closely modeled on \cite{IA} and many intermediate steps may be found there. 

\subsection{ Near $C_0$ }
Near $C_0$ the dynamics reduce to a 2-D Center Manifold
\begin{equation}\label{eq:c0cm}
 Y = A \zeta_0 + B \zeta_1 + \Psi(\epsilon,A,B)
\end{equation}
and the corresponding normal form is
\begin{subequations}
\begin{eqnarray}\label{eq:c0nf}
\frac{dA}{dz} &=& B \\
\frac{dB}{dz} &=& b \epsilon A + \tilde{c} A^2
\end{eqnarray}
\end{subequations}
Here,
\begin{equation}
\epsilon = \left( \frac{q^2}{9} - p\right) - \left(\frac{q}{3}\right)^2 = - p 
\end{equation}
measures the perturbation around $C_0$, and

\begin{subequations}
\begin{eqnarray}\label{eq:lineareigs}
\zeta_0 &=& \left<1,0,-q/3,0\right>^T\\
\zeta_1 &=& \left<0,1,0,-2 q/3\right>^T 
\end{eqnarray}
\end{subequations}

The linear eigenvalue of \eqref{eq:c0nf} satisfies 
\begin{equation}\label{eq:lineig}
\lambda^2 = b \epsilon 
\end{equation}
The characteristic equation of the linear part of 
\eqref{eq:bilinear2} is 
\begin{equation}\label{eq:charlinear}
\lambda^4 - q \lambda^2 - \epsilon =  0 
\end{equation}
Hence, the eigenvalues near zero ( the Center Manifold ) satisfy $\lambda^4 \ll \lambda^2$ and hence 
\begin{equation}\label{eq:lindominant}
\lambda^2 \sim -\frac{\epsilon}{q}
\end{equation}
Matching \eqref{eq:lineig} and \eqref{eq:lindominant} 
\begin{equation}
b = - \frac{1}{q}
\end{equation}
and only the nonlinear coefficient $\tilde{c}$ remains to be determined in the normal form \eqref{eq:c0nf}.

In order to determine $\tilde{c}$ (the coefficient of $A^2$) in \eqref{eq:c0nf} we calculate $\frac{dY}{dz}$ in two ways and match the 
$\mathcal{O}(A^2)$ terms. 

To this end, using the standard 'suspension' trick of treating the perturbation parameter $\epsilon$ as a variable, we expand the function
$\Psi$ in \eqref{eq:c0cm} as 

\begin{equation}\label{eq:psiexp}
\Psi(\epsilon,A,B) = \epsilon A \Psi_{10}^1 + \epsilon B \Psi_{01}^1 + A^2 \Psi_{20}^0 + A B \Psi_{11}^0 + B^2 \Psi_{02}^0 + \cdots
\end{equation}
where the subscripts denote powers of $A$ and $B$, respectively, and the superscript denotes the power of $\epsilon$. 

In the first way of computing $dY/dz$, we take
the $z$ derivative of \eqref{eq:c0cm} ( using \eqref{eq:c0nf} and \eqref{eq:psiexp} ). 
The coefficient of $A^2$ in the resulting expression is $\tilde{c} \zeta_1 $. In the second way of computing $dY/dz$, use \eqref{eq:c0cm} and \eqref{eq:psiexp} in \eqref{eq:bilinear}. The coefficient of $A^2$ in the resulting expression is 
$ L_{0,q} \Psi_{20}^0 - F_2\left(\zeta_0,\zeta_0\right)$.  Hence
\begin{equation}\label{eq:A2coef}
 \tilde{c} \zeta_1 = L_{0q} \Psi_{20}^0 - F_2(\zeta_0,\zeta_0) \end{equation}

Using \eqref{eq:lineareigs} and \eqref{eq:nonlinear} and denoting $\Psi_{20}^0 = \left<x_1,x_2,x_3,x_4\right>$ in \eqref{eq:A2coef} yields the equations

\begin{subequations}
\begin{eqnarray}
0 &=& x_2 \\
\tilde{c} &=& \frac{q}{3} x_1 + x_3 \label{eq:A2coefb}\\
0 &=& \frac{q}{3} x_2 + x_4 \implies x_4 = 0
\textrm{ using \eqref{eq:A2coefb} }
\end{eqnarray}
\end{subequations}
and
\begin{equation}
-\frac{2q}{3} \tilde{c} = \frac{q}{3}\left(\frac{q}{3} x_1 + x_3 \right) + \frac{2q}{3} = \frac{q}{3} \tilde{c} + \frac{b \Delta_1 }{3} 
\textrm{ using \eqref{eq:A2coefb} }
\end{equation}
Hence we obtain 
\begin{equation}
\tilde{c} = - \frac{b \Delta_1}{3} 
\end{equation}
 
Therefore, the normal form for \eqref{eq:MS} near $C_0$ is
\begin{subequations}
\begin{eqnarray}
\frac{dA}{dz} &=& B \\
\frac{dB}{dz} &=& -\frac{\epsilon}{q} A - \frac{ b \Delta_1}{3}  A^2
\end{eqnarray}
\end{subequations}

The normal form \eqref{eq:c0nf} admits a homoclinic solution (near $C_0$) of the form 
\begin{equation} \label{eq:soliton1}
A\left(z\right) = \ell \space sech^2\left(k z\right)
\end{equation}

with 
\begin{subequations} 
\begin{eqnarray}
k &=& \sqrt{\frac{-\epsilon}{4q}} \label{eq:keq} \\
\ell &=& \frac{ 6 k^2 }{ b \Delta_1 } 
\end{eqnarray}
\end{subequations}


Hence, since $\epsilon = - p $, and the curve $C_0$ corresponds to $p=0,q>0$, solitary waves of the 
form \eqref{eq:soliton1} exist in the vicinity of $C_0$ for 

\begin{equation}
p > 0, q > 0 
\end{equation}
which implies that $  \frac{c^2 - b}{\delta\left(\beta c^2 - \gamma\right)} > 0 $

(such that $k$ in \eqref{eq:keq} is real.)  As mentioned in section 2, one may show the persistence
of this homoclinic solution in the original travelling wave ODE \eqref{eq:linode}. Thus, we have 
demonstrated the existence of solitary waves of \eqref{eq:MS} for $p=0^+, q>0$. 

