\section{Solitary waves; local bifurcations}

Solitary waves of \eqref{eq:MS} of the form 
$u(x,t) = \phi\left(x - c t\right) = \phi\left(z\right)$
 satisfy the fourth-order travelling wave ODE
\begin{equation} \label{eq:ode} \phi_{zzzz} - q \phi_{zz} + p \phi = \mathcal{N}[\phi]
\end{equation}

where the notation $\mathcal{N[\phi]}$ means that the operator $\mathcal{N}$ operates on $\phi$ and all of it's derivatives, and 
\begin{equation}
\mathcal{N}\left[\phi\right] = -\Delta_1 \phi_z^2 - b \Delta_1 \phi \phi_{zz}
\end{equation}

where 
\begin{subequations}
\begin{eqnarray}
\Delta_1 &=& \frac{\mu}{ \delta\left( \beta c^2 - \gamma\right) } \\
p &=& 0 
\end{eqnarray}
\end{subequations}

The linearized system corresponding to \eqref{eq:ode}
\begin{equation}
 \label{eq:linode} \phi_{zzzz} - q \phi_{zz} + p \phi = 0
\end{equation}
has a fixed point $\phi = \phi_z = \phi_{zz} = \phi_{zzz} = 0 $.

Solutions $\phi = k e^{\lambda x}$ satisfy the characteristic equation
$\lambda^4 - q \lambda^2 + p = 0 $ from which one may deduce that the structure
of the eigenvalues is distinct in two regions of $\left(p,q\right)$-space.
Since $p=0$ we have only two possible regions of eigenvalues.  We denote $C_0$
as the positive $q$ axis and $C_1$ the negative $q$-axis.  On $C_0$ the
eigenvalues have the structure $\lambda_{1-4} = 0,0,\pm \lambda$, ($\lambda \in
\mathbb{R}$) while on $C_1$ we have $\lambda_{1-4} = 0,0,\pm i \omega $,
($\omega \in \mathbb{R} $) .


\subsection{ Near $C_0$ }
Near $C_0$ the dynamics reduce to a 2-D Center Manifold
\begin{equation} Y = A \zeta_0 + B \zeta_1 + \Psi(\epsilon,A,B)
\end{equation}

with 
\begin{subequations}
\begin{eqnarray}
\frac{dA}{dz} &=& B \\
\frac{dB}{dz} &=& b \epsilon A + \tilde{c} A^2
\end{eqnarray}
\end{subequations}
and the eigenvectors are 
$\zeta_0 = \left<1,0,-q/3,0\right>^T, \zeta_1 = \left<0,1,0,-2 q/3\right>^T $ , $\epsilon$ determines the distance from $C_0$ .
A simple dominant balance argument on the linearized equation yields $b=-\frac{1}{q}$.
To determine $\tilde{c}$ we calculate $\frac{dY}{dz}$ in two ways and match the 
$\mathcal{O}(A^2)$ terms. Expanding $\Psi(\epsilon,A,B)$ in series 

\begin{equation}
\Psi(\epsilon,A,B) = \epsilon A \Psi_{10}^1 + \epsilon B \Psi_{01}^1 + A^2 \Psi_{20}^0 + A B \Psi_{11}^0 + B^2 \Psi_{02}^0 + \cdots
\end{equation}

we find \begin{equation} \tilde{c} \zeta_1 = L_{0q} \Psi_{20}^0 - F_2(\zeta_0,\zeta_0) \end{equation}

We find that $F_2(\zeta_0,\zeta_0) = \left<0,0,0,-\frac{ b \Delta_1 }{3} q \right>^T$ which implies $\tilde{c} = - \frac{b \Delta_1}{3} $.

Therefore, the normal form for \eqref{eq:MS} near $C_0$ is
\begin{subequations}
\begin{eqnarray}
\frac{dA}{dz} &=& B \\
\frac{dB}{dz} &=& -\frac{\epsilon}{q} A - \frac{ b \Delta_1}{3}  A^2
\end{eqnarray}
\end{subequations}
