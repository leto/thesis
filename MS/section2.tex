\section{Solitary waves; local bifurcations}

Solitary waves of \eqref{eq:MS} of the form 
$u(x,t) = \phi\left(x - c t\right) = \phi\left(z\right)$
 satisfy the fourth-order travelling wave ODE
\begin{equation} \label{eq:ode} \phi_{zzzz} - q \phi_{zz} + p \phi = \mathcal{N}[\phi]
\end{equation}

where the notation $\mathcal{N[\phi]}$ means that the operator $\mathcal{N}$ operates on $\phi$ and all of it's derivatives, and 
\begin{equation}
\mathcal{N}\left[\phi\right] = -\Delta_1 \phi_z^2 - b \Delta_1 \phi \phi_{zz}
\end{equation}

where 
\begin{subequations}
\begin{eqnarray}
\Delta_1 &=& \frac{\mu}{ \delta\left( \beta c^2 - \gamma\right) } \\
p &=& 0 
\end{eqnarray}
\end{subequations}

The linearized system corresponding to \eqref{eq:ode}
\begin{equation}
 \label{eq:ode} \phi_{zzzz} - q \phi_{zz} + p \phi = 0
\end{equation}
has a fixed point $\phi = \phi_z = \phi_{zz} = \phi_{zzz} = 0 $.

Solutions $\phi = k e^{\lambda x}$ satisfy the characteristic equation
$\lambda^4 - q \lambda^2 + p = 0 $ from which one may deduce that the structure of the eigenvalues is distinct in two regions of 
$p,q)$-space. Since $p=0$ we have only two
possible regions of eigenvalues. On $C_0$ the eigenvalues have the structure
We denote $C_0$ as the positive $q$ axis and $C_1$ the negative $q$-axis. 
$\lambda_{1-4} = 0,0,\pm \lambda$, ($\lambda \in \mathbb{R}$) while on $C_1$ we
have $\lambda_{1-4} = 0,0,\pm i \omega $, ($\omega \in \mathbb{R} $) .


\subsection{ Near $C_0$ }
Near $C_0$ the dynamics reduce to a 2-D Center Manifold
\begin{equation} Y = A \zeta_0 + B \zeta_1 + \Psi(\epsilon,A,B)
\end{equation}

with 
\begin{subequations}
\begin{eqnarray}
\frac{dA}{dz} &=& B \\
\frac{dB}{dz} &=& b \epsilon A + \tilde{c} A^2
\end{eqnarray}
\end{subequations}
and the eigenvectors are $\zeta_0 = \left<1,0,-q/3,0\right>^T, \zeta_1 = \left<0,1,0,-2 q/3\right>^T $ , $\epsilon$ determines the distance from $C_0$ .
A simple dominant balance argument on the linearized equation yields $b=-\frac{1}{q}$.
To determine $\tilde{c}$ we calculate $\frac{dY}{dz}$ in two ways and match the $\mathcal{O}(A^2)$ terms. Expanding $\Psi(\epsilon,A,B)$ in series 

\begin{equation}
\Psi(\epsilon,A,B) = \epsilon A \Psi_{10}^1 + \epsilon B \Psi_{01}^1 + A^2 \Psi_{20}^0 + A B \Psi_{11}^0 + B^2 \Psi_{02}^0 + \cdots
\end{equation}

we find \begin{equation} \tilde{c} \zeta_1 = L_{0q} \Psi_{20}^0 - F_2(\zeta_0,\zeta_0) \end{equation}

We find that $F_2(\zeta_0,\zeta_0) = \left<0,0,0,-\frac{ b \Delta_1 }{3} q \right>^T$ which implies $\tilde{c} = - \frac{b \Delta_1}{3} $.

Therefore, the normal form for the MS PDE near $C_0$ is
\begin{subequations}
\begin{eqnarray}
\frac{dA}{dz} &=& B \\
\frac{dB}{dz} &=& -\frac{\epsilon}{q} A - \frac{ b \Delta_1}{3}  A^2
\end{eqnarray}
\end{subequations}
\subsection{ Near $C_1$ }
Near $C_1$ the dynamics reduce to a 4-D Center Manifold
\begin{equation} Y = A \zeta_0 + B \zeta_0 + C \zeta_+ + \bar{C} \zeta_- + \Psi(\epsilon,A,B,C,\bar{C})
\end{equation}

with 
\begin{subequations}
\begin{eqnarray}
\frac{dA}{dz} &=& B \\ \label{eq:aq}
\frac{dB}{dz} &=& \bar{\nu} A + b_* A^2 + c_* \left|C\right|^2 \\ \label{eq:bq}
\frac{dC}{dz} &=& i d_0 C + i \bar{\nu} d_1 C + i d_2 A C \label{eq:cq}
\end{eqnarray}
\end{subequations}

where  $\zeta_\pm = \left< 1, \lambda_\pm, 2 q / 3, \frac{\lambda_\pm}{3} q\right>^T$ where 

$\bar{\nu} = -\frac{\epsilon}{q}$, $b_* = -\frac{b \Delta1}{3}$  and $\lambda_\pm = \pm i \sqrt{-q} $. Comparing to the linearized equations gives $d_0 = \sqrt{-q} $.

If we do a dominant balance argument after the change of variable $\tilde{\epsilon} = \sqrt{-3 \mu}$ on the characteristic equation as $\lambda \rightarrow 0 $ then we find $d_1 = \frac{ \sqrt{-3 \mu} }{18 \mu^2 } $. Using $\mu=q/3$ we find $d_1 = \frac{\sqrt{-q}}{2 q^2} $ 

To determine $b_*,c_*$ and $d_2$ we expand 
\begin{equation}\label{eq:psiexp}
\Psi(\epsilon,A,B,C,\bar{C}) = \epsilon A \Psi_{1000}^1 + \epsilon B \Psi_{0100}^1 + A^2 \Psi_{2000}^0 + A B \Psi_{1100}^0 + A C \Psi_{1010}^0 + \epsilon C \Psi_{0010}^1 + \cdots 
\end{equation}

Now we use \eqref{eq:aq}, \eqref{eq:bq}, \eqref{eq:cq} in  \eqref{eq:psiexp} when we calculate $\frac{dY}{dz}$. Matching coefficients yields


\begin{subequations}
\begin{eqnarray}
\mathcal{O}(A^2): &		b_* \zeta_1 &= L_{0q} \Psi_{2000}^0 - F_2(\zeta_0,\zeta_0) \\
\mathcal{O}(\left|C\right|^2):&	c_* \zeta_1 &= L_{0q} \Psi_{0011}^0 -2 F_2(\zeta_+,\zeta_-) \label{eq:cstar} \\
\mathcal{O}(\epsilon C): &-\frac{i}{q} \left(d_1 \zeta_+ +  d_0 \Psi_{0010}^1\right) &= L_{0q} \Psi_{0010}^1 - F_2(\Psi_{0010}^1,\Psi_{0010}^1) \\
\mathcal{O}(A C): 	&i d_2 \zeta_+ + i d_0 \Psi_{1010}^0 &= L_{0q} \Psi_{1010}^0 - 2 F_2(\zeta_0,\zeta_+) \\ \label{eq:AC}
\end{eqnarray}
\end{subequations}
where we have used the fact that $F_2$ is a symmetric bilinear form. Equation \eqref{eq:cstar} is decoupled and yields 
$ c_* = 2 \Delta_1 \left(\frac{1}{q} + \frac{2 b}{3} \right)$ The only coefficient left to determine is $d_2$ which we shall compute now. 

Put $\Psi_{1010}^0 = \left<x_1,x_2,x_3,x_4\right>^T$  into \eqref{eq:AC} implies 

\begin{subequations}
\begin{eqnarray}
i d_2 + i d_0 x_1 &=& x_2 \label{eq:one} \\
- d_0 d_2 + i d_0 x_2 &=& \frac{q}{3} x_1 + x_3 \label{eq:two} \\
\frac{2 i q}{3} d_2 + i d_0 x_3 &=& \frac{q}{3} x_2 + x_4  \label{eq:three} \\
- \frac{q}{3} d_0 d_2 + i d_0 x_4 &=& \frac{q}{3}\left(\frac{q}{3} x_1 + x_3 \right) - \frac{ 2 b q \Delta_1} {3} \label{eq:four}
\end{eqnarray}
\end{subequations}

Using \eqref{eq:one} in \eqref{eq:two} , \eqref{eq:two} in \eqref{eq:four} and using these in \eqref{eq:three} yields $ d_2 = \frac{ b \Delta_1 }{ 3 \sqrt{-q} } $.

Therefore the normal form for the MS PDE near $C_1$ is 

\begin{subequations}
\begin{eqnarray}
\frac{dA}{dz} &=& B \\ \label{eq:normalA}
\frac{dB}{dz} &=& -\frac{\epsilon}{q} A - \frac{b \Delta_1 }{3} A^2 + 2 \Delta_1 \left(\frac{1}{q} + \frac{2 b }{3} \right) \left|C\right|^2 \\ \label{eq:normalB}
\frac{dC}{dz} &=& i \sqrt{-q} C - i \frac{\epsilon \sqrt{-q} }{q^3} C + i \frac{b \Delta_1}{3 \sqrt{-q}} A C \label{eq:normalC}
\end{eqnarray}
\end{subequations}
