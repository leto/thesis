\section{Introduction}

Propagation in microstructured solids is currently a topic of great interest.
This phenomenon has recently been modeled \cite{STE} by an equation

\begin{equation}\label{eq:MS}
v_{tt} - b v_{xx} - \frac{\mu}{2} \left( v^2 \right)_{xx} - \delta \left( \beta v_{tt} - \gamma v_{xx}\right)_{xx} = 0 
\end{equation}

with complicated dispersive and nonlinear terms. Here $b, \mu, \beta, \delta$
and $\gamma$ are dimensionless parameters, $v$ denotes the macrodeformation,
and $x$ and $t$ denote space and time coordinates respectively.  Equation
\eqref{eq:MS} is derived, using the so-called Mindlin Model, in
\cite{JE1},\cite{JE2},\cite{STE}.  It is non-integrable. However, analytic
conditions for the existence of solitary waves of \eqref{eq:MS} have been
derived in \cite{JE2} and \cite{STE}. These references also numerically
construct asymmetric solitary wave solutions of the form $ v\left(x - c t
\right)$ of \eqref{eq:MS}.

More recently (\cite{EP},\cite{EBS}) pulse trains in \eqref{eq:MS} have been
numerically constructed.

In this paper, we initiate a fresh approach to the solitary wave solutions of
\eqref{eq:MS}.  We invoke the theory of reversible systems and the method of
normal forms to categorize the possible solitary waves of \eqref{eq:MS}.  As we
shall see, several families of solitary waves exist in various regions of
parameter space. Our main focus here will be on delineating the possible
occurence and multiplicity of solitary waves in different parameter regimes.
Certain delicate questions relating to specific waves or wave families will
form the basis of future work. 

The remainder of this paper is organized as follows. In Section 2, we delineate
the possible families of solitary waves in various parameter domains and on
certain important curves using the theory of reversible systems. In Section 3
and 4, we next focus on the various transition curves and derive normal forms
in their vicinity to confirm the existence of families of regular or
delocalized solitary-wave solutions in their vicinity.



