\documentclass[10pt]{beamer}

\usepackage{beamerthemeHannover, graphicx, clrscode, amsmath, amssymb, multicol}
\setbeamercolor{sidebar}{use=structure,bg=brown!50!yellow!30!white}

\title{Solitary Wave Families in Two Non-Integrable Models Using Reversible Systems Theory}
\author[J.A. Leto]{Jonathan Leto}
\date{}

\begin{document}
\frame{\titlepage}

\section{Overview}

\frame{
    \frametitle{Overview}
    \begin{itemize}
    \item Definitions
    \item Background
    \item Literature Review
    \item Method of Solution
    \item The Generalized Pochammer-Chree Equations
    \item A Generalized Microstructure Equation
    \end{itemize}
}
\section{Definitions}
\subsection{Reversible System}
\frame{
    \frametitle{Reversible Dynamical System (Iooss \& Adelmeyer)}
    Consider \begin{equation}\label{reversible}
    \frac{dz}{dt} = F\left(z;\mu\right), z \in \mathbb{R}^n, \mu \in \mathbb{R}
    \end{equation} where
    \[F\left(0;0\right)=0\]. 
    If there exits a {\bf unitary map} \[S: \mathbb{R}^n \mapsto \mathbb{R}^n, S \neq I\] such
    that
    \[ F\left(S z; \mu \right) = - S F \left( z;\mu \right)  \forall z, \mu \]
then \eqref{reversible} is a reversible system.
}

\subsection{Solitary Wave}
\frame{
    \frametitle{Families of Solitary Waves}

    Here we will use the term solitary wave or "solitons" in the broadest 
    sense, as a solution to a nonlinear equation which ETC
}

%		\begin{center}
%	\includegraphics[width=10cm,bb=0 0 1530 666]{ocb1.png}
% ocb1.png: 72dpi, width=40.96cm, height=19.90cm, bb=0 0 1161 564
%	\end{center}

\section{Background}
\subsection{Normal Forms}

\frame
{
    \frametitle{Normal Form Theory}
}

\subsection{Bilinear Functions}
\frame {
    \frametitle{Properties of Bilinear Functions} 

}
\section{Literature Review}

\frame
{
  \frametitle{Selected Literature Review}

}

\section{The Models}
\subsection{GPC}
\frame
{
  \frametitle{The Generalized Pochammer-Chree Equations}
}
\subsection{MS}
\frame
{
  \frametitle{ A Generalized Microstructure PDE}
}
\section{Results}
\subsection{GPC}
\frame
{
  \frametitle{Results:The Generalized Pochammer-Chree Equations}
}
\subsection{MS}
\frame
{
  \frametitle{Results: A Generalized Microstructure PDE}
}

\end{document}
