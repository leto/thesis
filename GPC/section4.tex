\section{Normal form near $C_1$: possible solitary-wave solutions}
Using \eqref{eq:linode}, the curve $C_1$, corresponding to $\lambda = 0, 0\pm i \omega$, is given by
\begin{equation}\label{eq:c1}
C_1 : { p = 0, q < 0 }
\end{equation}
Which implies
\begin{equation}
a_1 > c^2
\end{equation}

In order to investigate the possibility of a $ sech^2 $  homoclinic orbit in the neighborhood of $C_1$ and delocalized solitary
waves, we next compute the normal form near $C_1$ following the procedure in \cite{IA}.

Near $C_1$ the dynamics reduce to a 4-D Center Manifold \cite{IA}
Since all the eigenvalues are non-hyperbolic, the Center Manifold has the form (a nonlinear coordinate change \cite{IA})
\begin{equation} \label{eq:c1cm}
Y = A \zeta_0 + B \zeta_0 + C \zeta_+ + \bar{C} \zeta_- + \Psi(\epsilon,A,B,C,\bar{C})
\end{equation}

with  a corresponding four-dimensional normal form
\begin{subequations}
\begin{eqnarray}\label{eq:c1nf}
\frac{dA}{dz} &=& B \\ \label{eq:aq}
\frac{dB}{dz} &=& \bar{\nu} A + b_* A^2 + c_* \left|C\right|^2 \\ \label{eq:bq}
\frac{dC}{dz} &=& i d_0 C + i \bar{\nu} d_1 C + i d_2 A C \label{eq:cq}
\end{eqnarray}
\end{subequations}
Here $C$ is complex, $\bar{C}$ is the complex conjugate of $C$, $\epsilon, \zeta_0, \zeta_1$ are given previously and the two new
complex eigenvectors co-spanning the Center Manifold are
\begin{equation}
\zeta_\pm	 = \left< 1, \lambda_\pm, 2 q / 3, \frac{\lambda_\pm}{3} q\right>^T 
\end{equation}

Using \eqref{eq:bq} and \eqref{eq:c0nf}
\begin{equation}
\bar{\nu} = b \epsilon = -\frac{\epsilon}{q} 
\end{equation}

Also from the characteristic equation \eqref{eq:charlinear}, the two non-zero 
(imaginary) roots are 
\begin{equation}
\lambda^2 = \frac{ q + \sqrt{q^2 + 4 \epsilon } }{2} \approx q \textrm{ for } \epsilon \textrm{ small }
\end{equation}

Hence
\begin{equation}
\lambda = \pm i \sqrt{-q}, q < 0
\end{equation}

Matching this to the linear part of \eqref{eq:cq} ( which corresponds to the imaginary eigenvalues), $\lambda = i d_0 = i \sqrt{-q}$ or 
\begin{equation}
d_0 = \sqrt{-q}
\end{equation}


If we do a dominant balance argument after the change of variable $\epsilon = \sqrt{-3 \alpha}$ on the characteristic equation as $\lambda \rightarrow 0 $ then we find $d_1 = \frac{ \sqrt{-3 \alpha} }{18 \alpha^2 } $. Using $\alpha=q/3$ we find $d_1 = \frac{\sqrt{-q}}{2 q^2} $ 

The remaining undetermined coefficients  in the normal form are the 
coefficients $b_*,c_*$ and $d_2$ 
which correspond to the $A^2, |C|^2$ and $AC$ terms respectively. In 
order to determine them, we follow the same procedure as 
in Section 3 and compute $dY/dz$ is two distinct ways. We expand the
function $\Psi$ as
\begin{equation}\label{eq:psiexp}
\Psi(\epsilon,A,B,C,\bar{C}) = \epsilon A \Psi_{1000}^1 + \epsilon B \Psi_{0100}^1 + A^2 \Psi_{2000}^0 + A B \Psi_{1100}^0 + A C \Psi_{1010}^0 + \epsilon C \Psi_{0010}^1 + \cdots 
\end{equation}
with subscripts denoting powers of $A$, $B$, $C$ and $\bar{C}$, respectively,
and the superscript is the power of $\epsilon$. In the first way of computing
$dY/dz$ is computed by taking the $z$ derivative of \eqref{eq:c1cm} ( 
using \eqref{eq:c1nf} and \eqref{eq:psiexp}) and read off the coefficients
of $A^2, \|C\|^2, C \epsilon$ and $AC$ terms.

In the second way, $dY/dz$ is computed using 
\eqref{eq:c1cm} and \eqref{eq:psiexp} in \eqref{eq:bilinear} 
(with $p=0$ on $C_1$ as given in \eqref{eq:c1})
and the coefficients of  $A$, $B$, $C$ and $\bar{C}$ are once
again read off.

Equating the coefficients of the corresponding terms in the two
separate expressions for $dY/dz$ yields the following equations:

\begin{subequations}
\begin{eqnarray}
\mathcal{O}(A^2): &		b_* \zeta_1 &= L_{0q} \Psi_{2000}^0 - G_{1,2}(\zeta_0,\zeta_0) \\
\mathcal{O}(\left|C\right|^2):&	c_* \zeta_1 &= L_{0q} \Psi_{0011}^0 -2 G_{1,2}(\zeta_+,\zeta_-) \label{eq:cstar} \\
\mathcal{O}(\epsilon C): &-\frac{i}{q} \left(d_1 \zeta_+ +  d_0 \Psi_{0010}^1\right) &= L_{0q} \Psi_{0010}^1 - G_{1,2}(\Psi_{0010}^1,\Psi_{0010}^1) \\
\mathcal{O}(A C): 	&i d_2 \zeta_+ + i d_0 \Psi_{1010}^0 &= L_{0q} \Psi_{1010}^0 - 2 G_{1,2}(\zeta_0,\zeta_+)  \label{eq:AC}
\end{eqnarray}
\end{subequations}
where we have used the fact that $G_{1,2}$ is a symmetric bilinear form. Equation \eqref{eq:cstar} is decoupled and yields 
$ c_* = \frac{8}{c^2}\left( 2 a_3 - a_2 \right)$ for \eqref{eq:GPC1} and 
$ c_* = \frac{1}{c^2}\left( 16 a_3 + \frac{140}{3} a_5 \right)$ for \eqref{eq:GPC2}. The only coefficient left to determine is $d_2$ which we shall compute now. 

Using $\Psi_{1010}^0 = \left<x_1,x_2,x_3,x_4\right>^T$ in \eqref{eq:AC} implies 
\begin{subequations}
\begin{eqnarray}
i d_2 + i d_0 x_1 &=& x_2 \label{eq:one} \\
- d_0 d_2 + i d_0 x_2 &=& \frac{q}{3} x_1 + x_3 \label{eq:two} \\
\frac{2 i q}{3} d_2 + i d_0 x_3 &=& \frac{q}{3} x_2 + x_4  \label{eq:three} \\
- \frac{q}{3} d_0 d_2 + i d_0 x_4 &=& \frac{q}{3}\left(\frac{q}{3} x_1 + x_3 \right) - \frac{ 2 q}{c^2}\left(\frac{7}{2} a_3 - \frac{i}{3} d_0 a_2\right) \label{eq:four}
\end{eqnarray}
\end{subequations}
for \eqref{eq:GPC1} and
\begin{subequations}
\begin{eqnarray}
i d_2 + i d_0 x_1 &=& x_2 \label{eq:one} \\
- d_0 d_2 + i d_0 x_2 &=& \frac{q}{3} x_1 + x_3 \label{eq:two} \\
\frac{2 i q}{3} d_2 + i d_0 x_3 &=& \frac{q}{3} x_2 + x_4  \label{eq:three} \\
- \frac{q}{3} d_0 d_2 + i d_0 x_4 &=& \frac{q}{3}\left(\frac{q}{3} x_1 + x_3 \right) - \frac{2 q}{c^2}\left( \frac{7}{2} a_3 + \frac{32}{3} a_5\right) \label{eq:four}
\end{eqnarray}
\end{subequations}
for \eqref{eq:GPC2}

Using \eqref{eq:one} in \eqref{eq:two} , \eqref{eq:two} in \eqref{eq:four} and using these in \eqref{eq:three} yields 
$ d_2 = \frac{1}{c^2}\left( \frac{7}{2 \sqrt{-q} } a_3 - \frac{i}{3} a_2 \right)$ for \eqref{eq:GPC1} and 
$ d_2 = \frac{1}{\sqrt{-q} c^2}\left( \frac{7}{2 } a_3 + \frac{32}{3} a_5 \right)$  for \eqref{eq:GPC2}. 

Therefore the normal form near $C_1$ is 
\begin{subequations}
\begin{eqnarray}
\frac{dA}{dz} &=& B \\ \label{eq:normalA}
\frac{dB}{dz} &=& -\frac{\epsilon}{q} A - b_* A^2 + \frac{1}{c^2}\left( \frac{7}{2 \sqrt{-q} } a_3 - \frac{i}{3} a_2 \right)  \left|C\right|^2 \\ \label{eq:normalB}
\frac{dC}{dz} &=& i \sqrt{-q} C - i \frac{\sqrt{-q} }{q^3} C\epsilon + i \frac{1}{c^2}\left( \frac{7}{2 \sqrt{-q} } a_3 - \frac{i}{3} a_2 \right)A C \label{eq:normalC}
\end{eqnarray}
\end{subequations}
for \eqref{eq:GPC1} and
\begin{subequations}
\begin{eqnarray}
\frac{dA}{dz} &=& B \\ \label{eq:normalA}
\frac{dB}{dz} &=& -\frac{\epsilon}{q} A - b_* A^2 + \frac{1}{c^2}\left( 16 a_3 + \frac{140}{3} a_5 \right)  \left|C\right|^2 \\ \label{eq:normalB}
\frac{dC}{dz} &=& i \sqrt{-q} C - i \frac{\sqrt{-q} }{q^3} C\epsilon + i \frac{1}{\sqrt{-q} c^2}\left( \frac{7}{2 } a_3 + \frac{32}{3} a_5 \right)A C \label{eq:normalC}
\end{eqnarray}
\end{subequations}
for \eqref{eq:GPC2}.
