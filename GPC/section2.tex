\section{Solitary waves: local bifurcation}

Solitary waves of \eqref{eq:GPC1} and \eqref{eq:GPC2} of the form 
$v(x,t) = \phi\left(x - c t\right) = \phi\left(z\right)$
 satisfy the fourth-order traveling wave ODE
\begin{equation} \label{eq:ode} \phi_{zzzz} - q \phi_{zz} + p \phi = \mathcal{N}_{1,2}[\phi]
\end{equation}

where 
\begin{subequations}
\begin{eqnarray}
\mathcal{N}_1\left[\phi\right] &=& - \frac{1}{c^2}\left[  3 a_3 \left( 2 \phi \phi_z^2 + \phi^2 \phi_{zz} \right) + 2 a_2\left( \phi_{zz} \phi_z + \phi_z^2\right) \right] \\
\mathcal{N}_2\left[\phi\right] &=& - \frac{1}{c^2}\left[ 3 a_3 \left( 2 \phi \phi_z^2 + \phi^2 \phi_{zz}\right) + 5 a_5 \left( 4 \phi^3 \phi_z^2 + \phi^4 \phi_{zz} \right) \right]
\end{eqnarray}
\end{subequations}

\begin{subequations}
\begin{eqnarray}
z &\equiv& x - c t\\
p &\equiv& 0\label{eq:pdef} \\
q &\equiv & 1 - \frac{a_1}{c^2} \label{eq:qdef} \\
\end{eqnarray}
\end{subequations}

Equation \eqref{eq:ode} is invariant under the transformation $ z \mapsto -z $ and is thus a reversible system. In this section we shall
use the theory of reversible systems to characterize the homoclinic orbits to the fixed point of \eqref{eq:ode}, which correspond to pulses
or solitary waves of \eqref{eq:GPC1} and \eqref{eq:GPC2} in various regions of the $(p,q)$ plane.

The linearized system corresponding to \eqref{eq:ode}
\begin{equation}
 \label{eq:linode} \phi_{zzzz} - q \phi_{zz} + p \phi = 0
\end{equation}
has a fixed point \begin{equation}\label{eq:fp} \phi = \phi_z = \phi_{zz} = \phi_{zzz} = 0 \end{equation}

Solutions $\phi = k e^{\lambda x}$ satisfy the characteristic equation
$\lambda^4 - q \lambda^2 + p = 0 $ from which one may deduce that the structure
of the eigenvalues is distinct in two regions of $\left(p,q\right)$-space.
Since $p=0$ we have only two possible regions of eigenvalues.  We denote $C_0$
as the positive $q$ axis and $C_1$ the negative $q$-axis. First we shall 
consider the bounding curves $C_0$ and $C_1$ and their neighborhoods, then we shall discuss the possible
occurrence and multiplicity of homoclinic orbits to \eqref{eq:fp}, corresponding
to pulse solitary waves of \eqref{eq:GPC1} and \eqref{eq:GPC2}, in each region:

\begin{description}
\item[Near $C_0$] 
The eigenvalues have the structure $\lambda_{1-4} = 0,0,\pm \lambda$, ($\lambda \in \mathbb{R}$) and the fixed point
\eqref{eq:fp} is a saddle-focus.
\item[Near $C_1$] 
Here the eigenvalues have the structure $\lambda_{1-4} = 0,0,\pm i \omega $, ($\omega \in \mathbb{R}$) . We will show by analysis of a
four-dimensional normal form in Section 4 that there exists a $\mathrm{sech}^2$ homoclinic orbit near $C_1$.
\end{description}

Having outlined the possible families of orbits homoclinic to the fixed point \eqref{eq:fp} of \eqref{eq:linode},
corresponding to pulse solitary waves of \eqref{eq:GPC1} and \eqref{eq:GPC2}, we now derive normal forms near the transition curves $C_0$ and $C_1$
to confirm the existence of regular or delocalized solitary waves in the corresponding regions of $\left(p,q\right)$ parameter space.

