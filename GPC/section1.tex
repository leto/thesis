\section{Introduction}

The propagation of longitudinal derformation waves in elastic rods is governed
(\cite{LCZ}, \cite{Runz}, \cite{WM}) by the Generalized Pochammer-Chree
Equations:

\begin{equation}\label{eq:GPC1}
\left( u - u_{xx} \right)_{tt} - \left( a_1 u + a_2 u^2 + a_3 u^3 \right)_{xx} =0  
\end{equation}

and

\begin{equation}  \label{eq:GPC2} 
\left( u - u_{xx} \right)_{tt} - \left( a_1 u + a_3 u^3 + a_5 u^5 \right)_{xx} =0
\end{equation}

corresponding to different constitutive relations.

References \cite{LCZ}, \cite{Runz}, \cite{WM} also discuss the primary
references, including derivations and applications of these equations in
various fields. In addition, motivated by experimental and numerical results,
there are derivations of special families of solitary wave solutions by the
extended $Tanh$ method \cite{LCZ}, and other ansatzen \cite{WM}. These extend
earlier solitary wave solutions given by Bogolubsky \cite{Bogo} and Clarkson
et. al \cite{CLVS} for special cases of \eqref{eq:GPC1} and \eqref{eq:GPC2}. In
addition, \cite{Runz} generalizes the existence results in \cite{Sax} for
solitary waves of \eqref{eq:GPC1} and \eqref{eq:GPC2}.  

In this paper, we initiate a fresh approach to the solitary wave solutions of
the Generalized Pochammer-Chree equations \eqref{eq:GPC1} and \eqref{eq:GPC2}.
We invoke the theory of reversible systems and the method of normal forms to
categorize the possible solitary waves of \eqref{eq:GPC1} and \eqref{eq:GPC2}
much more completely than done so far.  As we shall see, several families of
solitary waves exist in various regions of parameter space. Our main focus here
will be on delineating the possible occurence and multiplicity of solitary
waves in different parameter regimes. Certain delicate questions relating to
specific waves or wave families will form the basis of future work. 

The remainder of this paper is organized as follows. In Section 2, we delineate
the possible families of solitary waves in various parameter domains and on
certain important curves using the theory of reversible systems. In Section 3
and 4, we next focus on the various transition curves and derive normal forms
in their vicinity to confirm the existence of families of regular or
delocalized solitary-wave solutions in their vicinity.
