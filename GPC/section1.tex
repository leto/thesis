\section{Introduction}

The propagation of longitudinal derformation waves in elastic rods is governed \cite{LCZ}, \cite{Runz}, \cite{WM})
by the Generalized Pochammer-Chree Equations:

\begin{equation}\label{eq:GPC1}
\left( u - u_{xx} \right)_{tt} - \left( a_1 u + a_2 u^2 + a_3 u^3 \right)_{xx} =0  
\end{equation}

and

\begin{equation}  \label{eq:GPC2} 
\left( u - u_{xx} \right)_{tt} - \left( a_1 u + a_3 u^3 + a_5 u^5 \right)_{xx} =0
\end{equation}

corresponding to different constitutive relations.

We use the theory of reversible systems and the method of normal forms to categorize the possible solitary waves of \eqref{eq:GPC1} 
and \eqref{eq:GPC2}.
As we shall see, several families of solitary waves exist in various regions of parameter space. Our main focus here will be on 
delineating the possible occurence and multiplicity of solitary waves in different parameter regimes. Certain delicate questions
relating to specific waves or wave families will form the basis of future work. 

The remainder of this paper is organized as follows. In Section 2, we delineate the possible families of solitary waves
in various parameter domains and on certain important curves using the theory of reversible systems. In Section 3 and 4, we next
focus on the various transition curves and derive normal forms in their vicinity to confirm the existence of families of 
regular or delocalized solitary-wave solutions in their vicinity.
